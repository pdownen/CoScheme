% This is samplepaper.tex, a sample chapter demonstrating the
% LLNCS macro package for Springer Computer Science proceedings;
% Version 2.21 of 2022/01/12
%

\documentclass[runningheads]{llncs}
%
\usepackage[T1]{fontenc}
% T1 fonts will be used to generate the final print and online PDFs,
% so please use T1 fonts in your manuscript whenever possible.
% Other font encondings may result in incorrect characters.
%
\usepackage{graphicx}
% Used for displaying a sample figure. If possible, figure files should
% be included in EPS format.
\usepackage{xcolor}

% llncs.cls clashes with amsthm.
% Save the LNCS proof environment defined by the class
\let\lncsproof\proof \let\lncsendproof\endproof \let\lncsqed\qed
% Remove the definitions in order to load amsthm
\let\proof\relax\let\endproof\relax
% Load AMS styles
\usepackage{amsmath}
\usepackage{amsthm}
\usepackage{amssymb}
% restore the LNCS class defined proof
\let\proof\lncsproof \let\endproof\lncsendproof \let\qed\lncsqed

\usepackage{stmaryrd}
\usepackage{braket}
\usepackage{proof}

\usepackage{hyperref}
\usepackage{cleveref}

\usepackage{preamble}

% If you use the hyperref package, please uncomment the following two lines
% to display URLs in blue roman font according to Springer's eBook style:
\usepackage{color}
\renewcommand\UrlFont{\color{blue}\rmfamily}
\urlstyle{rm}

\begin{document}
%
\title{CoScheme: Compositional Copatterns in Scheme}
%
%\titlerunning{Abbreviated paper title}
% If the paper title is too long for the running head, you can set
% an abbreviated paper title here
%
\author{Paul Downen\inst{1}\orcidID{0000-0003-0165-9387}}
%
\authorrunning{P. Downen}
% First names are abbreviated in the running head.
% If there are more than two authors, 'et al.' is used.
%
\institute{University of Massachusetts Lowell, Lowell MA 01854, USA \\
\email{Paul\_Downen@uml.edu}}
%
\maketitle              % typeset the header of the contribution
%
\begin{abstract}
The abstract should briefly summarize the contents of the paper in
150--250 words.

\keywords{Codata \and Copatterns \and Scheme \and Macros \and Composition \and Expression Problem.}
\end{abstract}
%
%
%
\section{Introduction}

\section{Key Idea}

\section{Translating Composable Copatterns}

\begin{figure}
\centering
\begin{alignat*}{2}
  \mathit{Term} &\ni{}& M, N
  &::= x
  \mid M ~ N
  \mid \lambda x. M
  \mid K
  \mid \Match M \With \set{\many{P \to N}}
  \mid \Rec x = M
  \\
  \mathit{Pattern} &\ni{}& P
  &::= x
  \mid K ~ \many{P}
\end{alignat*}

\caption{Target language: pure $\lambda$-calculus with pattern-matching and recursion.}
\label{fig:target-syntax}
\end{figure}

\begin{figure}
\centering

\begin{alignat*}{2}
  \mathit{Term} &\ni{}& M, N
  &::= \dots
  \mid \lamstar B
  \mid \Template B
  \mid \Extension O
  \\
  \mathit{Template} &\ni{}& B
  &::= \varepsilon
  \mid O; B
  \mid \Continue x \to M
  \\
  \mathit{Extension} &\ni{}& O
  &::= \varepsilon
  \mid O; O'
  \mid Q[x] ~ O
  \mid \lambda P.~ O
  % \mid \If M ~ O
  \mid \Match P \gets M ~ O
  \mid \Nest O
  \mid \Try x \to B
  \\
  \mathit{Copattern} &\ni{}& Q
  &::= \hole
  \mid Q ~ P
  \\
  \mathit{Pattern} &\ni{}& P
  &::= x
  \mid K ~ \many{P}
\end{alignat*}

Syntactic sugar:
\begin{align*}
  \Else \to M
  &=
  \Continue \_ \to M
  \\
  \Do M
  &=
  \Try \_ \to \Else \to M
  \\
  (= M)
  &=
  \Do M
  \\
  \If M ~ O
  &=
  \Match \True \gets M ~ O
  \\
  \Let x = M ~ O
  &=
  \Match x \gets M ~ O
\end{align*}
\caption{Source language: target extended with nested copatterns,
  self-referential objects, recursion templates, and composable extensions.}
\label{fig:source-syntax}
\end{figure}

\begin{figure}
\centering
Translating new terms:  
\begin{align*}
  \den{\lamstar B}
  &=
  \Rec \mathit{self} = T\den{B} ~ (\lambda x. \mathit{self} ~ x)
  \\
  \den{\Template B}
  &=
  T\den{B}
  \\
  \den{\Extension O}
  &=
  E\den{O}
  \\
  \den{M}
  &=
  \text{by induction}
  &\text{otherwise}
\end{align*}
Translating templates:
\begin{align*}
  T\den{\varepsilon}
  &=
  \mathit{fail}
  \\
  T\den{O; B}
  &=
  E\den{O} ~ T\den{B}
  \\
  T\den{\Continue x \to M}
  &=
  \lambda x. \den{M}
\end{align*}

Translating copattern-matching and pattern-matching functions:
\begin{align*}
  E\den{(Q[x] ~ P) ~ O}
  &=
  E\den{Q[x] ~ (\lambda P. O)}
  \\
  E\den{x ~ O}
  &=
  \lambda x. E\den{O} ~ x
  \\
  E\den{\lambda P. O}
  &=
  E\den{\lambda x. \Match P \gets x ~ O}
  &(\text{if} P \notin \mathit{Variable})
\end{align*}

Translating other extensions:
\begin{align*}
  E\den{\varepsilon}
  &=
  \lambda b. b
  \\
  E\den{O; O'}
  &=
  E\den{O} \comp E\den{O'}
  \\
  E\den{\lambda x. O}
  &=
  \lambda b. \lambda s. (\lambda x. E\den{O} ~ (\lambda s'. b ~ s' ~ x) ~ s)
  \\
  E\den{\Match P \gets M ~ O}
  &=
  \Match \den{M} \With \set{P \to E\den{O}; \_ \to \lambda b. b}
  \\
  E\den{\Nest O}
  &=
  \lambda b. \lambda s. \Rec s' = E\den{O} ~ (\lambda \_. b ~ s) ~ (\lambda x. s' ~ x)
  \\
  E\den{\Try x \to B}
  &=
  \lambda x. T\den{B}
\end{align*}
\caption{Translating copattern-based source code to the target language.}
\label{fig:translation}
\end{figure}


\section{Macro Definition}

\section{Correctness}

\begin{figure}
\centering
  
\begin{alignat*}{2}
  \mathit{Value} &\ni{}& V, W
  &::= x
  \mid \lambda x. M
  \mid K~\many{V}
  \\
  \mathit{EvalCxt} &\ni{}& E
  &::= \hole
  \mid E ~ M
  \mid V ~ E
  \mid \Match E \With \set{\many{P \to N}}
  \mid \Rec x = E
\end{alignat*}

Operational axioms:
\begin{align*}
  (\lambda x. M) ~ V
  &=
  M\subst{x}{V}
  % \\
  % \begin{aligned}
  %   &\Match V \With \\
  %   &\qquad\set{\many[i]{P_i \to N_i}}
  % \end{aligned}
  % &=
  % N_k\subst{BV(P_k)}{\many{W}}
  % &
  % \begin{aligned}
  %   (&\text{if } && V = P_k\subst{BV(P_k)}{\many{W}} \\
  %   &\text{and } &&\forall 1 \leq j < k, \\
  %   &&&\not\exists \many{W'}, V = P_j\subst{BV(P_j)}{\many{W'}})
  % \end{aligned}
  \\
  \begin{aligned}
    &\Match V \With \\
    &\qquad
    \begin{aligned}[t]
    \{~ &P \to N; \\
    &\many{P' \to N'}~\}
    \end{aligned}
  \end{aligned}
  &=
  N\subst{BV(P)}{\many{W}}
  &(\text{if } \exists \many{W},~ V &= P\subst{BV(P)}{\many{W}})
  \\
  \begin{aligned}
    &\Match V \With \\
    &\qquad
    \begin{aligned}[t]
    \{~ &P \to N; \\
    &\many{P' \to N'}~\}
    \end{aligned}
  \end{aligned}
  &=
  \begin{aligned}
    &\Match V \With \\
    &\qquad \set{\many{P' \to N'}}
  \end{aligned}
  &(\text{if}\!\not\exists \many{W},~ V &= P\subst{BV(P)}{\many{W}})
  \\
  \Rec x = V
  &=
  V\subst{x}{(\Rec x = V)}
\end{align*}

Observational axioms:
\begin{align*}
  (\lambda x. E[x]) ~ M
  &=
  E[M]
  \\
  E\left[
    \begin{aligned}
      &\Match M \With \\
      &\qquad\set{\many{P \to N}}
    \end{aligned}
  \right]
  &=
  \begin{aligned}
    &\Match M \With \\
    &\qquad \set{\many{P \to E[N]}}
  \end{aligned}
  &(\text{if } BV(P) \cap FV(E) = \emptyset)
\end{align*}
\caption{Untyped equational axioms of the target language.}
\label{fig:target-equality}
\end{figure}

\begin{figure}
\centering

\begin{align*}
  \mathit{NonRecTemplate} &\ni{}& B_{nr}
  &::= O_{nr}; B_{nr}
  \mid \Else \to M
  \\
  \mathit{NonRecExtension} &\ni{}& O_{nr}
  &::= \varepsilon
  \mid O_{nr}; O'_{nr}
  \mid Q[\_] ~ O_{nr}
  \mid \lambda P. O_{nr}
  \\
  &&&\phantom{:=}
  \mid \Match P \gets M ~ O_{nr}
  \mid \Nest O
  \mid \Try x \to B_{nr}
  \\
  \mathit{RecCxt} &\ni{}& R
  &::= \hole
  \mid R; O
  \mid O; R
  \mid Q[x] ~ R
  \mid \lambda P. R
  \mid \Match P \gets M ~ O
  \mid \Try x \to R
\end{align*}

Identity, associativity, and annihilation laws of composition:
\begin{align*}
  \varepsilon; O &= O = O; \varepsilon
  &
  (O_1; O_2); O_3 &= O_1; (O_2; O_3)
  &
  \Do M; O &= \do M
  \\
  \varepsilon; B &= B
  &
  (O_1; O_2); B &= O_1; (O_2; B)
  &
  \Do M; B &= \do M
\end{align*}

% Decomposing patterns and copatterns:
% \begin{align*}
%   (Q[x] ~ P) ~ O
%   &=
%   Q[x] ~ (\lambda P. O)
%   &
%   \_ ~ O
%   &=
%   O
%   &
%   \lambda P. O
%   &=
%   \lambda x. (\Match P \gets x ~ O)
% \end{align*}

% Factoring out recursion ($x \neq y$ and $x \notin BV(P)$):
% \begin{align*}
%   \lambda y. (x ~ O)
%   &=
%   x ~ (\lambda y. O)
%   &
%   \Match P \gets M ~ (x ~ O)
%   &=
%   x ~ (\Match P \gets M ~ O)
%   \\
%   (x ~ O); O'
%   &=
%   x ~ (O; O')
%   &
%   O; (x ~ O')
%   &=
%   x ~ (O; O')
% \end{align*}

Instantiating templates, extensions, and recursive $\lamstar$:
\begin{align*}
  (\Extension O) ~ V
  &=
  \Template{} (O; \Continue x \to (V ~ x))
  \\
  (\Template R[Q[x] ~ O]) ~ V
  &=
  (\Template R[Q[\_] ~ O\subst{x}{V}]) ~ V
  \\
  (\Template R[\Continue x \to M]) ~ V
  &=
  (\Template R[\Else \to M\subst{x}{V}]) ~ V
  \\
  (\Template B_{nr}) ~ V
  &=
  (\Template B_{nr}) ~ W
  \\
  (\Template \varepsilon) ~ V
  &=
  \mathit{fail}~V
  \\
  \lamstar B
  &=
  (\Template B) ~ (\lambda x. (\lamstar B) ~ x)
\end{align*}

Pattern and copattern matching:
\begin{align*}
  \Match P \gets P\subst{\many{x}}{\many{V}} ~ O
  &=
  O\subst{\many{x}}{\many{V}}
  \\
  \Match{} (K~\many{P}) \gets (K'~\many{V}) ~ O
  &=
  \varepsilon
  \\
  (\Template{} (\lambda P. \Do M); B) ~ V ~ P\subst{\many{x}}{\many{W}}
  &=
  M\subst{\many{x}}{\many{W}}
  \\
  (\Template{} (\lambda (K~\many{x}). O); B) ~ V ~ (K'~\many{V})
  &=
  (\Template B) ~ V ~ (K' ~ \many{V})
  &(\text{if } K \neq K')
  \\
  Q[(\Template{} (Q[y] = M); B) ~ V]\subst{\many{x}}{\many{W}}
  &=
  M\subst{y}{V}\subst{\many{x}}{\many{W}}
  \\
  C[(\Template{} (Q[y] ~ O); B) ~ V]
  &=
  C[(\Template B) ~ V]
  &(\text{if } Q \apart C)
  \\
  (\Template \Continue x \to M) ~ V
  &=
  M\subst{x}{V}
\end{align*}

Copattern context apartness ($Q \apart C$):
\begin{align*}
  \infer
  {\hole ~ \many[i]{P_i} ~ (K~\many{P}) \apart \hole ~ \many[i]{V_i} ~ (K'~\many{W})}
  {K \neq K' & |\many[i]{P_i}| = |\many[i]{V_i}|}
  \qqqquad
  \infer
  {Q ~ P \apart C}
  {Q \apart C}
\end{align*}

\caption{Some equalities of copattern extensions.}
\label{fig:source-equality}
\end{figure}

\begin{proposition}[Soundness]
  The equational axioms given in \cref{fig:source-equality} are sound with
  respect to the translation in \cref{fig:translation},
  \begin{align*}
    M &= M' &&\implies & \den{M} &= \den{M'} \\
    B &= B' &&\implies & T\den{B} &= T\den{B'} \\
    O &= O' &&\implies & E\den{O} &= E\den{O'}
  \end{align*}
  up to the equational theory of the target language in
  \cref{fig:source-equality}.
\end{proposition}

\section{Optimizing Away Administrative Reductions}

\begin{figure}
\centering
Translating new terms:  
\begin{align*}
  I\den{\lamstar B}
  &=
  \Rec \mathit{self} = T\den{B}(\lambda x. \mathit{self} ~ x)
  \\
  I\den{\Template B}
  &=
  T\den{B}()
  \\
  I\den{\Extension O}
  &=
  E\den{O}()
  \\
  I\den{M}
  &=
  \text{by induction}
  &\text{otherwise}
\end{align*}
Translating templates:
\begin{align*}
  T\den{\varepsilon}(\many{V})
  &=
  \mathit{fail}~\many{V}
  \\
  T\den{O; B}(\many{V})
  &=
  E\den{O}(T\den{B}, \many{V})
  \\
  T\den{\Continue x \to M}()
  &=
  \lambda x. I\den{M}
  \\
  T\den{\Continue x \to M}(V)
  &=
  I\den{M}\subst{x}{V}
\end{align*}

Translating copattern-matching and pattern-matching functions:
\begin{align*}
  E\den{(Q[x] ~ P) ~ O}(\many{V})
  &=
  E\den{Q[x] ~ (\lambda P. O)}(\many{V})
  \\
  E\den{x ~ O}()
  &=
  \lambda x. E\den{O}(x)
  \\
  E\den{x ~ O}(W, \many{V})
  &=
  E\den{O}(W, \many{V})\subst{x}{W}
  \\
  E\den{\lambda P. O}(\many{V})
  &=
  E\den{\lambda x. \Match P \gets x ~ O}(\many{V})
  &(\text{if } P \notin \mathit{Variable})
\end{align*}

Translating other extensions:
\begin{align*}
  E\den{\varepsilon}()
  &=
  \lambda b. b
  \\
  E\den{\varepsilon}(W, \many{V})
  &=
  W(\many{V})
  \\
  E\den{O; O'}()
  &=
  E\den{O}() \comp E\den{O'}()
  \\
  E\den{O; O'}(W)
  &=
  E\den{O}() ~ E\den{O'}(W)
  \\
  E\den{O; O'}(W, V)
  &=
  (\lambda s. E\den{O}(s, V)) ~ E\den{O'}(W)
  \\
  E\den{\lambda x. O}()
  &=
  \lambda b. \lambda s. (\lambda x. E\den{O}((\lambda s'. b ~ s' ~ x), s))
  \\
  E\den{\lambda x. O}(W)
  &=
  \lambda s. (\lambda x. E\den{O}((\lambda s'. W(s', x)), s))
  \\
  E\den{\lambda x. O}(W, V)
  &=
  \lambda x. E\den{O}((\lambda s'. W(s', x)), V)
  \\
  E\den{\Match P \gets M ~ O}(\many{V})
  &=
  \Match I\den{M} \With
  \set{P \to E\den{O}(\many{V}); \_ \to E\den{\varepsilon}(\many{V})}
  \\
  E\den{\Nest O}()
  &=
  \lambda b. \lambda s. \Rec s' = E\den{O}((\lambda \_. b ~ s), (\lambda x. s' ~ x))
  \\
  E\den{\Nest O}(W)
  &=
  \lambda s. \Rec s' = E\den{O}((\lambda \_. W(s)), (\lambda x. s' ~ x))
  \\
  E\den{\Nest O}(W, V)
  &=
  \Rec s' = E\den{O}((\lambda \_. W(V)), (\lambda x. s' ~ x))
  \\
  E\den{\Try x \to B}()
  &=
  \lambda x. T\den{B}()
  \\
  E\den{\Try x \to B}(W, \many{V})
  &=
  T\den{B}(\many{V})\subst{x}{W}
\end{align*}

Inlining administrative $\lambda$-abstractions:
\begin{align*}
  (\lambda x. M)(W, \many{V})
  &=
  M(\many{V})\subst{x}{W}
  \\
  M(\many{V})
  &=
  M ~ \many{V}
  &
  \text{otherwise}
\end{align*}

\caption{Inlining version of the translation.}
\label{fig:inline-translation}
\end{figure}

\begin{proposition}
  Up to the equational theory of the target language in
  \cref{fig:target-equality},
  \begin{align*}
    I\den{M} &= \den{M}
    \\
    T\den{B}(\many{V}) &= T\den{B} ~ \many{V}
    \\
    E\den{O}(\many{V}) &= E\den{O} ~ \many{V}
  \end{align*}  
\end{proposition}

\section{Related and Future Work}

\section{Conclusion}

\begin{credits}
\subsubsection{\ackname}
% 
This material is based upon work supported by the National Science Foundation
under Grant No. 2245516.

\subsubsection{\discintname}
%
The authors have no competing interests to declare that are
relevant to the content of this article.
\end{credits}
%
% ---- Bibliography ----
%
% BibTeX users should specify bibliography style 'splncs04'.
% References will then be sorted and formatted in the correct style.
%
\bibliographystyle{splncs04}
\bibliography{refs}

\end{document}
