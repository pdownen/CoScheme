% This is samplepaper.tex, a sample chapter demonstrating the
% LLNCS macro package for Springer Computer Science proceedings;
% Version 2.21 of 2022/01/12
%

\documentclass[runningheads]{llncs}
%
\usepackage[T1]{fontenc}
% T1 fonts will be used to generate the final print and online PDFs,
% so please use T1 fonts in your manuscript whenever possible.
% Other font encondings may result in incorrect characters.
%
\usepackage{graphicx}
% Used for displaying a sample figure. If possible, figure files should
% be included in EPS format.
\usepackage{xcolor}
\usepackage[shortlabels,inline]{enumitem}

% llncs.cls clashes with amsthm.
% Save the LNCS proof environment defined by the class
\let\lncsproof\proof \let\lncsendproof\endproof \let\lncsqed\qed
% Remove the definitions in order to load amsthm
\let\proof\relax\let\endproof\relax
% Load AMS styles
\usepackage{amsmath}
\usepackage{amsthm}
\usepackage{amssymb}
% restore the LNCS class defined proof
\let\proof\lncsproof \let\endproof\lncsendproof \let\qed\lncsqed

\usepackage{thmtools}
\usepackage{stmaryrd}
\usepackage{braket}
\usepackage{proof}

\usepackage{minted}

\usepackage{hyperref}
\usepackage{cleveref}

\usepackage{preamble}

% If you use the hyperref package, please uncomment the following two lines
% to display URLs in blue roman font according to Springer's eBook style:
\usepackage{color}
\renewcommand\UrlFont{\color{blue}\rmfamily}
\urlstyle{rm}

% Unicode characters:
\DeclareUnicodeCharacter{3BB}{$\lambda$}

% Code settings:
\setminted{fontsize=\footnotesize}

\begin{document}
%
\title{CoScheme: Compositional Copatterns in Scheme}
%
%\titlerunning{Abbreviated paper title}
% If the paper title is too long for the running head, you can set
% an abbreviated paper title here
%
\author{
  Paul Downen\inst{1}\orcidID{0000-0003-0165-9387}
  \and \\
  Adriano Corbelino II\inst{1}\orcidID{0000-0002-6014-6189}
}
%
\authorrunning{P. Downen \and A. Corbelino II}
% First names are abbreviated in the running head.
% If there are more than two authors, 'et al.' is used.
%
\institute{
  University of Massachusetts Lowell, Lowell MA 01854, USA \\
  \email{Paul\_Downen@uml.edu} \\
  \email{Adriano\_VilargaCorbelino@uml.edu}
}
%
\maketitle              % typeset the header of the contribution
%
\begin{abstract}
The abstract should briefly summarize the contents of the paper in
150--250 words.

\keywords{Codata \and Copatterns \and Scheme \and Macros \and Composition \and Expression Problem.}
\end{abstract}
%
%
%
\section{Introduction} \label{sec-intro}

For decades, functional programmers have had a reliable and versatile method for representing tree-shaped structures: inductive data types.
These can model data of any size --- for example, lists of an arbitrary length --- but each instance must be \emph{finite}.
Infinite information --- like a stream of input that can go on forever --- does not fit into an inductive type, so the programmer must use some other representation to model potentially infinite objects.

Fortunately, the inductive data types used by functional programmers every day have a polar opposite: \emph{coinductive codata types}.
The \emph{coinductive} descriptor signals that values of the type may contain infinite information.
Haskell programmers are already well-versed in these types, since non-strict languages blur the line between induction and coinduction.
For example, consider the usual example of the infinite list of Fibonacci numbers in Haskell:
\begin{minted}{haskell}
fibs = 0 : 1 : zipWith (+) fib (tail fib)
\end{minted}
\hs|fibs| cannot be fully evaluated because it has no base case --- it would eventually expand out to \hs|0 : 1 : 1 : 2 : 3 : 5 : 8 : ...| forever --- but this is no problem in a non-strict language that only evaluates as much as needed.

In contrast, \emph{codata} describes types that are defined by primitive \emph{destructors} that \emph{use} values of the codata type, as opposed to the primitive constructors that define how to build values of a data type.
For example, the usual \agda|Stream a| codata type of infinite \agda|a|'s is defined by two destructors: \agda|Head : Stream a -> a| extracts the first element and \agda|Tail : Stream a -> Stream a| discards the first element and returns the rest.
To define new streams, we can describe how they react to different combinations of \agda|Head| and \agda|Tail| destructions using \emph{copatterns} \cite{APTS2013C}.  The copattern-based definition of the \agda|fibs| function above is:
\begin{minted}{agda}
fibs : Stream Nat
Head fibs = 0
Head (Tail fibs) = 1
Tail (Tail fibs) = zipWith _+_ fibs (Tail fibs)
\end{minted}
\adriano{This example is funny, it works but emits 3 warnings, two related to the termination checker and one suggestion to turn on a flag (--guardedness). However, if you turn this flag on those warnings become errors. After typesetting the proofs, I will try to look for more failing examples in Agda.}
Even though copatterns provide an elegant way to define codata, their adoption is not universal.
To illustrate, this previous definition does not fully satisfy Agda's termination checker.
Although we can force Agda to accept this instance, not every copattern is valid due to technical limitations.
OCaml is another language that supports copatterns.
However, the fork that added this feature was never merged and stopped being updated to newer OCaml versions.

In this scenario, we have chosen Scheme as our implementation language.
Customizability is one of the key features of lisps dialects, and by taking leverage on that, we can produce a new solution to the expression problem \cite{ExpressionProblem}.
Therefore, by choosing a well-established dialect, we produce a standalone, composable, and reusable implementation of copatterns for a new audience.
\adriano{Should we talk about the expression problem here in the introduction?}


Our primary contributions are:
\begin{itemize}
    \item An encoding of copatterns in Scheme and Racket.
    Our abstraction provides a syntax that enables equational reasoning and emphasizes both the vertical and horizontal composition of copatterns.
    We provide three flavors for our implementation: A R6RS version where we implement our own pattern matching, A naive Racket version, and a cleaver Racket version where we optimize the number of administrative reductions;
    \item A core calculus with high-level features such as nested copatterns, self-referential objects, recursion templates, and composable extensions.
    This calculus depicts our implementation, and we prove that it is a conservative extension of the $\lambda$-calculus with pattern-matching and recursion;
\end{itemize} 
The remainder of the article has the following structure: First, we introduce our implementation by explaining meaningful examples (Section \ref{sec-examples}).
Second, we specify a core language with high-level features representing our implementation. Then we describe a translation into a target $\lambda$-calculus (Section \ref{sec-translation}).
Third, we scrutinize our implementation, comparing each provided flavor (Section \ref{sec-macro}).
Fourth, we present the properties of our system (Section \ref{sec-correctness}).
Last, we explain the details of our optimized racket implementation (Section \ref{sec-opt}).


%%% Local Variables:
%%% mode: LaTeX
%%% TeX-master: "coscheme"
%%% End:


\section{Programming with Composable Copatterns in Scheme}
\label{sec-examples}
All examples shown below are executable Scheme and Racket code.
You can follow along and interact with the code using the supporting library found online at \url{https://github.com/pdownen/CoScheme}.

\subsection{Infinite streams}

% One of the uses of Copatterns is to define codata, and infinite streams are the cliché example.
% Before starting our stream journey, let us recapitulate what codata is.
% For decades, functional programmers have had a reliable and versatile method for representing tree-shaped structures: inductive data types.
% These can model data of any size --- for example, lists of an arbitrary length --- but each instance must be \emph{finite}.
% Infinite information --- like a stream of input that can go on forever --- does not fit into an inductive type, so the programmer must use some other representation to model potentially infinite objects.
% Fortunately, the inductive data types used by functional programmers every day have a polar opposite: \emph{coinductive codata types}.
% The \emph{coinductive} descriptor signals that values of the type may contain infinite information.

For decades, functional programmers have had a reliable and versatile method for representing tree-shaped structures: inductive data types.
These can model data of any size --- for example, lists of an arbitrary length --- but each instance must be \emph{finite}.
But how does a program handle infinite amounts of information, that cannot possibly occupy a finite memory space?

One method of modeling infinite information is through laziness, as in the Haskell programming language.
For example, consider the usual infinite list of Fibonacci numbers in Haskell:
\begin{minted}{haskell}
fibs = 0 : 1 : zipWith (+) fib (tail fib)
\end{minted}
\hs|fibs| cannot be fully evaluated because it has no base case --- it would eventually expand out to \hs|0 : 1 : 1 : 2 : 3 : 5 : 8 : ...| forever --- but this is no problem in a non-strict language that only evaluates as much as needed.
But what if we are working within a strict language without laziness built in?
Must we give up on the approach entirely, or is there an alternate solution that works just as well with eager and lazy evaluation?

In contrast, \emph{codata} describes types defined by primitive \emph{destructors} that \emph{use} values of the codata type --- as opposed to the primitive constructors that define how to build values of a data type --- and lets us easily model infinite data in eager languages, too.
For example, the usual \agda|Stream a| codata type of infinite \agda|a|'s is defined by two destructors: \agda|Head : Stream a -> a| extracts the first element and \agda|Tail : Stream a -> Stream a| discards the first element and returns the rest.
To define new streams, we can describe how they react to different combinations of \agda|Head| and \agda|Tail| destructors using \emph{copatterns}~\cite{APTS2013C}.
Borrowing Agda's syntax, a possible copattern-based definition of the same \agda|fibs| function above is:
\begin{minted}{agda}
fibs : Stream Nat
Head fibs = 0
Head (Tail fibs) = 1
Tail (Tail fibs) = zipWith _+_ fibs (Tail fibs)
\end{minted}
However, at the moment, Agda currently does not understand if \agda|fibs| is well-founded --- it is --- and so \agda|fibs| is rejected.
As a proof assistant, Agda has demanding requirements on all definitions to ensure well-foundedness:
they must never have unproductive infinite loops, and they must cover every possible case (when matching on arguments or copatterns, as in \agda|fibs|).
But for a general-purpose programming language, we expect to be able to write arbitrary loops that may or may not terminate.
Copattern-based definitions need to gracefully handle cases that fail to return --- either due to an infinite loop or an exception, like an unhandled case, which are semantically similar \cite{ImpreciseExceptions} --- and should generate code based on whatever is given.
In this kind of setting, the language does not enforce --- and indeed, our implementation does not check --- coverage, which is instead up to the programmer to determine.

Let us now consider some examples of programming by equational reasoning to get familiar with copatterns and how we can use them in Scheme.
% Deletion candidate
% In these examples, it can help to think about types informally to keep ourselves oriented.
For example, even in a dynamically-typed language like Scheme, linked lists can be thought of as an inductively-defined type combining two constructed forms: \scm#List a = null | (cons a (List a))#.
Likewise, infinite streams can be understood as the type of a procedure that exhibits two different behaviors at the same time: \scm#Stream a = 'head -> a & 'tail -> Stream a#.
In other words, any \scm|Stream a| is a procedure that takes one argument, and its response depends on the exact value: given \scm|'head| an \scm|a| is returned, and given \scm|'tail| another \scm|Stream a| is returned.

In order to define new coinductive processes, one of the main entry points is the top-level, multi-line \scm{define*} macro.
This macro enables us to declare codata objects through a list of equations between a copattern on the left-hand side and an expression on the right-hand side.
At the root of every copattern is a name for the object \emph{itself}, which can be inside any number of applications --- the applications may just list parameter names or more specific patterns, narrowing down the concrete arguments that match.
Using \scm{define*}, we can define the trivial \scm|zeroes| stream  --- whose \scm|'head| is \scm|0| and whose \scm|'tail| is more \scm|zeroes| --- as:
\begin{minted}{scheme}
;; zeroes : Stream nat
(define* [(zeroes 'head) = 0]
         [(zeroes 'tail) = zeroes])
\end{minted}
Streams like \scm|zeroes| are black boxes that can only be observed by passing \scm|'head| or \scm|'tail| as arguments to get their response.
Still, this is enough for many useful operations, like taking the first \scm|n| elements, which can be \scm|define*|d as:
\begin{minted}{scheme}
;; takes : (Stream a, nat) -> List a
(define* [(takes s 0) = '()]
         [(takes s n) = (cons (s 'head) (takes (s 'tail) (- n 1)))])
\end{minted}
% So that \scm|(takes zeroes 5) = '(0 0 0 0 0)|.
A constant stream is not particularly useful; more interesting streams will change over time.
For example, imagine a ``stuttering'' stream ($0, 0, 1, 1, 2, 2, 3, 3, \dots$) that repeats numbers twice before moving on.
This stream can be defined by copattern matching equations:
\begin{minted}{scheme}
;; stutter : nat -> Stream nat
(define* [ ((stutter n) 'head)        = n]
         [(((stutter n) 'tail) 'head) = n]
         [(((stutter n) 'tail) 'tail) = (stutter (+ n 1))])
\end{minted}
So that \scm|(takes (stutter 1) 10) = '(1 1 2 2 3 3 4 4 5 5)|,% 
\footnote{
  Try it!
  \url{https://github.com/pdownen/CoScheme} has implementations of \scm|define*| and related macros used by these examples.
  % All examples shown here are executable Scheme and Racket code.
}
because the first and second elements --- \scm|((stutter n) 'head)| and \scm|(((stutter n) 'tail) 'head)| respectively --- return the same \scm|n| before incrementing.

But why is \scm|stutter| well-defined, and how can we understand its meaning?
As in many functional languages, the \scm|=| in code really implies equality between the two sides, and this equality still holds when we plug in real values for placeholder variables like \scm|n|.
So to determine the first \scm|'head| element, of \scm|(stutter 1)|, we match the left-hand side and replace it with the right to get \scm|((stutter 1) 'head) = 1|.
Similarly, the second element is \scm|(((stutter 1) 'tail) 'head) = 1| as well.
The third element is accessed by two \scm|'tail| projections and then a \scm|'head| as the nested applications \scm|((((stutter 1) 'tail) 'tail) 'head)|, which doesn't exactly match any left-hand side.
However, equality holds in any context, and the inner application \scm|(((stutter 1) 'tail) 'tail)| \emph{does} match the third equation.
Thus, we can apply a few steps of equational reasoning to derive the expected answer \scm|2|:
\begin{minted}{scheme}
((((stutter 1) 'tail) 'tail) 'head) = ((stutter (+ 1 1)) 'tail)  ; line 3
                                    = ((stutter 2) 'head)        ; +
                                    = 2                          ; line 1
\end{minted}
% \begin{minted}{scheme}
%    ((((stutter 1) 'tail) 'tail) 'head) = ((stutter (+ 1 1)) 'tail)
%   = ((stutter 2) 'head)                =  2                                      
%   \end{minted}
So these three examples work, but is every case really covered?
The \scm|Stream Nat| interface that \scm|stutter|'s output follows allows for any number of \scm|'tail| projections followed by a final application to \scm|'head| that returns a natural number.
\scm|stutter| works its way through these projections in groups of two, eliminating a pair of \scm|'tail| projections at a time until it gets to the end case, which is either a \scm|'head| (if the total number of \scm|'tail|s is even) or a \scm|'tail| followed by \scm|'head| (if the total number of \scm|'tail|s is odd).
So, \scm|stutter| behavior is defined no matter what is asked of it.
Even with other observations like \scm|takes|, which passes around partial applications of \scm|stutter| as a first-class value, internally \scm|stutter| only ``sees'' the \scm|'head| and \scm|'tail| applications from \scm|takes|, and is dormant otherwise.

Reasoning about the coverage of our copatterns is important since our implementation does not provide coverage analysis.
If we encounter an uncovered case, our implementation emits a runtime error, explaining that this is an uncovered copattern.
Non-total configurations, akin to partial functions, are not always undesirable. They can simplify the development during a prototyping phase, and if the missing case does not make sense, they can be the most semantically meaningful.

% This framework is not limited by matching a single value in the initial group.
% To illustrate, we can define a stream that intercalates elements from two different streams using a similar configuration, but observing two arguments instead of one.

% \adriano{Maybe we can put those examples side by side in a single figure if we decrease the font size}

% \begin{minted}{scheme}
% ;; zigzag : (Stream a, Stream a) -> Stream a
% (define*
%   [ ((zigzag xs ys) 'head)        = (xs 'head)]
%   [(((zigzag xs ys) 'tail) 'head) = (ys 'head)]
%   [(((zigzag xs ys) 'tail) 'tail) = (zigzag (xs 'tail) (ys 'tail))])
% \end{minted}

With this practice under our belt, we can now directly translate canonical Fibonacci example from Agda to Scheme like so:
\begin{minted}{scheme}
;; zips-with : ((a, b) -> c, Stream a, Stream b) -> Stream c
(define*
  [((zips-with f xs ys) 'head) = (f (xs 'head) (ys 'head))]
  [((zips-with f xs ys) 'tail) = (zips-with f (xs 'tail) (ys 'tail))])

;; fibs : Stream nat
(define*
  [ (fibs 'head)        = 0]
  [((fibs 'tail) 'head) = 1]
  [((fibs 'tail) 'tail) = (zips-with + fibs (fibs 'tail))])
\end{minted}
so that \scm|(takes fibs 10)| is \scm|'(0 1 1 2 3 5 8 13 21 34)|.

\subsection{Self-referential objects}

Codata can also be used to implement an abstract interface over regular finite data.
As an alternate syntax for \scm|define*|, we can explicitly give a top-level name to bind the definition to for external use, and on each equation give a hidden internal for self-reference and recursion.
To illustrate this, consider the following queue example, which internally refers to itself by the name \scm|self| for an object-oriented feel:
\begin{minted}{scheme}
(define* queue
  [ (self 'new)            = (self '() '())]
  [((self in  out) 'enq x) = (self (cons x in) out)]
  [((self '() '()) 'deq)   = (error "Invalid dequeue: empty queue")]
  [((self in  '()) 'deq)   = ((self '() (reverse in)) 'deq)]
  [((self in  out) 'deq)   = (cons (car out) (self in (cdr out)))])
\end{minted}
This reflects the purely functional queue implementation in using two lists (an inbox and an outbox) as internal states.
We externally bound this declaration to the name \scm{queue}, but the internal recursion is through the name \scm|self|.
This \scm|queue| object responds to three methods: \scm|'new| returns a new empty queue, \scm|'enq x| puts the \scm|x| to the end of the queue (\ie the top of the inbox), and \scm|'deq| returns the oldest enqueued element (from the top of the outbox or bottom of the inbox, as appropriate).
Thus, \scm|((((queue 'new) 'enq 1) 'enq 2) 'deq)| returns the oldest element \scm|1| and a queue object containing only \scm|2|.

Visualizing what we are defining through the lens of the object-oriented paradigm can give a new perspective here.
With this metaphor, we can view our definitions as describing the protocols of objects, where the equations specify how an object should respond to a sequence of messages.
Here, \scm|queue| itself can only directly respond to one message --- \scm|'new| --- that initializes the object with two empty lists for its private internal state.
From there, the initialized \scm|queue| object now only responds to the \scm|'enq x| and \scm|'deq| messages which can read and update the object's internal state.
However, besides these two messages, there is no other way to manipulate the internal state of an initialized \scm|queue| object;
the \scm|in| and \scm|out| lists are completely hidden within an opaque procedural abstraction, enforcing an encapsulation of private state.
Given an initialized \scm|queue| object, it would not be possible, for instance, to break the first-in-first-out ordering by taking an element from the \scm|in| list or to put an element on the \scm|out| list.

Since we can already use encapsulation in copattern-based definitions, can we also use a functional model \cite{abadi96} of inheritance and dynamic dispatch?
Our implementation of copattern matching in Scheme includes new facilities for composing code snippets compared to current functional (or object-oriented) languages.
However, to avoid unwanted surprises, the programmer does have to ask for them.
This is a small request, and can be done by replacing \scm|define*| with \scm|define-object|, as in the following file system example:
\begin{minted}{scheme}
(define-object
  [((fs-object p . _) 'path) = p])

(define-object (<: (import-object fs-object))
  [((file p txt) 'text) = txt]
  [((file p txt) 'size) = (string-length ((file p txt) 'text))])

(define-object (directory <: (import-object fs-object))
  [((apply dir p cts) 'contents) = cts]
  [((apply dir p cts) 'overhead) = 8]
  [((apply dir p cts) 'size)
   = (apply + ((apply dir p cts) 'overhead)
              (map (λ(o) (o 'size)) ((apply dir p cts) 'contents)))])
\end{minted}
This example emulates some functionality of a filesystem, specifically calculating various sizes.
Every filesystem object has a path, which is captured by the \scm|fs-object| object that only knows how to calculate its \scm|'path| by returning the first piece of information it was given and ignoring the rest of the object's internal data.
Filesystem objects all also have a size, but calculating it requires more object-specific information.
This additional functionality is spelled out by more specific filesystem objects:
\begin{itemize}
\item A \scm|file| contains some \scm|'text| and its \scm|'size| is the length of that text.
\item A \scm|directory| contains any number of additional filesystem objects, stored as its \scm|'contents|, and its \scm|'size| is the sum of it's \scm|'contents| size plus an additional \scm|'overhead| that defaults to \scm|8|.
\end{itemize}
Both \scm|file| and \scm|directory| objects inherit the code for calculating it's \scm|'path| by importing \scm|fs-object| with the extension clause \scm|<: (import-object fs-object)|.
Note that these three object definitions exercise three of the four possible definition forms, based on whether the external name is given explicitly or inferred, and on whether there is an extension clause is included:
\begin{itemize}
\item \scm|fs-object|'s external name is inferred from the internal name in its copattern equation, and it has no listed extension clause,
\item \scm|file|'s external name is inferred from its internal one and it extends \scm|fs-object|,
\item \scm|directory|'s external name is given explicitly (and is different from its internal name) and it extends \scm|fs-object|.
\end{itemize}
The last possibility is an explicit external name with no extensions, such as the following equivalent definition of \scm|fs-object| that elaborates the naming inference:
\begin{minted}{scheme}
(define-object fs-object
  [((fs-object p . _) 'path) = p])
\end{minted}

While these definitions are functional, they contain some undesirable redundancy.
In particular, we have to repeat the same initialization forms --- \scm|(file p txt)| and \scm|(apply dir p cts)| --- in front of every equational definition because the object must be initialized with parameters before it is used.
What is worse, every time we want to ask a question about the object itself by recursively passing it a message, we have to repeat this same initialization again exactly as it occurred, leading to longer and more error-prone code.
It would be better here to follow the common object-oriented factorization of steps:
\emph{first} the object is initialized with some internal data at the time of its construction, and \emph{then} we get an object that can (recursively) respond to methods.
This can be done by factoring out the common construction phase in the above definitions using a \scm|construct| clause like so:
\begin{minted}{scheme}
(define-object (<: (import-object fs-object))
  [(file p txt) (construct (list p txt))
   (object
    [(self 'text) = txt]
    [(self 'size) = (string-length (self 'text))])])

(define-object (directory* <: (import-object fs-object))
  [(apply dir p cts) (construct (cons p cts))
   (object
    [(self 'contents) = cts]
    [(self 'overhead) = 8]
    [(self 'size)
     = (apply + (self 'overhead)
                (map (λ(o) (o 'size)) (self 'contents)))])])
\end{minted}
The \scm|construct| operation lists its internal parameters given at the time of
initialization.
After this step, we describe a first-class anonymous \scm|object| that knows
how to refer to its fully-constructed form by an internal name (here we use the
name \scm|self|), so there is no need to re-construct \scm|(file p txt)| or \scm|(apply dir p cts)| to recursively call other methods.
Factoring out the common copattern leads to shorter code, the definitions describe objects with exactly the same behavior as before.

So far, we have only shaved off small parts of shared code: the common \scm|'path| method and the initial initialization copattern.
Where this coding style starts to pay off is when we override some methods to automatically influence the result of others.
For example, we might have a fancier type of directory structure that replicates the exact same behavior as a normal directory, but its \scm|'overhead| is \scm|128| instead of \scm|8|.
Or we might have static links that act like a normal \scm|file|, except that they only contain a path to the real place its text is stored, so it always has a fixed size (\scm|8|).
These specialized revisions of directories and files can be implemented by importing from the original definitions and modifying certain lines like so (where we omit the code for looking up the text for a static link):
\begin{minted}{scheme}
(define-object (fancy-directory <: (import-object directory))
  [((fancy-dir p . cts) 'overhead) = 128])

(define-object (<: (import-object file))
  [(static-link p lnk) (construct (list p lnk))
   (object
    [(_ 'link) = lnk]
    [(_ 'text) = "..."]
    [(_ 'size) = 8])])
\end{minted}
Although \scm|fancy-directory| has only one defining equation about its \scm|'overhead|, the implication is that its \scm|'size| should be \scm|120| larger than a regular \scm|directory| due to the \scm|'overhead| increase.
We can test out this use of inheritance and dynamic dispatch by simulating a small directory structure:
\begin{minted}{scheme}
(define ham (file "hamlet.txt" "Words, words, words...."))
(define guide (file "guide.txt" "Don't Panic"))
(define books (directory* "Books" ham guide))
(define shortcut (static-link "shortcut.txt" "Books/guide.txt"))
(define docs (fancy-directory "Documents" shortcut books))
\end{minted}
and calculating the sizes of each file system object:
\begin{minted}{scheme}
(ham      'size) = 23  = (string-length "Words, words, words....")
(guide    'size) = 11  = (string-length "Don't Panic")
(books    'size) = 42  = (+ 8 23 11)
(shortcut 'size) = 8
(docs     'size) = 178 = (+ 128 8 42)
\end{minted}

The question is: how were we able to inject new code in the middle of an object like \scm|directory| to change its behavior?
The key issue is that, within \scm|directory|, recursive calls to \scm|((apply dir p cts) 'overhead)| --- or just simply \scm|(self 'overhead)| in the second version --- cannot be tied to \emph{this} definition of \scm|directory|.
Instead, we employ \emph{open recursion}: the internal references to the recursive object itself are left as unbound parameters that will only be bound to the full object value when the final definition is ready to use.
This is why the internal and external names can be different --- like in \scm|queue| and \scm|directory| --- since the external name is bound to the final object value while the internal names are left (temporarily) open-ended and will be filled in later.
In the same way, the internal names used in each clause are fully independent and can also differ from one another.
This difference of using open recursion to leave internal names temporarily unbound also applies to definitions where the external name is inferred: although the external names \scm|file| and \scm|static-link| are inferred from the internal names in the copatterns, the internal variables are left unbound until the externally-visible name gets bound to the object value.

Thankfully, this whole framework is still built on purely functional idioms, which makes it easier to reason about code.
How can we understand what inheritance should do?
The answer should be familiar to functional programmers: inheritance is composition and substitution!
For example, the inheritance dependencies can be fully inlined as-is into \scm|fancy-directory| to bring the full definition into one place by copying the inherited code into place like so:
\begin{minted}{scheme}
(define-object fancy-directory
  [((fancy-dir p . cts) 'overhead) = 128]
  [((apply dir p cts)   'overhead) = 8]
  [((apply dir p cts)   'contents) = cts]
  [((apply dir p cts)   'size)
   = (apply + ((apply dir p cts) 'overhead)
              (map (λ(o) (o 'size)) ((apply dir p cts) 'contents)))]
  [((fs-object p . _)   'path)     = p])
\end{minted}
After na\"ive inlining, there are some irrelevant differences: there are three different internal names (\scm|fancy-dir|, \scm|dir|, and \scm|fs-object|) used in various clauses, and we use two equivalent syntaxes  (\scm|(fancy-dir p . cts)| and \scm|(apply dir p cts)|) for copatterns that bind arbitrary-length argument sequences to \scm|cts|.
Cleaning up these differences by rewriting each initializing copattern to \scm|(apply self p cts)| and renaming as necessary gives a more uniform code:
\begin{minted}{scheme}
(define-object fancy-directory
  [((apply self p cts) 'overhead) = 128]
  [((apply self p cts) 'overhead) = 8]
  [((apply self p cts) 'contents) = cts]
  [((apply self p cts)   'size)
   = (apply + ((apply self p cts) 'overhead)
              (map (λ(o) (o 'size)) ((apply self p cts) 'contents)))]
  [((apply self p _)   'path)     = p])
\end{minted}
From here, it becomes more obvious that the two different equations defining \scm|(... 'overhead)| overlap, so the first one takes precedence and the second one is dead code that can be completely erased.
Furthermore, when there are no more future extensions, we can inline the recursive calls at this point, to get the simpler closed definition that reveals exactly what each method will do:
\begin{minted}{scheme}
(define* fancy-directory
  [((apply self p cts) 'overhead) = 128]
  [((apply self p cts) 'contents) = cts]
  [((apply self p cts) 'size)
   = (apply + 128 (map (λ(o) (o 'size)) cts))]
  [((apply self p _)   'path)     = p])
\end{minted}

\subsection{Decomposing the expression problem}

% However, definitions by copatterns are useful for more programming tasks than just streams and other infinite objects.
% In particular, we can define a depth-first search on a finite binary tree.
% For this goal, we need to specify what should happen in contexts containing leaves and nodes of a tree.
% We can create observations that match a specific shape of the input.
% Therefore, when we see an evaluation context with a leaf --- \scm{((search ('leaf e)) 'dfs)} ---, we return a singleton, and when we see a node --- \scm{((search ('node l e r)) 'dfs)} ---, we recurse on both children and append their results.
% \adriano{Do I need to talk about how DFS works?}
% \begin{minted}{scheme}
%   (define* [((search ('leaf e)) 'dfs)     = (list e)]
%            [((search ('node l e r)) 'dfs) = (append ((search l) 'dfs)
%                                                     (list e)
%                                                     ((search r) 'dfs))])
% \end{minted}

Our notion of compositional copatterns can capture some object-oriented styles of code (de)composition with encapsulation, inheritance, and dynamic dispatch.
How can this new capability for composition influence the kinds of functional programs we write?
For example, consider the usual definition of a simple arithmetic expression evaluator in typed functional languages like Haskell and OCaml (we use Haskell syntax here):
\begin{minted}{haskell}
data Expr = Num Int | Add Expr Expr

eval :: Expr -> Int
eval (Num n)   = n
eval (Add l r) = eval l + eval r
\end{minted}
While Scheme does not have algebraic data types, we can encode complex constructor expressions as a list starting with the constructor name as a quoted symbol and the arguments as the remainder of the list, and when unambiguous, leave atomic data alone.
So \hs|Num 5| could just be represented as the plain number \scm|5|, and \scm|Add l r| would be represented as the \emph{quasiquote} \scm|`(add ,l ,r)| which plugs in the values bound to variables \scm|l| and \scm|r| as the second and third elements of the list (denoted by the ``unquote'' comma \scm|,| before the variable names).
We can then use the facilities of \scm|define*| to write almost identical code in Scheme like so, using the \emph{guard} \scm|try-if| to test if the argument is a number:
\begin{minted}{scheme}
;; eval : Expr -> Number
(define*
  [(eval n) (try-if (number? n)) = n]
  [(eval `(add ,l ,r))           = (+ (eval l) (eval r))])
\end{minted}
Fantastic, it works!
Both the Scheme and Haskell code have the same structure.
And on the surface, they both share the same strengths and weaknesses.
From the lens of the \emph{expression problem}~\cite{ExpressionProblem}, it is easy to add new operations to existing expressions --- such as listing the numeric literals in an expression
\begin{minted}{scheme}
;; list-nums : Expr -> List num
(define*
  [(list-nums n)
   (try-if (number? n))     = (list n)]
  [(list-nums `(add ,l ,r)) = (append (list-nums l) (list-nums r))])
\end{minted}
--- but adding new classes of expressions is hard.
For example, if we wanted to support multiplication, we could add a \hs|Mult| constructor to the \hs|Expr| data type, but this would require modifying \emph{all} existing operations and case-splitting expressions over \hs|Expr| values.
Even worse, if we wanted to support both expression languages --- with or without multiplication --- we would have to copy the code and maintain both versions.

Thankfully, our implementation of copattern matching in Scheme includes new facilities for composing code snippets.
As we previously saw with the object-oriented examples, we can turn ordinary functional code into a more extensional form by using \scm|define-object| instead of \scm|define*|.
\begin{minted}{scheme}
;; list-nums* : Expr -> List num
(define-object
  [(list-nums* n)
   (try-if (number? n))      = (list n)]
  [(list-nums* `(add ,l ,r)) = (append (list-nums* l) (list-nums* r))])
\end{minted}
The \scm|list-nums*| object behaves exactly like \scm|list-nums| in all the same contexts it works in, but in addition, it implicitly inherits additional functionality for composition defined elsewhere.
This new composition lets us break existing multi-line definitions into individual parts, and recompose them later.
For example, the evaluator can be composed in terms of separate objects for each line like so:
\begin{minted}{scheme}
(define-object
  [(eval-num n) (try-if (number? n)) = n])

(define-object
  [(eval-add `(add ,l ,r)) = (+ (eval-add l) (eval-add r))])

;; eval* : Expr -> num
(define eval* (eval-num 'compose eval-add))
\end{minted}
So \scm|(eval expr)| is the same as \scm|(eval* expr)| for any well-formed expression argument.
Why program in this way?
Now, if we want to extend the functionality of existing operations --- like evaluation and listing literals --- to support a new class of expression, we can define the new special cases separately as a patch and then \emph{compose} them with the existing code as-is like so:
\begin{minted}{scheme}
(define-object
  [(eval-mul `(mul ,l ,r)) = (* (eval-mul l) (eval-mul r))])
(define-object
  [(list-mul `(mul ,l ,r)) = (append (list-mul l) (list-mul r))])

;; eval-arith : Expr+Mul -> num
(define eval-arith (eval* 'compose eval-mul))

;; eval-arith : Expr+Mul -> List num
(define list-nums-arith (list-nums* 'compose list-mul))
\end{minted}
So for an expression \scm|(define expr1 '(add (mul 2 3) 4))|, the extended code correctly yields \scm|(eval-arith expr1) = 10| and \scm|(list-nums-arith expr1) = '(2 3 4)| whereas the original code fails at the \scm|'mul| case.%
\footnote{
  The astute reader might notice the open recursion at work here: the recursive calls to \scm|eval-mul| cannot be specifically tied to this definition because it only says what to do with multiplication and fails to handle the other cases.
  Instead, recursive calls to \scm|eval-mul| must \emph{also} open to invoking the other code associated with \scm|eval-num| and \scm|eval-add| even though it is not known to be associated with them yet.}
%
Note that this composition automatically generates \emph{new} functions and leaves the original code intact, which can still be used for the smaller expression language with only numbers and addition.

This example emphasizes our guiding principle: \emph{composition}.
We call combinations like \scm|(eval-num 'compose eval-add eval-mul)| \emph{vertical composition} since they behave as if we simply stacked their internal cases vertically, like in the original definition of \scm|eval|.

Not all types of language extensions are this simple, though.
Consider what happens if we want to support algebraic expressions which might have variables in them.
To evaluate a variable, we need a given environment --- mapping names to numbers --- which we can use to look up the variable's value.
\begin{minted}{scheme}
(define-object [(eval-var env `(var ,x)) = (lookup env x)])
\end{minted}
However, it is wrong to just vertically compose this variable evaluator with the previous code because the arithmetic evaluator only takes a single expression as an argument, whereas the variable evaluator needs \emph{both} an environment and an expression.
The manual way to perform this extension is routine for functional programmers: in addition to adding a new case, we have to add an extra parameter to each case, which gets passed along on all recursive calls.
% On an individual equation, this transformation looks like:
% \begin{minted}{scheme}
% [(eval     some-expr-pattern) (... (eval     sub-expr) ... )]
% ==>
% [(eval env some-expr-pattern) (... (eval env sub-expr) ...)]
% \end{minted}

It would be highly disappointing to have to rewrite our existing code in-place to do this extension.
Fortunately, our copattern language allows for another type of composition --- \emph{horizontal composition} --- which allows us to combine sequences of steps, one after another, and automatically fall through to the next case if something fails.
For this example, we can define a general procedure \scm|with-environment| to perform the above transformation, taking any extensible evaluator object expecting just an expression and threading an environment along each recursive call.
This lets us patch our existing arithmetic evaluator with an environment and then compose it with variable evaluation like so:
\begin{minted}{scheme}
(define (with-environment eval-ext)
  (object [(self env expr)
           (with-self (override-lambda* self
                        [(_ sub-expr) = (self env sub-expr)])
             (try-apply-forget eval-ext expr))]))

;; Env = List (Symbol . num)

;; eval-alg : (Env, Expr+Mul+Var) -> num
(define eval-alg
  ((with-environment (eval-arith 'unplug)) 'compose eval-var))
\end{minted}
% TODO: Maybe resume this part
The \scm|with-environment| function is the most complex code we have seen so far, but it just spells out the usual steps a functional programmer uses to modify existing code with an environment.
\begin{itemize}
\item Given the evaluator \scm|eval-ext|, it returns a new first-class \scm|object| (which is the same as \scm|define-object| without assigning a name) that expects both an environment and expression to process.
\item This new object then invokes \scm|eval-ext| by passing just the expression, except that if \scm|eval-ext| ever tries to recur with a sub-expression, the calls \scm|(self sub-expr)| gets replaced with \scm|(self env sub-expr)| just like the template transformation.
\item This transformation of the evaluator's notion of self is done by the \scm|with-self| operation, which can override the original recursive \scm|self|.
\item Finally, if none of the clauses of \scm|eval-ext| succeed, then this updated evaluator also falls through as before, forgetting the application had ever happened via \scm|try-apply-forget|.
\end{itemize}
The complete algebraic evaluator can then be made from an open-ended, extensible version of the arithmetic evaluator --- retrieved from \scm|(eval-arith 'unplug)| --- horizontally composed to take an environment and vertically composed with the single-line \scm|eval-var|.
It can now successfully evaluate algebraic expressions, such as \scm|(define expr2 '(add (var x) (mul 3 (var y))))|, so that running the evaluation \scm|(eval-arith '((x . 10) (y . 20)) expr2)| returns \scm|70| because the environment maps \scm|x| to \scm|10| and \scm|y| to \scm|20|.

Another possible way to evaluate expressions with variables is \emph{constant folding}, a common optimization where operations are simplified unless they are blocked by variables whose values are unknown.
In other words, the evaluator might return a blocked expression if it cannot fully calculate the final number.
Ideally, we would like to extend our existing evaluator as-is, with the additional cases when blocked expressions are encountered.  
However, as written, the equation handling \scm|(eval `(add ,l ,r))| already commits to a real numeric addition, even if evaluating \scm|l| or \scm|r| does not give a numeric result.

To avoid over-committing before we know whether evaluation will successfully calculate a final number or not, we can --- at first glance --- rewrite the basic clauses of evaluation in a more defensive style.
Essentially, this splits evaluation into two separate steps:
\begin{enumerate*}[(1)]
\item check which operation we are supposed to do and evaluate the two sub-expressions,
\item combine the two expressions according to that operation.
\end{enumerate*}
For example, the two steps for addition and multiplication look like:
\begin{minted}{scheme} 
(define-object eval-add-safe
  [(self 'eval ('add l r))
  = (self 'add (self 'eval l) (self 'eval r))]
  [(self 'add x y) (try-if (and (number? x) (number? y)))
  = (+ x y)])

(define-object eval-mul-safe
  [(self `(mul ,l ,r))
   = (self 'mul (self l) (self r))]
  [(self 'mul x y)
   (try-if (and (number? x) (number? y)))
   = (* x y)])
\end{minted}
Here, the evaluation step is explicated by a \scm|'eval| tag, to help distinguish from the other operation \scm|'add| for adding the left and right results.
Note that in this code, the \scm|'add| clause only performs a numeric addition \scm|+| if it knows for sure that \emph{both} of the arguments are actually numbers.
We can now compose the original base-case for evaluating numbers with this ``safer'' version of addition that fails to match cases where sub-expressions don't evaluate to numbers (multiplication could be added as well in a similar style):
\begin{minted}{scheme}
;; eval-arith-safe : ('eval, Expr+Mul) -> num
;;                 & ('add, num, num) -> num
;;                 & ('mul, num, num) -> num
(define eval-arith-safe (eval-num 'compose eval-add-safe eval-mul-safe))
\end{minted}
So \scm|(eval-arith-safe expr1)| still evaluates to \scm|70|, but \scm|(eval-arith-safe expr2)| fails when it finds a variable sub-expression.

If it finds a variable, constant folding will just leave it alone and return an unevaluated expression rather than a final number.
Because the \scm|'eval| operation might return a (partially) unevaluated expression, we now need to handle cases where the left or right (or both) sub-expressions do not evaluate to numbers.
In each of those cases, we must reform the addition expression out of what we find, converting numbers \scm|n| into a syntax tree of the form \scm|`(num ,n)|.
\begin{minted}{scheme}
(define-object
  [(leave-variables 'eval ('var x)) = (list 'var x)])

(define-object reform-operations
  [(reform 'add l r) = (list 'add l r)]
  [(reform 'mul l r) = (list 'mul l r)])
\end{minted}
The final constant-folding algorithm can be composed from this ``safe'' version of evaluation, along with the cases for leaving variables alone and reforming partially-evaluated additions and multiplications.
\begin{minted}{scheme}
;; constant-fold : ('eval, Expr+Mul+Var) -> Expr+Mul+Var
;;               & ('add, Expr+Mul+Var, Expr+Mul+Var) -> Expr+Mul+Var
;;               & ('mul, Expr+Mul+Var, Expr+Mul+Var) -> Expr+Mul+Var
(define constant-fold
  (eval-arith-safe 'compose leave-variables reform-operations))
\end{minted}
So now \scm|(constant-fold 'eval expr2)| successfully returns \scm|expr2| itself (because there are no operations to perform without knowing the values of variables \scm|x| and \scm|y|).
And running \scm|(constant-fold 'eval expr3)| on the expression
\begin{minted}{scheme}
(define expr3
  '(add (add 1 1)
        (mul (var x)
             (mul 2 (add 2 3)))))
\end{minted}
simplifies it down to \scm|'(add 2 (mul (var x) 10))|.
To add other operations, like subtraction, we can easily define similar \scm|eval-sub-safe| and \scm|reform-subtraction|, and \scm|'compose| them with \scm|constant-fold| without having to rewrite any code.
% All examples shown here are in the supplemental materials.

%%% Local Variables:
%%% mode: LaTeX
%%% TeX-master: "coscheme"
%%% End:



\section{Translating Composable Copatterns} \label{sec-translation}

\subsection{Challenges}

Even though the behavior of small examples may be straightforward to understand, there are several challenges to correctly implementing copatterns in the general case.
Some of these challenges are specific to Scheme --- a dynamically-typed, call-by-value language --- which forces us to carefully resolve the timing of when and which copatterns are matched.
Other challenges are specific to our extensions to copatterns --- the ability to compose copattern matching in two different directions --- which also brings in the notion of the recursive ``self.''

\subsubsection{Timing and the order of copattern matching}
\label{sec:timing-challenges}

Copatterns may have ambiguous cases where two different overlapping copattern equations match the same application.
For example, this following function moves a number by \scm|1| away from \scm|0| --- positives are incremented and negatives are decremented:
\begin{minted}{scheme}
(define* [(away-from0 x) (try-if (>= x 0)) = (+ x 1)]
 [(away-from0 x) (try-if (<= x 0)) = (- x 1)])
\end{minted}
Consequently, we must interpret the programmer's code as it is written since we cannot gain any information from a static type system.
In the previous example, the two different equations overlap for \scm|0| itself: either one matches the call \scm|(away-from0 0)|.
To disambiguate overlapping copatterns, the listed equations are always tried top-down, and the first full match ``wins,'' as is typical in functional languages.
In this case, the first line wins, so \scm|(away-from0 0)| is \scm|1|.
Furthermore, guards like \scm|try-if| and \scm|try-match| are run left-to-right with shortcircuiting --- the moment a copattern or a guard fails, everything to the right is skipped.
This makes it possible to protect potentially-erroneous guards with another safety guard to its left, such as \scm|(try-if (not (= y 0)))| followed by \scm|(try-if (> (/ x y) z))|.

However, there are more timing issues besides these usual choices for disambiguation and short-circuiting.
First of all, since we are in a call-by-value language, we have to handle cases where an object is used in a context that doesn't fully match a copattern \emph{yet}, but could in the future --- and possibly multiple different times.
This can happen for instances like curried functions that take arguments in multiple different calls.
Just like with ordinary curried functions, using such an object in a calling context passing only the first list of arguments --- but not the second --- builds a \emph{value} which closes over the parameters so far.
For example, consider this simple counter object that can add or get its current internal state.
\begin{minted}{scheme}
(define* [((counter x) 'add y) = (counter (+ x y))]
         [((counter x) 'get)   = x])
\end{minted}
The call \scm|((counter 4) 'get)| matches the second equation, which is \scm|4|, but \scm|(counter 4)| on its own is not enough information to definitively match either copattern, so it is just a value remembering that \scm|x = 4| and waiting for another call.
Similarly, the call \scm|(counter (+ x y))| on the right-hand side is \emph{also} incomplete in the same sense, so it, too, is a value.
This definition gives us an object with the following behavior:
\begin{minted}{scheme}
> (define c0 (counter 4))
> (define c1 (c0 'add 1))
> ((c1 'add 2) 'get)
7
> (c1 'get)
5
\end{minted}

So far, what we have seen so far seems similar to pattern-matching functions in languages that are curried-by-default.
One way in which copatterns generalize curried functions is that each equation can take a \emph{different} number of arguments.
For example, consider this reordering of the \scm|stutter| stream from \cref{sec-examples}:
\begin{minted}{scheme}
(define* [(((stutter n) 'tail) 'tail) = (stutter (+ n 1))]
         [(((stutter n) 'tail) 'head) = n]
         [ ((stutter n) 'head)        = n])
\end{minted}
Since none of the copatterns overlap, its behavior is exactly the same as before.
But notice the extra complication here: calling \scm|((stutter 10) 'head)| with two arguments (\scm|10| and \scm|'head|) should immediately return \scm|10|.
However, the first equation is waiting for three arguments (an \scm|n| and two \scm|'tail|s passed separately).
That means the underlying code implementing \scm|stutter| \emph{cannot} ask for three arguments in three different calls and then check that the last two are \scm|'tail|.
Instead, it has to eagerly match the arguments its given against the patterns and try each of the guards to see if the current line fails --- and only \emph{after} that all succeed, it may ask for more arguments and continue the copattern match.

\subsubsection{Composition and the dimensions of extensibility}
\label{sec:composition-challenges}

The second set of challenges is due to the new notions of object composition that we develop here.
In particular, we want to be able to combine objects in two different directions:
\begin{itemize}
\item
  \emph{vertical composition} is an ``either or'' combination of two or more objects, such as \scm|(o1 'compose o2 ...)| that acts like \scm|o1| or \scm|o2|, \etc[,] depending on which one knows how to respond to the context.
  Textually, vertical composition of \scm|(object line-a1 ...)| and \scm|(object line-b1 ...)| behaves as if we copied all each line of copattern-matching equations internally used to define the two objects and pasted them vertically into the newly-composed object as:
\begin{minted}{scheme}
(object line-a1 ...
        line-b1 ...)
\end{minted}
\item
  \emph{horizontal composition} is an ``and then'' combination of objects in a copattern-matching line, such as \scm|[(self 'method1) (try-object o1)]| defining a \scm|'method1| that continues to act like \scm|o1| when \scm|o1| knows how to respond to the surrounding context, and otherwise tries the next line.
  Textually, the vertical composition of a \scm|'method1| followed by trying another object with its own copattern-matching contexts \scm|Q1 Q2 ...| acts as if the two copatterns are combined, and the inner object is inlined into the outer one like so:
\begin{minted}{scheme}
(object [(self 'method1) (try-object (object [Q1 = response1]
                                             [Q2 = response2]
                                             ...))]
        ...)
=
(object [(self 'method1) (comatch Q1) = response1]
        [(self 'method1) (comatch Q2) = response2]
        ...)
\end{minted}
\end{itemize}

Even though we can visually understand the two directions of composition by the textual manipulations above, in reality, both of these compositions are done at run-time (\ie with arbitrary procedural values), as opposed to ``compile-time'' transformation (\ie macro-expansion time manipulations of code).
This means we need an extensible representation of run-time object values that allows for automatically switching from one object to another in the case of copattern-match failure, as well as correctly keeping track of what to try next.

The basic idea of this representation can be understood as an extension of an idiom in ordinary functional programming.
In order to define an open-ended, pattern-matching function, we can give the cases we know how to handle now by matching on the arguments and include a default ``catch-all'' case at the end for the other behavior.
In Haskell, this might look like
\begin{minted}{haskell}
f next PatA1 PatA2 ... = expr1
f next PatB1 PatB2 ... = expr2
...
f next x1    x2    ... = next x1 x2 ...
\end{minted}

For example, consider the single-line \scm|eval-add| evaluator object from \cref{sec-examples}.
In order to compose \scm|eval-add| with another evaluator handling a different case, like \scm|eval-mul|, its internal extensible code takes an extra hidden argument saying what to try \scm|next| if its line does not match, analogous to:
\begin{minted}{scheme}
(define (eval-add-ext1 next)
  (lambda* [(self 'eval `(add ,l ,r)) = (+ (self 'eval l) (self 'eval r))]
           [ self                     = (next self)]))
\end{minted}
Note that, unlike the Haskell code above, the hidden \scm|next| parameter also takes \emph{another} hidden parameter: \scm|self|.
Why?
Because if the \scm|next| set of equations needs to recurse, it cannot actually jump to itself directly --- that would skip the \scm|eval-add| code entirely --- but needs to jump back to the very first equation to try.
This \scm|self| parameter holds the value of the \emph{whole} object after all compositions have been done, as it appeared in the original call site.
Thus, the internal extensible code \scm|eval-add-ext| \emph{also} takes this second \scm|self| parameter for the same reason: it may be the second component of a composition, and it is similar to the following Racket definition:
\begin{minted}{scheme}
(define ((eval-add-ext2 next) self)
  (match-lambda* [(list 'eval `(add ,l ,r)) (+ (self 'eval l) (self 'eval r))]
                 [args                      ((next self) args)]))
\end{minted}

% One final detail to note: unless otherwise stated, the \scm|self| parameter --- which is visible as the root of any copattern --- is \emph{always} the same view of the entire object.
% That means nesting multiple copatterns in sequence might not give the expected result because the \scm|self| parameter in the hole of every copattern context will be bound to the same value.
% If we instead want the parameter in the hole of every copattern context to reflect the object at that point in time --- that is, be assigned the value given by the partial applications given by the preceding copatterns --- we can use the \scm|nest| operation.
% For example, nesting copatterns in a sequence gives us a shorthand for the common functional idiom of a ``local'' loop that closes over some parameters that never change, such as this definition of mapping a function over a list:
% \begin{minted}{scheme}
% (define* [(map* f xs) = ((map* f) xs)]
%          [(map* f) (nest)
%            (extension
%             [(go null)        = null]
%             [(go (cons x xs)) = (cons (f x) (go xs))])])
% \end{minted}
% The \scm|map*| function supports both curried and uncurried applications, and they are defined to be equal.
% Its real code is given in the curried case, where the function parameter \scm|f| is bound first and never changes.
% Then, in a second step, we have the internal looping function \scm|go|, which matches over its list parameter and recurses with a new list.
% By using \scm|nest|, we have \scm|go = (map* f)|, so that \scm|f| is visible from the closure but does not need to be passed again at every step of the loop.

\subsection{Double-barrel translation}

To explain the correctness and behavior of composable copattern matching, we give a high-level translation into a conventional $\lambda$-calculus with recursion and pattern matching (given in \cref{fig:target-syntax}).
Our pattern language is modeled after a small common core found among various implementations of Scheme, which includes normal variable wildcards $x$ that can match anything, quoted symbols $\q{x}$, and lists of the form $\Null$ or $(\Cons P \, P')$.
Note that we assume all bound variables $x$ in a pattern are distinct.
% TODO: Deletion Candidate
As shorthand, we write a list of patterns $P_1 ~ P_2 ~ \dots ~ P_n$ for $(\Cons P_1 ~ (\Cons P_2 ~ \dots (\Cons P_n \Null)))$.
To model the patterns found in typed functional languages like ML and Haskell, such as constructor applications $K ~ \many{P}$, we can represent the constructor  as a quoted symbol $\q{K}$ and the application as a list $\q{K} ~ \many{P}$.
% TODO: Deletion Candidate
The patterns' specifics are surprisingly not essential to the main copattern translation and could be extended with other features found in more specific implementations.  

\begin{figure}[t]
\centering
\begin{alignat*}{2}
  % \mathit{Variable} &\ni{}& x, y, z
  % \\
  \mathit{Term} &\ni{}& M, N
  &::= x
  \mid M ~ N
  \mid \lambda x. M
  \mid K
  \mid \Match M \With \set{\many{P \to N}}
  \mid \Rec x = M
  \\
  \mathit{Pattern} &\ni{}& P
  &::= x
  \mid \q{x}
  \mid \Null
  \mid \Cons P \, P'
\end{alignat*}

\caption{Target language: pure $\lambda$-calculus with pattern-matching and recursion.}
\label{fig:target-syntax}
\end{figure}

For simplicity, this translation begins from a small source language with copatterns (given in \cref{fig:source-syntax}) separated into three main syntactic categories:
\begin{itemize}
\item[($M, N$)] \emph{Terms} represent ordinary first-class values as well as applications.
  The new forms of terms are $\lamstar B$, which gives a self-referential copattern-matching object, along with $\Template B$ and $\Extension O$ which include the other two syntactic categories as first-class values.
\item[($B$)] \emph{Templates} represent self-referential code without a fixed self.
  Instead, the ``self'' placeholder remains unbound for now, and it can be instantiated later as $\Template B ~ V$ (where the ``self'' of the template is bound to $V$) or $\lamstar B$ (where the ``self'' of the template is recursively bound to $\lamstar B$).
\item[($O$)] \emph{Extensions} represent extensible code that can be composed together both vertically and horizontally.
  Instead of failing on an unsuccessful match, will try an as-of-yet unspecified ``next'' option.
  To support recursion, the ``self'' placeholder is also unbound for now --- just like with templates --- and can be bound later when the whole object is finished being composed.
  The ``next'' thing to try can be given by the vertical composition with another extension $O; O'$ or a base-case template $O; B$.
  Arbitrary first-class values can passed in as the next option ($V$) and the self object ($W$) as $\Extension O ~ V ~ W$.
\end{itemize}

\begin{figure}[t]
\centering
\small
\begin{alignat*}{2}
  % \mathit{Variable} &\ni{}& x, y, z
  % \\
  \mathit{Term} &\ni{}& M, N
  &::= \dots
  \mid \lamstar B
  \mid \Template B
  \mid \Extension O
  \\
  \mathit{Template} &\ni{}& B
  &::= \varepsilon
  \mid O; B
  \mid \Continue x \to M
  \\
  \mathit{Extension} &\ni{}& O
  &::= \varepsilon
  \mid O; O'
  \mid Q[x] ~ O
  \mid \lambda P.~ O
  % \mid \If M ~ O
  \mid \Match P \gets M ~ O
  % \mid \Nest O
  \mid \Try x \to B
  \\
  \mathit{Copattern} &\ni{}& Q
  &::= \hole
  \mid Q ~ P
  \\
  \mathit{Pattern} &\ni{}& P
  &::= x
  \mid \q{x}
  \mid \Null
  \mid \Cons P \, P'
\end{alignat*}

Syntactic sugar:
\begin{align*}
  \Else M
  &=
  \Continue \_ \to M
  &
  (= M)
  =
  \Do M
  &=
  \Try \_ \to \Else M
  \\
  \If M ~ O
  &=
  \Match \True \gets M ~ O
  &
  (\Let x = M ~ O)
  &=
  \Match x \gets M ~ O
\end{align*}
\caption{Source language: target extended with nested copatterns,
  self-referential objects, recursion templates, and composable extensions.}
\label{fig:source-syntax}
\end{figure}

The remaining new syntax gives ways to define and combine copattern-matching expressions.
Copatterns $Q[x]$ themselves are a subset of contexts, $Q$, surrounding an object internally named $x$.
Two lines separated by a semicolon ($O; O'$) is vertical composition that tries either $O$ or $O'$, and prefixing with a copattern-matching expression ($Q[x] O$) is horizontal composition that tries $Q[x]$ and then $O$.
The $\varepsilon$ represents an empty extension with respect to vertical composition: it immediately refers to the next option.
Smaller special cases of matching include pattern lambdas ($\lambda P. O$) that try to match a new argument against $P$, and pattern guards ($\Match P \gets M ~ O$) that try to match a given expression $M$ against $P$; both of which continue as $O$ if they succeed.
% $\Nest O$ allows for nesting multiple copatterns with a partially applied self object from this point.

Finally, we have the terminators for ending a sequence of matching.
A template can end in the empty $\varepsilon$ (which just fails, because there is no code to handle the case) or a $\Continue x \to M$ which serves as the default ``catch-all'' case.
The parameter $x$ bound by $\Continue x \to M$ is another way to introduce a name for the recursive reference to the object itself at the end of a template  and allows for $M$ to restart from the top and continue the computation.
The syntactic sugar $\Else M$ covers the common case where $M$ give an answer without recursively continuing.
Similarly, an extension can end with $\Try x \to B$. This gives a ``catch-all'' case that runs some other (non-extensible) template $B$.
The parameter $x$ bound by $\Try x \to B$ gives a name to the next option that would have been tried after this one and allows $B$ to explicitly move on to the next option if it needs to.
The syntactic sugar $\Do M$ covers the most common case of $\Try$ which definitively commits to a particular term $M$ to return as the result without trying any further options.
To write examples in a similar style to ML-family languages, we also use the syntactic sugar $(= M)$ with the same meaning, which looks odd out of context but expresses the equational nature of copattern matching when used in examples.

Thus, the full translation from the source (\cref{fig:source-syntax}) to target (\cref{fig:target-syntax}) is given in \cref{fig:translation}.
This translation shares many similarities to continuation-passing style (CPS) translations.
However, we explicitly avoid converting the entire program to CPS.
Notably, every syntactic form for the source language is unchanged; for example, $\den{M~N} = \den{M} ~ \den{N}$.
Instead, the only time we need to introduce an extra parameter is for the two new syntactic categories.
All templates are translated to functions that take a value for the whole object itself to a new version of that object.
Similarly, all extensions are translated to functions that take both a template as the ``base case'' to try next and a value for the whole object itself.
Even though this is dynamically-typed, we can view the type of templates as object transformers and extensions as template transformers:
\begin{align*}
  Object &= \text{some type of function}
  \\
  Template &= Object \to Object'
  \\
  Extension &= Template \to Template'
  = Template \to Object \to Object'
\end{align*}

\begin{figure}[t]
\centering
\small
Translating new terms:  
\begin{align*}
  \den{\lamstar B}
  &=
  (\Rec \mathit{self} = T\den{B} ~ \mathit{self})
  &=_\eta
  (\Rec \mathit{self} = T\den{B} ~ (\lambda x. \mathit{self} ~ x))
  \\
  \den{\Template B}
  &=
  T\den{B}
  \\
  \den{\Extension O}
  &=
  E\den{O}
  \\
  \den{M}
  &=
  \text{by induction}
  &(\text{otherwise})
\end{align*}
Translating templates:
\begin{align*}
  T\den{\varepsilon}
  &=
  \mathit{fail}
  &
  &=_\eta
  \lambda s. \mathit{fail}~s
  \\
  T\den{O; B}
  &=
  E\den{O} ~ T\den{B}
  &
  &=_\eta
  \lambda s. E\den{O} ~ T\den{B} ~ s
  \\
  T\den{\Continue x \to M}
  &=
  \lambda x. \den{M}
\end{align*}

Translating copattern-matching and pattern-matching functions:
\begin{align*}
  E\den{(Q[x] ~ P) ~ O}
  &=
  E\den{Q[x] ~ (\lambda P. O)}
  \\
  E\den{x ~ O}
  &=
  \lambda b. \lambda x. E\den{O} ~ b ~ x
  \\
  E\den{\lambda P. O}
  &=
  E\den{\lambda x. \Match P \gets x ~ O}
  &(\text{if } P \notin \mathit{Variable})
\end{align*}

Translating other extensions:
\begin{align*}
  E\den{\varepsilon}
  &=
  \lambda b. b
  &
  &=_\eta
  \lambda b. \lambda s. b ~ s
  \\
  E\den{O; O'}
  &=
  E\den{O} \comp E\den{O'}
  &
  &=_\eta
  \lambda b. \lambda s. E\den{O} ~ (E\den{O'}~b) ~ s
  \\
  E\den{\lambda x. O}
  &=
  \lambda b. \lambda s. (\lambda x. E\den{O} ~ (\lambda s'. b ~ s' ~ x) ~ s)
  \\
  E\den{\Match P \gets M ~ O}
  &=
  \rlap{$
    \lambda b. \lambda s.
    \Match \den{M} \With \set{P \to E\den{O}~b~s; \_ \to b~s}
  $}
  % \lambda b. \lambda s.
  % \begin{aligned}[t]
  %   &\Match \den{M} \With \\
  %   &\quad
  %   \begin{aligned}[t]
  %     \{~
  %     P &\to E\den{O}~b~s; \\
  %     \_ &\to b~s
  %     ~\}
  %   \end{aligned}
  % \end{aligned}
  \\
  % E\den{\Nest O}
  % &=
  % \lambda b. \lambda s. \Rec s' = E\den{O} ~ (\lambda \_. b ~ s) ~ (\lambda x. s' ~ x)
  % \\
  E\den{\Try x \to B}
  &=
  \lambda x. T\den{B}
\end{align*}
\caption{Translating copattern-based source code to the target language.}
\label{fig:translation}
\end{figure}

The interesting cases for translating terms are the new forms.
$\Template B$ and $\Extension O$ are just translated to their given forms as transformation functions.
With $\lamstar B$, we need to recursively plug its translation in for its self parameter.
Note the one detail that the recursive $\mathit{self}$ is $\eta$-expanded to in this application.
This ensures that $\lambda x. \mathit{self} ~ x$ is treated as a value in a real implementation, and is always safe assuming that $B$ describes a function (non-functional cases of $\lamstar B$ are undefined user error).

For templates and extensions, the terminators $\Continue$ and $\Try$ are translated to plain $\lambda$-abstractions that allow the programmer direct access to their implicit parameters.
% Other cases are specific to each form.
Complex copatterns ($x ~ \many{P_1} P_n ~ O$) are reduced down to a simpler sequence of pattern lambdas ($x ~ \lambda P_1. \dots \lambda P_n. ~ O$), and pattern lambdas ($\lambda P. O$) are reduced down to a simpler non-matching lambda followed by an explicit match guard ($\lambda x. \Match P \gets x ~ O$).

This leaves just the base cases of simple extension forms.
Each time an extension (of form $\lambda b. \lambda s. \dots$) ``fails,'' it calls the given next template with the given self object ($b~s$).
A match guard $\den{\Match P \gets M ~ O}$ will try to match the translation of $M$ against the pattern $P$; the success case continues as $E\den{O}$ with the same next template and self.
A non-matching lambda $\den{\lambda x. O}$ always succeeds (for now), but note that the next template to try on failure has to be changed to include the given argument.
Why?
Because the lambda has already consumed the next argument from its context, it would be gone if, later on, the following operations fail and move on to the next option.
So instead of invoking the given $b$ directly as $b~s'$ (for a potentially different future $s'$), they need to invoke $b$ applied to this argument $x$ as $b~s'~x$.
% Finally, the $\den{\Nest O}$ operation is defined by recursively creating a new value for the self parameter by recursively taking a new snapshot of how the object looks now after all the preceding applications and matchings have already occurred.

In this translation, we also give the $\eta$-reduced forms on the right-hand side when available.
This shows that the empty extension $\varepsilon$ is just the identity function (given the next thing $b$ to try, $\varepsilon$ does nothing and immediately moves on to $b$), and horizontal composition $O; O'$ is just ordinary function composition.


%%% Local Variables:
%%% mode: LaTeX
%%% TeX-master: "coscheme"
%%% End:


\section{Macro Definition} \label{sec-macro}

The real implementation of copattern matching in the Scheme macro system is quite similar to the high-level translation given in \cref{fig:translation}
However, there are some important differences which have to do with integrating the new feature with the rest of the language, as well as practical implementation details.
For example, note the definition of $\den{\lamstar B}$ in particular.
While the $\eta$-equality simplifying $\lambda x. \mathit{self} ~ x$ to just $\mathit{self}$ is theoretically sound, it does not work in practice:
when a Scheme interpreter tries to evaluate the right-hand side ($T\den{B}~\mathit{self}$) of the recursive binding, it first tries to lookup the value bound to $\mathit{self}$ which has not been defined yet, leading to an error.
This one level of $\eta$-expansion delays the evaluation step so that $\lambda x. \mathit{self} ~ x$ returns a closure around the location where $\mathit{self}$ will be placed, which is passed to $T\den{B}$ whose result is bound to $\mathit{self}$.

Happily, instead of a single big recursive macro, first-class templates and extensions make it possible to implement the various parts of copattern matching as many independent macros that can be used separately and composed by the programmer. 
For example, $\lambda P. O$, $\If M ~ O$, $\Match P \gets M ~ O$, \etc are all implemented as self-contained macros that create new extension values around other extensions.
These forms need to be macros because they either bind variables around an expression (like $\lambda P$ or $\Match$) or do not evaluate a sub-expression in some cases (like $\If$).
% Other simpler forms, like the empty object, or even $\Nest$ or the composition $O; O'$, are just ordinary procedural values and not defined as macros.
Other simpler forms, like the empty object or the composition $O; O'$, are just ordinary procedural values and not defined as macros.
The macro for copattern matching, $Q[x]~O$, is the only main recursive step, which decomposes a copattern into a sequence of more basic matching $\lambda$s.

Additionally, the source language, as implemented, is more flexible than presented in \cref{fig:source-syntax}, in the sense that there are not as many syntactic categories.  
So the $O$ in forms like $\lambda P. O$ or $\If M ~ O$ can be \emph{any} host language expression as long as it evaluates to a procedure following the calling convention of extensions (otherwise a run-time error may be encountered).
The implementation also supports other standard Scheme expressions, including functions of multiple arguments (corresponding to \scm|(lambda (P ...) O)| or the copattern \scm|(self P ...)|) and variable numbers of arguments (corresponding to \scm|(lambda (P ... . rest) O)| or the copatterns \scm|(self P ... . rest)| or \scm|(apply self P ... rest)|).
The main points where the syntactic restrictions are used are in the macros implementing $\Extension O$ or $\Template B$.
For example, the \scm|extension| macro definition is:
\begin{minted}{scheme}
(define-syntax-rule
  (extension [copat step ...] ...)
  (merge [chain (comatch copat) step ...] ...))
\end{minted}
where \scm|merge| is the regular definition of first-class function composition, \scm|comatch| is the macro for the copattern matching form $Q[self] ~ O$, and \scm|chain| is a macro for right-associating any chain of operations to avoid overly-nested parentheses, with special support for unparenthesized terminators:
\begin{minted}{scheme}
(define-syntax chain
  (syntax-rules (= try)
    [(chain ext)                   ext]
    [(chain (op ...) step ... ext) (op ... (chain step ... ext))]
    [(chain = expr)                (always-do expr)]
    [(chain try ext)               ext]))
\end{minted}

One concern for a real implementation is to consider what kind of pattern-matching facilities the host language already provides.
Unfortunately, the answer is not standard across different languages in the Scheme family.
For example, the R${}^6$RS standard does not require any built-in support for pattern matching to be fully compliant, but specific languages like Racket may include a library for pattern matching by default.
Thus, we provide two different implementations to illustrate how copatterns may be implemented depending on their host language:
\begin{itemize}
\item
  A Racket implementation that uses its standard pattern-matching constructs \rkt|match| and \rkt|match-lambda*|.
  Thus, the $\Match$  from the target language in \cref{fig:target-syntax} is interpreted as Racket's \rkt|match|, and the translation of $E\den{\lambda P. O}$ is implemented directly as \rkt|match-lambda*| instead of separating the $\lambda$ from the pattern as in \cref{fig:translation}.
  This choice ensures the pattern language implemented is exactly the same as the pattern language already used in Racket programs.
\item
  A general implementation intended for any R${}^6$RS-compliant Scheme,%
  \footnote{We have explicitly tested this implementation against Chez Scheme.}
  %
  which internally implements its own pattern-matching macro, \scm|try-match|, by expanding into other primitives like \scm|if| and comparison predicates.
  Of note, due to only having to handle a single line of pattern-matching at a time, this implementation is 75 lines of Scheme and supports quasiquoting forms of patterns.
  This gives a sufficiently expressive intersection between Racket's pattern-matching syntax and the manually implemented R${}^6$RS version.
\end{itemize}


%%% Local Variables:
%%% mode: LaTeX
%%% TeX-master: "coscheme"
%%% End:


\section{Correctness} \label{sec-correctness}

\begin{figure}[t!]
\centering
\small
\begin{alignat*}{2}
  \mathit{Value} &\ni{}& V, W
  &::= x
  \mid \lambda x. M
  \mid \Null
  \mid \Cons V \, W
  \mid \q{x}
  \\
  \mathit{EvalCxt} &\ni{}& E
  &::= \hole
  \mid E ~ M
  \mid V ~ E
  \mid \Match E \With \set{\many{P \to N}}
  \mid \Rec x = E
\end{alignat*}
% Operational axioms:
\begin{align*}
  (\beta)
  &&
  (\lambda x. M) ~ V
  &=
  M\subst{x}{V}
  % \\
  % \begin{aligned}
  %   &\Match V \With \\
  %   &\qquad\set{\many[i]{P_i \to N_i}}
  % \end{aligned}
  % &=
  % N_k\subst{BV(P_k)}{\many{W}}
  % &
  % \begin{aligned}
  %   (&\text{if } && V = P_k\subst{BV(P_k)}{\many{W}} \\
  %   &\text{and } &&\forall 1 \leq j < k, \\
  %   &&&\not\exists \many{W'}, V = P_j\subst{BV(P_j)}{\many{W'}})
  % \end{aligned}
  \\
  (\mathit{match})
  &&
  \begin{aligned}
    &\Match V \With
    \begin{aligned}[t]
    \{~ &P \to N; \\
    &\many{P' \to N'}~\}
    \end{aligned}
  \end{aligned}
  &=
  N\subst{\many{x}}{\many{W}}
  &(\text{if } P\subst{\many{x}}{\many{W}} &= V)
  % &(\text{if } \exists \many{W},~ V &= P\subst{BV(P)}{\many{W}})
  \\
  (\mathit{apart})
  &&
  \begin{aligned}
    &\Match V \With
    \begin{aligned}[t]
    \{~ &P \to N; \\
    &\many{P' \to N'}~\}
    \end{aligned}
  \end{aligned}
  &=
  \begin{aligned}
    &\Match V \With \\
    &\qquad \set{\many{P' \to N'}}
  \end{aligned}
  &(\text{if } P &\apart V)
  % &(\text{if}\!\not\exists \many{W},~ V &= P\subst{BV(P)}{\many{W}})
  \\
  (\mathit{rec})
  &&
  (\Rec x = V)
  &=
  V\subst{x}{(\Rec x = V)}
\end{align*}

% Observational axioms:
% \begin{align*}
%   % \lambda x. (V ~ x)
%   % &=
%   % V
%   % &(\text{if } x &\notin FV(V))
%   % \\
%   (\lambda x. E[x]) ~ M
%   &=
%   E[M]
%   \\
%   E\left[
%     \begin{aligned}
%       &\Match M \With \\
%       &\qquad\set{\many{P \to N}}
%     \end{aligned}
%   \right]
%   &=
%   \begin{aligned}
%     &\Match M \With \\
%     &\qquad \set{\many{P \to E[N]}}
%   \end{aligned}
%   &(\text{if } BV(P) \cap FV(E) = \emptyset)
% \end{align*}

Apartness between patterns and values ($P \apart V$):
\begin{gather*}
  \infer
  {\q{x} \apart V}
  {V \notin \mathit{Variable} \cup \set{\q{x}}}
  \qquad
  \infer
  {\Null \apart V}
  {V \notin \mathit{Variable} \cup \set{\Null}}
  \\[1ex]
  \infer
  {\Cons P ~ P' \apart V}
  {V \notin \mathit{Variable} \cup \set{\Cons W ~ W' \mid W, W' \in \mathit{Value}}}
  \\[1ex]
  \infer
  {\Cons P ~ P' \apart \Cons W ~ W'}
  {P \apart W}
  \qquad
  \infer
  {\Cons P ~ P' \apart \Cons W ~ W'}
  {P' \apart W'}
  % \infer
  % {K ~ P_1 \dots P_n \apart V}
  % {V \neq K ~ W_1 \dots W_n}
  % \qqqquad
  % \infer
  % {K~P_1 \dots P_n \apart K ~ V_1 \dots V_n}
  % {1 \leq j \leq n & P_j \apart V_j}
\end{gather*}

\caption{Untyped equational axioms of the target language.}
\label{fig:target-equality}
\end{figure}

We already used the translation to a core $\lambda$-calculus as a specification for implementing compositional copatterns, but the translation is also useful for another purpose: checking the expected meaning of copattern-matching code.
With that in mind, we now look for some laws that let us equationally reason about some programs to make sure they behave as expected.

First, the core target language --- a standard call-by-value $\lambda$-calculus extended with pattern-matching and recursion --- has the equational theory shown in \cref{fig:target-equality}, which is the \emph{reflexive}, \emph{symmetric}, \emph{transitive}, and \emph{compatible} (\ie equalities can be applied in \emph{any} context) closure of the listed rules.
It has the usual $\beta$ axiom (restricted to substituting value arguments), two axioms for handling pattern-match success ($\mathit{match}$) and failure ($\mathit{apart}$), and an axiom for unrolling recursive values ($\mathit{rec}$).
Values ($V, W$) include the usual ones for call-by-value $\lambda$-calculus ($x$ and $\lambda x. M$) as well as lists ($\Null$ and $\Cons V ~ W$) and symbolic literals ($\q x$).
Matching a value $V$ against a pattern $P$ will succeed if the variables ($\many{x}$) in the pattern can be replaced by other values ($\many{W}$) to generate exactly that $V$: $P\subst{\many{x}}{\many{W}} = V$.
In contrast, matching fails if the two are known to be \emph{apart}, written $P \apart V$ and defined in \cref{fig:target-equality}, which implies that all possible substitutions of $P$ will \emph{never} generate $V$.
Note that while matching and apartness are mutually exclusive, there are some values that are neither matching nor apart from some patterns.
For example, compare the variable $x$ against the pattern $\Null$; $x$ may indeed stand for $\Null$ or another value like $\lambda y. M$.

The first usual property is that the translation specified in \cref{fig:translation} is a \emph{conservative extension}: any two terms that are equal by the target equational theory are still equal after translation.
Because the translation is hygienic and compositional by definition,  we can follow the proof strategy in \cite{DownenAriola2014CSCC}.

\begin{restatable}[Conservative Extension]{proposition}{thmconservativeextension}
  \label{thm:conservative-extension}
  If $M = N$ in the equational theory of the target
  (\cref{fig:target-equality}), then so too does $\den{M} = \den{N}$.
\end{restatable}

\begin{figure}[t!]
\centering
\small
\begin{alignat*}{2}
  % \mathit{TemplateValue} &\ni{}& B_v
  % &::= \varepsilon
  % \mid O_v; B_v
  % \mid \Continue x \to V
  % \\
  % \mathit{ExtensionValue} &\ni{}& O_v
  % &::= O_f
  % \mid O_f; O_v
  % \mid \Nest O_v
  % \mid \Try x \to B_v
  % \\
  \mathit{ExtensionFunc} &\ni{}& F
  &::= Q[x ~ P] ~ O
  \mid \lambda P. O
  \\
  \mathit{Value} &\ni{}& V
  &::= \dots
  \mid \lamstar (F; B)
  \mid \Template B
  \mid \Extension O
  % \mathit{NonRecTemplate} &\ni{}& B_{nr}
  % &::= O_{nr}; B_{nr}
  % \mid \Else \to M
  % \\
  % \mathit{NonRecExtension} &\ni{}& O_{nr}
  % &::= \varepsilon
  % \mid O_{nr}; O'_{nr}
  % \mid Q[\_] ~ O_{nr}
  % \mid \lambda P. O_{nr}
  % \\
  % &&&\phantom{:=}
  % \mid \Match P \gets M ~ O_{nr}
  % \mid \Nest O
  % \mid \Try x \to B_{nr}
  % \\
  % \mathit{RecCxt} &\ni{}& R
  % &::= \hole
  % \mid R; O
  % \mid O; R
  % \mid Q[x] ~ R
  % \mid \lambda P. R
  % \mid \Match P \gets M ~ O
  % \mid \Try x \to R
\end{alignat*}
Identity, associativity, and annihilation laws of composition:
\begin{align*}
  \varepsilon; O &= O % = O; \varepsilon
  &
  (O_1; O_2); O_3 &= O_1; (O_2; O_3)
  &
  \Do M; O &= \Do M
  \\
  \varepsilon; B &= B
  &
  (O_1; O_2); B &= O_1; (O_2; B)
  &
  \Do M; B &= \Else M
\end{align*}

% Decomposing patterns and copatterns:
% \begin{align*}
%   (Q[x] ~ P) ~ O
%   &=
%   Q[x] ~ (\lambda P. O)
%   &
%   \_ ~ O
%   &=
%   O
%   &
%   \lambda P. O
%   &=
%   \lambda x. (\Match P \gets x ~ O)
% \end{align*}

% Factoring out recursion ($x \neq y$ and $x \notin BV(P)$):
% \begin{align*}
%   \lambda y. (x ~ O)
%   &=
%   x ~ (\lambda y. O)
%   &
%   \Match P \gets M ~ (x ~ O)
%   &=
%   x ~ (\Match P \gets M ~ O)
%   \\
%   (x ~ O); O'
%   &=
%   x ~ (O; O')
%   &
%   O; (x ~ O')
%   &=
%   x ~ (O; O')
% \end{align*}

% Instantiating templates and recursive $\lamstar$:
% \begin{align*}
%   % (\Extension O) ~ V
%   % &=
%   % \Template{} (O; \Continue x \to (V ~ x))
%   % \\
%   % (\Template R[Q[x] ~ O]) ~ V
%   % &=
%   % (\Template R[Q[\_] ~ O\subst{x}{V}]) ~ V
%   % \\
%   % (\Template R[\Continue x \to M]) ~ V
%   % &=
%   % (\Template R[\Else \to M\subst{x}{V}]) ~ V
%   % \\
%   % (\Template B_{nr}) ~ V
%   % &=
%   % (\Template B_{nr}) ~ W
%   % \\
%   \lamstar (F; B)
%   &=
%   (\Template F; B) ~ (\lamstar (F; B))
%   % \\
%   % \lambda x. (\lamstar (F; B)) ~ x
%   % &=
%   % \lamstar (F; B)
%   \\
%   (\Template O; B) ~ V
%   &=
%   (\Extension O) ~ (\Template B) ~ V
%   \\
%   (\Template \varepsilon) ~ V
%   &=
%   \mathit{fail}~V
%   \\
%   (\Template \Continue x \to M) ~ V
%   &=
%   M\subst{x}{V}
%   \\
%   (\Extension \Try x \to B) ~ V
%   &=
%   \Template B\subst{x}{V}
% \end{align*}

Pattern and copattern matching:
\begin{align*}
  \Match P \gets V ~ O
  &=
  O\subst{\many{x}}{\many{W}}
  &(\text{if } P\subst{\many{x}}{\many{W}} &= V)
  \\
  \Match{} P \gets V ~ O
  &=
  \varepsilon
  &(\text{if } P &\apart V)
  \\[1ex]
  (\Template{} (\lambda P. \Do M); B) ~ V' ~ V
  &=
  M\subst{\many{x}}{\many{W}}
  &(\text{if } P\subst{\many{x}}{\many{W}} &= V)
  \\
  (\Template{} (\lambda P. O); B) ~ V' ~ V
  &=
  (\Template B) ~ V' ~ V
  &(\text{if } P &\apart V)
  \\[1ex]
  C[(\Template{} (Q[y] = M); B) ~ V]
  &=
  M\subst{y}{V}\subst{\many{x}}{\many{W}}
  &(\text{if } Q\subst{\many{x}}{\many{W}} &= C)
  \\
  C[(\Template{} (Q[y] ~ O); B) ~ V]
  &=
  C[(\Template B) ~ V]
  &(\text{if } Q &\apart C)
  \\[1ex]
  C[\lamstar (Q[y] = M); B]
  &=
  \begin{aligned}[t]
    M
    &\subst{y}{(\lamstar (Q[y] = M); B)}
    \\
    &\subst{\many{x}}{\many{W}}
  \end{aligned}
  &(\text{if } Q\subst{\many{x}}{\many{W}} &= C)
  \\
  C[\lamstar (Q[y] ~ O); \Else M]
  &=
  C[M]
  &(\text{if } Q &\apart C)
\end{align*}

Apartness between copatterns and contexts ($Q \apart C$):
\begin{gather*}
  \infer
  {Q ~ P \apart C ~ V}
  {Q\subst{\many{x}}{\many{W}} = C & P \apart V}
  \qqqquad
  \infer
  {Q ~ P \apart C}
  {Q \apart C}
  \qqqquad
  \infer
  {Q \apart C ~ V}
  {Q \apart C}
\end{gather*}

\caption{Some equalities of copattern extensions.}
\label{fig:source-equality}
\end{figure}

To reason about the new features in the source language --- introduced by $\lamstar$, $\Template$, and $\Extension$ --- we introduce additional axioms given in \cref{fig:source-equality}, so that the source equational theory is the \emph{reflexive}, \emph{symmetric}, \textit{transitive}, and \emph{compatible} closure of these rules in both \cref{fig:target-equality,fig:source-equality}.
The purpose of these new equalities is to perform some reasoning about programs using copatterns, and in particular, to check that the syntactic use of \scm|=| really means equality.
For example, a special case is
\begin{math}
  Q[\lamstar(Q[y] = M); B] = M\subst{y}{\lamstar(Q[y]=M);B}
  ,
\end{math}
which says a $\lamstar$ appearing in \emph{exactly} the same context as the left-hand side of an equation will unroll (recursively) to the right-hand side.
Other equations describe algebraic laws of copattern alternatives and how to fill in templates and extensions when applied.
This source equational theory is \emph{sound} with respect to translation.

\begin{restatable}[Soundness]{proposition}{thmsoundness}
  \label{thm:soundness}
  The translation is \emph{sound} w.r.t. the source and target equational theories (\eg $M = N$ in \cref{fig:source-equality} implies $\den{M} = \den{N}$ in \cref{fig:target-equality}).
  % The equational axioms given in \cref{fig:source-equality} are sound with
  % respect to the translation in \cref{fig:translation},
  % \begin{align*}
  %   M &= M' &&\implies & \den{M} &= \den{M'} \\
  %   B &= B' &&\implies & T\den{B} &= T\den{B'} \\
  %   O &= O' &&\implies & E\den{O} &= E\den{O'}
  % \end{align*}
  % up to the equational theory of the target language in
  % \cref{fig:target-equality}.
\end{restatable}


%%% Local Variables:
%%% mode: LaTeX
%%% TeX-master: "coscheme"
%%% End:


% \section{Optimizing Away Administrative Reductions} \label{sec-opt}

% \begin{figure}
\centering
Translating new terms:  
\begin{align*}
  \den{\lamstar B}()
  &=
  \Rec \mathit{self} = T\den{B}(\lambda x. \mathit{self} ~ x)
  \\
  \den{\Template B}()
  &=
  \lambda s. T\den{B}(s)
  \\
  \den{\Extension O}()
  &=
  \lambda b. \lambda s. E\den{O}(b, s)
  \\
  \den{M}()
  &=
  \text{by induction}
  &(\text{otherwise})
\end{align*}
Translating templates:
\begin{align*}
  T\den{\varepsilon}(V)
  &=
  \mathit{fail}~V
  \\
  T\den{O; B}(V)
  &=
  (\lambda b. E\den{O}(b, V)) ~ (\lambda s. T\den{B}(s))
  \\
  T\den{\Continue x \to M}(V)
  &=
  \den{M}()\subst{x}{V}
\end{align*}

Translating copattern-matching and pattern-matching functions:
\begin{align*}
  E\den{(Q[x] ~ P) ~ O}(W, V)
  &=
  E\den{Q[x] ~ (\lambda P. O)}(W, V)
  \\
  E\den{x ~ O}(W, V)
  &=
  E\den{O}(W, V)\subst{x}{W}
  \\
  E\den{\lambda P. O}(W, V)
  &=
  E\den{\lambda x. \Match P \gets x ~ O}(W, V)
  &(\text{if } P \notin \mathit{Variable})
\end{align*}

Translating other extensions:
\begin{align*}
  E\den{\varepsilon}(W, V)
  &=
  W(V)
  \\
  E\den{O; O'}(W, V)
  &=
  (\lambda b. E\den{O}(b, V)) ~ (\lambda s. E\den{O'}(W, s))
  \\
  E\den{\lambda x. O}(W, V)
  &=
  \lambda x. E\den{O}((\lambda s'. W(s') ~ x), V)
  \\
  E\den{\Match P \gets M ~ O}(W, V)
  &=
  \Match \den{M}() \With
  \set{P \to E\den{O}(W, V); \_ \to W(V)}
  \\
  E\den{\Nest O}(W, V)
  &=
  \Rec s' = E\den{O}((\lambda \_. W(V)), (\lambda x. s' ~ x))
  \\
  E\den{\Try x \to B}(W, V)
  &=
  T\den{B}(V)\subst{x}{W}
\end{align*}

Inlining administrative $\lambda$-abstractions:
\begin{align*}
  (\lambda x. M)(W, \many{V})
  &=
  M(\many{V})\subst{x}{W}
  \\
  M(\many{V})
  &=
  M ~ \many{V}
  &(\text{otherwise})
\end{align*}

\caption{Inlining version of the translation.}
\label{fig:inline-translation}
\end{figure}

\begin{proposition}
  Up to the equational theory of the target language in
  \cref{fig:target-equality},
  \begin{align*}
    \den{M}() &= \den{M}
    \\
    \lambda s. T\den{B}(s) &= T\den{B}
    \\
    \lambda b. \lambda s. E\den{O}(b, s) &= E\den{O}
  \end{align*}  
\end{proposition}


%%% Local Variables:
%%% mode: LaTeX
%%% TeX-master: "coscheme"
%%% End:


\section{Related and Future Work} \label{sec-future}

\section{Conclusion} \label{sec-conclusion}

\begin{credits}
\subsubsection{\ackname}
% 
This material is based upon work supported by the National Science Foundation
under Grant No. 2245516.

\subsubsection{\discintname}
%
The authors have no competing interests to declare that are
relevant to the content of this article.
\end{credits}
%
% ---- Bibliography ----
%
% BibTeX users should specify bibliography style 'splncs04'.
% References will then be sorted and formatted in the correct style.
%
\bibliographystyle{splncs04}
\bibliography{refs}

\section{Appendix A: Proofs}
\allowdisplaybreaks

\adriano{Review the copattern case (probably use induction)}

\adriano{Figure out the correct argument to assure things about $\den{M}$'s}


\thmvaluetranslation*
\begin{proof}
  % The listed instances of translation are all values up to the equational theory of the target language in \cref{fig:target-equality}:

  % The proof follows
  By mutual induction on the syntax of templates $B$, extensions $O$, extension functions $F$, and values $V$.
  % By definition, we know that our translation returns terms in the target language.
  % In other words, we can say that $\den{M}=M'$ for some target term $M'$.
  \begin{enumerate}[(a)]
  \item $T\den{B} = \lambda s. M$ for some term $M$, as shown by the following cases:
    % \begin{proof}
    %   By induction on $B$.
    \begin{itemize}
    \item $(B = \varepsilon)$
      % By our translation we have:
      $T\den{\varepsilon} = \lambda s. ~\mathit{fail} ~s$, so $M=\mathit{fail} ~s$.
    \item ($B = O; B'$)
      $T\den{O; B'} = \lambda s. ~\den{O} ~\den{B'} ~s$, so $M = \den{O} ~\den{B'} ~s$.
      % By our translation we have: $T\den{O; B'} = \lambda s. ~\den{O} ~\den{B'} ~s$.
      % Our induction hypothesis provides us that $T\den{B'} = \lambda s. M'$ for some term $M'$.
      % % By (b) we know that $E\den{O}=\lambda b. \lambda s. M_O$ for some term $M_O$.
      % \begin{align*}
      %   & \quad \lambda s. ~\den{O} ~\den{B'} ~s \\
      %   % =& \quad \lambda s.  ~(\lambda b. \lambda s. M_O) ~(\lambda s. M') ~s & (IH, (b))
      %   =& \quad \lambda s. ~M_O ~(\lambda s. M') ~s & (IH)
      % \end{align*}
      % Where $M=M_O ~(\lambda s. M') ~s$.
    \item $(B = \Continue x {\,\to\,} N)$
      $T\den{\Continue x {\,\to\,} N} = \lambda x. \den{N}$, so $M = \den{N}$.
      % \begin{align*}
      %   & \quad T\den{\Continue x \to M'} = \lambda x. ~\den{M'} \\
      %   =& \quad \lambda x. ~(\lambda s. ~M'') \\
      %   =& \quad \lambda s. ~(\lambda s. ~M'') \subst{x}{s} & (=_{\alpha})
      % \end{align*} 
    \end{itemize}
    % \qed
    % \end{proof}
  \item $E\den{O} = \lambda b. \lambda s. M$ for some term $M$, as shown by the following cases: 
    % \begin{proof}
    %   By induction on $O$.
    \begin{itemize}
    \item $(O = \varepsilon)$
      % By our translation we have
      $E\den{\varepsilon} = \lambda b. \lambda s. ~b ~s$, so $M= b ~s$.
    \item $(O = O_1;O_2)$
      $E\den{O_1;O_2} = \lambda b. \lambda s. ~\den{O_1} ~(\den{O_2} ~b) ~s$, so $M = \den{O_1} ~(\den{O_2} ~b) ~s$.
      % \begin{align*}
      %   & \quad E\den{O_1;O_2} = \lambda b. \lambda s. ~\den{O_1} ~(\den{O_2} ~b) ~s\\
      %   =& \quad \lambda b. \lambda s. ~(\lambda b. \lambda s. ~M_{O_1}) ~((\lambda b. \lambda s. ~M_{O_2}) ~b) ~s & (IH)
      % \end{align*}
      % Where $M=(\lambda b. \lambda s. ~M_{O_1}) ~((\lambda b. \lambda s. ~M_{O_2}) ~b) ~s$.
    \item $(O = Q[x] ~O)$ follows by induction on $Q$ (generalizing $O$):
      \begin{itemize}
      \item $(Q=\hole)$ for all $O$,
        % According to our translation:
        \begin{align*}
          E\den{x ~O}
          &=
          \lambda b. \lambda x. E\den{O} ~ b ~ x
          \\
          &=
          \lambda b. \lambda s. E\den{O}\subst{x}{s} ~ b ~ s
          &(\alpha)
        \end{align*}
        so $M = E\den{O}\subst{x}{s} ~ b ~ s$ for the given $O$.
        % By our induction hypothesis we can infer that $E\den{O}=\lambda b. \lambda s. M_O$.
        % Therefore, we can conclude that $\lambda b. \lambda x. ~(\lambda b. \lambda s. M_O) ~ b ~ x$, where $ M = (\lambda b. \lambda s. M_O) ~ b ~ x$.
      \item $(Q=Q'~P)$
        assuming the inductive hypothesis (IH) that, for all $O$, there is an $N_O$ such that $E\den{Q[x] ~ O} = \lambda b. \lambda s. N_O$.
        For all $O$,
        \begin{align*}
          E\den{(Q'[x] ~ P) O}
          &=
          E\den{Q'[x] ~ (\lambda P. O)}
          \\
          &=
          \lambda b. \lambda s. N_{(\lambda P. O)}
          &(IH)
        \end{align*}
        so $M = N_{(\lambda P. O)}$ given by the inductive hypothesis applied to $\lambda P. O$.
        % According to our translation:
        % $E\den{(Q ~P) ~ [x] ~O} = E\den{(Q[x] ~P) ~O} = E\den{Q[x] ~ (\lambda P. O)}$.
        % Eventually, we will hit the previous case after converting all patterns in the copattern into pattern $\lambda$'s.
        % After converting all patterns $O$ will have a shape of form: $\lambda P_0. ~\lambda P_1 ... \lambda P_n. ~O$.
        % \adriano{I think I need to use induction here.}
      \end{itemize}
    \item $(O = \lambda P. ~O)$ follows by cases if $P$ is a variable or another pattern:
      \begin{itemize}
      \item $(P \in \mathit{Variable})$
        \begin{math}
          E\den{\lambda x. ~O}
          =
          \lambda b. \lambda s.
          (\lambda x. E\den{O} ~ (\lambda s'. b ~ s' ~ x) ~ s)
          ,
        \end{math}
        so
        \\
        $M = \lambda x. E\den{O} ~ (\lambda s'. b ~ s' ~ x) ~ s$.
      \item $(P \notin \mathit{Variable})$
      \begin{math}
        E\den{\lambda P. ~O}
        =
        E\den{\lambda x. \Match P \gets x ~ O}
        ,
      \end{math}
      which follows by the above case.
      \end{itemize}
      % According to our translation:
      % \begin{align*} 
      %   & \quad E\den{\lambda P. ~O} = E\den{\lambda x. \Match P \gets x ~ O}\\
      %   =& \quad \lambda b. \lambda s. (\lambda x. E\den{\Match P \gets x ~ O} ~ (\lambda s'. b ~ s' ~ x) ~ s) \\
      %   =& \quad \lambda b. \lambda s. (\lambda x. \lambda b. \lambda s. (
      %   \begin{aligned}[t]
      %     &\Match \den{M} \With \\
      %     &\quad
      %     \begin{aligned}[t]
      %       \{~
      %       P &\to E\den{O}~b~s; \\
      %       \_ &\to b~s
      %       ~\}
      %     \end{aligned}
      %   \end{aligned}
      %   )~ (\lambda s'. b ~ s' ~ x) ~ s) \\
      %   =& \quad \lambda b. \lambda s. (\lambda x. \lambda b. \lambda s. (
      %   \begin{aligned}[t]
      %     &\Match \den{M} \With \\
      %     &\quad
      %     \begin{aligned}[t]
      %       \{~
      %       P &\to (\lambda b. \lambda s. M_O)~b~s; \\
      %       \_ &\to b~s
      %       ~\}
      %     \end{aligned}
      %   \end{aligned}
      %   )~ (\lambda s'. b ~ s' ~ x) ~ s) & (IH)
      % \end{align*}
    \item $(O = \Match P \gets N ~ O)$
        \begin{align*}
          & \quad E\den{\Match P \gets N ~ O}
          =
          \lambda b. \lambda s.
          \begin{aligned}[t]
            &\Match \den{N} \With \\
            &\quad
            \begin{aligned}[t]
              \{~
              P &\to E\den{O}~b~s; \\
              \_ &\to b~s
              ~\}
            \end{aligned}
          \end{aligned}
          % \\
          % =& \quad \lambda b. \lambda s.
          % \begin{aligned}[t]
          %   &\Match \den{N} \With \\
          %   &\quad
          %   \begin{aligned}[t]
          %     \{~
          %     P &\to (\lambda b. \lambda s.~ M_O) ~b~s; \\
          %     \_ &\to b~s
          %     ~\}
          %   \end{aligned}
          % \end{aligned}  & (IH)
        \end{align*}
        so $M = \Match \den{N} \With \set{P \to E\den{O}~b~s; \_ \to b~s}$.
      \item $(O = \Nest O)$
        \begin{math}
          E\den{\Nest O}
          =
          \lambda b. \lambda s.
          (\Rec s' = E\den{O} ~ (\lambda \_. b ~ s) ~ (\lambda x. s' ~ x))
          ,
        \end{math}
        so $M = (\Rec s' = E\den{O} ~ (\lambda \_. b ~ s) ~ (\lambda x. s' ~ x))$
        % \begin{align*}
        %   & \quad E\den{\Nest O} = \lambda b. \lambda s. \Rec s' = E\den{O} ~ (\lambda \_. b ~ s) ~ (\lambda x. s' ~ x)\\
        %   =& \quad \lambda b. \lambda s. \Rec s' = (\lambda b. \lambda s.~ M_O) ~ (\lambda \_. b ~ s) ~ (\lambda x. s' ~ x) & (IH)
        % \end{align*}
      \item $(O = \Try x \to B)$
        assuming the inductive hypothesis $(IH)$ from part (a) that
        $T\den{B} = \lambda s. N$ for some $N$,
        \begin{align*}
          E\den{\Try x \to B}
          &= \lambda x. T\den{B}
          \\
          &= \lambda x. (\lambda s.~ N)
          & (IH)
          \\
          &=\lambda b. (\lambda s.~ N)\subst{x}{b}
          & (\alpha)
        \end{align*}
        so $M = N\subst{x}{b}$
    \end{itemize}
    % \qed
    % \end{proof}
  \item $E\den{F} = \lambda b. \lambda s. \lambda x. M$ for some term $M$, as shown by the following cases:
    % \begin{proof}
    %   Let us consider the possible values of $F$.
    \begin{itemize}
    \item $(F= \lambda P. ~O)$
      Following the same calculation in the matching special case of part (b) above,
      $E\den{\lambda P. O} = \lambda b. \lambda s. (\lambda x. M)$
      for some $M$.
    %   Let' consider if $P$ is a variable pattern or not:
    %   \begin{itemize}
    %   \item $(P = x)$ According to our translation $E\den{\lambda x. ~O}=\lambda b. \lambda s. (\lambda x. E\den{O} ~ (\lambda s'. b ~ s' ~ x) ~ s)$.
    %     Therefore, by (b), we can conclude that $\lambda b. \lambda s. (\lambda x. (\lambda b. \lambda s. M_O) ~ (\lambda s'. b ~ s' ~ x) ~ s)$
    %   \item $(P \neq x)$ We have that $E\den{\lambda P. ~O}=E\den{\lambda x. \Match P \gets x ~ O}$ which is a specific case of the previous case, where $O=\Match P \gets x ~ O$.
    %   \end{itemize} 
    \item $(F= Q[x ~P] ~O)$
      follows by induction on $Q$ (generalizing $O$):
      \begin{itemize}
      \item $(Q = \hole)$ for all $O$,
        \begin{align*}
          E\den{(y ~ P) ~ O}
          =
          E\den{y ~ (\lambda P. O)}
          &=
          \lambda b. \lambda y. E\den{\lambda P. O} ~ b ~ y
          % \\
          % &=
          % \lambda b. \lambda s. E\den{\lambda P. O}\subst{y}{s} ~ b ~ s
          % &(\alpha)
        \end{align*}
        Following the same calculation in the previous case $(F = \lambda P. O)$ gives us
        some $N$ such that $E\den{\lambda P. O} = \lambda b. \lambda s'. \lambda x. N$, so continuing we have
        \begin{align*}
          E\den{(y ~ P) ~ O}
          &=
          \lambda b. \lambda y. (\lambda b. \lambda s'. \lambda x. N) ~ b ~ y
          \\
          &=
          \lambda b. \lambda y. \lambda x. N\subst{s'}{y}
          &(\beta)
          \\
          &=
          \lambda b. \lambda s. \lambda x. N\subst{s'}{y}\subst{y}{s}
          &(\alpha)
        \end{align*}
        so $M = N\subst{s'}{y}\subst{y}{s}$.
      \item $(Q = Q' ~ P')$
        assuming the inductive hypothesis that,
        for all $O$, there is an $N_O$ such that
        $E\den{Q'[y ~ P] ~ O} = \lambda b. \lambda s. \lambda x. N_O$.
        For all $O$,
        \begin{align*}
          E\den{(Q'[y ~ P] ~ P') ~ O}
          &=
          E\den{Q'[y ~ P] ~ (\lambda P'. O)}
          \\
          &=
          \lambda b. \lambda s. \lambda x. N_{(\lambda P'. O)}
          &(IH)
        \end{align*}
        so $M = N_{(\lambda P'. O)}$ given by the inductive hypothesis applied to $\lambda P'. O$.
      \end{itemize}
      % Eventually, after converting the whole copattern we will hit $E\den{x ~O}$, where $O$ is some sequence of nested $\lambda P$'s.
      % \begin{align*}
      %   & \quad E\den{x ~O} =  \lambda b. \lambda x. E\den{\lambda P. O'} ~ b ~ x\\
      %   =& \quad \lambda b. \lambda x. (\lambda b. \lambda s. \lambda x. M') ~ b ~ x & (IH) \\
      %   =& \quad \lambda b. \lambda x. (\lambda x. M'\subst{s}{x}) & (\beta) \\
      %   =& \quad \lambda b. \lambda s. (\lambda x. M'\subst{s}{x})\subst{x}{s} & (\alpha)
      % \end{align*}
    \end{itemize}
    % \qed
    % \end{proof}
  \item $\den{V} = W$ for some value $W$, as shown by the following cases:
    % \begin{proof}
    %   By induction on $V$.
    \begin{itemize}
    \item $(V = x)$
      $\den{x} = x$, so $W=x$.
    \item $(V = \lambda x. M)$
      $\den{\lambda x. M} = \lambda x. \den{M}$, so $W = \lambda x. \den{M}$.
    \item $(V = \Null)$
      $\den{\Null} = \Null$, so $W=\Null$.
    \item $(V = \Cons V_1 ~ V_2)$
      $\den{\Cons V_1 ~ V_2} = \Cons \den{V_1} ~ \den{V_2}$,
      where $\den{V_1} = W_1$ and $\den{V_2} = W_2$ by the inductive hypotheses,
      so $W = \Cons W_1 ~ W_2$.
    \item $(V = \Template B)$
      $\den{\Template B} = T\den{B} = \lambda s. M$, for some $M$,
      by the inductive hypothesis part (a),
      so $W = \lambda s. M$.
    \item $(V = \Extension O)$
      $\den{\Extension O} = E\den{O} = \lambda b. \lambda s. M$, for some $M$,
      by the inductive hypothesis part (b),
      so $W = \lambda b. \lambda s. M$.
    \item $(V = \lamstar (F; B))$
      assuming the inductive hypotheses that 
      \begin{itemize}
      \item[$IH_1$] there is some $N_1$ such that
        $E\den{F} = \lambda b. \lambda s. \lambda x. N_1$, and
      \item[$IH_2$] there is some $N_2$ such that
        $T\den{B} = \lambda s. N_2$,
      \end{itemize}
      \begin{align*}
        \den{\lamstar (F; B)}
        &=
        (\Rec \mathit{self} = T\den{F; B} ~ (\lambda x. \mathit{self} ~ x))
        \\
        &=
        (\Rec \mathit{self} = E\den{F} ~ T\den{B} ~ (\lambda x. \mathit{self} ~ x))
        &(\beta)
        \\
        &=
        (\Rec \mathit{self}
        = (\lambda b. \lambda s. \lambda x. N_1)
        ~ T\den{B}
        ~ (\lambda x. \mathit{self} ~ x))
        &(IH_1)
        \\
        &=
        (\Rec \mathit{self}
        = (\lambda b. \lambda s. \lambda x. N_1)
        ~ (\lambda s. N_2)
        ~ (\lambda x. \mathit{self} ~ x))
        &(IH_2)
        \\
        &=
        (\Rec \mathit{self}
        = \lambda x.N_1\subst{b}{(\lambda s.N_2)}\subst{s}{(\lambda x.\mathit{self}~x)})
        &(\beta)
        \\
        &=
        \begin{aligned}[t]
          \lambda x.
          N_1
          &\subst{b}{(\lambda s.N_2)}
          \\
          &\subst{s}{(\lambda x.\mathit{self}~x)}
          \\
          &\subst
          {\mathit{self}}
          {
            (\Rec \mathit{self}
            =
            \lambda x.N_1\subst{b}{(\lambda s.N_2)}\subst{s}{(\lambda x.\mathit{self}~x)})
          }
        \end{aligned}
        &(rec)
      \end{align*}
    % \qed
    % \end{proof}
    \end{itemize}
  \end{enumerate}
\end{proof}

\thmsoundness*
\begin{proof}
  Each axiom is derived up to the target equational theory in the following lemmas.
\end{proof}

\begin{lemma}[Extension Composition Identity Left]
  \label{thm:ext-compose-id-left}
    $ E\den{\varepsilon; O} = E\den{O}.$
\end{lemma}
    \begin{proof}
        \begin{align*}
            &\quad E\den{\varepsilon; O} \\
            =& \quad \lambda b. \lambda s.~ E\den{\varepsilon} ~(E\den{O} ~b) ~s \\
            =&  \quad \lambda b. \lambda s.~ E\den{\varepsilon} ~((\lambda b. \lambda s.~ M_0) ~b) ~s & (\cref{l-value})\\
            =& \quad \lambda b. \lambda s.~ (\lambda b. \lambda s.~ b ~s) ~((\lambda b. \lambda s.~ M_0) ~b) ~s \\
            =& \quad \lambda b. \lambda s.~ (\lambda b. \lambda s.~ b ~s) ~(\lambda s.~ M_0) ~s & (\beta)\\
            =& \quad \lambda b. \lambda s.~ (\lambda s.~ (\lambda s.~ M_0) ~s) ~s & (\beta)\\
            =& \quad \lambda b. \lambda s.~ (\lambda s.~ M_0) ~s & (\beta)\\
            =& \quad \lambda b. \lambda s.~ M_0 & (\beta) \\
            =& \quad E\den{O} & (\cref{l-value})
        \end{align*}
        \qed
    \end{proof}

%% Paul: Removing this lemma since it is the only one that requires eta.
    
% \begin{lemma}[Extension Composition Identity Right]
%   \label{thm:ext-compose-id-right}
%   $ E\den{O} = E\den{O;\varepsilon}.$
% \end{lemma}
%     \begin{proof}
%         \begin{align*}
%             &\quad E\den{O;\varepsilon} \\
%             =& \quad \lambda b. \lambda s.~ E\den{O} ~(E\den{\varepsilon} ~b) ~s \\
%             =& \quad \lambda b. \lambda s.~ E\den{O} ~((\lambda b. \lambda s.~ b ~s) ~b) ~s \\
%             =& \quad \lambda b. \lambda s.~ E\den{O} ~(\lambda s.~ b ~s) ~s & (\beta)\\
%             =& \quad \lambda b. \lambda s.~ (\lambda b. \lambda s.~ M_0) ~(\lambda s.~ b ~s) ~s & (\cref{l-value})\\
%             =& \quad \color{red}{\lambda b. \lambda s.~ (\lambda b. \lambda s.~ M_0) ~b ~s} & \color{red}{(\eta)}\\
%             =& \quad \color{red}{\lambda b. \lambda s.~ (\lambda s.~ M_0) ~s} & (\beta)\\
%             =& \quad \color{red}{\lambda b. \lambda s.~ M_0} & (\beta) \\
%             =& \quad \color{red}{E\den{O}} & (\cref{l-value})
%         \end{align*}
%         \qed
%     \end{proof}

\begin{lemma}[Template Composition Identity Left]
  \label{thm:tmpl-compose-id-left}
  $ T\den{\varepsilon; B} = T\den{B}.$
\end{lemma}
    \begin{proof}
        \begin{align*}
            &\quad T\den{\varepsilon; B} \\
            =& \quad \lambda s. ~ E\den{\varepsilon} ~T\den{B} ~s \\
            =& \quad \lambda s. ~ (\lambda b. \lambda s.~ b ~s) ~T\den{B} ~s \\
            =& \quad \lambda s. ~ (\lambda b. \lambda s.~ b ~s) ~(\lambda s.~ M_0) ~s & (\cref{l-value})\\
            =& \quad \lambda s. ~ (\lambda s.~ (\lambda s.~ M_0) ~s) ~s & (\beta)\\
            =& \quad \lambda s. ~ (\lambda s.~ M_0) ~s & (\beta)\\
            =& \quad \lambda s. ~ M_0 & (\beta) \\
            =& \quad T\den{B} & (\cref{l-value})
        \end{align*}
        \qed
    \end{proof}

\begin{lemma}[Extension Composition Associativity]
  \label{thm:ext-compose-assoc}
  \\
  $ E\den{(O_1;O_2);O_3} = E\den{O_1;(O_2;O_3)}.$
\end{lemma}
    \begin{proof}
        \begin{align*}
            &\quad E\den{(O_1;O_2);O_3} \\
            =& \quad \lambda b. \lambda s.~ E\den{O_1;O_2} ~(E\den{O_3} ~b) ~s \\
            =& \quad \lambda b. \lambda s.~ (\lambda b. \lambda s.~ E\den{O_1} ~(E\den{O_2} ~b) ~s) ~(E\den{O_3} ~b) ~s \\
            =& \quad \lambda b. \lambda s.~ (\lambda b. \lambda s.~ E\den{O_1} ~(E\den{O_2} ~b) ~s) ~((\lambda b. \lambda s.~ M_3) ~b) ~s & (\cref{l-value})\\
            =& \quad \lambda b. \lambda s.~ (\lambda b. \lambda s.~ E\den{O_1} ~(E\den{O_2} ~b) ~s) ~(\lambda s.~ M_3) ~s & (\beta)\\
            =& \quad \lambda b. \lambda s.~ (\lambda b. \lambda s.~ (\lambda b. \lambda s.~ M_1) ~(E\den{O_2} ~b) ~s) ~(\lambda s.~ M_3) ~s & (\cref{l-value})\\
            =& \quad \lambda b. \lambda s.~ (\lambda b. \lambda s.~ (\lambda b. \lambda s.~ M_1) ~((\lambda b. \lambda s.~ M_2) ~b) ~s) ~(\lambda s.~ M_3) ~s & (\cref{l-value})\\
            =& \quad \lambda b. \lambda s.~ (\lambda b. \lambda s.~ (\lambda b. \lambda s.~ M_1) ~(\lambda s.~ M_2) ~s) ~(\lambda s.~ M_3) ~s & (\beta) \\
            =& \quad \lambda b. \lambda s.~ (\lambda b. \lambda s.~ (\lambda s.~ M_1\subst{b}{\lambda s.~ M_2}) ~s) ~(\lambda s.~ M_3) ~s & (\beta) \\
            =& \quad \lambda b. \lambda s.~ (\lambda s.~ (\lambda s.~ M_1\subst{b}{\lambda s.~ M_2}\subst{b}{\lambda s.~ M_3}) ~s) ~s & (\beta) \\
            =& \quad \lambda b. \lambda s.~ (\lambda s.~ M_1\subst{b}{\lambda s.~ M_2}\subst{b}{\lambda s.~ M_3}) ~s & (\beta) \\
            =& \quad \lambda b. \lambda s.~ \color{red}{M_1\subst{b}{\lambda s.~ M_2}\subst{b}{\lambda s.~ M_3}} & (\beta)
        \end{align*}

        \begin{align*}
            &\quad E\den{O_1;(O_2;O_3)} \\
            =& \quad \lambda b. \lambda s.~ E\den{O_1} ~(\den{O_2;O_3} ~b) ~s \\
            =& \quad \lambda b. \lambda s.~ E\den{O_1} ~((\lambda b. \lambda s.~ E\den{O_2} ~(E\den{O_3} ~b) ~s) ~b) ~s \\
            =& \quad \lambda b. \lambda s.~ E\den{O_1} ~(\lambda s.~ E\den{O_2} ~(E\den{O_3} ~b) ~s) ~s & (\beta) \\
            =& \quad \lambda b. \lambda s.~ (\lambda b. \lambda s.~ M_1) ~(\lambda s.~ E\den{O_2} ~(E\den{O_3} ~b) ~s) ~s & (\cref{l-value})\\
            =& \quad \lambda b. \lambda s.~ (\lambda b. \lambda s.~ M_1) ~(\lambda s.~ (\lambda b. \lambda s.~ M_2) ~(E\den{O_3} ~b) ~s) ~s & (\cref{l-value})\\
            =& \quad \lambda b. \lambda s.~ (\lambda b. \lambda s.~ M_1) ~(\lambda s.~ (\lambda b. \lambda s.~ M_2) ~((\lambda b. \lambda s.~ M_3) ~b) ~s) ~s & (\cref{l-value})\\
            =& \quad \lambda b. \lambda s.~ (\lambda b. \lambda s.~ M_1) ~(\lambda s.~ (\lambda b. \lambda s.~ M_2) ~(\lambda s.~ M_3) ~s) ~s & (\beta)\\
            =& \quad \lambda b. \lambda s.~ (\lambda b. \lambda s.~ M_1) ~(\lambda s.~ (\lambda s.~ M_2\subst{b}{\lambda s.~ M_3}) ~s) ~s & (\beta)\\
            =& \quad \lambda b. \lambda s.~ (\lambda b. \lambda s.~ M_1) ~(\lambda s.~ M_2\subst{b}{\lambda s.~ M_3}) ~s & (\beta)\\
            =& \quad \lambda b. \lambda s.~ (\lambda s.~ M_1\subst{b}{\lambda s.~ M_2\subst{b}{\lambda s.~ M_3}}) ~s & (\beta)\\
            =& \quad \lambda b. \lambda s.~ \color{red}{M_1\subst{b}{\lambda s.~ M_2\subst{b}{\lambda s.~ M_3}}} & (\beta)
        \end{align*}
        \qed
    \end{proof}

\begin{lemma}[Template Composition Associativity]
  \label{thm:templ-compose-assoc}
  $ T\den{(O_1;O_2);B} = T\den{O_1;(O_2;B)}.$
\end{lemma}
    \begin{proof}
        \begin{align*}
            &\quad T\den{(O_1;O_2);B} \\
            =& \quad \lambda s.~ E\den{O_1;O_2} ~T\den{B} ~s \\
            =& \quad \lambda s.~ (\lambda b. \lambda s.~ E\den{O_1} ~(E\den{O_2} ~b) ~s) ~T\den{B} ~s \\
            =& \quad \lambda s.~ (\lambda b. \lambda s.~ (\lambda b. \lambda s.~ M_1) ~(E\den{O_2} ~b) ~s) ~T\den{B} ~s & (\cref{l-value})\\
            =& \quad \lambda s.~ (\lambda b. \lambda s.~ (\lambda b. \lambda s.~ M_1) ~((\lambda b. \lambda s.~ M_2) ~b) ~s) ~T\den{B} ~s & (\cref{l-value})\\
            =& \quad \lambda s.~ (\lambda b. \lambda s.~ (\lambda b. \lambda s.~ M_1) ~(\lambda s.~ M_2) ~s) ~T\den{B} ~s & (\beta)\\
            =& \quad \lambda s.~ (\lambda b. \lambda s.~ (\lambda s.~ M_1\subst{b}{\lambda s.~ M_2}) ~s) ~T\den{B} ~s & (\beta)\\
            =& \quad \lambda s.~ (\lambda b. \lambda s.~ M_1\subst{b}{\lambda s.~ M_2}) ~T\den{B} ~s & (\beta)\\
            =& \quad \lambda s.~ (\lambda b. \lambda s.~ M_1\subst{b}{\lambda s.~ M_2}) ~(\lambda s.~ M_B) ~s & (\cref{l-value})\\
            =& \quad \lambda s.~ (\lambda s.~ M_1\subst{b}{\lambda s.~ M_2}\subst{b}{\lambda s.~ M_B}) ~s & (\beta)\\
            =& \quad \color{red}{\lambda s.~ M_1\subst{b}{\lambda s.~ M_2}\subst{b}{\lambda s.~ M_B}} & (\beta)
        \end{align*}
        \begin{align*}
            &\quad T\den{O_1;(O_2;B)} \\
            =& \quad \lambda s.~ E\den{O_1} ~T\den{O_2;B} ~s \\
            =& \quad \lambda s.~ E\den{O_1} ~(\lambda s. ~E\den{O_2} ~T\den{B} ~s) ~s \\
            =& \quad \lambda s.~ E\den{O_1} ~(\lambda s. ~E\den{O_2} ~T\den{B} ~s) ~s \\
            =& \quad \lambda s.~ E\den{O_1} ~(\lambda s. ~(\lambda b. \lambda s.~ M_2) ~T\den{B} ~s) ~s & (\cref{l-value})\\
            =& \quad \lambda s.~ E\den{O_1} ~(\lambda s. ~(\lambda b. \lambda s.~ M_2) ~(\lambda s.~ M_B) ~s) ~s & (\cref{l-value})\\
            =& \quad \lambda s.~ E\den{O_1} ~(\lambda s. ~(\lambda s.~ M_2\subst{b}{\lambda s.~ M_B}) ~s) ~s & (\beta) \\
            =& \quad \lambda s.~ E\den{O_1} ~(\lambda s. ~M_2\subst{b}{\lambda s.~ M_B}) ~s & (\beta)\\
            =& \quad \lambda s.~ (\lambda b. \lambda s.~ M_1) ~(\lambda s. ~M_2\subst{b}{\lambda s.~ M_B}) ~s & (\cref{l-value})\\
            =& \quad \lambda s.~ (\lambda s.~ M_1\subst{b}{\lambda s. ~M_2\subst{b}{\lambda s.~ M_B}}) ~s & (\beta)\\
            =& \quad \color{red}{\lambda s.~ M_1\subst{b}{\lambda s. ~M_2\subst{b}{\lambda s.~ M_B}}} & (\beta)
        \end{align*}
        \qed
    \end{proof}

\begin{lemma}[Extension Commit]
  \label{thm:ext-commit}
  $ E\den{\Do M; O} = E\den{\Do M}.$
\end{lemma}
    \begin{proof}
        \begin{align*}
            &\quad E\den{\Do M; O} \\
            =& \quad \lambda b. \lambda s.~ E\den{\Do M} ~(E\den{O} ~b) ~s \\
            =& \quad \lambda b. \lambda s.~ E\den{\Do M} ~((\lambda b. \lambda s.~ M_O) ~b) ~s & (\cref{l-value})\\
            =& \quad \lambda b. \lambda s.~ E\den{\Do M} ~(\lambda s.~ M_O) ~s & (\beta)\\
            =& \quad \lambda b. \lambda s.~ E\den{\Try \_ \to \Else \_ \to M} ~(\lambda s.~ M_O) ~s \\
            =& \quad \lambda b. \lambda s.~ (\lambda (x/\_). ~T\den{\Else \_ \to M}) ~(\lambda s.~ M_O) ~s \\
            =& \quad \lambda b. \lambda s.~ (T\den{\Else \_ \to M}) ~s & (\beta)\\
            =& \quad \lambda b. \lambda s.~ (T\den{\Continue \_ \to M}) ~s \\
            =& \quad \lambda b. \lambda s.~ (\lambda(x/\_) ~ \den{M}) ~s \\
            =& \quad \color{red}{\lambda b. \lambda s.~ \den{M}} & (\beta)
        \end{align*}
        \begin{align*}
            &\quad E\den{\Do M} \\
            =&\quad E\den{\Try \_ \to \Continue \_ \to M} \\
            =& \quad \lambda \_. ~T\den{\Continue \_ \to M} \\
            =& \quad \color{red}{\lambda \_. ~\lambda \_. ~  \den{M}}
        \end{align*}
        \qed
    \end{proof}

\begin{lemma}[Template Commit]
  \label{thm:template-commit}
  $ T\den{\Do M; B} = T\den{\Else M}.$
\end{lemma}
    \begin{proof}
        \begin{align*}
            &\quad T\den{\Else M} \\
            =&\quad T\den{\Continue \_ \to M} \\
            =&\quad \color{red}{\lambda \_. ~\den{M}} \\
        \end{align*}
        \begin{align*}
            &\quad E\den{\Do M} \\
            =&\quad E\den{\Try \_ \to \Continue \_ \to M} \\
            =& \quad \lambda \_. ~T\den{\Continue \_ \to M} \\
            =& \quad \lambda \_. ~\lambda \_. ~  \den{M}  
        \end{align*}
        \begin{align*}
            &\quad T\den{\Do M; B} \\
            =& \quad \lambda s. ~ E\den{\Do M} ~T\den{B} ~s \\
            =& \quad \lambda s. ~ (\lambda \_. ~\lambda \_. ~  \den{M}) ~T\den{B} ~s \\
            =& \quad \lambda s. ~ (\lambda \_. ~  \den{M}) ~s  & (\beta)\\
            =& \quad \color{red}{\lambda s. ~ \den{M}}  & (\beta)
        \end{align*}
        \qed
    \end{proof}

\begin{lemma}[$\eta\lamstar$]
  \label{thm:eta-lamstar}
  $E\den{\lambda x.~ (\lamstar (F; B)) ~ x} = E\den{\lamstar (F; B)}$
\end{lemma}
\begin{proof}
  From \cref{thm:value-translation}, we have
  $E\den{F} = \lambda b.\lambda s. \lambda z. M$ for some term $M$.  In the
  following, let
  $M' = M\subst{b}{T\den{B}}\subst{s}{(\lambda y.\mathit{self}~y)}$,
  \begin{align*}
    &\quad
    E\den{\lambda x.~ (\lamstar (F; B)) ~ x}
    \\
    =&\quad
    \lambda x. E\den{(\lamstar (F; B))} ~ x
    \\
    =&\quad
    \lambda x.
    (\Rec \mathit{self}
    = (\lambda s. E\den{F} ~ T\den{B} ~ s)
    ~ (\lambda y. \mathit{self}~y))
    ~ x
    \\
    =&\quad
    \lambda x.
    (\Rec \mathit{self} = E\den{F} ~ T\den{B} ~ (\lambda y. \mathit{self}~y))
    ~ x
    &(\beta)
    \\
    =&\quad
    \lambda x.
    (\Rec \mathit{self}
    = (\lambda b.\lambda s.\lambda z. M) ~ T\den{B}
    ~ (\lambda y. \mathit{self}~y))
    ~ x
    &(\cref{thm:value-translation})
    \\
    =&\quad
    \lambda x.
    (\Rec \mathit{self}
    = \lambda z. M\subst{b}{T\den{B}}\subst{s}{(\lambda y.\mathit{self}~y)})
    ~ x
    \\
    =&\quad
    \lambda x.
    (\Rec \mathit{self} = \lambda z. M')
    ~ x
    &(\beta)
    \\
    =&\quad
    \lambda x.
    (\lambda z.
    M'
    \subst{\mathit{self}}{\Rec \mathit{self} = \lambda z. M'})
    ~ x
    &(rec)
    \\
    =&\quad
    \lambda x.
    M'
    \subst{\mathit{self}}{\Rec \mathit{self} = \lambda z. M'}
    \subst{z}{x}
    &(\beta)
    \\
    =&\quad
    \lambda z.
    M'
    \subst{\mathit{self}}{\Rec \mathit{self} = \lambda z. M'}
    &(\alpha)
    \\
    =&\quad
    \Rec \mathit{self} = \lambda z. M'
    &(rec)
    \\
    =&\quad
    \Rec \mathit{self}
    = \lambda z. M\subst{b}{T\den{B}}\subst{s}{(\lambda y.\mathit{self}~y)}
    \\
    =&\quad
    \Rec \mathit{self}
    = (\lambda b. \lambda s. \lambda z. M) ~ T\den{B}
    ~ (\lambda x. \mathit{self} ~ x)
    &(\beta)
    \\
    =&\quad
    \Rec \mathit{self} = E\den{F} ~ T\den{B} ~ (\lambda x. \mathit{self} ~ x)
    &(\cref{thm:value-translation})
    \\
    =&\quad
    \Rec \mathit{self}
    = (\lambda s. E\den{F} ~ T\den{B} ~ s)
    ~ (\lambda x. \mathit{self} ~ x)
    &(\beta)
    \\
    =&\quad
    \Rec \mathit{self} = T\den{F; B} ~ (\lambda x. \mathit{self} ~ x)
    \\
    =&\quad
    E\den{\lamstar(F; B)}
  \end{align*}
  \qed
\end{proof}

\begin{lemma}[Unfold $\lamstar$]
  \label{thm:unfold-lamstar}
    $ E\den{ \lamstar (F; B)} = \den{(\Template F; B) ~ (\lamstar (F; B))}.$
\end{lemma}
\begin{proof}
  Note,
  \begin{math}
    T\den{F; B} ~ (\lambda x. \mathit{self} ~ x)
    =
    E\den{F} ~ T\den{B} ~ (\lambda x. \mathit{self} ~ x)    
  \end{math}
  is $\beta$-equal to some value of the form $\lambda z. M'$ because
  $T\den{F; B} = \lambda b. \lambda s. \lambda z. M$ from
  \cref{thm:value-translation}.  So
  $\Rec \mathit{self} = T\den{F; B} ~ (\lambda x. \mathit{self} ~ x)$ unfolds
  via $\beta$ and $rec$ in the following,
  \begin{align*}
    &\quad
    E\den{ \lamstar (F; B)}
    \\
    =& \quad
    (\Rec \mathit{self} = T\den{F; B} ~ (\lambda x. \mathit{self} ~ x))
    \\
    =& \quad
    T\den{F; B}
    ~
    (\lambda x.
    (\Rec \mathit{self} = T\den{F; B} ~ (\lambda x. \mathit{self} ~ x))
    ~ x)
    )
    & (\beta rec)
    \\
    =& \quad
    T\den{F; B} ~ (\lambda x. \den{\lamstar(F; B)} ~ x)
    \\
    =& \quad
    T\den{F; B} ~ \den{\lamstar(F; B)}
    & (\cref{thm:eta-lamstar})
    \\
    =& \quad
    \den{\Template (F; B)} ~ \den{\lamstar (F; B)}
    \\
    =& \quad
    \den{(\Template (F; B)) ~ (\lamstar (F; B))}
  \end{align*}
  \qed
\end{proof}

\begin{lemma}[Template Extension]
  \label{thm:templ-ext}
  \\
  $ \den{(\Template O; B) ~ V} = \den{(\Extension O) ~ (\Template B) ~ V}.$
\end{lemma}
    \begin{proof}
        \begin{align*}
            &\quad \den{(\Template O; B) ~ V} \\
            =& \quad \den{(\Template O; B)} ~ \den{V} \\
            =& \quad \den{(\Template O; B)} ~ W & (\cref{l-value}) \\
            =& \quad (\lambda s. ~E\den{O} ~T\den{B} ~s) ~ W \\
            =& \quad E\den{O} ~T\den{B} ~W
        \end{align*}
        \begin{align*}
            &\quad  \den{(\Extension O) ~ (\Template B) ~ V} \\
            =& \quad (\den{(\Extension O)} ~ \den{(\Template B)}) ~ \den{V} \\
            =& \quad (\den{(\Extension O)} ~ \den{(\Template B)}) ~ W & (\cref{l-value})\\
            =& \quad E\den{O} ~ T\den{B} ~ W
        \end{align*}
        \qed
    \end{proof}

\begin{lemma}[Template Failure]
  \label{thm:templ-fail}
  $ \den{(\Template \varepsilon) ~ V} = \den{\mathit{fail}~V}.$
\end{lemma}
    \begin{proof}
        \begin{align*}
            &\quad \den{(\Template \varepsilon) ~ V} \\
            =& \quad \den{(\Template \varepsilon)} ~ \den{V}\\
            =& \quad \den{(\Template \varepsilon)} ~ W & (\cref{l-value})\\
            =& \quad T\den{\varepsilon} ~ W \\
            =& \quad (\lambda s. ~fail ~s) ~ W \\
            =& \quad ~fail ~W 
        \end{align*}
        \begin{align*}
            &\quad \den{\mathit{fail}~V} \\
            =& \quad \den{\mathit{fail}} ~\den{V}\\
            =& \quad \den{\mathit{fail}} ~W  & (\cref{l-value})\\
            =& \quad \mathit{fail} ~W
        \end{align*}
        \qed
    \end{proof}

\begin{lemma}[Template Continue]
  \label{thm:templ-continue}
  $ \den{(\Template \Continue x \to M) ~ V} = \den{M\subst{x}{V}}.$
\end{lemma}
    \begin{proof}
        \begin{align*}
            &\quad \den{(\Template \Continue x \to M) ~ V} \\
            =& \quad T\den{(\Continue x \to M)} ~ \den{V}\\
            =& \quad T\den{(\Continue x \to M)} ~ W & (\cref{l-value})\\
            =& \quad (\lambda x. ~\den{M}) ~ W \\
            =& \quad \den{M}\subst{x}{W} & (\beta)
        \end{align*}
        \qed
    \end{proof}

\adriano{$\forall M V x, \den{M}\subst{x}{V} = \den{M\subst{x}{V}}$}

\begin{lemma}[Extension Try]
  \label{thm:ext-try}
    $ \den{(\Extension \Try x \to B) ~ V} = \Template B\subst{x}{V}.$
\end{lemma}
    \begin{proof}
        \begin{align*}
            &\quad \den{(\Extension \Try x \to B) ~ V}  \\
            =& \quad \den{(\Extension \Try x \to B)} ~ \den{V}\\
            =& \quad E\den{(\Try x \to B)} ~ \den{V}\\
            =& \quad (\lambda x. ~T\den{B}) ~ \den{V}\\
            =& \quad T\den{B}\subst{x}{\den{V}}
        \end{align*}
        \qed
    \end{proof}

\begin{lemma}[Try Match]
  \label{thm:try-match}
  \\
  $ \den{\Match P \gets V ~ O} = \den{O\subst{\many{x}}{\many{W}}} (\text{if } P\subst{\many{x}}{\many{W}} = V).$
\end{lemma}
    \begin{proof}
        \begin{align*}
            &\quad \den{\Match P \gets V ~ O}  \\
            =&\quad \lambda b. \lambda s.
            \begin{aligned}[t]
              &\Match \den{V} \With \\
              &\quad
              \begin{aligned}[t]
                \{~
                P &\to E\den{O}~b~s; \\
                \_ &\to b~s
                ~\}
              \end{aligned}
            \end{aligned}  \\
            =&\quad  \lambda b. \lambda s.~(E\den{O}~b~s)\subst{\many{x}}{\many{W}}  & (match) \\
            =&\quad  \lambda b. \lambda s.~((\lambda b. \lambda s. ~M_O)~b~s)\subst{\many{x}}{\many{W}}  & (\cref{l-value}) \\
            =&\quad  \lambda b. \lambda s.~M_O\subst{\many{x}}{\many{W}}  & (\beta)
        \end{align*}
        \begin{align*}
            &\quad \den{O\subst{\many{x}}{\many{W}}}  \\
            =& \quad \den{O}\subst{\many{x}}{\many{W}} \\
            =& \quad (\lambda b. \lambda s. ~M_O)\subst{\many{x}}{\many{W}} & (\cref{l-value}) \\
            =& \quad (\lambda b. \lambda s. ~M_O\subst{\many{x}}{\many{W}}) 
        \end{align*}
        \qed
    \end{proof}


%%% Local Variables:
%%% mode: LaTeX
%%% TeX-master: "coscheme"
%%% End:


\end{document}
