% This is samplepaper.tex, a sample chapter demonstrating the
% LLNCS macro package for Springer Computer Science proceedings;
% Version 2.21 of 2022/01/12
%

\documentclass[runningheads]{llncs}
%
\usepackage[T1]{fontenc}
% T1 fonts will be used to generate the final print and online PDFs,
% so please use T1 fonts in your manuscript whenever possible.
% Other font encondings may result in incorrect characters.
%
\usepackage{graphicx}
% Used for displaying a sample figure. If possible, figure files should
% be included in EPS format.
\usepackage{xcolor}
\usepackage[shortlabels]{enumitem}

% llncs.cls clashes with amsthm.
% Save the LNCS proof environment defined by the class
\let\lncsproof\proof \let\lncsendproof\endproof \let\lncsqed\qed
% Remove the definitions in order to load amsthm
\let\proof\relax\let\endproof\relax
% Load AMS styles
\usepackage{amsmath}
\usepackage{amsthm}
\usepackage{amssymb}
% restore the LNCS class defined proof
\let\proof\lncsproof \let\endproof\lncsendproof \let\qed\lncsqed

\usepackage{stmaryrd}
\usepackage{braket}
\usepackage{proof}

\usepackage{minted}

\usepackage{hyperref}
\usepackage{cleveref}

\usepackage{preamble}

% If you use the hyperref package, please uncomment the following two lines
% to display URLs in blue roman font according to Springer's eBook style:
\usepackage{color}
\renewcommand\UrlFont{\color{blue}\rmfamily}
\urlstyle{rm}

% Unicode characters:
\DeclareUnicodeCharacter{3BB}{$\lambda$}

\begin{document}
%
\title{CoScheme: Compositional Copatterns in Scheme}
%
%\titlerunning{Abbreviated paper title}
% If the paper title is too long for the running head, you can set
% an abbreviated paper title here
%
\author{Paul Downen\inst{1}\orcidID{0000-0003-0165-9387}}
%
\authorrunning{P. Downen}
% First names are abbreviated in the running head.
% If there are more than two authors, 'et al.' is used.
%
\institute{University of Massachusetts Lowell, Lowell MA 01854, USA \\
\email{Paul\_Downen@uml.edu}}
%
\maketitle              % typeset the header of the contribution
%
\begin{abstract}
The abstract should briefly summarize the contents of the paper in
150--250 words.

\keywords{Codata \and Copatterns \and Scheme \and Macros \and Composition \and Expression Problem.}
\end{abstract}
%
%
%
\section{Introduction} \label{sec-intro}


Composition is one of the great promises of functional programming to combat complexity.
As opposed to monolothic solutions, functional programming languages encourage us to decompose large problems into small, reusable, and reliable parts and then to recompose them back into a whole solution \cite{Hughes1989WFPM}.
This practice is encouraged through tools like higher-order functions to abstract out common patterns and laziness to separate generation, selection, and consumption of information.
Rather than implementing a complex algorithm as a single special-purpose loop, functional programming lets us express the same solution as the composition of simple domain-specific operations and generic combinators: maps, filters, folds, and unfolds.

However, the \emph{expression problem}~\cite{ExpressionProblem} is a familiar foe that still resists this (de)compositional approach.
It captures the common problem that arises when we want to maintain code --- such as an evaluator for the syntax trees of an expression language --- by extending it in two different directions: adding new forms of data (\ie classes of objects) and new operations (\ie methods) on them.
Traditionally, functional languages can easily add new operations over any given data type, but adding a new constructor requires a major rewrite that can potentially alter the rest of the code.
Conversely, object-oriented languages make it easy to add a new class of object, but extending a base class with a new method again requires major rewriting.
Being a common obstacle in the way of maintaining, extending, and decomposing code, the expression problem has garnered many solutions in the object-oriented \cite{GangOfFour,wehr_javagi_2011} and functional \cite{swierstra_data_2008,keep_nanopass_2013} worlds, and especially hybrid languages that mix both \cite{BrachaC90Mixins,flatt1998mixin}.

This work presents a novel solution to the expression problem: composable copatterns.
Copatterns~\cite{APTS2013C} are often associated with codata types for expressing infinite objects, but their use is not limited to just that.
Their composition, in particular, allows us to define programs by performing equational reasoning in the evaluation context.
Performing the ``substitution of equals for equals''~\cite{wadler_critique_1987} enhances the predictability and composability of our programs since we can analyze our part code in isolation.

Previous implementations of copatterns can be found in strongly typed languages which impose prescribed restrictions on their use.
For example, Agda gives the most full-fledged implementation of copatterns in a real system~\cite{ElaboratingDependentCopatterns}.
However, Agda is primarily a proof assistant rather than a general-purpose programming language, and as such, has different concerns than an ordinary functional programmer.
There is also some support for copatterns in OCaml~\cite{LaforgueR17}, but as an unofficial extension that has not been merged into the main compiler.

The copatterns implemented here are also implemented as macros like \cite{LaforgueR17}, however, we present a different encoding that focuses on providing new methods of extensibility that were not available before, and can be desugared without any static typing information.
To achieve that, and to fully integrate it into a practical general-purpose programming language, we choose a programmable programming language~\cite{ProgrammablePL} and provide a new language extension as a library~\cite{LanguageLibrary}.
We focus, in particular, on Scheme and Racket, which offers a robust macro system for seamlessly implementing new language features.

Our extension presents three different composition flavors, allowing us to capture some ``design patterns'' used by functional programmers as first-class abstractions.
First, we have \emph{vertical} composition, which permits us to gather a collection of alternative options with failure handling.
Second, we have \emph{horizontal} composition, which permits us to compose a sequence of steps, parameters, matching, or guards.
Third, we have \emph{circular}, which allows us to recurse back on the entire composition itself.

Our primary contributions are organized as follows:
\begin{itemize}
\item \Cref{sec-examples} shows examples of programming with copattern equations in Scheme-like languages, including new forms of program composition --- vertical and horizontal --- that allows us to solve familiar examples of the expression problem~\cite{ExpressionProblem} through a fusion of functional and object-oriented techniques.
\item \Cref{sec-api} exposes the challenges related to implementing copatterns in this scenario, introduces our library API, shows how we can desugar our abstractions into a set of primitives and how the implementations differ between Racket and a standard R${}^6$RS-compliant Scheme.
% \item \Cref{sec-macro} explains how to implement the high-level translation above in real code, and specifically how the implementation differs between Racket and a standard R${}^6$RS-compliant Scheme.
\item \Cref{sec-translation} presents a theory for how to translate copatterns into a small core target language --- untyped $\lambda$-calculus with recursion and patterns --- with a local double-barrel transformation reminiscent of selective continuation-passing style transformation.
  Importantly, only the new language constructs are transformed, while existing ones in the target language are unchanged.
\item \Cref{sec-correctness} demonstrates correctness in terms of an equational theory for reasoning about copattern-matching code in the source language, which is a conservative extension of the target language, and we prove that it is sound with respect to translation.
\end{itemize} 
% The remainder of the article has the following structure: First, we introduce our implementation by explaining meaningful examples (Section \ref{sec-examples}).
% Second, we specify a core language with high-level features representing our implementation. Then we describe a translation into a target $\lambda$-calculus (Section \ref{sec-translation}).
% Third, we scrutinize our implementation, comparing each provided flavor (Section \ref{sec-macro}).
% Fourth, we present the properties of our system (Section \ref{sec-correctness}).
% Last, we explain the details of our optimized racket implementation (Section \ref{sec-opt}).


%%% Local Variables:
%%% mode: LaTeX
%%% TeX-master: "coscheme"
%%% End:


\section{Programming with Composable Copatterns in Scheme}
\label{sec-examples}

\section{Translating Composable Copatterns} \label{sec-translation}


To help study the behavior and correctness of composable copattern matching, we model a simplified version of the library API in the form of an extended $\lambda$-calculus, and
give a high-level translation into a conventional $\lambda$-calculus with recursion and pattern matching (given in \cref{fig:target-syntax}).
Our pattern language is modeled after a small common core found among various implementations of Scheme, which includes normal variable wildcards $x$ that can match anything, quoted symbols $\q{x}$, and lists of the form $\Null$ or $(\Cons P \, P')$.
Note that we assume all bound variables $x$ in a pattern are distinct.
As shorthand, we write a list of patterns $P_1 ~ P_2 ~ \dots ~ P_n$ for $(\Cons P_1 ~ (\Cons P_2 ~ \dots (\Cons P_n \Null)))$.
To model the patterns found in typed functional languages like ML and Haskell, such as constructor applications $K ~ \many{P}$, we can represent the constructor as a quoted symbol $\q{K}$ and the application as a list $\q{K} ~ \many{P}$.
The patterns' specifics are surprisingly not essential to the main copattern translation and could be extended with other features found in more specific implementations.  

\begin{figure}[t]
\centering
\begin{alignat*}{2}
  % \mathit{Variable} &\ni{}& x, y, z
  % \\
  \mathit{Term} &\ni{}& M, N
  &::= x
  \mid M ~ N
  \mid \lambda x. M
  \mid K
  \mid \Match M \With \set{\many{P \to N}}
  \mid \Rec x = M
  \\
  \mathit{Pattern} &\ni{}& P
  &::= x
  \mid \q{x}
  \mid \Null
  \mid \Cons P \, P'
\end{alignat*}

\caption{Target language: pure $\lambda$-calculus with pattern-matching and recursion.}
\label{fig:target-syntax}
\end{figure}

For simplicity, this translation begins from a smaller source language with copatterns (given in \cref{fig:source-syntax}) separated into three main syntactic categories that reflect the different groups of values from the macro library:
\begin{itemize}
\item[($M, N$)] \emph{Term} syntax represents all ordinary expressions of the host langauge as well as the new first-class \emph{objects} of the library.
  The new forms of terms are $\lamstar B$, which gives a self-referential copattern-matching object, along with $\Template B$ and $\Extension O$ which include the other two syntactic categories as first-class values that can be applied as functions to instantiate their open-ended recursion and composition.
\item[($B$)] \emph{Template} syntax represents a simplified grammar supported by \scm|template| and similar macros specified as \scm|TemplateStx| in \cref{fig:macro-syntax}.
  Including some extension cases in a template is written as $O; B$, the catch-all clause which may continue the loop again via a recursive object bound to $x$ is written as $\Continue x \to M$, and the closed case where the catch-all clause raises an error is the empty string $\varepsilon$.
  Since the simpler final $\Else$ clause is a special case of $\Continue$, we treat it as syntactic sugar.
\item[($O$)] \emph{Extension} syntax represents a simplified grammar supported by \scm|extension| and similar macros specified as \scm|ExtensionStx| in \cref{fig:macro-syntax} with terser notation.
  Vertical composition is written as $O; O'$, similar to $O; B$, with the empty string $\varepsilon$ as its neutral element.
  Copattern-matching is written as $Q[x] O$, where $Q$ is a copattern context with $x$ as the root identifier naming the recursive object itself.
  The more basic forms are written as $\lambda P. O$ for a functional abstraction over an extension, $\Match P \gets M ~ O$ for a pattern-matching guard, and $\Try x \to B$ for the statement which binds the following cases to $x$ before running a template specified by $B$.
  We treat if-guards and the form $(= M)$ as syntactic sugar for special cases of the more general forms, and also sometimes use the alternative notation $\Do M$ in place of $(= M)$ in contexts where the latter notation appears awkward.
\end{itemize}

\begin{figure}[t]
\centering
\small
\begin{alignat*}{2}
  % \mathit{Variable} &\ni{}& x, y, z
  % \\
  \mathit{Term} &\ni{}& M, N
  &::= \dots
  \mid \lamstar B
  \mid \Template B
  \mid \Extension O
  \\
  \mathit{Template} &\ni{}& B
  &::= \varepsilon
  \mid O; B
  \mid \Continue x \to M
  \\
  \mathit{Extension} &\ni{}& O
  &::= \varepsilon
  \mid O; O'
  \mid Q[x] ~ O
  \mid \lambda P.~ O
  % \mid \If M ~ O
  \mid \Match P \gets M ~ O
  % \mid \Nest O
  \mid \Try x \to B
  \\
  \mathit{Copattern} &\ni{}& Q
  &::= \hole
  \mid Q ~ P
  \\
  \mathit{Pattern} &\ni{}& P
  &::= x
  \mid \q{x}
  \mid \Null
  \mid \Cons P \, P'
\end{alignat*}

Syntactic sugar:
\begin{align*}
  \Else M
  &=
  \Continue \_ \to M
  &
  (= M)
  =
  \Do M
  &=
  \Try \_ \to \Else M
  \\
  \If M ~ O
  &=
  \Match \True \gets M ~ O
  &
  (\Let x = M ~ O)
  &=
  \Match x \gets M ~ O
\end{align*}
\caption{Source language: target extended with nested copatterns,
  self-referential objects, recursion templates, and composable extensions.}
\label{fig:source-syntax}
\end{figure}

The syntax in $B$ and $O$ directly reflects the core operations for forming and combining copattern-matching expressions of the library API.
Here, the copattern syntax $Q[x]$ itself is expressed as a subset of contexts, $Q$, surrounding an object internally named $x$.
Two lines separated by a semicolon ($O; O'$) represents a binary vertical composition \scm|compose| that tries either $O$ or $O'$, 
and $\varepsilon$ represents an empty extension \scm|(extend)| with respect to vertical composition: it immediately refers to the next option.
Prefixing with a copattern-matching expression ($Q[x] ~ O$) represents the \scm|(comatch Q[x] O)| form that tries $Q[x]$ and then $O$.
Smaller special cases of matching include pattern lambdas ($\lambda P. O$) for \scm|try-λ|, and pattern guards ($\Match P \gets M ~ O$) for \scm|try-match|.
Other operations use the same names as in \cref{fig:api}.

This simplified grammar makes it easier to define the full macro expansion as a translation from the source (\cref{fig:source-syntax}) to target (\cref{fig:target-syntax}) as given in \cref{fig:translation}.
This translation shares many similarities to continuation-passing style (CPS) translations.
However, we explicitly avoid converting the entire program to CPS.
Notably, every syntactic form for the source language is unchanged; for example, $\den{M~N} = \den{M} ~ \den{N}$.
Instead, the only time we need to introduce an extra parameter is for the two new syntactic categories.
All templates are translated to functions that take a value for the whole object itself to a new version of that object.
Similarly, all extensions are translated to functions that take both a template as the ``base case'' to try next and a value for the whole object itself.
Even though this is dynamically-typed, we can view the type of templates as object transformers and extensions as template transformers:
\begin{align*}
  Object &= \text{some type of function}
  \\
  Template &= Object \to Object'
  \\
  Extension &= Template \to Template'
  = Template \to Object \to Object'
\end{align*}

\begin{figure}[t]
\centering
\small
Translating new terms:  
\begin{align*}
  \den{\lamstar B}
  &=
  (\Rec \mathit{self} = T\den{B} ~ \mathit{self})
  &=_\eta
  (\Rec \mathit{self} = T\den{B} ~ (\lambda x. \mathit{self} ~ x))
  \\
  \den{\Template B}
  &=
  T\den{B}
  \\
  \den{\Extension O}
  &=
  E\den{O}
  \\
  \den{M}
  &=
  \text{by induction}
  &(\text{otherwise})
\end{align*}
Translating templates:
\begin{align*}
  T\den{\varepsilon}
  &=
  \mathit{fail}
  &
  &=_\eta
  \lambda s. \mathit{fail}~s
  \\
  T\den{O; B}
  &=
  E\den{O} ~ T\den{B}
  &
  &=_\eta
  \lambda s. E\den{O} ~ T\den{B} ~ s
  \\
  T\den{\Continue x \to M}
  &=
  \lambda x. \den{M}
\end{align*}

Translating copattern-matching and pattern-matching functions:
\begin{align*}
  E\den{(Q[x] ~ P) ~ O}
  &=
  E\den{Q[x] ~ (\lambda P. O)}
  \\
  E\den{x ~ O}
  &=
  \lambda b. \lambda x. E\den{O} ~ b ~ x
  \\
  E\den{\lambda P. O}
  &=
  E\den{\lambda x. \Match P \gets x ~ O}
  &(\text{if } P \notin \mathit{Variable})
\end{align*}

Translating other extensions:
\begin{align*}
  E\den{\varepsilon}
  &=
  \lambda b. b
  &
  &=_\eta
  \lambda b. \lambda s. b ~ s
  \\
  E\den{O; O'}
  &=
  E\den{O} \comp E\den{O'}
  &
  &=_\eta
  \lambda b. \lambda s. E\den{O} ~ (E\den{O'}~b) ~ s
  \\
  E\den{\lambda x. O}
  &=
  \lambda b. \lambda s. (\lambda x. E\den{O} ~ (\lambda s'. b ~ s' ~ x) ~ s)
  \\
  E\den{\Match P \gets M ~ O}
  &=
  \rlap{$
    \lambda b. \lambda s.
    \Match \den{M} \With \set{P \to E\den{O}~b~s; \_ \to b~s}
  $}
  % \lambda b. \lambda s.
  % \begin{aligned}[t]
  %   &\Match \den{M} \With \\
  %   &\quad
  %   \begin{aligned}[t]
  %     \{~
  %     P &\to E\den{O}~b~s; \\
  %     \_ &\to b~s
  %     ~\}
  %   \end{aligned}
  % \end{aligned}
  \\
  % E\den{\Nest O}
  % &=
  % \lambda b. \lambda s. \Rec s' = E\den{O} ~ (\lambda \_. b ~ s) ~ (\lambda x. s' ~ x)
  % \\
  E\den{\Try x \to B}
  &=
  \lambda x. T\den{B}
\end{align*}
\caption{Translating copattern-based source code to the target language.}
\label{fig:translation}
\end{figure}

The interesting cases for translating terms are the new forms.
$\Template B$ and $\Extension O$ are just translated to their given forms as transformation functions.
With $\lamstar B$, we need to recursively plug its translation in for its self parameter.
Note the one detail that the recursive $\mathit{self}$ is $\eta$-expanded to in this application.
This ensures that $\lambda x. \mathit{self} ~ x$ is treated as a value in a real implementation, and is always safe assuming that $B$ describes a function (non-functional cases of $\lamstar B$ are undefined user error).

For templates and extensions, the terminators $\Continue$ and $\Try$ are translated to plain $\lambda$-abstractions that allow the programmer direct access to their implicit parameters.
% Other cases are specific to each form.
Complex copatterns ($x ~ \many{P_1} P_n ~ O$) are reduced down to a simpler sequence of pattern lambdas ($x ~ \lambda P_1. \dots \lambda P_n. ~ O$), and pattern lambdas ($\lambda P. O$) are reduced down to a simpler non-matching lambda followed by an explicit match guard ($\lambda x. \Match P \gets x ~ O$).

This leaves just the base cases of simple extension forms.
Each time an extension (of form $\lambda b. \lambda s. \dots$) ``fails,'' it calls the given next template with the given self object ($b~s$).
A match guard $\den{\Match P \gets M ~ O}$ will try to match the translation of $M$ against the pattern $P$; the success case continues as $E\den{O}$ with the same next template and self.
A non-matching lambda $\den{\lambda x. O}$ always succeeds (for now), but note that the next template to try on failure has to be changed to include the given argument.
Why?
Because the lambda has already consumed the next argument from its context, it would be gone if, later on, the following operations fail and move on to the next option.
So instead of invoking the given $b$ directly as $b~s'$ (for a potentially different future $s'$), they need to invoke $b$ applied to this argument $x$ as $b~s'~x$.
% Finally, the $\den{\Nest O}$ operation is defined by recursively creating a new value for the self parameter by recursively taking a new snapshot of how the object looks now after all the preceding applications and matchings have already occurred.

In this translation, we also give the $\eta$-reduced forms on the right-hand side when available.
This shows that the empty extension $\varepsilon$ is just the identity function (given the next thing $b$ to try, $\varepsilon$ does nothing and immediately moves on to $b$), and horizontal composition $O; O'$ is just ordinary function composition.


%%% Local Variables:
%%% mode: LaTeX
%%% TeX-master: "coscheme"
%%% End:


\section{Macro Definition} \label{sec-macro}

\section{Correctness} \label{sec-correctness}

\begin{figure}
\centering
  
\begin{alignat*}{2}
  \mathit{Value} &\ni{}& V, W
  &::= x
  \mid \lambda x. M
  \mid \Null
  \mid \Cons V \, W
  \mid \q{x}
  \\
  \mathit{EvalCxt} &\ni{}& E
  &::= \hole
  \mid E ~ M
  \mid V ~ E
  \mid \Match E \With \set{\many{P \to N}}
  \mid \Rec x = E
\end{alignat*}

Operational axioms:
\begin{align*}
  (\lambda x. M) ~ V
  &=
  M\subst{x}{V}
  % \\
  % \begin{aligned}
  %   &\Match V \With \\
  %   &\qquad\set{\many[i]{P_i \to N_i}}
  % \end{aligned}
  % &=
  % N_k\subst{BV(P_k)}{\many{W}}
  % &
  % \begin{aligned}
  %   (&\text{if } && V = P_k\subst{BV(P_k)}{\many{W}} \\
  %   &\text{and } &&\forall 1 \leq j < k, \\
  %   &&&\not\exists \many{W'}, V = P_j\subst{BV(P_j)}{\many{W'}})
  % \end{aligned}
  \\
  \begin{aligned}
    &\Match V \With \\
    &\qquad
    \begin{aligned}[t]
    \{~ &P \to N; \\
    &\many{P' \to N'}~\}
    \end{aligned}
  \end{aligned}
  &=
  N\subst{\many{x}}{\many{W}}
  &(\text{if } P\subst{\many{x}}{\many{W}} &= V)
  % &(\text{if } \exists \many{W},~ V &= P\subst{BV(P)}{\many{W}})
  \\
  \begin{aligned}
    &\Match V \With \\
    &\qquad
    \begin{aligned}[t]
    \{~ &P \to N; \\
    &\many{P' \to N'}~\}
    \end{aligned}
  \end{aligned}
  &=
  \begin{aligned}
    &\Match V \With \\
    &\qquad \set{\many{P' \to N'}}
  \end{aligned}
  &(\text{if } P &\apart V)
  % &(\text{if}\!\not\exists \many{W},~ V &= P\subst{BV(P)}{\many{W}})
  \\
  \Rec x = V
  &=
  V\subst{x}{(\Rec x = V)}
\end{align*}

Observational axioms:
\begin{align*}
  % \lambda x. (V ~ x)
  % &=
  % V
  % &(\text{if } x &\notin FV(V))
  % \\
  (\lambda x. E[x]) ~ M
  &=
  E[M]
  \\
  E\left[
    \begin{aligned}
      &\Match M \With \\
      &\qquad\set{\many{P \to N}}
    \end{aligned}
  \right]
  &=
  \begin{aligned}
    &\Match M \With \\
    &\qquad \set{\many{P \to E[N]}}
  \end{aligned}
  &(\text{if } BV(P) \cap FV(E) = \emptyset)
\end{align*}

Apartness between patterns and values ($P \apart V$):
\begin{gather*}
  \infer
  {\q{x} \apart V}
  {V \notin \mathit{Variable} \cup \set{\q{x}}}
  \qquad
  \infer
  {\Null \apart V}
  {V \notin \mathit{Variable} \cup \set{\Null}}
  \\[1ex]
  \infer
  {\Cons P ~ P' \apart V}
  {V \notin \mathit{Variable} \cup \set{\Cons W ~ W' \mid W, W' \in \mathit{Value}}}
  \\[1ex]
  \infer
  {\Cons P ~ P' \apart \Cons W ~ W'}
  {P \apart W}
  \qquad
  \infer
  {\Cons P ~ P' \apart \Cons W ~ W'}
  {P' \apart W'}
  % \infer
  % {K ~ P_1 \dots P_n \apart V}
  % {V \neq K ~ W_1 \dots W_n}
  % \qqqquad
  % \infer
  % {K~P_1 \dots P_n \apart K ~ V_1 \dots V_n}
  % {1 \leq j \leq n & P_j \apart V_j}
\end{gather*}

\caption{Untyped equational axioms of the target language.}
\label{fig:target-equality}
\end{figure}

\begin{figure}
\centering

\begin{alignat*}{2}
  % \mathit{TemplateValue} &\ni{}& B_v
  % &::= \varepsilon
  % \mid O_v; B_v
  % \mid \Continue x \to V
  % \\
  % \mathit{ExtensionValue} &\ni{}& O_v
  % &::= O_f
  % \mid O_f; O_v
  % \mid \Nest O_v
  % \mid \Try x \to B_v
  % \\
  \mathit{ExtensionFunc} &\ni{}& F
  &::= Q[x ~ P] ~ O
  \mid \lambda P. O
  \\
  \mathit{Value} &\ni{}& V
  &::= \dots
  \mid \lamstar (F; B)
  \mid \Template B
  \mid \Extension O
  \\
  % \mathit{NonRecTemplate} &\ni{}& B_{nr}
  % &::= O_{nr}; B_{nr}
  % \mid \Else \to M
  % \\
  % \mathit{NonRecExtension} &\ni{}& O_{nr}
  % &::= \varepsilon
  % \mid O_{nr}; O'_{nr}
  % \mid Q[\_] ~ O_{nr}
  % \mid \lambda P. O_{nr}
  % \\
  % &&&\phantom{:=}
  % \mid \Match P \gets M ~ O_{nr}
  % \mid \Nest O
  % \mid \Try x \to B_{nr}
  % \\
  % \mathit{RecCxt} &\ni{}& R
  % &::= \hole
  % \mid R; O
  % \mid O; R
  % \mid Q[x] ~ R
  % \mid \lambda P. R
  % \mid \Match P \gets M ~ O
  % \mid \Try x \to R
\end{alignat*}

Identity, associativity, and annihilation laws of composition:
\begin{align*}
  \varepsilon; O &= O = O; \varepsilon
  &
  (O_1; O_2); O_3 &= O_1; (O_2; O_3)
  &
  \Do M; O &= \Do M
  \\
  \varepsilon; B &= B
  &
  (O_1; O_2); B &= O_1; (O_2; B)
  &
  \Do M; B &= \Else M
\end{align*}

% Decomposing patterns and copatterns:
% \begin{align*}
%   (Q[x] ~ P) ~ O
%   &=
%   Q[x] ~ (\lambda P. O)
%   &
%   \_ ~ O
%   &=
%   O
%   &
%   \lambda P. O
%   &=
%   \lambda x. (\Match P \gets x ~ O)
% \end{align*}

% Factoring out recursion ($x \neq y$ and $x \notin BV(P)$):
% \begin{align*}
%   \lambda y. (x ~ O)
%   &=
%   x ~ (\lambda y. O)
%   &
%   \Match P \gets M ~ (x ~ O)
%   &=
%   x ~ (\Match P \gets M ~ O)
%   \\
%   (x ~ O); O'
%   &=
%   x ~ (O; O')
%   &
%   O; (x ~ O')
%   &=
%   x ~ (O; O')
% \end{align*}

Instantiating templates and recursive $\lamstar$:
\begin{align*}
  % (\Extension O) ~ V
  % &=
  % \Template{} (O; \Continue x \to (V ~ x))
  % \\
  % (\Template R[Q[x] ~ O]) ~ V
  % &=
  % (\Template R[Q[\_] ~ O\subst{x}{V}]) ~ V
  % \\
  % (\Template R[\Continue x \to M]) ~ V
  % &=
  % (\Template R[\Else \to M\subst{x}{V}]) ~ V
  % \\
  % (\Template B_{nr}) ~ V
  % &=
  % (\Template B_{nr}) ~ W
  % \\
  \lamstar (F; B)
  &=
  (\Template F; B) ~ (\lamstar (F; B))
  \\
  (\Template O; B) ~ V
  &=
  (\Extension O) ~ (\Template B) ~ V
  \\
  (\Template \varepsilon) ~ V
  &=
  \mathit{fail}~V
  \\
  (\Template \Continue x \to M) ~ V
  &=
  M\subst{x}{V}
  \\
  (\Extension \Try x \to B) ~ V
  &=
  \Template B\subst{x}{V}
\end{align*}

Pattern and copattern matching:
\begin{align*}
  \Match P \gets V ~ O
  &=
  O\subst{\many{x}}{\many{W}}
  &(\text{if } P\subst{\many{x}}{\many{W}} &= V)
  \\
  \Match{} P \gets V ~ O
  &=
  \varepsilon
  &(\text{if } P &\apart V)
  \\[1ex]
  (\Template{} (\lambda P. \Do M); B) ~ V' ~ V
  &=
  M\subst{\many{x}}{\many{W}}
  &(\text{if } P\subst{\many{x}}{\many{W}} &= V)
  \\
  (\Template{} (\lambda P. O); B) ~ V' ~ V
  &=
  (\Template B) ~ V' ~ V
  &(\text{if } P &\apart V)
  \\[1ex]
  C[(\Template{} (Q[y] = M); B) ~ V]
  &=
  M\subst{y}{V}\subst{\many{x}}{\many{W}}
  &(\text{if } Q\subst{\many{x}}{\many{W}} &= C)
  \\
  C[(\Template{} (Q[y] ~ O); B) ~ V]
  &=
  C[(\Template B) ~ V]
  &(\text{if } Q &\apart C)
  \\[1ex]
  C[\lamstar (Q[y] = M); B]
  &=
  \begin{aligned}[t]
    M
    &\subst{y}{(\lamstar (Q[y] = M); B)}
    \\
    &\subst{\many{x}}{\many{W}}
  \end{aligned}
  &(\text{if } Q\subst{\many{x}}{\many{W}} &= C)
  \\
  C[\lamstar (Q[y] ~ O); B]
  &=
  C[\lamstar B]
  &(\text{if } Q &\apart C)
\end{align*}

Apartness between copatterns and contexts ($Q \apart C$):
\begin{gather*}
  \infer
  {\hole ~ P_1 \dots P_n \apart \hole ~ V_1 \dots V_n}
  {P_n \apart V_n}
  \qqqquad
  \infer
  {Q ~ P \apart C}
  {Q \apart C}
  \qqqquad
  \infer
  {Q \apart C ~ V}
  {Q \apart C}
\end{gather*}

\caption{Some equalities of copattern extensions.}
\label{fig:source-equality}
\end{figure}

\begin{lemma}
  The following instances of translation are all values up to the equational
  theory of the target language in \cref{fig:target-equality}:
  \begin{enumerate}[(a)]
  \item $T\den{B} = \lambda s. M$ for some term $M$,
  \item $E\den{O} = \lambda b. \lambda s. M$ for some term $M$,
  \item $E\den{F} = \lambda b. \lambda s. \lambda x. M$ for some term $M$,
  \item $\den{V} = W$ for some value $W$.
  \end{enumerate}
\end{lemma}

\begin{proposition}[Conservative Extension]
  If $M = N$ in the equational theory of the target
  (\cref{fig:target-equality}), then so too does $\den{M} = \den{N}$.
\end{proposition}

\begin{proposition}[Soundness]
  The equational axioms given in \cref{fig:source-equality} are sound with
  respect to the translation in \cref{fig:translation},
  \begin{align*}
    M &= M' &&\implies & \den{M} &= \den{M'} \\
    B &= B' &&\implies & T\den{B} &= T\den{B'} \\
    O &= O' &&\implies & E\den{O} &= E\den{O'}
  \end{align*}
  up to the equational theory of the target language in
  \cref{fig:target-equality}.
\end{proposition}

\section{Optimizing Away Administrative Reductions} \label{sec-opt}

\begin{figure}
\centering
Translating new terms:  
\begin{align*}
  \den{\lamstar B}()
  &=
  \Rec \mathit{self} = T\den{B}(\lambda x. \mathit{self} ~ x)
  \\
  \den{\Template B}()
  &=
  \lambda s. T\den{B}(s)
  \\
  \den{\Extension O}()
  &=
  \lambda b. \lambda s. E\den{O}(b, s)
  \\
  \den{M}()
  &=
  \text{by induction}
  &(\text{otherwise})
\end{align*}
Translating templates:
\begin{align*}
  T\den{\varepsilon}(V)
  &=
  \mathit{fail}~V
  \\
  T\den{O; B}(V)
  &=
  (\lambda b. E\den{O}(b, V)) ~ (\lambda s. T\den{B}(s))
  \\
  T\den{\Continue x \to M}(V)
  &=
  \den{M}()\subst{x}{V}
\end{align*}

Translating copattern-matching and pattern-matching functions:
\begin{align*}
  E\den{(Q[x] ~ P) ~ O}(W, V)
  &=
  E\den{Q[x] ~ (\lambda P. O)}(W, V)
  \\
  E\den{x ~ O}(W, V)
  &=
  E\den{O}(W, V)\subst{x}{W}
  \\
  E\den{\lambda P. O}(W, V)
  &=
  E\den{\lambda x. \Match P \gets x ~ O}(W, V)
  &(\text{if } P \notin \mathit{Variable})
\end{align*}

Translating other extensions:
\begin{align*}
  E\den{\varepsilon}(W, V)
  &=
  W(V)
  \\
  E\den{O; O'}(W, V)
  &=
  (\lambda b. E\den{O}(b, V)) ~ (\lambda s. E\den{O'}(W, s))
  \\
  E\den{\lambda x. O}(W, V)
  &=
  \lambda x. E\den{O}((\lambda s'. W(s') ~ x), V)
  \\
  E\den{\Match P \gets M ~ O}(W, V)
  &=
  \Match \den{M}() \With
  \set{P \to E\den{O}(W, V); \_ \to W(V)}
  \\
  E\den{\Nest O}(W, V)
  &=
  \Rec s' = E\den{O}((\lambda \_. W(V)), (\lambda x. s' ~ x))
  \\
  E\den{\Try x \to B}(W, V)
  &=
  T\den{B}(V)\subst{x}{W}
\end{align*}

Inlining administrative $\lambda$-abstractions:
\begin{align*}
  (\lambda x. M)(W, \many{V})
  &=
  M(\many{V})\subst{x}{W}
  \\
  M(\many{V})
  &=
  M ~ \many{V}
  &(\text{otherwise})
\end{align*}

\caption{Inlining version of the translation.}
\label{fig:inline-translation}
\end{figure}

\begin{proposition}
  Up to the equational theory of the target language in
  \cref{fig:target-equality},
  \begin{align*}
    \den{M}() &= \den{M}
    \\
    \lambda s. T\den{B}(s) &= T\den{B}
    \\
    \lambda b. \lambda s. E\den{O}(b, s) &= E\den{O}
  \end{align*}  
\end{proposition}

\section{Related and Future Work} \label{sec-future}

\section{Conclusion} \label{sec-conclusion}

\begin{credits}
\subsubsection{\ackname}
% 
This material is based upon work supported by the National Science Foundation
under Grant No. 2245516.

\subsubsection{\discintname}
%
The authors have no competing interests to declare that are
relevant to the content of this article.
\end{credits}
%
% ---- Bibliography ----
%
% BibTeX users should specify bibliography style 'splncs04'.
% References will then be sorted and formatted in the correct style.
%
\bibliographystyle{splncs04}
\bibliography{refs}

\end{document}
