% This is samplepaper.tex, a sample chapter demonstrating the
% LLNCS macro package for Springer Computer Science proceedings;
% Version 2.21 of 2022/01/12
%

\documentclass[runningheads]{llncs}
%
\usepackage[T1]{fontenc}
% T1 fonts will be used to generate the final print and online PDFs,
% so please use T1 fonts in your manuscript whenever possible.
% Other font encondings may result in incorrect characters.
%
\usepackage{graphicx}
% Used for displaying a sample figure. If possible, figure files should
% be included in EPS format.
\usepackage{xcolor}
\usepackage[shortlabels]{enumitem}

% llncs.cls clashes with amsthm.
% Save the LNCS proof environment defined by the class
\let\lncsproof\proof \let\lncsendproof\endproof \let\lncsqed\qed
% Remove the definitions in order to load amsthm
\let\proof\relax\let\endproof\relax
% Load AMS styles
\usepackage{amsmath}
\usepackage{amsthm}
\usepackage{amssymb}
% restore the LNCS class defined proof
\let\proof\lncsproof \let\endproof\lncsendproof \let\qed\lncsqed

\usepackage{thmtools}
\usepackage{stmaryrd}
\usepackage{braket}
\usepackage{proof}

\usepackage{minted}

\usepackage{hyperref}
\usepackage{cleveref}

\usepackage{preamble}

% If you use the hyperref package, please uncomment the following two lines
% to display URLs in blue roman font according to Springer's eBook style:
\usepackage{color}
\renewcommand\UrlFont{\color{blue}\rmfamily}
\urlstyle{rm}

% Unicode characters:
\DeclareUnicodeCharacter{3BB}{$\lambda$}

\begin{document}
%
\title{CoScheme: Compositional Copatterns in Scheme}
%
%\titlerunning{Abbreviated paper title}
% If the paper title is too long for the running head, you can set
% an abbreviated paper title here
%
\author{
  Paul Downen\inst{1}\orcidID{0000-0003-0165-9387}
  \and \\
  Adriano Corbelino II\inst{1}\orcidID{0000-0002-6014-6189}
}
%
\authorrunning{P. Downen \and A. Corbelino II}
% First names are abbreviated in the running head.
% If there are more than two authors, 'et al.' is used.
%
\institute{
  University of Massachusetts Lowell, Lowell MA 01854, USA \\
  \email{Paul\_Downen@uml.edu} \\
  \email{Adriano\_VilargaCorbelino@uml.edu}
}
%
\maketitle              % typeset the header of the contribution
%
\begin{abstract}
The abstract should briefly summarize the contents of the paper in
150--250 words.

\keywords{Codata \and Copatterns \and Scheme \and Macros \and Composition \and Expression Problem.}
\end{abstract}
%
%
%
\section{Introduction} \label{sec-intro}


Composition is one of the great promises of functional programming to combat complexity.
As opposed to monolothic solutions, functional programming languages encourage us to decompose large problems into small, reusable, and reliable parts and then to recompose them back into a whole solution \cite{Hughes1989WFPM}.
This practice is encouraged through tools like higher-order functions to abstract out common patterns and laziness to separate generation, selection, and consumption of information.
Rather than implementing a complex algorithm as a single special-purpose loop, functional programming lets us express the same solution as the composition of simple domain-specific operations and generic combinators: maps, filters, folds, and unfolds.

However, the \emph{expression problem}~\cite{ExpressionProblem} is a familiar foe that still resists this (de)compositional approach.
It captures the common problem that arises when we want to maintain code --- such as an evaluator for the syntax trees of an expression language --- by extending it in two different directions: adding new forms of data (\ie classes of objects) and new operations (\ie methods) on them.
Traditionally, functional languages can easily add new operations over any given data type, but adding a new constructor requires a major rewrite that can potentially alter the rest of the code.
Conversely, object-oriented languages make it easy to add a new class of object, but extending a base class with a new method again requires major rewriting.
Being a common obstacle in the way of maintaining, extending, and decomposing code, the expression problem has garnered many solutions in the object-oriented \cite{GangOfFour,wehr_javagi_2011} and functional \cite{swierstra_data_2008,keep_nanopass_2013} worlds, and especially hybrid languages that mix both \cite{BrachaC90Mixins,flatt1998mixin}.

This work presents a novel solution to the expression problem: composable copatterns.
Copatterns~\cite{APTS2013C} are often associated with codata types for expressing infinite objects, but their use is not limited to just that.
Their composition, in particular, allows us to define programs by performing equational reasoning in the evaluation context.
Performing the ``substitution of equals for equals''~\cite{wadler_critique_1987} enhances the predictability and composability of our programs since we can analyze our part code in isolation.

Previous implementations of copatterns can be found in strongly typed languages which impose prescribed restrictions on their use.
For example, Agda gives the most full-fledged implementation of copatterns in a real system~\cite{ElaboratingDependentCopatterns}.
However, Agda is primarily a proof assistant rather than a general-purpose programming language, and as such, has different concerns than an ordinary functional programmer.
There is also some support for copatterns in OCaml~\cite{LaforgueR17}, but as an unofficial extension that has not been merged into the main compiler.

The copatterns implemented here are also implemented as macros like \cite{LaforgueR17}, however, we present a different encoding that focuses on providing new methods of extensibility that were not available before, and can be desugared without any static typing information.
To achieve that, and to fully integrate it into a practical general-purpose programming language, we choose a programmable programming language~\cite{ProgrammablePL} and provide a new language extension as a library~\cite{LanguageLibrary}.
We focus, in particular, on Scheme and Racket, which offers a robust macro system for seamlessly implementing new language features.

Our extension presents three different composition flavors, allowing us to capture some ``design patterns'' used by functional programmers as first-class abstractions.
First, we have \emph{vertical} composition, which permits us to gather a collection of alternative options with failure handling.
Second, we have \emph{horizontal} composition, which permits us to compose a sequence of steps, parameters, matching, or guards.
Third, we have \emph{circular}, which allows us to recurse back on the entire composition itself.

Our primary contributions are organized as follows:
\begin{itemize}
\item \Cref{sec-examples} shows examples of programming with copattern equations in Scheme-like languages, including new forms of program composition --- vertical and horizontal --- that allows us to solve familiar examples of the expression problem~\cite{ExpressionProblem} through a fusion of functional and object-oriented techniques.
\item \Cref{sec-api} exposes the challenges related to implementing copatterns in this scenario, introduces our library API, shows how we can desugar our abstractions into a set of primitives and how the implementations differ between Racket and a standard R${}^6$RS-compliant Scheme.
% \item \Cref{sec-macro} explains how to implement the high-level translation above in real code, and specifically how the implementation differs between Racket and a standard R${}^6$RS-compliant Scheme.
\item \Cref{sec-translation} presents a theory for how to translate copatterns into a small core target language --- untyped $\lambda$-calculus with recursion and patterns --- with a local double-barrel transformation reminiscent of selective continuation-passing style transformation.
  Importantly, only the new language constructs are transformed, while existing ones in the target language are unchanged.
\item \Cref{sec-correctness} demonstrates correctness in terms of an equational theory for reasoning about copattern-matching code in the source language, which is a conservative extension of the target language, and we prove that it is sound with respect to translation.
\end{itemize} 
% The remainder of the article has the following structure: First, we introduce our implementation by explaining meaningful examples (Section \ref{sec-examples}).
% Second, we specify a core language with high-level features representing our implementation. Then we describe a translation into a target $\lambda$-calculus (Section \ref{sec-translation}).
% Third, we scrutinize our implementation, comparing each provided flavor (Section \ref{sec-macro}).
% Fourth, we present the properties of our system (Section \ref{sec-correctness}).
% Last, we explain the details of our optimized racket implementation (Section \ref{sec-opt}).


%%% Local Variables:
%%% mode: LaTeX
%%% TeX-master: "coscheme"
%%% End:


\section{Programming with Composable Copatterns in Scheme}
\label{sec-examples}
To increase familiarity with copatterns in Scheme, let us stress their main traits and introduce our syntax.
One of the entry points of our extension is the macro \scm{define*}.
This macro enables us to declare a codata object by a list of observations.
Those observations are pairs of a copattern and a possible outcome.
In each observation, the first element of the copattern is the object's name.

One key feature of copatterns is that they can be defined upon an arbitrarily long sequence of patterns.
In other words, they can be \emph{nested}.
Hence, after the object's name, we can have more patterns dictating a more specific evaluation context to be matched.

\adriano{Insert the explanation that methods/destructors are represented with atoms}
To illustrate this concept, let us materialize a simple example from the ground up.
Still considering the classic stream examples, we can define streams in terms of their response to the \scm{head} and \scm{tail} destructors. 
So, to create fancier streams, it suffices to specify copatterns containing those destructors.
Let us target a stream that, starting from a number \scm{n}, counts the same number twice before incrementing it.
Since we want to start from a given number, we need to specify a context that represents this idea: \scm{stutter n}.
We could have other roots, such as \scm{stutter} and \scm{stutter n m}, but we want to state contexts with only a single value.
Now, we need to describe the behavior of our destructors.
For the \scm{head} destructor, we can return our current state \scm{n}.
However, what should we do with \emph{tail}?
We want our second element to be \scm{m}, yet we want our third element to be the successor of \scm{n}.
There is no way to specify the desired behavior, considering that the outcome depends on the \emph{next} action. 
This is where copatterns take the spotlight.
\adriano{Remove the repeated information above} Copatterns can define contexts not only based on the next action but with any finite number of sequential actions.  
Therefore, we can spell out that when we ask for the second element --- \scm{(((stutter n) 'tail) 'head)} --- we return n.
Despite that, we did not define where the increment should happen.
We want to depict the behavior that we increment the current state in every two elements. 
\adriano{bad sentence}
We can do this by defining a recursive case when we observe that we passed by two elements --- \scm{(((stutter n) 'tail) 'tail)} ---. 
It may not seem to, but this is enough to define our desired stream. This specific combination of copatterns of the stream's destructors covers all possible cases.
As a user, this is important since our implementation does not provide coverage analysis.

This framework is not limited by matching a single value in each group.
For example, we can define a stream that intercalates elements from two different streams using a similar configuration, taking two arguments instead of one.

\adriano{Maybe we can put those examples side by side in a single figure if we decrease the font size}
\begin{minted}{scheme}
;; stutter : Stream Nat
(define*
  [ ((stutter n) 'head)        = n]
  [(((stutter n) 'tail) 'head) = n]
  [(((stutter n) 'tail) 'tail) = (stutter (+ n 1))])
\end{minted}

\begin{minted}{scheme}
;; zigzag : (Stream a, Stream a) -> Stream a
(define*
  [ ((zigzag xs ys) 'head)        = (xs 'head)]
  [(((zigzag xs ys) 'tail) 'head) = (ys 'head)]
  [(((zigzag xs ys) 'tail) 'tail) = (zigzag (xs 'tail) (ys 'tail))])
\end{minted}

However, copatterns are not exclusively used with streams and infinite data.
In particular, we can define a depth-first search on a finite binary tree.
For this goal, we need to specify what should happen in the leaves and nodes of the tree.
We can create copatterns that match on a specific pattern of the input.
Therefore, when we see an evaluation context with a leaf, we return a singleton, and when we see a node, we recurse.
\adriano{Do I need to talk about how DFS works?}

\begin{minted}{scheme}
  (define*
    [((search ('leaf e)) 'dfs) = (list e)]
    [((search ('node l e r)) 'dfs) = (append ((search l) 'dfs) (list e)
                                             ((search r) 'dfs))])
  \end{minted}

Now let us consider an arithmetic evaluator:

\begin{minted}{scheme}
;; eval* : Expr -> Number
(define eval*
  (lambda*
  [(eval ('num n))   = n]
  [(eval ('add l r)) = (+ (eval l) (eval r))]))
\end{minted}

Fantastic, it works! But what if we wish to add a multiplication, a new operation, to our evaluator? 
We could definitely add one more line to the previous definition with a copattern matching a \scm{'Mult} term.
Nevertheless, we do not always want to modify existing working code, which is one of the questions concerning the expression problem.
One approach we can take is to define each operator as an individual evaluator who only knows how to deal with one operation and compose them all.
\begin{minted}{scheme}
  ;; eval-num : ('num Number) -> Number
  (define-object
    [(eval-num 'eval ('num n)) = n])
  
  ;; eval-add : ('add e e) <: e -> Number
  (define-object (eval-add <: meta)
    [(self 'eval ('add l r)) = (+ (self 'eval l) (self 'eval r))])
  
  ;; eval-mul : ('mul e e) <: e -> Number
  (define eval-mul
    (object [(self 'eval ('mul l r)) = (* (self 'eval l) (self 'eval r))]))
  
  (define eval-arith
    (eval-num 'compose eval-add eval-mul))

\end{minted}

This is our crown jewel: \emph{Composition}.
To be more precise, vertical composition since we are adding more cases to the list of observations.
To help visualize, you can think that the list of possible observations has one entry per line, and we are adding more lines to that list, thus vertical composition.
Another approach to composition is to extend some copattern with a series of other copatterns.
In that case, we are extending the size of a line, thus creating a horizontal composition.

\adriano{I have three topics: constant folding, evaluator with variables, and a horizontal composition example.
The dream would be to encode one of the evaluators with a horizontal composition so I could explain the other and finish with the fancy one.}


\section{Translating Composable Copatterns} \label{sec-translation}


To help study the behavior and correctness of composable copattern matching, we model a simplified version of the library API in the form of an extended $\lambda$-calculus, and
give a high-level translation into a conventional $\lambda$-calculus with recursion and pattern matching (given in \cref{fig:target-syntax}).
Our pattern language is modeled after a small common core found among various implementations of Scheme, which includes normal variable wildcards $x$ that can match anything, quoted symbols $\q{x}$, and lists of the form $\Null$ or $(\Cons P \, P')$.
Note that we assume all bound variables $x$ in a pattern are distinct.
As shorthand, we write a list of patterns $P_1 ~ P_2 ~ \dots ~ P_n$ for $(\Cons P_1 ~ (\Cons P_2 ~ \dots (\Cons P_n \Null)))$.
To model the patterns found in typed functional languages like ML and Haskell, such as constructor applications $K ~ \many{P}$, we can represent the constructor as a quoted symbol $\q{K}$ and the application as a list $\q{K} ~ \many{P}$.
The patterns' specifics are surprisingly not essential to the main copattern translation and could be extended with other features found in more specific implementations.  

\begin{figure}[t]
\centering
\begin{alignat*}{2}
  % \mathit{Variable} &\ni{}& x, y, z
  % \\
  \mathit{Term} &\ni{}& M, N
  &::= x
  \mid M ~ N
  \mid \lambda x. M
  \mid K
  \mid \Match M \With \set{\many{P \to N}}
  \mid \Rec x = M
  \\
  \mathit{Pattern} &\ni{}& P
  &::= x
  \mid \q{x}
  \mid \Null
  \mid \Cons P \, P'
\end{alignat*}

\caption{Target language: pure $\lambda$-calculus with pattern-matching and recursion.}
\label{fig:target-syntax}
\end{figure}

For simplicity, this translation begins from a smaller source language with copatterns (given in \cref{fig:source-syntax}) separated into three main syntactic categories that reflect the different groups of values from the macro library:
\begin{itemize}
\item[($M, N$)] \emph{Term} syntax represents all ordinary expressions of the host langauge as well as the new first-class \emph{objects} of the library.
  The new forms of terms are $\lamstar B$, which gives a self-referential copattern-matching object, along with $\Template B$ and $\Extension O$ which include the other two syntactic categories as first-class values that can be applied as functions to instantiate their open-ended recursion and composition.
\item[($B$)] \emph{Template} syntax represents a simplified grammar supported by \scm|template| and similar macros specified as \scm|TemplateStx| in \cref{fig:macro-syntax}.
  Including some extension cases in a template is written as $O; B$, the catch-all clause which may continue the loop again via a recursive object bound to $x$ is written as $\Continue x \to M$, and the closed case where the catch-all clause raises an error is the empty string $\varepsilon$.
  Since the simpler final $\Else$ clause is a special case of $\Continue$, we treat it as syntactic sugar.
\item[($O$)] \emph{Extension} syntax represents a simplified grammar supported by \scm|extension| and similar macros specified as \scm|ExtensionStx| in \cref{fig:macro-syntax} with terser notation.
  Vertical composition is written as $O; O'$, similar to $O; B$, with the empty string $\varepsilon$ as its neutral element.
  Copattern-matching is written as $Q[x] O$, where $Q$ is a copattern context with $x$ as the root identifier naming the recursive object itself.
  The more basic forms are written as $\lambda P. O$ for a functional abstraction over an extension, $\Match P \gets M ~ O$ for a pattern-matching guard, and $\Try x \to B$ for the statement which binds the following cases to $x$ before running a template specified by $B$.
  We treat if-guards and the form $(= M)$ as syntactic sugar for special cases of the more general forms, and also sometimes use the alternative notation $\Do M$ in place of $(= M)$ in contexts where the latter notation appears awkward.
\end{itemize}

\begin{figure}[t]
\centering
\small
\begin{alignat*}{2}
  % \mathit{Variable} &\ni{}& x, y, z
  % \\
  \mathit{Term} &\ni{}& M, N
  &::= \dots
  \mid \lamstar B
  \mid \Template B
  \mid \Extension O
  \\
  \mathit{Template} &\ni{}& B
  &::= \varepsilon
  \mid O; B
  \mid \Continue x \to M
  \\
  \mathit{Extension} &\ni{}& O
  &::= \varepsilon
  \mid O; O'
  \mid Q[x] ~ O
  \mid \lambda P.~ O
  % \mid \If M ~ O
  \mid \Match P \gets M ~ O
  % \mid \Nest O
  \mid \Try x \to B
  \\
  \mathit{Copattern} &\ni{}& Q
  &::= \hole
  \mid Q ~ P
  \\
  \mathit{Pattern} &\ni{}& P
  &::= x
  \mid \q{x}
  \mid \Null
  \mid \Cons P \, P'
\end{alignat*}

Syntactic sugar:
\begin{align*}
  \Else M
  &=
  \Continue \_ \to M
  &
  (= M)
  =
  \Do M
  &=
  \Try \_ \to \Else M
  \\
  \If M ~ O
  &=
  \Match \True \gets M ~ O
  &
  (\Let x = M ~ O)
  &=
  \Match x \gets M ~ O
\end{align*}
\caption{Source language: target extended with nested copatterns,
  self-referential objects, recursion templates, and composable extensions.}
\label{fig:source-syntax}
\end{figure}

The syntax in $B$ and $O$ directly reflects the core operations for forming and combining copattern-matching expressions of the library API.
Here, the copattern syntax $Q[x]$ itself is expressed as a subset of contexts, $Q$, surrounding an object internally named $x$.
Two lines separated by a semicolon ($O; O'$) represents a binary vertical composition \scm|compose| that tries either $O$ or $O'$, 
and $\varepsilon$ represents an empty extension \scm|(extend)| with respect to vertical composition: it immediately refers to the next option.
Prefixing with a copattern-matching expression ($Q[x] ~ O$) represents the \scm|(comatch Q[x] O)| form that tries $Q[x]$ and then $O$.
Smaller special cases of matching include pattern lambdas ($\lambda P. O$) for \scm|try-λ|, and pattern guards ($\Match P \gets M ~ O$) for \scm|try-match|.
Other operations use the same names as in \cref{fig:api}.

This simplified grammar makes it easier to define the full macro expansion as a translation from the source (\cref{fig:source-syntax}) to target (\cref{fig:target-syntax}) as given in \cref{fig:translation}.
This translation shares many similarities to continuation-passing style (CPS) translations.
However, we explicitly avoid converting the entire program to CPS.
Notably, every syntactic form for the source language is unchanged; for example, $\den{M~N} = \den{M} ~ \den{N}$.
Instead, the only time we need to introduce an extra parameter is for the two new syntactic categories.
All templates are translated to functions that take a value for the whole object itself to a new version of that object.
Similarly, all extensions are translated to functions that take both a template as the ``base case'' to try next and a value for the whole object itself.
Even though this is dynamically-typed, we can view the type of templates as object transformers and extensions as template transformers:
\begin{align*}
  Object &= \text{some type of function}
  \\
  Template &= Object \to Object'
  \\
  Extension &= Template \to Template'
  = Template \to Object \to Object'
\end{align*}

\begin{figure}[t]
\centering
\small
Translating new terms:  
\begin{align*}
  \den{\lamstar B}
  &=
  (\Rec \mathit{self} = T\den{B} ~ \mathit{self})
  &=_\eta
  (\Rec \mathit{self} = T\den{B} ~ (\lambda x. \mathit{self} ~ x))
  \\
  \den{\Template B}
  &=
  T\den{B}
  \\
  \den{\Extension O}
  &=
  E\den{O}
  \\
  \den{M}
  &=
  \text{by induction}
  &(\text{otherwise})
\end{align*}
Translating templates:
\begin{align*}
  T\den{\varepsilon}
  &=
  \mathit{fail}
  &
  &=_\eta
  \lambda s. \mathit{fail}~s
  \\
  T\den{O; B}
  &=
  E\den{O} ~ T\den{B}
  &
  &=_\eta
  \lambda s. E\den{O} ~ T\den{B} ~ s
  \\
  T\den{\Continue x \to M}
  &=
  \lambda x. \den{M}
\end{align*}

Translating copattern-matching and pattern-matching functions:
\begin{align*}
  E\den{(Q[x] ~ P) ~ O}
  &=
  E\den{Q[x] ~ (\lambda P. O)}
  \\
  E\den{x ~ O}
  &=
  \lambda b. \lambda x. E\den{O} ~ b ~ x
  \\
  E\den{\lambda P. O}
  &=
  E\den{\lambda x. \Match P \gets x ~ O}
  &(\text{if } P \notin \mathit{Variable})
\end{align*}

Translating other extensions:
\begin{align*}
  E\den{\varepsilon}
  &=
  \lambda b. b
  &
  &=_\eta
  \lambda b. \lambda s. b ~ s
  \\
  E\den{O; O'}
  &=
  E\den{O} \comp E\den{O'}
  &
  &=_\eta
  \lambda b. \lambda s. E\den{O} ~ (E\den{O'}~b) ~ s
  \\
  E\den{\lambda x. O}
  &=
  \lambda b. \lambda s. (\lambda x. E\den{O} ~ (\lambda s'. b ~ s' ~ x) ~ s)
  \\
  E\den{\Match P \gets M ~ O}
  &=
  \rlap{$
    \lambda b. \lambda s.
    \Match \den{M} \With \set{P \to E\den{O}~b~s; \_ \to b~s}
  $}
  % \lambda b. \lambda s.
  % \begin{aligned}[t]
  %   &\Match \den{M} \With \\
  %   &\quad
  %   \begin{aligned}[t]
  %     \{~
  %     P &\to E\den{O}~b~s; \\
  %     \_ &\to b~s
  %     ~\}
  %   \end{aligned}
  % \end{aligned}
  \\
  % E\den{\Nest O}
  % &=
  % \lambda b. \lambda s. \Rec s' = E\den{O} ~ (\lambda \_. b ~ s) ~ (\lambda x. s' ~ x)
  % \\
  E\den{\Try x \to B}
  &=
  \lambda x. T\den{B}
\end{align*}
\caption{Translating copattern-based source code to the target language.}
\label{fig:translation}
\end{figure}

The interesting cases for translating terms are the new forms.
$\Template B$ and $\Extension O$ are just translated to their given forms as transformation functions.
With $\lamstar B$, we need to recursively plug its translation in for its self parameter.
Note the one detail that the recursive $\mathit{self}$ is $\eta$-expanded to in this application.
This ensures that $\lambda x. \mathit{self} ~ x$ is treated as a value in a real implementation, and is always safe assuming that $B$ describes a function (non-functional cases of $\lamstar B$ are undefined user error).

For templates and extensions, the terminators $\Continue$ and $\Try$ are translated to plain $\lambda$-abstractions that allow the programmer direct access to their implicit parameters.
% Other cases are specific to each form.
Complex copatterns ($x ~ \many{P_1} P_n ~ O$) are reduced down to a simpler sequence of pattern lambdas ($x ~ \lambda P_1. \dots \lambda P_n. ~ O$), and pattern lambdas ($\lambda P. O$) are reduced down to a simpler non-matching lambda followed by an explicit match guard ($\lambda x. \Match P \gets x ~ O$).

This leaves just the base cases of simple extension forms.
Each time an extension (of form $\lambda b. \lambda s. \dots$) ``fails,'' it calls the given next template with the given self object ($b~s$).
A match guard $\den{\Match P \gets M ~ O}$ will try to match the translation of $M$ against the pattern $P$; the success case continues as $E\den{O}$ with the same next template and self.
A non-matching lambda $\den{\lambda x. O}$ always succeeds (for now), but note that the next template to try on failure has to be changed to include the given argument.
Why?
Because the lambda has already consumed the next argument from its context, it would be gone if, later on, the following operations fail and move on to the next option.
So instead of invoking the given $b$ directly as $b~s'$ (for a potentially different future $s'$), they need to invoke $b$ applied to this argument $x$ as $b~s'~x$.
% Finally, the $\den{\Nest O}$ operation is defined by recursively creating a new value for the self parameter by recursively taking a new snapshot of how the object looks now after all the preceding applications and matchings have already occurred.

In this translation, we also give the $\eta$-reduced forms on the right-hand side when available.
This shows that the empty extension $\varepsilon$ is just the identity function (given the next thing $b$ to try, $\varepsilon$ does nothing and immediately moves on to $b$), and horizontal composition $O; O'$ is just ordinary function composition.


%%% Local Variables:
%%% mode: LaTeX
%%% TeX-master: "coscheme"
%%% End:


\section{Macro Definition} \label{sec-macro}


The real implementation of copattern-matching in the Scheme macro system is quite similar to the high-level translation given in \cref{fig:translation}
However, there are some important differences which have to do with integrating the new feature with the rest of the language, as well as practical implementation details.
For example, note the definition of $\den{\lamstar B}$ in particular.
While the $\eta$-equality simplifying $\lambda x. \mathit{self} ~ x$ to just $\mathit{self}$ is theoretically sound, it does not work in practice:
when a Scheme interpreter tries to evaluate the right-hand side ($T\den{B}~\mathit{self}$) of the recursive binding, it first tries to lookup the value bound to $\mathit{self}$ which has not been defined yet, leading to an error.
This one level of $\eta$-expansion delays the evaluation step so that $\lambda x. \mathit{self} ~ x$ returns a closure around the location where $\mathit{self}$ will be placed, which is passed to $T\den{B}$ whose result is bound to $\mathit{self}$.

Happily, instead of a single big recursive macro, the first-class nature of templates and extensions make it possible to implement the various parts of copattern-matching as many independent macros that can be used separately and composed together by the programmer.  For example, $\lambda P. O$, $\If M ~ O$, $\Match P \gets M ~ O$, \etc are all implemented as self-contained macros that create new extension values around other extensions.
These forms need to be macros because they either bind variables around an expression (like $\lambda P$ or $\Match$) or don't evaluate a sub-expression in some cases (like $\If$).
Other simpler forms, like the empty object, or even $\Nest$ or the composition $O; O'$, are just ordinary procedural values and not defined as macros.
The macro for copattern matching, $Q[x]~O$, is the only main recursive step, which decomposes a copattern into a sequence of more basic matching $\lambda$s.

Additionally, the source language as implemented is more flexible than presented in \cref{fig:source-syntax}, in the sense that there are not as many syntactic categories.  
So the $O$ in forms like $\lambda P. O$ or $\If M ~ O$ can be \emph{any} host language expression as long as it evaluates to a procedure following the calling convention of extensions (otherwise a run-time error may be encountered).
The implementation also supports other standard Scheme expressions, including functions of multiple arguments (corresponding to \scm|(lambda (P ...) O)| or the copattern \scm|(self P ...)|) and variable numbers of arguments (corresponding to \scm|(lambda (P ... . rest) O)| or the copatterns \scm|(self P ... . rest)| or \scm|(apply self P ... rest)|).
The main points where the syntactic restrictions are used are in the macros implementing $\Extension O$ or $\Template B$.
For example, the \scm|extension| macro definition is:
\begin{minted}{scheme}
(define-syntax-rule
  (extension [copat step ...] ...)
  (merge [chain (comatch copat) step ...] ...))
\end{minted}
where \scm|merge| is the regular definition of first-class function composition, \scm|comatch| is the macro for the copattern-matching form $Q[self] ~ O$, and \scm|chain| is a macro for right-assocating any chain of operations to avoid overly-nested parentheses, with special support for unparenthesized terminators:
\begin{minted}{scheme}
(define-syntax chain
  (syntax-rules (= try)
    [(chain ext)                   ext]
    [(chain (op ...) step ... ext) (op ... (chain step ... ext))]
    [(chain = expr)                (always-do expr)]
    [(chain try ext)               ext]))
\end{minted}

One concern for a real implementation is to consider what kind of pattern-matching facilities the host language already provides.
Unfortunately, the answer is not standard across different languages in the Scheme family.
For example, the R${}^6$RS standard does not require any built-in support for pattern matching to be fully compliant, but specific languages like Racket may include a library for pattern matching by default.
Thus, we provide two different implementations to illustrate how copatterns may be implemented depending on their host language:
\begin{itemize}
\item
  A Racket implementation that uses its standard pattern-matching constructs \rkt|match| and \rkt|match-lambda*|.
  Thus, the $\Match$  from the target language in \cref{fig:target-syntax} is interpreted as Racket's \rkt|match|, and the translation of $E\den{\lambda P. O}$ is implemented directly as \rkt|match-lambda*| instead of separating the $\lambda$ from the pattern as in \cref{fig:translation}.
  This choice ensures the pattern language implemented is exactly the same as the pattern language already used in Racket programs.
\item
  A general implementation intended to work for any R${}^6$RS-compliant Scheme,%
  \footnote{We have explicitly tested this implementation against Chez Scheme.}
  %
  which internally implements its own pattern-matching macro, \scm|try-match|, by expanding into other primitives like \scm|if| and comparison predicates.
  Of note, due to only having to handle a single line of pattern-matching at a time, this implementation is 75 lines of Scheme and supports quasiquoting forms of patterns.
  This gives a sufficiently expressive intersection between Racket's pattern-matching syntax and the manually implemented R${}^6$RS version.
\end{itemize}


%%% Local Variables:
%%% mode: LaTeX
%%% TeX-master: "coscheme"
%%% End:


\section{Correctness} \label{sec-correctness}

\begin{figure}[t!]
\centering
\small
\begin{alignat*}{2}
  \mathit{Value} &\ni{}& V, W
  &::= x
  \mid \lambda x. M
  \mid \Null
  \mid \Cons V \, W
  \mid \q{x}
  \\
  \mathit{EvalCxt} &\ni{}& E
  &::= \hole
  \mid E ~ M
  \mid V ~ E
  \mid \Match E \With \set{\many{P \to N}}
  \mid \Rec x = E
\end{alignat*}
% Operational axioms:
\begin{align*}
  (\beta)
  &&
  (\lambda x. M) ~ V
  &=
  M\subst{x}{V}
  % \\
  % \begin{aligned}
  %   &\Match V \With \\
  %   &\qquad\set{\many[i]{P_i \to N_i}}
  % \end{aligned}
  % &=
  % N_k\subst{BV(P_k)}{\many{W}}
  % &
  % \begin{aligned}
  %   (&\text{if } && V = P_k\subst{BV(P_k)}{\many{W}} \\
  %   &\text{and } &&\forall 1 \leq j < k, \\
  %   &&&\not\exists \many{W'}, V = P_j\subst{BV(P_j)}{\many{W'}})
  % \end{aligned}
  \\
  (\mathit{match})
  &&
  \begin{aligned}
    &\Match V \With
    \begin{aligned}[t]
    \{~ &P \to N; \\
    &\many{P' \to N'}~\}
    \end{aligned}
  \end{aligned}
  &=
  N\subst{\many{x}}{\many{W}}
  &(\text{if } P\subst{\many{x}}{\many{W}} &= V)
  % &(\text{if } \exists \many{W},~ V &= P\subst{BV(P)}{\many{W}})
  \\
  (\mathit{apart})
  &&
  \begin{aligned}
    &\Match V \With
    \begin{aligned}[t]
    \{~ &P \to N; \\
    &\many{P' \to N'}~\}
    \end{aligned}
  \end{aligned}
  &=
  \begin{aligned}
    &\Match V \With \\
    &\qquad \set{\many{P' \to N'}}
  \end{aligned}
  &(\text{if } P &\apart V)
  % &(\text{if}\!\not\exists \many{W},~ V &= P\subst{BV(P)}{\many{W}})
  \\
  (\mathit{rec})
  &&
  (\Rec x = V)
  &=
  V\subst{x}{(\Rec x = V)}
\end{align*}

% Observational axioms:
% \begin{align*}
%   % \lambda x. (V ~ x)
%   % &=
%   % V
%   % &(\text{if } x &\notin FV(V))
%   % \\
%   (\lambda x. E[x]) ~ M
%   &=
%   E[M]
%   \\
%   E\left[
%     \begin{aligned}
%       &\Match M \With \\
%       &\qquad\set{\many{P \to N}}
%     \end{aligned}
%   \right]
%   &=
%   \begin{aligned}
%     &\Match M \With \\
%     &\qquad \set{\many{P \to E[N]}}
%   \end{aligned}
%   &(\text{if } BV(P) \cap FV(E) = \emptyset)
% \end{align*}

Apartness between patterns and values ($P \apart V$):
\begin{gather*}
  \infer
  {\q{x} \apart V}
  {V \notin \mathit{Variable} \cup \set{\q{x}}}
  \qquad
  \infer
  {\Null \apart V}
  {V \notin \mathit{Variable} \cup \set{\Null}}
  \\[1ex]
  \infer
  {\Cons P ~ P' \apart V}
  {V \notin \mathit{Variable} \cup \set{\Cons W ~ W' \mid W, W' \in \mathit{Value}}}
  \\[1ex]
  \infer
  {\Cons P ~ P' \apart \Cons W ~ W'}
  {P \apart W}
  \qquad
  \infer
  {\Cons P ~ P' \apart \Cons W ~ W'}
  {P' \apart W'}
  % \infer
  % {K ~ P_1 \dots P_n \apart V}
  % {V \neq K ~ W_1 \dots W_n}
  % \qqqquad
  % \infer
  % {K~P_1 \dots P_n \apart K ~ V_1 \dots V_n}
  % {1 \leq j \leq n & P_j \apart V_j}
\end{gather*}

\caption{Untyped equational axioms of the target language.}
\label{fig:target-equality}
\end{figure}

We already used the translation to a core $\lambda$-calculus as a specification for implementing compositional copatterns, but the translation is also useful for another purpose: checking the expected meaning of copattern-matching code.
With that in mind, we now look for some laws that let us equationally reason about some programs to make sure they behave as expected.

First, the core target language --- a standard call-by-value $\lambda$-calculus extended with pattern-matching and recursion --- has the equational theory shown in \cref{fig:target-equality}, which is the \emph{reflexive}, \emph{symmetric}, \emph{transitive}, and \emph{compatible} (\ie equalities can be applied in \emph{any} context) closure of the listed rules.
It has the usual $\beta$ axiom (restricted to substituting value arguments), two axioms for handling pattern-match success ($\mathit{match}$) and failure ($\mathit{apart}$), and an axiom for unrolling recursive values ($\mathit{rec}$).
Values ($V, W$) include the usual ones for call-by-value $\lambda$-calculus ($x$ and $\lambda x. M$) as well as lists ($\Null$ and $\Cons V ~ W$) and symbolic literals ($\q x$).
Matching a value $V$ against a pattern $P$ will succeed if the variables ($\many{x}$) in the pattern can be replaced by other values ($\many{W}$) to generate exactly that $V$: $P\subst{\many{x}}{\many{W}} = V$.
In contrast, matching fails if the two are known to be \emph{apart}, written $P \apart V$ and defined in \cref{fig:target-equality}, which implies that all possible substitutions of $P$ will \emph{never} generate $V$.
Note that while matching and apartness are mutually exclusive, there are some values that are neither matching nor apart from some patterns.
For example, compare the variable $x$ against the pattern $\Null$; $x$ may indeed stand for $\Null$ or another value like $\lambda y. M$.

The first usual property is that the translation specified in \cref{fig:translation} is a \emph{conservative extension}: any two terms that are equal by the target equational theory are still equal after translation.
Because the translation is hygienic and compositional by definition,  we can follow the proof strategy in \cite{DownenAriola2014CSCC}.

\begin{restatable}[Conservative Extension]{proposition}{thmconservativeextension}
  \label{thm:conservative-extension}
  If $M = N$ in the equational theory of the target
  (\cref{fig:target-equality}), then so too does $\den{M} = \den{N}$.
\end{restatable}

\begin{figure}[t!]
\centering
\small
\begin{alignat*}{2}
  % \mathit{TemplateValue} &\ni{}& B_v
  % &::= \varepsilon
  % \mid O_v; B_v
  % \mid \Continue x \to V
  % \\
  % \mathit{ExtensionValue} &\ni{}& O_v
  % &::= O_f
  % \mid O_f; O_v
  % \mid \Nest O_v
  % \mid \Try x \to B_v
  % \\
  \mathit{ExtensionFunc} &\ni{}& F
  &::= Q[x ~ P] ~ O
  \mid \lambda P. O
  \\
  \mathit{Value} &\ni{}& V
  &::= \dots
  \mid \lamstar (F; B)
  \mid \Template B
  \mid \Extension O
  % \mathit{NonRecTemplate} &\ni{}& B_{nr}
  % &::= O_{nr}; B_{nr}
  % \mid \Else \to M
  % \\
  % \mathit{NonRecExtension} &\ni{}& O_{nr}
  % &::= \varepsilon
  % \mid O_{nr}; O'_{nr}
  % \mid Q[\_] ~ O_{nr}
  % \mid \lambda P. O_{nr}
  % \\
  % &&&\phantom{:=}
  % \mid \Match P \gets M ~ O_{nr}
  % \mid \Nest O
  % \mid \Try x \to B_{nr}
  % \\
  % \mathit{RecCxt} &\ni{}& R
  % &::= \hole
  % \mid R; O
  % \mid O; R
  % \mid Q[x] ~ R
  % \mid \lambda P. R
  % \mid \Match P \gets M ~ O
  % \mid \Try x \to R
\end{alignat*}
Identity, associativity, and annihilation laws of composition:
\begin{align*}
  \varepsilon; O &= O % = O; \varepsilon
  &
  (O_1; O_2); O_3 &= O_1; (O_2; O_3)
  &
  \Do M; O &= \Do M
  \\
  \varepsilon; B &= B
  &
  (O_1; O_2); B &= O_1; (O_2; B)
  &
  \Do M; B &= \Else M
\end{align*}

% Decomposing patterns and copatterns:
% \begin{align*}
%   (Q[x] ~ P) ~ O
%   &=
%   Q[x] ~ (\lambda P. O)
%   &
%   \_ ~ O
%   &=
%   O
%   &
%   \lambda P. O
%   &=
%   \lambda x. (\Match P \gets x ~ O)
% \end{align*}

% Factoring out recursion ($x \neq y$ and $x \notin BV(P)$):
% \begin{align*}
%   \lambda y. (x ~ O)
%   &=
%   x ~ (\lambda y. O)
%   &
%   \Match P \gets M ~ (x ~ O)
%   &=
%   x ~ (\Match P \gets M ~ O)
%   \\
%   (x ~ O); O'
%   &=
%   x ~ (O; O')
%   &
%   O; (x ~ O')
%   &=
%   x ~ (O; O')
% \end{align*}

% Instantiating templates and recursive $\lamstar$:
% \begin{align*}
%   % (\Extension O) ~ V
%   % &=
%   % \Template{} (O; \Continue x \to (V ~ x))
%   % \\
%   % (\Template R[Q[x] ~ O]) ~ V
%   % &=
%   % (\Template R[Q[\_] ~ O\subst{x}{V}]) ~ V
%   % \\
%   % (\Template R[\Continue x \to M]) ~ V
%   % &=
%   % (\Template R[\Else \to M\subst{x}{V}]) ~ V
%   % \\
%   % (\Template B_{nr}) ~ V
%   % &=
%   % (\Template B_{nr}) ~ W
%   % \\
%   \lamstar (F; B)
%   &=
%   (\Template F; B) ~ (\lamstar (F; B))
%   % \\
%   % \lambda x. (\lamstar (F; B)) ~ x
%   % &=
%   % \lamstar (F; B)
%   \\
%   (\Template O; B) ~ V
%   &=
%   (\Extension O) ~ (\Template B) ~ V
%   \\
%   (\Template \varepsilon) ~ V
%   &=
%   \mathit{fail}~V
%   \\
%   (\Template \Continue x \to M) ~ V
%   &=
%   M\subst{x}{V}
%   \\
%   (\Extension \Try x \to B) ~ V
%   &=
%   \Template B\subst{x}{V}
% \end{align*}

Pattern and copattern matching:
\begin{align*}
  \Match P \gets V ~ O
  &=
  O\subst{\many{x}}{\many{W}}
  &(\text{if } P\subst{\many{x}}{\many{W}} &= V)
  \\
  \Match{} P \gets V ~ O
  &=
  \varepsilon
  &(\text{if } P &\apart V)
  \\[1ex]
  (\Template{} (\lambda P. \Do M); B) ~ V' ~ V
  &=
  M\subst{\many{x}}{\many{W}}
  &(\text{if } P\subst{\many{x}}{\many{W}} &= V)
  \\
  (\Template{} (\lambda P. O); B) ~ V' ~ V
  &=
  (\Template B) ~ V' ~ V
  &(\text{if } P &\apart V)
  \\[1ex]
  C[(\Template{} (Q[y] = M); B) ~ V]
  &=
  M\subst{y}{V}\subst{\many{x}}{\many{W}}
  &(\text{if } Q\subst{\many{x}}{\many{W}} &= C)
  \\
  C[(\Template{} (Q[y] ~ O); B) ~ V]
  &=
  C[(\Template B) ~ V]
  &(\text{if } Q &\apart C)
  \\[1ex]
  C[\lamstar (Q[y] = M); B]
  &=
  \begin{aligned}[t]
    M
    &\subst{y}{(\lamstar (Q[y] = M); B)}
    \\
    &\subst{\many{x}}{\many{W}}
  \end{aligned}
  &(\text{if } Q\subst{\many{x}}{\many{W}} &= C)
  \\
  C[\lamstar (Q[y] ~ O); \Else M]
  &=
  C[M]
  &(\text{if } Q &\apart C)
\end{align*}

Apartness between copatterns and contexts ($Q \apart C$):
\begin{gather*}
  \infer
  {Q ~ P \apart C ~ V}
  {Q\subst{\many{x}}{\many{W}} = C & P \apart V}
  \qqqquad
  \infer
  {Q ~ P \apart C}
  {Q \apart C}
  \qqqquad
  \infer
  {Q \apart C ~ V}
  {Q \apart C}
\end{gather*}

\caption{Some equalities of copattern extensions.}
\label{fig:source-equality}
\end{figure}

To reason about the new features in the source language --- introduced by $\lamstar$, $\Template$, and $\Extension$ --- we introduce additional axioms given in \cref{fig:source-equality}, so that the source equational theory is the \emph{reflexive}, \emph{symmetric}, \textit{transitive}, and \emph{compatible} closure of these rules in both \cref{fig:target-equality,fig:source-equality}.
The purpose of these new equalities is to perform some reasoning about programs using copatterns, and in particular, to check that the syntactic use of \scm|=| really means equality.
For example, a special case is
\begin{math}
  Q[\lamstar(Q[y] = M); B] = M\subst{y}{\lamstar(Q[y]=M);B}
  ,
\end{math}
which says a $\lamstar$ appearing in \emph{exactly} the same context as the left-hand side of an equation will unroll (recursively) to the right-hand side.
Other equations describe algebraic laws of copattern alternatives and how to fill in templates and extensions when applied.
This source equational theory is \emph{sound} with respect to translation.

\begin{restatable}[Soundness]{proposition}{thmsoundness}
  \label{thm:soundness}
  The translation is \emph{sound} w.r.t. the source and target equational theories (\eg $M = N$ in \cref{fig:source-equality} implies $\den{M} = \den{N}$ in \cref{fig:target-equality}).
  % The equational axioms given in \cref{fig:source-equality} are sound with
  % respect to the translation in \cref{fig:translation},
  % \begin{align*}
  %   M &= M' &&\implies & \den{M} &= \den{M'} \\
  %   B &= B' &&\implies & T\den{B} &= T\den{B'} \\
  %   O &= O' &&\implies & E\den{O} &= E\den{O'}
  % \end{align*}
  % up to the equational theory of the target language in
  % \cref{fig:target-equality}.
\end{restatable}


%%% Local Variables:
%%% mode: LaTeX
%%% TeX-master: "coscheme"
%%% End:


% \section{Optimizing Away Administrative Reductions} \label{sec-opt}

% \begin{figure}
\centering
Translating new terms:  
\begin{align*}
  \den{\lamstar B}()
  &=
  \Rec \mathit{self} = T\den{B}(\lambda x. \mathit{self} ~ x)
  \\
  \den{\Template B}()
  &=
  \lambda s. T\den{B}(s)
  \\
  \den{\Extension O}()
  &=
  \lambda b. \lambda s. E\den{O}(b, s)
  \\
  \den{M}()
  &=
  \text{by induction}
  &(\text{otherwise})
\end{align*}
Translating templates:
\begin{align*}
  T\den{\varepsilon}(V)
  &=
  \mathit{fail}~V
  \\
  T\den{O; B}(V)
  &=
  (\lambda b. E\den{O}(b, V)) ~ (\lambda s. T\den{B}(s))
  \\
  T\den{\Continue x \to M}(V)
  &=
  \den{M}()\subst{x}{V}
\end{align*}

Translating copattern-matching and pattern-matching functions:
\begin{align*}
  E\den{(Q[x] ~ P) ~ O}(W, V)
  &=
  E\den{Q[x] ~ (\lambda P. O)}(W, V)
  \\
  E\den{x ~ O}(W, V)
  &=
  E\den{O}(W, V)\subst{x}{W}
  \\
  E\den{\lambda P. O}(W, V)
  &=
  E\den{\lambda x. \Match P \gets x ~ O}(W, V)
  &(\text{if } P \notin \mathit{Variable})
\end{align*}

Translating other extensions:
\begin{align*}
  E\den{\varepsilon}(W, V)
  &=
  W(V)
  \\
  E\den{O; O'}(W, V)
  &=
  (\lambda b. E\den{O}(b, V)) ~ (\lambda s. E\den{O'}(W, s))
  \\
  E\den{\lambda x. O}(W, V)
  &=
  \lambda x. E\den{O}((\lambda s'. W(s') ~ x), V)
  \\
  E\den{\Match P \gets M ~ O}(W, V)
  &=
  \Match \den{M}() \With
  \set{P \to E\den{O}(W, V); \_ \to W(V)}
  \\
  E\den{\Nest O}(W, V)
  &=
  \Rec s' = E\den{O}((\lambda \_. W(V)), (\lambda x. s' ~ x))
  \\
  E\den{\Try x \to B}(W, V)
  &=
  T\den{B}(V)\subst{x}{W}
\end{align*}

Inlining administrative $\lambda$-abstractions:
\begin{align*}
  (\lambda x. M)(W, \many{V})
  &=
  M(\many{V})\subst{x}{W}
  \\
  M(\many{V})
  &=
  M ~ \many{V}
  &(\text{otherwise})
\end{align*}

\caption{Inlining version of the translation.}
\label{fig:inline-translation}
\end{figure}

\begin{proposition}
  Up to the equational theory of the target language in
  \cref{fig:target-equality},
  \begin{align*}
    \den{M}() &= \den{M}
    \\
    \lambda s. T\den{B}(s) &= T\den{B}
    \\
    \lambda b. \lambda s. E\den{O}(b, s) &= E\den{O}
  \end{align*}  
\end{proposition}


%%% Local Variables:
%%% mode: LaTeX
%%% TeX-master: "coscheme"
%%% End:


\section{Related and Future Work} \label{sec-future}

\section{Conclusion} \label{sec-conclusion}

\begin{credits}
\subsubsection{\ackname}
% 
This material is based upon work supported by the National Science Foundation
under Grant No. 2245516.

\subsubsection{\discintname}
%
The authors have no competing interests to declare that are
relevant to the content of this article.
\end{credits}
%
% ---- Bibliography ----
%
% BibTeX users should specify bibliography style 'splncs04'.
% References will then be sorted and formatted in the correct style.
%
\bibliographystyle{splncs04}
\bibliography{refs}

\section{Appendix A: Proofs}
\allowdisplaybreaks

\adriano{Review the copattern case (probably use induction)}

\adriano{Figure out the correct argument to assure things about $\den{M}$'s}

\subsection{Conservative extension of the target}

\begin{restatable}{lemma}{thmvaluetranslation}
  \label{l-value}
  \label{thm:value-translation}
  The following instances of translation are all values up to the equational
  theory of the target language in \cref{fig:target-equality}:
  \begin{enumerate}[(a)]
  \item $T\den{B} = \lambda s. M$ for some term $M$,
  \item $E\den{O} = \lambda b. \lambda s. M$ for some term $M$,
  \item $E\den{F} = \lambda b. \lambda s. \lambda x. M$ for some term $M$,
  \item $\den{V} = W$ for some value $W$.
  \end{enumerate}
\end{restatable}
% \thmvaluetranslation*
\begin{proof}
  % The listed instances of translation are all values up to the equational theory of the target language in \cref{fig:target-equality}:

  % The proof follows
  By mutual induction on the syntax of templates $B$, extensions $O$, extension functions $F$, and values $V$.
  % By definition, we know that our translation returns terms in the target language.
  % In other words, we can say that $\den{M}=M'$ for some target term $M'$.
  \begin{enumerate}[(a)]
  \item $T\den{B} = \lambda s. M$ for some term $M$, as shown by the following cases:
    % \begin{proof}
    %   By induction on $B$.
    \begin{itemize}
    \item $(B = \varepsilon)$
      % By our translation we have:
      $T\den{\varepsilon} = \lambda s. ~\mathit{fail} ~s$, so $M=\mathit{fail} ~s$.
    \item ($B = O; B'$)
      $T\den{O; B'} = \lambda s. ~\den{O} ~\den{B'} ~s$, so $M = \den{O} ~\den{B'} ~s$.
      % By our translation we have: $T\den{O; B'} = \lambda s. ~\den{O} ~\den{B'} ~s$.
      % Our induction hypothesis provides us that $T\den{B'} = \lambda s. M'$ for some term $M'$.
      % % By (b) we know that $E\den{O}=\lambda b. \lambda s. M_O$ for some term $M_O$.
      % \begin{align*}
      %   & \quad \lambda s. ~\den{O} ~\den{B'} ~s \\
      %   % =& \quad \lambda s.  ~(\lambda b. \lambda s. M_O) ~(\lambda s. M') ~s & (IH, (b))
      %   =& \quad \lambda s. ~M_O ~(\lambda s. M') ~s & (IH)
      % \end{align*}
      % Where $M=M_O ~(\lambda s. M') ~s$.
    \item $(B = \Continue x {\,\to\,} N)$
      $T\den{\Continue x {\,\to\,} N} = \lambda x. \den{N}$, so $M = \den{N}$.
      % \begin{align*}
      %   & \quad T\den{\Continue x \to M'} = \lambda x. ~\den{M'} \\
      %   =& \quad \lambda x. ~(\lambda s. ~M'') \\
      %   =& \quad \lambda s. ~(\lambda s. ~M'') \subst{x}{s} & (=_{\alpha})
      % \end{align*} 
    \end{itemize}
    % \qed
    % \end{proof}
  \item $E\den{O} = \lambda b. \lambda s. M$ for some term $M$, as shown by the following cases: 
    % \begin{proof}
    %   By induction on $O$.
    \begin{itemize}
    \item $(O = \varepsilon)$
      % By our translation we have
      $E\den{\varepsilon} = \lambda b. \lambda s. ~b ~s$, so $M= b ~s$.
    \item $(O = O_1;O_2)$
      $E\den{O_1;O_2} = \lambda b. \lambda s. ~\den{O_1} ~(\den{O_2} ~b) ~s$, so $M = \den{O_1} ~(\den{O_2} ~b) ~s$.
      % \begin{align*}
      %   & \quad E\den{O_1;O_2} = \lambda b. \lambda s. ~\den{O_1} ~(\den{O_2} ~b) ~s\\
      %   =& \quad \lambda b. \lambda s. ~(\lambda b. \lambda s. ~M_{O_1}) ~((\lambda b. \lambda s. ~M_{O_2}) ~b) ~s & (IH)
      % \end{align*}
      % Where $M=(\lambda b. \lambda s. ~M_{O_1}) ~((\lambda b. \lambda s. ~M_{O_2}) ~b) ~s$.
    \item $(O = Q[x] ~O)$ follows by induction on $Q$ (generalizing $O$):
      \begin{itemize}
      \item $(Q=\hole)$ for all $O$,
        % According to our translation:
        \begin{align*}
          E\den{x ~O}
          &=
          \lambda b. \lambda x. E\den{O} ~ b ~ x
          \\
          &=
          \lambda b. \lambda s. E\den{O}\subst{x}{s} ~ b ~ s
          &(\alpha)
        \end{align*}
        so $M = E\den{O}\subst{x}{s} ~ b ~ s$ for the given $O$.
        % By our induction hypothesis we can infer that $E\den{O}=\lambda b. \lambda s. M_O$.
        % Therefore, we can conclude that $\lambda b. \lambda x. ~(\lambda b. \lambda s. M_O) ~ b ~ x$, where $ M = (\lambda b. \lambda s. M_O) ~ b ~ x$.
      \item $(Q=Q'~P)$
        assuming the inductive hypothesis (IH) that, for all $O$, there is an $N_O$ such that $E\den{Q[x] ~ O} = \lambda b. \lambda s. N_O$.
        For all $O$,
        \begin{align*}
          E\den{(Q'[x] ~ P) O}
          &=
          E\den{Q'[x] ~ (\lambda P. O)}
          \\
          &=
          \lambda b. \lambda s. N_{(\lambda P. O)}
          &(IH)
        \end{align*}
        so $M = N_{(\lambda P. O)}$ given by the inductive hypothesis applied to $\lambda P. O$.
        % According to our translation:
        % $E\den{(Q ~P) ~ [x] ~O} = E\den{(Q[x] ~P) ~O} = E\den{Q[x] ~ (\lambda P. O)}$.
        % Eventually, we will hit the previous case after converting all patterns in the copattern into pattern $\lambda$'s.
        % After converting all patterns $O$ will have a shape of form: $\lambda P_0. ~\lambda P_1 ... \lambda P_n. ~O$.
        % \adriano{I think I need to use induction here.}
      \end{itemize}
    \item $(O = \lambda P. ~O)$ follows by cases if $P$ is a variable or another pattern:
      \begin{itemize}
      \item $(P \in \mathit{Variable})$
        \begin{math}
          E\den{\lambda x. ~O}
          =
          \lambda b. \lambda s.
          (\lambda x. E\den{O} ~ (\lambda s'. b ~ s' ~ x) ~ s)
          ,
        \end{math}
        so
        \\
        $M = \lambda x. E\den{O} ~ (\lambda s'. b ~ s' ~ x) ~ s$.
      \item $(P \notin \mathit{Variable})$
      \begin{math}
        E\den{\lambda P. ~O}
        =
        E\den{\lambda x. \Match P \gets x ~ O}
        ,
      \end{math}
      which follows by the above case.
      \end{itemize}
      % According to our translation:
      % \begin{align*} 
      %   & \quad E\den{\lambda P. ~O} = E\den{\lambda x. \Match P \gets x ~ O}\\
      %   =& \quad \lambda b. \lambda s. (\lambda x. E\den{\Match P \gets x ~ O} ~ (\lambda s'. b ~ s' ~ x) ~ s) \\
      %   =& \quad \lambda b. \lambda s. (\lambda x. \lambda b. \lambda s. (
      %   \begin{aligned}[t]
      %     &\Match \den{M} \With \\
      %     &\quad
      %     \begin{aligned}[t]
      %       \{~
      %       P &\to E\den{O}~b~s; \\
      %       \_ &\to b~s
      %       ~\}
      %     \end{aligned}
      %   \end{aligned}
      %   )~ (\lambda s'. b ~ s' ~ x) ~ s) \\
      %   =& \quad \lambda b. \lambda s. (\lambda x. \lambda b. \lambda s. (
      %   \begin{aligned}[t]
      %     &\Match \den{M} \With \\
      %     &\quad
      %     \begin{aligned}[t]
      %       \{~
      %       P &\to (\lambda b. \lambda s. M_O)~b~s; \\
      %       \_ &\to b~s
      %       ~\}
      %     \end{aligned}
      %   \end{aligned}
      %   )~ (\lambda s'. b ~ s' ~ x) ~ s) & (IH)
      % \end{align*}
    \item $(O = \Match P \gets N ~ O)$
        \begin{align*}
          & \quad E\den{\Match P \gets N ~ O}
          =
          \lambda b. \lambda s.
          \begin{aligned}[t]
            &\Match \den{N} \With \\
            &\quad
            \begin{aligned}[t]
              \{~
              P &\to E\den{O}~b~s; \\
              \_ &\to b~s
              ~\}
            \end{aligned}
          \end{aligned}
          % \\
          % =& \quad \lambda b. \lambda s.
          % \begin{aligned}[t]
          %   &\Match \den{N} \With \\
          %   &\quad
          %   \begin{aligned}[t]
          %     \{~
          %     P &\to (\lambda b. \lambda s.~ M_O) ~b~s; \\
          %     \_ &\to b~s
          %     ~\}
          %   \end{aligned}
          % \end{aligned}  & (IH)
        \end{align*}
        so $M = \Match \den{N} \With \set{P \to E\den{O}~b~s; \_ \to b~s}$.
      \item $(O = \Nest O)$
        \begin{math}
          E\den{\Nest O}
          =
          \lambda b. \lambda s.
          (\Rec s' = E\den{O} ~ (\lambda \_. b ~ s) ~ (\lambda x. s' ~ x))
          ,
        \end{math}
        so $M = (\Rec s' = E\den{O} ~ (\lambda \_. b ~ s) ~ (\lambda x. s' ~ x))$
        % \begin{align*}
        %   & \quad E\den{\Nest O} = \lambda b. \lambda s. \Rec s' = E\den{O} ~ (\lambda \_. b ~ s) ~ (\lambda x. s' ~ x)\\
        %   =& \quad \lambda b. \lambda s. \Rec s' = (\lambda b. \lambda s.~ M_O) ~ (\lambda \_. b ~ s) ~ (\lambda x. s' ~ x) & (IH)
        % \end{align*}
      \item $(O = \Try x \to B)$
        assuming the inductive hypothesis $(IH)$ from part (a) that
        $T\den{B} = \lambda s. N$ for some $N$,
        \begin{align*}
          E\den{\Try x \to B}
          &= \lambda x. T\den{B}
          \\
          &= \lambda x. (\lambda s.~ N)
          & (IH)
          \\
          &=\lambda b. (\lambda s.~ N)\subst{x}{b}
          & (\alpha)
        \end{align*}
        so $M = N\subst{x}{b}$
    \end{itemize}
    % \qed
    % \end{proof}
  \item $E\den{F} = \lambda b. \lambda s. \lambda x. M$ for some term $M$, as shown by the following cases:
    % \begin{proof}
    %   Let us consider the possible values of $F$.
    \begin{itemize}
    \item $(F= \lambda P. ~O)$
      Following the same calculation in the matching special case of part (b) above,
      $E\den{\lambda P. O} = \lambda b. \lambda s. (\lambda x. M)$
      for some $M$.
    %   Let' consider if $P$ is a variable pattern or not:
    %   \begin{itemize}
    %   \item $(P = x)$ According to our translation $E\den{\lambda x. ~O}=\lambda b. \lambda s. (\lambda x. E\den{O} ~ (\lambda s'. b ~ s' ~ x) ~ s)$.
    %     Therefore, by (b), we can conclude that $\lambda b. \lambda s. (\lambda x. (\lambda b. \lambda s. M_O) ~ (\lambda s'. b ~ s' ~ x) ~ s)$
    %   \item $(P \neq x)$ We have that $E\den{\lambda P. ~O}=E\den{\lambda x. \Match P \gets x ~ O}$ which is a specific case of the previous case, where $O=\Match P \gets x ~ O$.
    %   \end{itemize} 
    \item $(F= Q[x ~P] ~O)$
      follows by induction on $Q$ (generalizing $O$):
      \begin{itemize}
      \item $(Q = \hole)$ for all $O$,
        \begin{align*}
          E\den{(y ~ P) ~ O}
          =
          E\den{y ~ (\lambda P. O)}
          &=
          \lambda b. \lambda y. E\den{\lambda P. O} ~ b ~ y
          % \\
          % &=
          % \lambda b. \lambda s. E\den{\lambda P. O}\subst{y}{s} ~ b ~ s
          % &(\alpha)
        \end{align*}
        Following the same calculation in the previous case $(F = \lambda P. O)$ gives us
        some $N$ such that $E\den{\lambda P. O} = \lambda b. \lambda s'. \lambda x. N$, so continuing we have
        \begin{align*}
          E\den{(y ~ P) ~ O}
          &=
          \lambda b. \lambda y. (\lambda b. \lambda s'. \lambda x. N) ~ b ~ y
          \\
          &=
          \lambda b. \lambda y. \lambda x. N\subst{s'}{y}
          &(\beta)
          \\
          &=
          \lambda b. \lambda s. \lambda x. N\subst{s'}{y}\subst{y}{s}
          &(\alpha)
        \end{align*}
        so $M = N\subst{s'}{y}\subst{y}{s}$.
      \item $(Q = Q' ~ P')$
        assuming the inductive hypothesis that,
        for all $O$, there is an $N_O$ such that
        $E\den{Q'[y ~ P] ~ O} = \lambda b. \lambda s. \lambda x. N_O$.
        For all $O$,
        \begin{align*}
          E\den{(Q'[y ~ P] ~ P') ~ O}
          &=
          E\den{Q'[y ~ P] ~ (\lambda P'. O)}
          \\
          &=
          \lambda b. \lambda s. \lambda x. N_{(\lambda P'. O)}
          &(IH)
        \end{align*}
        so $M = N_{(\lambda P'. O)}$ given by the inductive hypothesis applied to $\lambda P'. O$.
      \end{itemize}
      % Eventually, after converting the whole copattern we will hit $E\den{x ~O}$, where $O$ is some sequence of nested $\lambda P$'s.
      % \begin{align*}
      %   & \quad E\den{x ~O} =  \lambda b. \lambda x. E\den{\lambda P. O'} ~ b ~ x\\
      %   =& \quad \lambda b. \lambda x. (\lambda b. \lambda s. \lambda x. M') ~ b ~ x & (IH) \\
      %   =& \quad \lambda b. \lambda x. (\lambda x. M'\subst{s}{x}) & (\beta) \\
      %   =& \quad \lambda b. \lambda s. (\lambda x. M'\subst{s}{x})\subst{x}{s} & (\alpha)
      % \end{align*}
    \end{itemize}
    % \qed
    % \end{proof}
  \item $\den{V} = W$ for some value $W$, as shown by the following cases:
    % \begin{proof}
    %   By induction on $V$.
    \begin{itemize}
    \item $(V = x)$
      $\den{x} = x$, so $W=x$.
    \item $(V = \lambda x. M)$
      $\den{\lambda x. M} = \lambda x. \den{M}$, so $W = \lambda x. \den{M}$.
    \item $(V = \Null)$
      $\den{\Null} = \Null$, so $W=\Null$.
    \item $(V = \Cons V_1 ~ V_2)$
      $\den{\Cons V_1 ~ V_2} = \Cons \den{V_1} ~ \den{V_2}$,
      where $\den{V_1} = W_1$ and $\den{V_2} = W_2$ by the inductive hypotheses,
      so $W = \Cons W_1 ~ W_2$.
    \item $(V = \Template B)$
      $\den{\Template B} = T\den{B} = \lambda s. M$, for some $M$,
      by the inductive hypothesis part (a),
      so $W = \lambda s. M$.
    \item $(V = \Extension O)$
      $\den{\Extension O} = E\den{O} = \lambda b. \lambda s. M$, for some $M$,
      by the inductive hypothesis part (b),
      so $W = \lambda b. \lambda s. M$.
    \item $(V = \lamstar (F; B))$
      assuming the inductive hypotheses that 
      \begin{itemize}
      \item[$IH_1$] there is some $N_1$ such that
        $E\den{F} = \lambda b. \lambda s. \lambda x. N_1$, and
      \item[$IH_2$] there is some $N_2$ such that
        $T\den{B} = \lambda s. N_2$,
      \end{itemize}
      \begin{align*}
        \den{\lamstar (F; B)}
        &=
        (\Rec \mathit{self} = T\den{F; B} ~ (\lambda x. \mathit{self} ~ x))
        \\
        &=
        (\Rec \mathit{self} = E\den{F} ~ T\den{B} ~ (\lambda x. \mathit{self} ~ x))
        &(\beta)
        \\
        &=
        (\Rec \mathit{self}
        = (\lambda b. \lambda s. \lambda x. N_1)
        ~ T\den{B}
        ~ (\lambda x. \mathit{self} ~ x))
        &(IH_1)
        \\
        &=
        (\Rec \mathit{self}
        = (\lambda b. \lambda s. \lambda x. N_1)
        ~ (\lambda s. N_2)
        ~ (\lambda x. \mathit{self} ~ x))
        &(IH_2)
        \\
        &=
        (\Rec \mathit{self}
        = \lambda x.N_1\subst{b}{(\lambda s.N_2)}\subst{s}{(\lambda x.\mathit{self}~x)})
        &(\beta)
        \\
        &=
        \begin{aligned}[t]
          \lambda x.
          N_1
          &\subst{b}{(\lambda s.N_2)}
          \\
          &\subst{s}{(\lambda x.\mathit{self}~x)}
          \\
          &\subst
          {\mathit{self}}
          {
            (\Rec \mathit{self}
            =
            \lambda x.N_1\subst{b}{(\lambda s.N_2)}\subst{s}{(\lambda x.\mathit{self}~x)})
          }
        \end{aligned}
        &(rec)
      \end{align*}
    % \qed
    % \end{proof}
    \end{itemize}
  \end{enumerate}
\end{proof}

\begin{lemma}[Substitution]
  \label{thm:substitution-translation}
  For all values $V$,
  \begin{enumerate}[(a)]
  \item $\den{M\subst{x}{V}} = \den{M}\subst{x}{\den{V}}$,
  \item $T\den{B\subst{x}{V}} = T\den{B}\subst{x}{\den{V}}$,
  \item $E\den{O\subst{x}{V}} = E\den{O}\subst{x}{\den{V}}$.
  \end{enumerate}
\end{lemma}
\begin{proof}
  By mutual induction on the syntax of terms $M$, tempaltes $B$, and extensions $O$.
\end{proof}

\begin{lemma}[Compositionality]
  \label{thm:compositional-translation}

  There exists a translation of contexts $\den{C}$ such that
  $\den{C[M]} = \den{C}{[\den{M}]}$.
\end{lemma}
\begin{proof}
  By induction on the context $C$.
\end{proof}

\thmconservativeextension*
\begin{proof}
  The proof strategy follows from proposition 1 of \cite{DownenAriola2014CSCC}.
  
  Reflexivity, transitivity, and symmetry of the equational theory follows immediately.

  Congruence---$M = N$ implies $C[M] = C[N]$ for all contexts $C$---follows from
  the fact that the translation is \emph{compositional}
  (\cref{thm:compositional-translation}).  An equation $C[M] = C[N]$, derived
  from congruence of $M = N$ inside $C$, can be derived by distributing the
  translation across contexts to apply the underlying equality
  $\den{M} = \den{N}$ gotten from soundness of $M = N$:
  \begin{align*}
    \den{C[M]}
    &=
    \den{C}{[\den{M}]}
    &(\cref{thm:compositional-translation})
    \\
    &=
    \den{C}{[\den{N}]}
    \\
    &=
    \den{C[N]}
    &(\cref{thm:compositional-translation})
  \end{align*}

  Finally, each individual axiom ($\beta$, $rec$, \etc) from the target language
  immediately holds by the inductive hypothesis---because the definition of
  translation for these cases does not change syntax---along with
  \cref{thm:value-translation,thm:substitution-translation} for cases which
  substitute.  For example, with the $\beta$ axiom, we have
  \begin{align*}
    \den{(\lambda x. M) ~ V}
    &=
    (\lambda x. \den{M}) ~ \den{V}
    \\
    &=
    (\lambda x. \den{M}) ~ W
    &(\cref{thm:value-translation})
    \\
    &=
    \den{M}\subst{x}{W}
    &(\beta)
    \\
    &=
    \den{M}\subst{x}{\den{V}}
    &(\cref{thm:value-translation})
    \\
    &=
    \den{M\subst{x}{V}}
    &(\cref{thm:substitution-translation})
  \end{align*}
  since values are only translated to values, $\den{V} = W$ up to the target equational theory (\cref{thm:value-translation}), which means that $\beta$-reduction still applies after translation.
\end{proof}

\subsection{Soundness of the source}

\begin{lemma}[Pattern Translation]
  \label{thm:pattern-translation}

  $\den{P} = P$.
\end{lemma}
\begin{proof}
  By induction on the syntax of $P$.
\end{proof}

\begin{lemma}[Apartness Translation]
  \label{thm:apartness-translation}

  If $P \apart V$ then $P \apart \den{V}$.
\end{lemma}
\begin{proof}
  By induction on the derivation of apartness $P \apart V$.
\end{proof}

\thmsoundness*
\begin{proof}
  Each axiom is derived up to the target equational theory in the following lemmas.
\end{proof}

\begin{lemma}[Extension Composition Identity Left]
  \label{thm:ext-compose-id-left}
    $ E\den{\varepsilon; O} = E\den{O}.$
\end{lemma}
    \begin{proof}
        \begin{align*}
            &\quad E\den{\varepsilon; O} \\
            =& \quad \lambda b. \lambda s.~ E\den{\varepsilon} ~(E\den{O} ~b) ~s \\
            =&  \quad \lambda b. \lambda s.~ E\den{\varepsilon} ~((\lambda b. \lambda s.~ M_0) ~b) ~s & (\cref{l-value})\\
            =& \quad \lambda b. \lambda s.~ (\lambda b. \lambda s.~ b ~s) ~((\lambda b. \lambda s.~ M_0) ~b) ~s \\
            =& \quad \lambda b. \lambda s.~ (\lambda b. \lambda s.~ b ~s) ~(\lambda s.~ M_0) ~s & (\beta)\\
            =& \quad \lambda b. \lambda s.~ (\lambda s.~ (\lambda s.~ M_0) ~s) ~s & (\beta)\\
            =& \quad \lambda b. \lambda s.~ (\lambda s.~ M_0) ~s & (\beta)\\
            =& \quad \lambda b. \lambda s.~ M_0 & (\beta) \\
            =& \quad E\den{O} & (\cref{l-value})
        \end{align*}
        \qed
    \end{proof}

%% Paul: Removing this lemma since it is the only one that requires eta.
    
% \begin{lemma}[Extension Composition Identity Right]
%   \label{thm:ext-compose-id-right}
%   $ E\den{O} = E\den{O;\varepsilon}.$
% \end{lemma}
%     \begin{proof}
%         \begin{align*}
%             &\quad E\den{O;\varepsilon} \\
%             =& \quad \lambda b. \lambda s.~ E\den{O} ~(E\den{\varepsilon} ~b) ~s \\
%             =& \quad \lambda b. \lambda s.~ E\den{O} ~((\lambda b. \lambda s.~ b ~s) ~b) ~s \\
%             =& \quad \lambda b. \lambda s.~ E\den{O} ~(\lambda s.~ b ~s) ~s & (\beta)\\
%             =& \quad \lambda b. \lambda s.~ (\lambda b. \lambda s.~ M_0) ~(\lambda s.~ b ~s) ~s & (\cref{l-value})\\
%             =& \quad \color{red}{\lambda b. \lambda s.~ (\lambda b. \lambda s.~ M_0) ~b ~s} & \color{red}{(\eta)}\\
%             =& \quad \color{red}{\lambda b. \lambda s.~ (\lambda s.~ M_0) ~s} & (\beta)\\
%             =& \quad \color{red}{\lambda b. \lambda s.~ M_0} & (\beta) \\
%             =& \quad \color{red}{E\den{O}} & (\cref{l-value})
%         \end{align*}
%         \qed
%     \end{proof}

\begin{lemma}[Template Composition Identity Left]
  \label{thm:tmpl-compose-id-left}
  $ T\den{\varepsilon; B} = T\den{B}.$
\end{lemma}
    \begin{proof}
        \begin{align*}
            &\quad T\den{\varepsilon; B} \\
            =& \quad \lambda s. ~ E\den{\varepsilon} ~T\den{B} ~s \\
            =& \quad \lambda s. ~ (\lambda b. \lambda s.~ b ~s) ~T\den{B} ~s \\
            =& \quad \lambda s. ~ T\den{B} ~s & (\beta, \cref{thm:value-translation})\\
            =& \quad T\den{B} & (\alpha, \beta, \cref{l-value})
        \end{align*}
        \qed
    \end{proof}

\begin{lemma}[Extension Composition Associativity]
  \label{thm:ext-compose-assoc}

  $ E\den{(O_1;O_2);O_3} = E\den{O_1;(O_2;O_3)}.$
\end{lemma}
\begin{proof}
  The left-hand side simplifies as follows:
  \begin{align*}
    &\quad E\den{(O_1;O_2);O_3} \\
    =& \quad \lambda b. \lambda s.~ E\den{O_1;O_2} ~(E\den{O_3} ~b) ~s \\
    =& \quad \lambda b. \lambda s.~ (\lambda b. \lambda s.~ E\den{O_1} ~(E\den{O_2} ~b) ~s) ~(E\den{O_3} ~b) ~s \\
    =& \quad \lambda b. \lambda s.~ (\lambda b. \lambda s.~ (\lambda b. \lambda s. M_1) ~((\lambda b. \lambda s. M_2) ~b) ~s) ~((\lambda b. \lambda s. M_3) ~b) ~s & (\cref{l-value}) \\
    =& \quad \lambda b. \lambda s.~ (\lambda b. \lambda s.~ (\lambda b. \lambda s. M_1) ~((\lambda b. \lambda s. M_2) ~b) ~s) ~(\lambda s. M_3) ~s & (\beta) \\
    =& \quad \lambda b. \lambda s.~ (\lambda b. \lambda s. M_1) ~((\lambda b. \lambda s. M_2) ~(\lambda s. M_3)) ~s & (\beta) \\
    =& \quad \lambda b. \lambda s.~ (\lambda b. \lambda s. M_1) ~(\lambda s. M_2\subst{b}{(\lambda s. M_3)}) ~s & (\beta) \\
    \\
    =& \quad \lambda b. \lambda s.~ M_1\subst{b}{(\lambda s. M_2\subst{b}{(\lambda s. M_3)})} & (\beta) \\
  \end{align*}
  The right-hand side simplifies to the same term as follows:
  \begin{align*}
    &\quad E\den{O_1;(O_2;O_3)} \\
    =& \quad \lambda b. \lambda s.~ E\den{O_1} ~(\den{O_2;O_3} ~b) ~s \\
    =& \quad \lambda b. \lambda s.~ E\den{O_1} ~((\lambda b. \lambda s.~ E\den{O_2} ~(E\den{O_3} ~b) ~s) ~b) ~s \\
    =& \quad \lambda b. \lambda s.~ E\den{O_1} ~(\lambda s.~ E\den{O_2} ~(E\den{O_3} ~b) ~s) ~s & (\beta)\\
    =& \quad \lambda b. \lambda s.~ (\lambda b. \lambda s. M_1) ~(\lambda s.~ (\lambda b. \lambda s. M_2) ~((\lambda b. \lambda s. M_3) ~b) ~s) ~s & (\cref{l-value}) \\
    =& \quad \lambda b. \lambda s.~ (\lambda b. \lambda s. M_1) ~(\lambda s.~ (\lambda b. \lambda s. M_2) ~(\lambda s. M_3) ~s) ~s & (\beta) \\
    =& \quad \lambda b. \lambda s.~ (\lambda b. \lambda s. M_1) ~(\lambda s. M_2\subst{b}{(\lambda s. M_3)}) ~s & (\beta) \\
    =& \quad \lambda b. \lambda s.~ M_1\subst{b}{(\lambda s. M_2\subst{b}{(\lambda s. M_3)})} & (\beta)
  \end{align*}
  \qed
\end{proof}

\begin{lemma}[Template Composition Associativity]
  \label{thm:templ-compose-assoc}

  $ T\den{(O_1;O_2);B} = T\den{O_1;(O_2;B)}.$
\end{lemma}
\begin{proof}
  The left-hand side simplifies as follows:
  \begin{align*}
    &\quad T\den{(O_1;O_2);B} \\
    =& \quad \lambda s.~ E\den{O_1;O_2} ~T\den{B} ~s \\
    =& \quad \lambda s.~ (\lambda b. \lambda s.~ E\den{O_1} ~(E\den{O_2} ~b) ~s) ~T\den{B} ~s \\
    =& \quad \lambda s.~ E\den{O_1} ~(E\den{O_2} ~T\den{B}) ~s & (\beta, \cref{thm:value-translation}) \\
    =& \quad \lambda s.~ (\lambda b.\lambda s. M_1) ~((\lambda b. \lambda s. M_2) ~T\den{B}) ~s & (\cref{l-value}) \\
    =& \quad \lambda s.~ (\lambda b.\lambda s. M_1) ~(\lambda s. M_2\subst{b}{T\den{B}}) ~s & (\beta, \cref{thm:value-translation}) \\
    =& \quad \lambda s.~ M_1\subst{b}{(\lambda s. M_2\subst{b}{T\den{B}})} & (\beta) \\
  \end{align*}
  The right-hand side simplifies to the same term as follows:
  \begin{align*}
    &\quad T\den{O_1;(O_2;B)} \\
    =& \quad \lambda s.~ E\den{O_1} ~T\den{O_2;B} ~s \\
    =& \quad \lambda s.~ E\den{O_1} ~(\lambda s. ~E\den{O_2} ~T\den{B} ~s) ~s \\
    =& \quad \lambda s.~ (\lambda b. \lambda s. M_1) ~(\lambda s. ~(\lambda b. \lambda s.~ M_2) ~T\den{B} ~s) ~s & (\cref{l-value}) \\
    =& \quad \lambda s.~ (\lambda b. \lambda s. M_1) ~(\lambda s. M_2\subst{b}{T\den{B}}) ~s & (\beta, \cref{thm:value-translation}) \\
    =& \quad \lambda s.~ M_1\subst{b}{(\lambda s. M_2\subst{b}{T\den{B}})} & (\beta) \\
  \end{align*}
  \qed
\end{proof}

\begin{lemma}[Extension Commit]
  \label{thm:ext-commit}
  $ E\den{\Do M; O} = E\den{\Do M}.$
\end{lemma}
\begin{proof}
  \begin{align*}
    &\quad
    E\den{\Do M; O}
    \\
    =& \quad
    \lambda b. \lambda s.~ E\den{\Do M} ~(E\den{O} ~b) ~s
    \\
    =& \quad
    \lambda b. \lambda s.~ E\den{\Do M} ~((\lambda b. \lambda s.~ M_O) ~b) ~s
    & (\cref{l-value})
    \\
    =& \quad
    \lambda b. \lambda s.~ E\den{\Do M} ~(\lambda s.~ M_O) ~s
    & (\beta)
    \\
    =& \quad
    \lambda b. \lambda s.~ E\den{\Try \_ \to \Continue \_ \to M} ~(\lambda s.~ M_O) ~s
    \\
    =& \quad
    \lambda b. \lambda s.~ (\lambda \_. ~T\den{\Continue \_ \to M}) ~(\lambda s.~ M_O) ~s
    \\
    =& \quad
    \lambda \_. \lambda s.~ T\den{\Continue \_ \to M} ~s
    & (\beta, b \notin FV(M))
    \\
    =& \quad
    \lambda \_. \lambda s.~ (\lambda \_. ~ \den{M}) ~s
    \\
    =& \quad
    \lambda \_. \lambda \_.~ \den{M}
    & (\beta, s \notin FV(M))
    \\
    =& \quad
    \lambda \_. ~T\den{\Continue \_ \to M}
    \\
    =&\quad
    E\den{\Try \_ \to \Continue \_ \to M}
    \\
    =&\quad
    E\den{\Do M}
  \end{align*}
  \qed
\end{proof}

\begin{lemma}[Template Commit]
  \label{thm:template-commit}
  $ T\den{\Do M; B} = T\den{\Else M}.$
\end{lemma}
\begin{proof}
  \begin{align*}
    &\quad
    T\den{\Do M; B}
    \\
    =& \quad
    \lambda s. ~ E\den{\Do M} ~T\den{B} ~s
    \\
    =& \quad
    \lambda s. ~ E\den{\Try \_ \to \Continue \_ \to M} ~T\den{B} ~s
    \\
    =& \quad
    \lambda s. ~ (\lambda \_. ~T\den{\Continue \_ \to M}) ~T\den{B} ~s
    \\
    =& \quad
    \lambda s. ~ (\lambda \_. ~\lambda \_. ~  \den{M}) ~T\den{B} ~s
    \\
    =& \quad
    \lambda s. ~ (\lambda \_. ~  \den{M}) ~s
    & (\beta)
    \\
    =& \quad
    \lambda \_. ~ \den{M}
    & (\beta, s \notin FV(M))
    \\
    =& \quad
    T\den{\Continue \_ \to M}
    \\
    =& \quad
    T\den{\Else M}
  \end{align*}
  \qed
\end{proof}

\begin{lemma}[$\eta\lamstar$]
  \label{thm:eta-lamstar}
  $E\den{\lambda x.~ (\lamstar (F; B)) ~ x} = E\den{\lamstar (F; B)}$
\end{lemma}
\begin{proof}
  From \cref{thm:value-translation}, we have
  $E\den{F} = \lambda b.\lambda s. \lambda z. M$ for some term $M$.  In the
  following, let
  $M' = M\subst{b}{T\den{B}}\subst{s}{(\lambda y.\mathit{self}~y)}$,
  \begin{align*}
    &\quad
    E\den{\lambda x.~ (\lamstar (F; B)) ~ x}
    \\
    =&\quad
    \lambda x. E\den{(\lamstar (F; B))} ~ x
    \\
    =&\quad
    \lambda x.
    (\Rec \mathit{self}
    = (\lambda s. E\den{F} ~ T\den{B} ~ s)
    ~ (\lambda y. \mathit{self}~y))
    ~ x
    \\
    =&\quad
    \lambda x.
    (\Rec \mathit{self} = E\den{F} ~ T\den{B} ~ (\lambda y. \mathit{self}~y))
    ~ x
    &(\beta)
    \\
    =&\quad
    \lambda x.
    (\Rec \mathit{self}
    = (\lambda b.\lambda s.\lambda z. M) ~ T\den{B}
    ~ (\lambda y. \mathit{self}~y))
    ~ x
    &(\cref{thm:value-translation})
    \\
    =&\quad
    \lambda x.
    (\Rec \mathit{self}
    = \lambda z. M\subst{b}{T\den{B}}\subst{s}{(\lambda y.\mathit{self}~y)})
    ~ x
    \\
    =&\quad
    \lambda x.
    (\Rec \mathit{self} = \lambda z. M')
    ~ x
    &(\beta)
    \\
    =&\quad
    \lambda x.
    (\lambda z.
    M'
    \subst{\mathit{self}}{\Rec \mathit{self} = \lambda z. M'})
    ~ x
    &(rec)
    \\
    =&\quad
    \lambda x.
    M'
    \subst{\mathit{self}}{\Rec \mathit{self} = \lambda z. M'}
    \subst{z}{x}
    &(\beta)
    \\
    =&\quad
    \lambda z.
    M'
    \subst{\mathit{self}}{\Rec \mathit{self} = \lambda z. M'}
    &(\alpha)
    \\
    =&\quad
    \Rec \mathit{self} = \lambda z. M'
    &(rec)
    \\
    =&\quad
    \Rec \mathit{self}
    = \lambda z. M\subst{b}{T\den{B}}\subst{s}{(\lambda y.\mathit{self}~y)}
    \\
    =&\quad
    \Rec \mathit{self}
    = (\lambda b. \lambda s. \lambda z. M) ~ T\den{B}
    ~ (\lambda x. \mathit{self} ~ x)
    &(\beta)
    \\
    =&\quad
    \Rec \mathit{self} = E\den{F} ~ T\den{B} ~ (\lambda x. \mathit{self} ~ x)
    &(\cref{thm:value-translation})
    \\
    =&\quad
    \Rec \mathit{self}
    = (\lambda s. E\den{F} ~ T\den{B} ~ s)
    ~ (\lambda x. \mathit{self} ~ x)
    &(\beta)
    \\
    =&\quad
    \Rec \mathit{self} = T\den{F; B} ~ (\lambda x. \mathit{self} ~ x)
    \\
    =&\quad
    E\den{\lamstar(F; B)}
  \end{align*}
  \qed
\end{proof}

\begin{lemma}[Unfold $\lamstar$]
  \label{thm:unfold-lamstar}
    $ E\den{ \lamstar (F; B)} = \den{(\Template F; B) ~ (\lamstar (F; B))}.$
\end{lemma}
\begin{proof}
  Note,
  \begin{math}
    T\den{F; B} ~ (\lambda x. \mathit{self} ~ x)
    =
    E\den{F} ~ T\den{B} ~ (\lambda x. \mathit{self} ~ x)    
  \end{math}
  is $\beta$-equal to some value of the form $\lambda z. M'$ because
  $T\den{F; B} = \lambda b. \lambda s. \lambda z. M$ from
  \cref{thm:value-translation}.  So
  $\Rec \mathit{self} = T\den{F; B} ~ (\lambda x. \mathit{self} ~ x)$ unfolds
  via $\beta$ and $rec$ in the following,
  \begin{align*}
    &\quad
    E\den{ \lamstar (F; B)}
    \\
    =& \quad
    (\Rec \mathit{self} = T\den{F; B} ~ (\lambda x. \mathit{self} ~ x))
    \\
    =& \quad
    T\den{F; B}
    ~
    (\lambda x.
    (\Rec \mathit{self} = T\den{F; B} ~ (\lambda x. \mathit{self} ~ x))
    ~ x)
    )
    & (\beta, rec)
    \\
    =& \quad
    T\den{F; B} ~ (\lambda x. \den{\lamstar(F; B)} ~ x)
    \\
    =& \quad
    T\den{F; B} ~ \den{\lamstar(F; B)}
    & (\cref{thm:eta-lamstar})
    \\
    =& \quad
    \den{\Template (F; B)} ~ \den{\lamstar (F; B)}
    \\
    =& \quad
    \den{(\Template (F; B)) ~ (\lamstar (F; B))}
  \end{align*}
  \qed
\end{proof}

\begin{lemma}[Template Extension]
  \label{thm:templ-ext}

  $ \den{(\Template O; B) ~ V} = \den{(\Extension O) ~ (\Template B) ~ V}.$
\end{lemma}
\begin{proof}
  \begin{align*}
    &\quad \den{(\Template O; B) ~ V} \\
    =& \quad \den{(\Template O; B)} ~ \den{V} \\
    =& \quad (\lambda s. ~E\den{O} ~T\den{B} ~s) ~ \den{V} \\
    =& \quad E\den{O} ~T\den{B} ~ \den{V} & (\beta, \cref{thm:value-translation}) \\
    =& \quad E\den{\Extension O} ~T\den{\Template B} ~ \den{V} \\
    =& \quad E\den{(\Extension O) ~ (\Template B) ~ V}
  \end{align*}
  \qed
\end{proof}

\begin{lemma}[Template Failure]
  \label{thm:templ-fail}
  $ \den{(\Template \varepsilon) ~ V} = \den{\mathit{fail}~V}.$
\end{lemma}
    \begin{proof}
        \begin{align*}
            &\quad \den{(\Template \varepsilon) ~ V} \\
            =& \quad \den{(\Template \varepsilon)} ~ \den{V}\\
            =& \quad T\den{\varepsilon} ~ \den{V} \\
            =& \quad (\lambda s. \mathit{fail} ~ s) ~ \den{V} \\
            =& \quad \mathit{fail} ~ \den{V} &(\beta, \cref{thm:value-translation}) \\
            =& \quad \den{\mathit{fail} ~ V}
        \end{align*}
        \qed
    \end{proof}

\begin{lemma}[Template Continue]
  \label{thm:templ-continue}
  
  $ \den{(\Template \Continue x \to M) ~ V} = \den{M\subst{x}{V}}.$
\end{lemma}
\begin{proof}
  \begin{align*}
    &\quad \den{(\Template \Continue x \to M) ~ V} \\
    =& \quad T\den{(\Continue x \to M)} ~ \den{V}\\
    =& \quad (\lambda x. ~\den{M}) ~ \den{V} \\
    =& \quad \den{M}\subst{x}{\den{V}} & (\beta, \cref{thm:value-translation}) \\
    =& \quad \den{M\subst{x}{V}} &(\cref{thm:substitution-translation})
  \end{align*}
  \qed
\end{proof}

\begin{lemma}[Extension Try]
  \label{thm:ext-try}
  
  $ \den{(\Extension \Try x \to B) ~ V} = \Template B\subst{x}{V}.$
\end{lemma}
\begin{proof}
  \begin{align*}
    &\quad \den{(\Extension \Try x \to B) ~ V}  \\
    =& \quad \den{(\Extension \Try x \to B)} ~ \den{V}\\
    =& \quad E\den{(\Try x \to B)} ~ \den{V}\\
    =& \quad (\lambda x. ~T\den{B}) ~ \den{V}\\
    =& \quad T\den{B}\subst{x}{\den{V}} &(\beta, \cref{thm:value-translation}) \\
    =& \quad T\den{B\subst{x}{V}} &(\cref{thm:substitution-translation}) \\
    =& \quad \den{\Template B\subst{x}{V}}
  \end{align*}
  \qed
\end{proof}

\begin{lemma}[Try Match]
  \label{thm:try-match}

  $\den{\Match P \gets V ~ O} = \den{O\subst{\many{x}}{\many{W}}}
  \qquad (\text{if } P\subst{\many{x}}{\many{W}} = V).$
\end{lemma}
\begin{proof}
  By \cref{thm:substitution-translation,thm:pattern-translation},
  \begin{math}
    \den{V}
    =
    \den{P\subst{\many{x}}{\many{W}}}
    =
    \den{P}\subst{\many{x}}{\many{\den{W}}}
    =
    P\subst{\many{x}}{\many{\den{W}}}
  \end{math}
  in the following:
  \begin{align*}
    &\quad \den{\Match P \gets V ~ O}  \\
    =&\quad \lambda b. \lambda s. \Match \den{V} \With \set{P \to E\den{O}~b~s; \_ \to b~s} \\
    =&\quad  \lambda b. \lambda s.~E\den{O}\subst{\many{x}}{\many{\den{W}}}~b~s  & (match, \cref{thm:substitution-translation,thm:pattern-translation}) \\
    =& \quad \lambda b. \lambda s.~E\den{O\subst{\many{x}}{\many{W}}}~b~s & (\cref{thm:substitution-translation})
    \\
    =& \quad E\den{O\subst{\many{x}}{\many{W}}} & (\alpha, \beta, \cref{thm:value-translation})
  \end{align*}
  \qed
\end{proof}

\begin{lemma}[Try Match Apart]
  \label{thm:try-match-apart}

  $\den{\Match P \gets V ~ O} = \den{\varepsilon}
  \qquad (\text{if } P \apart V).$
\end{lemma}
\begin{proof}
  By \cref{thm:apartness-translation}, $P \apart \den{V}$ in the following:
  \begin{align*}
    &\quad \den{\Match P \gets V ~ O}  \\
    =&\quad \lambda b. \lambda s. \Match \den{V} \With \set{P \to E\den{O}~b~s; \_ \to b~s} \\
    =&\quad  \lambda b. \lambda s.~b~s  & (apart, \cref{thm:apartness-translation}) \\
    =& \quad E\den{\varepsilon}
  \end{align*}
  \qed
\end{proof}

\begin{lemma}[Template Match]
  \label{thm:template-match}
  \begin{align*}
    \den{\Template{} (\lambda P. O; B) ~ V' ~ V}
    &=
    \denbig{
      \left(
        \begin{aligned}
          &\Template{} \\
          &\quad O\subst{\many{x}}{\many{W}}; \\
          &\quad \Continue s' \to (\Template B) ~ s' ~ V
        \end{aligned}
      \right)
      ~ V'
    }
    \\
    &
    (\text{if } P\subst{\many{x}}{\many{W}} = V)
    .
  \end{align*}
\end{lemma}
\begin{proof}
  By cases on whether $P$ is a variable or not:
  \begin{itemize}
  \item If $P = y$, then $P\subst{\many{x}}{\many{W}} = y\subst{y}{V}$ with $\many{x} = y$ and $\many{W}=V$ and:
  \begin{align*}
    &\quad
    \den{(\Template{} (\lambda y. O); B) ~ V' ~ V}
    \\
    =&\quad (\lambda s. E\den{\lambda y. O} ~T\den{B} ~s) ~\den{V'} ~\den{V}
    \\
    =&\quad
    E\den{\lambda y. \Do M} ~T\den{B} ~\den{V'} ~\den{V}
    & (\beta, \cref{thm:value-translation})
    \\
    =&\quad
    (\lambda b. \lambda s. \lambda y. E\den{O} ~(\lambda s'. b ~s' ~y) ~s) ~T\den{B} ~\den{V'} ~\den{V}
    \\
    =&\quad
    E\den{O}\subst{y}{\den{V}} ~(\lambda s'. T\den{B} ~s' ~\den{V}) ~\den{V'}
    & (\beta, \cref{thm:value-translation})
    \\
    =&\quad
    E\den{O\subst{y}{V}} ~ (\lambda s'. T\den{B} ~ s' ~ \den{V}) ~ \den{V'}
    &(\cref{thm:substitution-translation})
    \\
    =&\quad
    \denbig{
      \left(
        \begin{aligned}
          \Template{}
          & O\subst{y}{V}; \\
          &\Continue s' \to (\Template B) ~ s' ~ V
        \end{aligned}
      \right)
      ~ V'
    }
    & (\beta, \cref{thm:value-translation})
  \end{align*}
  \item Otherwise, by \cref{thm:substitution-translation,thm:pattern-translation},
  \begin{math}
    \den{V}
    =
    \den{P\subst{\many{x}}{\many{W}}}
    =
    \den{P}\subst{\many{x}}{\many{\den{W}}}
    =
    P\subst{\many{x}}{\many{\den{W}}}
  \end{math}
  in the following:
  \begin{align*}
    &\quad
    \den{(\Template{} (\lambda P. O); B) ~ V' ~ V}
    \\
    =&\quad
    (\lambda s. E\den{\lambda P. O} ~T\den{B} ~s) ~\den{V'} ~\den{V}
    \\
    =&\quad
    E\den{\lambda P. O} ~T\den{B} ~\den{V'} ~\den{V}
    & (\beta,\cref{thm:value-translation})
    \\
    =&\quad
    E\den{\lambda y. \Match P \gets y ~ O} ~T\den{B} ~\den{V'} ~\den{V}
    \\
    =&\quad
    E\den{\Match P \gets y ~ O}\subst{y}{\den{V}} ~(\lambda s'. T\den{B} ~s' ~\den{V}) ~\den{V'}
    & (\beta, \cref{thm:value-translation})
    \\
    =&\quad
    \begin{aligned}[t]
      &\Match \den{V} \With{} \{ \\
      &\quad P \to E\den{O}~(\lambda s'. T\den{B} ~s' ~\den{V})~\den{V'}; \\
      &\quad\_ \to (\lambda s'. T\den{B} ~s' ~\den{V}) ~ \den{V'}
      ~\}
    \end{aligned}
    &(\beta, \cref{thm:value-translation})
    \\
    =&\quad E\den{O}\subst{\many{x}}{\many{\den{W}}} ~(\lambda s'. T\den{B} ~s' ~\den{V}) ~\den{V'} & (match)
    \\
    =&\quad
    (E\den{O\subst{\many{x}}{\many{W}}} ~ (\lambda s'. T\den{B} ~ s' ~ \den{V})) ~ \den{V'}
    &(\cref{thm:substitution-translation})
    \\
    =&\quad
    \denbig{
      \left(
        \begin{aligned}
          &\Template{} \\
          &\quad O\subst{\many{x}}{\many{W}}; \\
          &\quad \Continue s' \to (\Template B) ~ s' ~ V
        \end{aligned}
      \right)
      ~ V'
    }
    &(\beta, \cref{thm:value-translation})
  \end{align*}
  \qed
  \end{itemize}
\end{proof}

\begin{lemma}[Template Do Match]
  \label{thm:template-do-match}

  $\den{(\Template{} (\lambda P. \Do M); B) ~ V' ~ V} = \den{M\subst{\many{x}}{\many{W}}}
  \quad (\text{if } P\subst{\many{x}}{\many{W}} = V)$.
\end{lemma}
\begin{proof}
  In terms of the source-level axioms above, we have
  \begin{align*}
    &\quad
    (\Template{} (\lambda P. \Do M); B) ~ V' ~ V
    \\
    =&\quad
    \left(
      \begin{aligned}
        &\Template \\
        &\quad \Do M\subst{\many{x}}{\many{W}}; \\
        &\quad \Continue s' \to (\Template B) ~ s' ~ V
      \end{aligned}
    \right)
    ~ V'
    &(\cref{thm:template-match})
    \\
    =&\quad
    \begin{aligned}[t]
      &(\Extension \Do M\subst{\many{x}}{\many{W}}) \\
      &\quad (\Template \Continue s' \to (\Template B) ~ s' ~ V) \\
      &\quad V'
    \end{aligned}
    &(\cref{thm:templ-ext})
    \\
    =&\quad
    \begin{aligned}[t]
      &(\Extension \Try \_ \to \Else M\subst{\many{x}}{\many{W}}) \\
      &\quad (\Template \Continue s' \to (\Template B) ~ s' ~ V) \\
      &\quad V'
    \end{aligned}
    \\
    =&\quad
    (\Template \Else M\subst{\many{x}}{\many{W}})
    ~ V'
    &(\cref{thm:ext-try})
    \\
    =&\quad
    (\Template \Continue \_ \to  M\subst{\many{x}}{\many{W}})
    ~ V'
    \\
    =&\quad
    M\subst{\many{x}}{\many{W}}    
    &(\cref{thm:templ-continue})
  \end{align*}
  and so the translation of the two sides are equal by congruence and transitivity of the equational theory (\cref{thm:conservative-extension}).
  \qed
  % By cases on whether $P$ is a variable or not:
  % \begin{itemize}
  % \item If $P = y$, then $P\subst{\many{x}}{\many{W}} = y\subst{y}{V}$ with $\many{x} = y$ and $\many{W}=V$ and:
  % \begin{align*}
  %   &\quad \den{(\Template{} (\lambda P. \Do M); B) ~ V' ~ V} \\
  %   =&\quad \den{(\Template{} (\lambda y. \Do M); B) ~ V' ~ V}\\
  %   =&\quad (\lambda s. E\den{\lambda y. \Do M} ~T\den{B} ~s) ~\den{V'} ~\den{V}\\
  %   =&\quad E\den{\lambda y. \Do M} ~T\den{B} ~\den{V'} ~\den{V} & (\beta, \cref{thm:value-translation})\\
  %   =&\quad (\lambda b. \lambda s. \lambda y. E\den{\Do M} ~(\lambda s'. b ~s' ~y) ~s) ~T\den{B} ~\den{V'} ~\den{V}\\
  %   =&\quad E\den{\Do M}\subst{y}{\den{V}} ~(\lambda s'. T\den{B} ~s' ~\den{V}) ~\den{V'} & (\beta, \cref{thm:value-translation})\\
  %   =&\quad (\lambda \_. \lambda  \_. \den{M}\subst{y}{\den{V}}) ~(\lambda s'. T\den{B} ~s' ~\den{V}) ~\den{V'}\\
  %   =&\quad \den{M}\subst{y}{\den{V}} & (\beta, \cref{thm:value-translation}) \\
  %   =&\quad \den{M\subst{y}{V}} & (\cref{thm:substitution-translation}) \\
  %   =&\quad \den{M\subst{\many{x}}{\many{W}}}
  % \end{align*}
  % \item Otherwise, by \cref{thm:substitution-translation,thm:pattern-translation},
  % \begin{math}
  %   \den{V}
  %   =
  %   \den{P\subst{\many{x}}{\many{W}}}
  %   =
  %   \den{P}\subst{\many{x}}{\many{\den{W}}}
  %   =
  %   P\subst{\many{x}}{\many{\den{W}}}
  % \end{math}
  % in the following:
  % \begin{align*}
  %   &\quad \den{(\Template{} (\lambda P. \Do M); B) ~ V' ~ V}\\
  %   =&\quad (\lambda s. E\den{\lambda P. \Do M} ~T\den{B} ~s) ~\den{V'} ~\den{V}\\
  %   =&\quad E\den{\lambda P. \Do M} ~T\den{B} ~\den{V'} ~\den{V} & (\beta,\cref{thm:value-translation})\\
  %   =&\quad E\den{\lambda y. \Match P \gets y ~\Do M} ~T\den{B} ~\den{V'} ~\den{V} \\
  %   % =&\quad (\lambda b. \lambda s. \lambda y. E\den{\Match P \gets y ~\Do M} ~(\lambda s'. b ~s' ~x) ~s) ~T\den{B} ~\den{V'} ~\den{V} \\
  %   =&\quad E\den{\Match P \gets y ~\Do M}\subst{y}{\den{V}} ~(\lambda s'. T\den{B} ~s' ~\den{V}) ~\den{V'} & (\beta, \cref{thm:value-translation}) \\
  %   =&\quad
  %   \begin{aligned}[t]
  %     &\Match \den{V} \With{} \{ \\
  %     &\quad P \to \den{\Do M}~(\lambda s'. T\den{B} ~s' ~\den{V})~\den{V'}; \\
  %     &\quad\_ \to (\lambda s'. T\den{B} ~s' ~\den{V}) ~ \den{V'}
  %     ~\}
  %   \end{aligned}
  %   &(\beta, \cref{thm:value-translation}) \\
  %   =&\quad E\den{\Do M}\subst{\many{x}}{\many{\den{W}}} ~(\lambda s'. T\den{B} ~s' ~\den{V}) ~\den{V'} & (match) \\
  %   =&\quad (\lambda \_. \lambda  \_. \den{M}\subst{\many{x}}{\many{\den{W}}}) ~(\lambda s'. T\den{B} ~s' ~\den{V}) ~\den{V'} & \\
  %   =&\quad \den{M}\subst{\many{x}}{\many{\den{W}}} & (\beta, \cref{thm:value-translation}) \\
  %   =&\quad \den{M\subst{\many{x}}{\many{W}}} & (\cref{thm:substitution-translation}) \\
  % \end{align*}
  % \qed
  % \end{itemize}
\end{proof}

\begin{lemma}[Template Apart]
  \label{thm:template-apart}

  $\den{(\Template{} (\lambda P. O); B) ~ V' ~ V} = \den{(\Template B) ~ V' ~ V}
  \qquad (\text{if } P \apart V)$.
\end{lemma}
\begin{proof}
  Since a variable can never be apart from a value, $P$ must be a non-variable pattern.
  By \cref{thm:apartness-translation}, $P \apart \den{V}$, and the translations are then equal as follows:
  \begin{align*}
    & \quad \den{(\Template{} (\lambda P. O); B) ~ V' ~ V} \\
    =& \quad E\den{\lambda P. O} ~T\den{B} ~\den{V'} ~\den{V} & (\beta,\cref{thm:value-translation}) \\
    =&\quad E\den{\lambda y. \Match P \gets y ~\Do M} ~T\den{B} ~\den{V'} ~\den{V} \\
    =&\quad E\den{\Match P \gets y ~\Do M}\subst{y}{\den{V}} ~(\lambda s'. T\den{B} ~s' ~\den{V}) ~\den{V'} & (\beta, \cref{thm:value-translation}) \\
    =& \quad
    \begin{aligned}[t]
      &\Match \den{V} \With{} \{ \\
      &\quad P \to \den{O}~(\lambda s'. T\den{B} ~s' ~\den{V})~\den{V'}; \\
      &\quad \_ \to (\lambda s'. T\den{B} ~s' ~\den{V})~\den{V'}
      ~\}
    \end{aligned}
    & (\beta, \cref{thm:value-translation}) \\
    =& \quad (\lambda s'. T\den{B} ~s' ~\den{V}) ~\den{V'} & (apart)\\
    =& \quad T\den{B} ~ \den{V'} ~ \den{V} & (\beta) \\
    =& \quad \den{(\Template B)} ~ \den{V'} ~ \den{V} \\
    =& \quad \den{(\Template B) ~ V' V}
  \end{align*}
  \qed
\end{proof}

\begin{lemma}[Template Self]
  \label{thm:template-self}
  
  $\den{(\Template{} (y ~ O); B) ~ V = (\Template O\subst{y}{V}; B) ~ V}$
\end{lemma}
\begin{proof}
  \begin{align*}
    &\quad
    \den{C[(\Template{} (y ~ O); B) ~ V]}
    \\
    =&\quad
    E\den{y ~ O} ~ T\den{B} ~ \den{V}
    &(\beta, \cref{thm:value-translation})
    \\
    =&\quad
    (\lambda b. \lambda y. E\den{O} ~ b ~ y) ~ T\den{B} ~ \den{V}
    \\
    =&\quad
    E\den{O}\subst{y}{\den{V}} ~ T\den{B} ~ \den{V}
    &(\beta, \cref{thm:value-translation})
    \\
    =&\quad
    E\den{O\subst{y}{V}} ~ T\den{B} ~ \den{V}
    &(\cref{thm:substitution-translation})
    \\
    =&\quad
    \den{(\Template O\subst{y}{V}; B) ~ V}
    &(\beta, \cref{thm:value-translation})
  \end{align*}
  \qed
\end{proof}

\begin{lemma}[Context Match]
  \label{thm:context-match}
  \begin{align*}
    \den{C[(\Template{} (Q[y] ~ O); B) ~ V]}
    &=
    \denbig{
      \begin{aligned}
        &\Template \\
        &\quad O\subst{y}{V}\subst{\many{x}}{\many{W}}; \\
        &\quad \Continue s' \to C[(\Template B) ~ s']
      \end{aligned}
    }
    \\
    &(\text{if } Q\subst{\many{x}}{\many{W}} = C \neq \hole)
    .
  \end{align*}
\end{lemma}
\begin{proof}
  By induction on the syntax of $Q$.
  \begin{itemize}
  \item $Q = \hole$ is impossible due to the assumption $Q\subst{\many{x}}{\many{W}} = C \neq \hole$.
  \item If $Q = \hole ~ P$ then $C = \hole ~ V'$ such that $P\subst{\many{x}}{\many{W}} = V$, and the equality holds by \cref{thm:template-self,thm:template-match} and congruence of the equational theory (\cref{thm:conservative-extension}):
    \begin{align*}
      &\quad
      \den{C[(\Template{} (Q[y] ~ O); B) ~ V]}
      \\
      =&\quad
      \den{(\Template{} ((y ~ P) ~ O); B) ~ V ~ V'}
      \\
      =&\quad
      \den{(\Template{} (y ~ (\lambda P. O)); B) ~ V ~ V'}
      \\
      =&\quad
      \den{(\Template{} (\lambda P. O\subst{y}{V}); B) ~ V ~ V'}
      &(\cref{thm:template-self})
      \\
      =&\quad
      \denbig{
        \left(
          \begin{aligned}
            &\Template \\
            &\quad O\subst{y}{V}\subst{\many{x}}{\many{W}}; \\
            &\quad \Continue s' \to (\Template B) ~ s' ~ V'
          \end{aligned}
        \right)
        ~ V
      }
      &(\cref{thm:template-match})
    \end{align*}
  \item If $Q = Q' ~ P$ then $C = C' ~ V'$ where $\subst{\many{x}}{\many{W}} = \subst{\many{x_1},\many{x_2}}{\many{W_1},\many{W_2}}$ such that $Q'\subst{\many{x_1}}{\many{W_2}} = C$ and $P\subst{\many{x_2}}{\many{W_2}} = V'$.
    Assume the inductive hypothesis
    \begin{align*}
      \den{C'[(\Template{} (Q'[y] ~ O); B) ~ V]}
      &=
      \denbig{
        \begin{aligned}
          &\Template \\
          &\quad O\subst{y}{V}\subst{\many{x}}{\many{W}}; \\
          &\quad \Continue s' \to C'[(\Template B) ~ s']
        \end{aligned}
      }
    \end{align*}
    The equality then holds by this inductive hypothesis, \cref{thm:template-match,thm:templ-continue}, and congruence of the equational theory (\cref{thm:conservative-extension}):
    \begin{align*}
      &\quad
      \den{C[(\Template{} (Q[y] ~ O); B) ~ V]}
      \\
      =&\quad
      \den{C'[(\Template{} ((Q'[y] ~ P) ~ O); B) ~ V] ~ V'}
      \\
      =&\quad
      \den{C'[(\Template{} (Q'[y] ~ (\lambda P. O)); B) ~ V] ~ V'}
      \\
      =&\quad
      \denbig{
        \left(
          \begin{aligned}
            &\Template \\
            &\quad \lambda P. O\subst{y}{V}\subst{\many{x_1}}{\many{W_1}}; \\
            &\quad \Continue s' \to C'[(\Template B) ~ s']
          \end{aligned}
        \right)
        ~ V ~ V'
      }
      &(IH)
      \\
      =&\quad
      \denbig{
        \left(
          \begin{aligned}
            &\Template \\
            &\quad O\subst{y}{V}\subst{\many{x_1}}{\many{W_1}}\subst{\many{x_2}}{\many{W_2}}; \\
            &\quad \Continue s'' \to \\
            &\qquad
            \left(
              \begin{aligned}
                &\Template \\
                &\quad \Continue s' \to \\
                &\qquad C'[(\Template B) ~ s']
              \end{aligned}
            \right)
            ~ s'' ~ V'
          \end{aligned}
        \right)
        ~ V
      }
      &(\cref{thm:template-match})
      \\
      =&\quad
      \denbig{
        \left(
          \begin{aligned}
            &\Template \\
            &\quad O\subst{y}{V}\subst{\many{x_1}}{\many{W_1}}\subst{\many{x_2}}{\many{W_2}}; \\
            &\quad \Continue s'' \to 
            \left(
              C'[(\Template B) ~ s'']
            \right)
            ~ V'
          \end{aligned}
        \right)
        ~ V
      }
      &(\cref{thm:templ-continue})
      \\
      =&\quad
      \denbig{
        \left(
          \begin{aligned}
            &\Template \\
            &\quad O\subst{y}{V}\subst{\many{x}}{\many{W}}; \\
            &\quad \Continue s' \to 
            C[(\Template B) ~ s']
          \end{aligned}
        \right)
        ~ V
      }
    \end{align*}
  \end{itemize}
\end{proof}
  
\begin{lemma}[Context Exact Match]
  \label{thm:context-exact-match}
  \begin{align*}
    \den{C[(\Template{} (Q[y] = M); B) ~ V]}
    &=
    \den{M\subst{y}{V}\subst{\many{x}}{\many{W}}}
    \\
    &(\text{if } Q\subst{\many{x}}{\many{W}} = C)
    .
  \end{align*}
\end{lemma}
\begin{proof}
  In terms of the source-level axioms above, we have:
  \begin{align*}
    &\quad
    C[(\Template{} (Q[y] = M); B) ~ V]
    \\
    =&\quad
    C[(\Template{} (Q[y] \Do M); B) ~ V]
    \\
    =&\quad
    \left(
      \begin{aligned}
        &\Template \\
        &\quad \Do M\subst{y}{V}\subst{\many{x}}{\many{W}}; \\
        &\quad \Continue s' \to C[(\Template B) ~ s']
      \end{aligned}
    \right)
    ~ V
    &(\cref{thm:context-match})
    \\
    =&\quad
    \begin{aligned}[t]
      &(\Extension \Do M\subst{y}{V}\subst{\many{x}}{\many{W}}) \\
      &\quad (\Template \Continue s' \to C[(\Template B) ~ s']) \\
      &\quad ~ V
    \end{aligned}
    &(\cref{thm:templ-ext})
    \\
    =&\quad
    \begin{aligned}[t]
      &(\Extension \Try \_ \to \Else M\subst{y}{V}\subst{\many{x}}{\many{W}}) \\
      &\quad (\Template \Continue s' \to C[(\Template B) ~ s']) \\
      &\quad ~ V
    \end{aligned}
    \\
    =&\quad
    (\Template \Else M\subst{y}{V}\subst{\many{x}}{\many{W}}) ~ V
    &(\cref{thm:ext-try})
    \\
    =&\quad
    (\Template \Continue \_ \to  M\subst{y}{V}\subst{\many{x}}{\many{W}}) ~ V
    \\
    =&\quad
    M\subst{y}{V}\subst{\many{x}}{\many{W}}
  \end{align*}
  and so the translation of the two sides are equal by congruence and transitivity of the equational theory (\cref{thm:conservative-extension}).
  \qed
  
  % First, note that every copattern has the form $Q = \hole ~ P_1 \dots P_n$, where each pattern $P_i$ binds a distinct set of variables $\many{x_i}$, and so the assumed matching equality $Q\subst{\many{x}}{\many{W}} = C$ implies
  % \begin{itemize}
  % \item $C = ~ V_1 \dots V_n$, where there are equal numbers of $V_i$ and $P_i$, and
  % \item for each $1 \leq i \leq n$, there are variables $\many{x_i}$ and values $\many{W_i}$ such that $P_i\subst{\many{x_i}}{\many{W_i}}$.
  % \end{itemize}

  % The equation then holds by induction on the copattern $Q = \hole ~ P_i \dots P_n$ and matching pairs $P_i\subst{\many{x_i}}{\many{W_i}} = V_i$as follows:
  % \begin{align*}
  %   &\quad
  %   \den{C[(\Template{} (Q[y] = M); B) ~ V]}
  %   \\
  %   =&\quad
  %   \den{(\Template{} (y ~ P_1 \dots P_n = M); B) ~ V ~ V_1 \dots V_n}
  %   \\
  %   =&\quad
  %   \den{(\Template{} (y ~ \lambda P_1. \dots \lambda P_n. \Do M); B) ~ V ~ V_1 \dots V_n}
  %   \\
  %   =&\quad
  %   \den{(\Template{} (\lambda P_1. \dots \lambda P_n. \Do M\subst{y}{V}); B) ~ V ~ V_1 \dots V_n}
  %   &(\cref{thm:template-self})
  %   \\
  %   =&\quad
  %   \denbig{
  %     \left(
  %       \begin{aligned}
  %         &\Template \\
  %         &\quad \Do M\subst{y}{V}\subst{\many{x}}{\many{W}}; \\
  %         &\quad \Continue s' \to (\Template B) ~ s' ~ V_1 \dots V_n
  %       \end{aligned}
  %     \right)
  %     ~ V
  %   }
  %   &(\cref{thm:template-match})
  %   \\
  %   =&\quad
  %   \denbig{
  %     \begin{aligned}
  %       &(\Extension \Do M\subst{y}{V}\subst{\many{x}}{\many{W}})\\
  %       &\quad (\Template \Continue s' \to (\Template B) ~ s' ~ V_1 \dots V_n) \\
  %       &\quad V
  %     \end{aligned}
  %   }
  %   &(\cref{thm:templ-ext})
  %   \\
  %   =&\quad
  %   \denbig{
  %     \begin{aligned}
  %       &(\Extension \Try \_ \to \Continue \_ \to M\subst{y}{V}\subst{\many{x}}{\many{W}})\\
  %       &\quad (\Template \Continue s' \to (\Template B) ~ s' ~ V_1 \dots V_n) \\
  %       &\quad V
  %     \end{aligned}
  %   }
  %   \\
  %   =&\quad
  %   \den{
  %     (\Template \Continue \_ \to M\subst{y}{V}\subst{\many{x}}{\many{W}}) ~ V
  %   }
  %   &(\cref{thm:ext-try})
  %   \\
  %   =&\quad
  %   \den{
  %     M\subst{y}{V}\subst{\many{x}}{\many{W}}
  %   }
  %   &(\cref{thm:templ-continue})
  % \end{align*}
\end{proof}

\begin{lemma}[Context Apart]
  \label{thm:context-apart}

  $\den{C[(\Template{} (Q[y] ~ O); B) ~ V]} = \den{C[(\Template B) ~ V]}
  \qquad (\text{if } Q \apart C)$
\end{lemma}
\begin{proof}
  % Note that derivation of apartness can be rewritten such that the base case, proving $\hole ~ P_1 \dots P_n \apart \hole ~ V_1 \dots V_n$ from $P_n \apart V_n$, is such that $P_n \apart V_n$ is the \emph{first} non-matching application, and each previous application (for $1 \leq i < n$) matches as some $P_i\subst{\many{x_i}}{\many{W_i}} = V_i$.
  % If there is an earlier non-matching application, then that can form a new smaller base case closer to $\hole$, and all following applications are added via the other two rules for copattern apartness.
  % The proof then proceeds by induction on the derivation of apartness $Q \apart C$ with the above invariant.
  By induction on the derivation of apartness $Q \apart C$:
  \begin{itemize}
  \item Suppose $Q \apart C$ because $Q = Q' ~ P$ and $C = C' ~ W$ such that $P \apart W$ and $Q'\subst{\many{x}}{\many{W}} = C'$.
    % The two sides are equal \dots by \cref{thm:template-match,thm:template-apart}.
    \begin{align*}
      & \quad \den{C'[(\Template{} ((Q'[y] ~P) ~ O); B) ~ V] ~W} \\
      =& \quad \den{C'[(\Template{} (Q'[y] ~(\lambda P. ~O)); B) ~ V] ~W} \\
      =& \quad \den{C'} ~\denbig{
        \left(
          \begin{aligned}
            &\Template{} \\
            &\quad (\lambda P. ~O)\subst{\many{x}}{\many{W}}; \\
            &\quad \Continue s' \to (\Template B) ~ s' ~ W
          \end{aligned}
        \right)
        ~ V
      } & (\cref{thm:compositional-translation,thm:template-match})\\
      =& \quad \den{C'} \den{\Continue s' \to (\Template B) ~ s' ~ W} ~\den{V} & (\cref{thm:template-apart}) \\
      =& \quad \den{C'} (\lambda s'. \den{(\Template B)} ~ s' ~\den{W}) ~\den{V} \\
      =& \quad \den{C'} \den{(\Template B) ~V ~W} & (\beta)\\
      =& \quad \den{C'[(\Template B) ~ V ~ W]} & (\cref{thm:compositional-translation}) \\
      =& \quad \den{C[(\Template B) ~ V]}
    \end{align*}

  \item Suppose $Q \apart C$ because $Q = Q' ~ P$ and $Q' \apart C$, and assume the inductive hypothesis
    \begin{math}
      \den{C[(\Template{} (Q'[y] ~ O); B) ~ V]} = \den{C[(\Template B) ~ V]}
      .
    \end{math}
    % The two sides are equal \dots by the inductive hypothesis.
    \begin{align*}
      & \quad \den{C[(\Template{} ((Q'[y] ~P) ~ O); B) ~ V]} \\
      =& \quad \den{C[(\Template{} (Q'[y] ~ (\lambda P. ~O)); B) ~ V]} \\
      =& \quad \den{C[(\Template{} (Q'[y] ~ (\lambda P. ~O)); B) ~ V]} & (IH)
    \end{align*}
  \item Suppose $Q \apart C$ because $C = C' ~ W$ and  $Q \apart C'$, and assume the inductive hypothesis
    \begin{math}
      \den{C'[(\Template{} (Q[y] ~ O); B) ~ V]} = \den{C'[(\Template B) ~ V]}
      .
    \end{math}
    % The two sides are equal \dots by the inductive hypothesis and congruence (\cref{thm:conservative-extension}).
    \begin{align*}
      & \quad \den{C[(\Template{} (Q[y] ~ O); B) ~ V]} \\
      =& \quad \den{C'[(\Template{} (Q[y] ~ (\lambda P. ~O)); B) ~ V]} ~\den{W}\\
      =& \quad \den{C'[(\Template B) ~ V]} ~\den{W} & (IH)\\
      =& \quad \den{C[(\Template B) ~ V]}
    \end{align*}
  \end{itemize}
  \qed
\end{proof}

\begin{lemma}[Context $\lamstar$ Match]
  \label{thm:context-lamstar-match}
  \begin{align*}
    \den{C[\lamstar (Q[y] = M); B]}
    &=
    \den{M\subst{y}{(\lamstar (Q[y] = M); B)}\subst{\many{x}}{\many{W}}}
    \\
    &(\text{if } Q\subst{\many{x}}{\many{W}} = C)
  \end{align*}
\end{lemma}
\begin{proof}
  In terms of the source-level axioms above, we have:
  \begin{align*}
    &\quad
    C[\lamstar (Q[y] = M); B]
    \\
    =&\quad
    C[(\Template{} (Q[y] = M); B)~(\lamstar (Q[y]=M); B)]
    &(\cref{thm:unfold-lamstar})
    \\
    =&\quad
    C[M\subst{y}{(\lamstar (Q[y]=M); B)}\subst{\many{x}}{\many{W}}]
    &(Q\subst{\many{x}}{\many{W}} = C, \cref{thm:context-exact-match})
  \end{align*}
  and so the translation of the two sides are equal by congruence and transitivity of the equational theory (\cref{thm:conservative-extension}).
  \qed
\end{proof}

\begin{lemma}[Context $\lamstar$ Apart]
  \label{thm:context-lamstar-apart}

  $\den{C[\lamstar (Q[y] ~ O); \Else M]} = \den{C[M]}
  \qquad (\text{if } Q \apart C)$.
\end{lemma}
\begin{proof}
  In terms of the source-level axioms above, we have:
  \begin{align*}
    &\quad
    C[\lamstar (Q[y] ~ O); \Else M]
    \\
    =&\quad
    C[(\Template{} (Q[y] ~ O); \Else M) ~ (\lamstar (Q[y] ~ O); \Else M)]
    &(\cref{thm:unfold-lamstar})
    \\
    =&\quad
    C[(\Template \Else M) ~ (\lamstar (Q[y] ~ O); \Else M)]
    &(Q \apart C, \cref{thm:context-apart})
    \\
    =&\quad
    C[(\Template \Continue \_ \to M) ~ (\lamstar (Q[y] ~ O); \Else M)]
    \\
    =&\quad
    C[M]
    &(\cref{thm:templ-continue})
  \end{align*}
  and so the translation of the two sides are equal by congruence and transitivity of the equational theory (\cref{thm:conservative-extension}).
  \qed
\end{proof}

%%% Local Variables:
%%% mode: LaTeX
%%% TeX-master: "coscheme"
%%% End:


\end{document}
