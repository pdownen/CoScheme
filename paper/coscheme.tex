% This is samplepaper.tex, a sample chapter demonstrating the
% LLNCS macro package for Springer Computer Science proceedings;
% Version 2.21 of 2022/01/12
%

\documentclass[runningheads]{llncs}
%
\usepackage[T1]{fontenc}
% T1 fonts will be used to generate the final print and online PDFs,
% so please use T1 fonts in your manuscript whenever possible.
% Other font encondings may result in incorrect characters.
%
\usepackage{graphicx}
% Used for displaying a sample figure. If possible, figure files should
% be included in EPS format.
\usepackage{xcolor}
\usepackage[shortlabels]{enumitem}

% llncs.cls clashes with amsthm.
% Save the LNCS proof environment defined by the class
\let\lncsproof\proof \let\lncsendproof\endproof \let\lncsqed\qed
% Remove the definitions in order to load amsthm
\let\proof\relax\let\endproof\relax
% Load AMS styles
\usepackage{amsmath}
\usepackage{amsthm}
\usepackage{amssymb}
% restore the LNCS class defined proof
\let\proof\lncsproof \let\endproof\lncsendproof \let\qed\lncsqed

\usepackage{stmaryrd}
\usepackage{braket}
\usepackage{proof}

\usepackage{minted}

\usepackage{hyperref}
\usepackage{cleveref}

\usepackage{preamble}

% If you use the hyperref package, please uncomment the following two lines
% to display URLs in blue roman font according to Springer's eBook style:
\usepackage{color}
\renewcommand\UrlFont{\color{blue}\rmfamily}
\urlstyle{rm}

\begin{document}
%
\title{CoScheme: Compositional Copatterns in Scheme}
%
%\titlerunning{Abbreviated paper title}
% If the paper title is too long for the running head, you can set
% an abbreviated paper title here
%
\author{Paul Downen\inst{1}\orcidID{0000-0003-0165-9387}}
%
\authorrunning{P. Downen}
% First names are abbreviated in the running head.
% If there are more than two authors, 'et al.' is used.
%
\institute{University of Massachusetts Lowell, Lowell MA 01854, USA \\
\email{Paul\_Downen@uml.edu}}
%
\maketitle              % typeset the header of the contribution
%
\begin{abstract}
The abstract should briefly summarize the contents of the paper in
150--250 words.

\keywords{Codata \and Copatterns \and Scheme \and Macros \and Composition \and Expression Problem.}
\end{abstract}
%
%
%
\section{Introduction} \label{sec-intro}

For decades, functional programmers have had a reliable and versatile method for representing tree-shaped structures: inductive data types.
These can model data of any size --- for example, lists of an arbitrary length --- but each instance must be \emph{finite}.
Infinite information --- like a stream of input that can go on forever --- does not fit into an inductive type, so the programmer must use some other representation to model potentially infinite objects.

Fortunately, the inductive data types used by functional programmers every day have a polar opposite: \emph{coinductive codata types}.
The \emph{coinductive} descriptor signals that values of the type may contain infinite information.
Haskell programmers are already well-versed in these types, since non-strict languages blur the line between induction and coinduction.
For example, consider the usual example of the infinite list of Fibonacci numbers in Haskell:
\begin{minted}{haskell}
fibs = 0 : 1 : zipWith (+) fib (tail fib)
\end{minted}
\hs|fibs| cannot be fully evaluated because it has no base case --- it would eventually expand out to \hs|0 : 1 : 1 : 2 : 3 : 5 : 8 : ...| forever --- but this is no problem in a non-strict language that only evaluates as much as needed.

In contrast, \emph{codata} describes types that are defined by primitive \emph{destructors} that \emph{use} values of the codata type, as opposed to the primitive constructors that define how to build values of a data type.
For example, the usual \agda|Stream a| codata type of infinite \agda|a|'s is defined by two destructors: \agda|Head : Stream a -> a| extracts the first element and \agda|Tail : Stream a -> Stream a| discards the first element and returns the rest.
To define new streams, we can describe how they react to different combinations of \agda|Head| and \agda|Tail| destructions using \emph{copatterns} \cite{APTS2013C}.  The copattern-based definition of the \agda|fibs| function above is:
\begin{minted}{agda}
fibs : Stream Nat
Head fibs = 0
Head (Tail fibs) = 1
Tail (Tail fibs) = zipWith _+_ fibs (Tail fibs)
\end{minted}
\adriano{This example is funny, it works but emits 3 warnings, two related to the termination checker and one suggestion to turn on a flag (--guardedness). However, if you turn this flag on those warnings become errors. After typesetting the proofs, I will try to look for more failing examples in Agda.}
Even though copatterns provide an elegant way to define codata, their adoption is not universal.
To illustrate, this previous definition does not fully satisfy Agda's termination checker.
Although we can force Agda to accept this instance, not every copattern is valid due to technical limitations.
OCaml is another language that supports copatterns.
However, the fork that added this feature was never merged and stopped being updated to newer OCaml versions.

In this scenario, we have chosen Scheme as our implementation language.
Customizability is one of the key features of lisps dialects, and by taking leverage on that, we can produce a new solution to the expression problem \cite{ExpressionProblem}.
Therefore, by choosing a well-established dialect, we produce a standalone, composable, and reusable implementation of copatterns for a new audience.
\adriano{Should we talk about the expression problem here in the introduction?}


Our primary contributions are:
\begin{itemize}
    \item An encoding of copatterns in Scheme and Racket.
    Our abstraction provides a syntax that enables equational reasoning and emphasizes both the vertical and horizontal composition of copatterns.
    We provide three flavors for our implementation: A R6RS version where we implement our own pattern matching, A naive Racket version, and a cleaver Racket version where we optimize the number of administrative reductions;
    \item A core calculus with high-level features such as nested copatterns, self-referential objects, recursion templates, and composable extensions.
    This calculus depicts our implementation, and we prove that it is a conservative extension of the $\lambda$-calculus with pattern-matching and recursion;
\end{itemize} 
The remainder of the article has the following structure: First, we introduce our implementation by explaining meaningful examples (Section \ref{sec-examples}).
Second, we specify a core language with high-level features representing our implementation. Then we describe a translation into a target $\lambda$-calculus (Section \ref{sec-translation}).
Third, we scrutinize our implementation, comparing each provided flavor (Section \ref{sec-macro}).
Fourth, we present the properties of our system (Section \ref{sec-correctness}).
Last, we explain the details of our optimized racket implementation (Section \ref{sec-opt}).


%%% Local Variables:
%%% mode: LaTeX
%%% TeX-master: "coscheme"
%%% End:


\section{Programming with Composable Copatterns in Scheme}
\label{sec-examples}

\section{Translating Composable Copatterns} \label{sec-translation}

\begin{figure}
\centering
\begin{alignat*}{2}
  \mathit{Term} &\ni{}& M, N
  &::= x
  \mid M ~ N
  \mid \lambda x. M
  \mid K
  \mid \Match M \With \set{\many{P \to N}}
  \mid \Rec x = M
  \\
  \mathit{Pattern} &\ni{}& P
  &::= x
  \mid K ~ \many{P}
\end{alignat*}

\caption{Target language: pure $\lambda$-calculus with pattern-matching and recursion.}
\label{fig:target-syntax}
\end{figure}

\begin{figure}
\centering

\begin{alignat*}{2}
  \mathit{Term} &\ni{}& M, N
  &::= \dots
  \mid \lamstar B
  \mid \Template B
  \mid \Extension O
  \\
  \mathit{Template} &\ni{}& B
  &::= \varepsilon
  \mid O; B
  \mid \Continue x \to M
  \\
  \mathit{Extension} &\ni{}& O
  &::= \varepsilon
  \mid O; O'
  \mid Q[x] ~ O
  \mid \lambda P.~ O
  % \mid \If M ~ O
  \mid \Match P \gets M ~ O
  \mid \Nest O
  \mid \Try x \to B
  \\
  \mathit{Copattern} &\ni{}& Q
  &::= \hole
  \mid Q ~ P
  \\
  \mathit{Pattern} &\ni{}& P
  &::= x
  \mid K ~ \many{P}
\end{alignat*}

Syntactic sugar:
\begin{align*}
  \Else M
  &=
  \Continue \_ \to M
  \\
  \Do M
  &=
  \Try \_ \to \Else \to M
  \\
  (= M)
  &=
  \Do M
  \\
  \If M ~ O
  &=
  \Match \True \gets M ~ O
  \\
  \Let x = M ~ O
  &=
  \Match x \gets M ~ O
\end{align*}
\caption{Source language: target extended with nested copatterns,
  self-referential objects, recursion templates, and composable extensions.}
\label{fig:source-syntax}
\end{figure}

\begin{figure}
\centering
Translating new terms:  
\begin{align*}
  \den{\lamstar B}
  &=
  \Rec \mathit{self} = T\den{B} ~ (\lambda x. \mathit{self} ~ x)
  \\
  \den{\Template B}
  &=
  T\den{B}
  \\
  \den{\Extension O}
  &=
  E\den{O}
  \\
  \den{M}
  &=
  \text{by induction}
  &(\text{otherwise})
\end{align*}
Translating templates:
\begin{align*}
  T\den{\varepsilon}
  &=
  \lambda s. \mathit{fail}~s
  &
  &=_\eta
  \mathit{fail}
  \\
  T\den{O; B}
  &=
  \lambda s. E\den{O} ~ T\den{B} ~ s
  &
  &=_\eta
  E\den{O} ~ T\den{B}
  \\
  T\den{\Continue x \to M}
  &=
  \lambda x. \den{M}
\end{align*}

Translating copattern-matching and pattern-matching functions:
\begin{align*}
  E\den{(Q[x] ~ P) ~ O}
  &=
  E\den{Q[x] ~ (\lambda P. O)}
  \\
  E\den{x ~ O}
  &=
  \lambda b. \lambda x. E\den{O} ~ b ~ x
  \\
  E\den{\lambda P. O}
  &=
  E\den{\lambda x. \Match P \gets x ~ O}
  &(\text{if } P \notin \mathit{Variable})
\end{align*}

Translating other extensions:
\begin{align*}
  E\den{\varepsilon}
  &=
  \lambda b. \lambda s. b ~ s
  &
  &=_\eta
  \lambda b. b
  \\
  E\den{O; O'}
  &=
  \lambda b. \lambda s. E\den{O} ~ (E\den{O'}~b) ~ s
  &
  &=_\eta
  E\den{O} \comp E\den{O'}
  \\
  E\den{\lambda x. O}
  &=
  \lambda b. \lambda s. (\lambda x. E\den{O} ~ (\lambda s'. b ~ s' ~ x) ~ s)
  \\
  E\den{\Match P \gets M ~ O}
  &=
  \lambda b. \lambda s.
  \begin{aligned}[t]
    &\Match \den{M} \With \\
    &\quad
    \begin{aligned}[t]
      \{~
      P &\to E\den{O}~b~s; \\
      \_ &\to b~s
      ~\}
    \end{aligned}
  \end{aligned}
  \\
  E\den{\Nest O}
  &=
  \lambda b. \lambda s. \Rec s' = E\den{O} ~ (\lambda \_. b ~ s) ~ (\lambda x. s' ~ x)
  \\
  E\den{\Try x \to B}
  &=
  \lambda x. T\den{B}
\end{align*}
\caption{Translating copattern-based source code to the target language.}
\label{fig:translation}
\end{figure}


\section{Macro Definition} \label{sec-macro}

\section{Correctness} \label{sec-correctness}

\begin{figure}
\centering
  
\begin{alignat*}{2}
  \mathit{Value} &\ni{}& V, W
  &::= x
  \mid \lambda x. M
  \mid K~\many{V}
  \\
  \mathit{EvalCxt} &\ni{}& E
  &::= \hole
  \mid E ~ M
  \mid V ~ E
  \mid \Match E \With \set{\many{P \to N}}
  \mid \Rec x = E
\end{alignat*}

Operational axioms:
\begin{align*}
  (\lambda x. M) ~ V
  &=
  M\subst{x}{V}
  % \\
  % \begin{aligned}
  %   &\Match V \With \\
  %   &\qquad\set{\many[i]{P_i \to N_i}}
  % \end{aligned}
  % &=
  % N_k\subst{BV(P_k)}{\many{W}}
  % &
  % \begin{aligned}
  %   (&\text{if } && V = P_k\subst{BV(P_k)}{\many{W}} \\
  %   &\text{and } &&\forall 1 \leq j < k, \\
  %   &&&\not\exists \many{W'}, V = P_j\subst{BV(P_j)}{\many{W'}})
  % \end{aligned}
  \\
  \begin{aligned}
    &\Match V \With \\
    &\qquad
    \begin{aligned}[t]
    \{~ &P \to N; \\
    &\many{P' \to N'}~\}
    \end{aligned}
  \end{aligned}
  &=
  N\subst{\many{x}}{\many{W}}
  &(\text{if } P\subst{\many{x}}{\many{W}} &= V)
  % &(\text{if } \exists \many{W},~ V &= P\subst{BV(P)}{\many{W}})
  \\
  \begin{aligned}
    &\Match V \With \\
    &\qquad
    \begin{aligned}[t]
    \{~ &P \to N; \\
    &\many{P' \to N'}~\}
    \end{aligned}
  \end{aligned}
  &=
  \begin{aligned}
    &\Match V \With \\
    &\qquad \set{\many{P' \to N'}}
  \end{aligned}
  &(\text{if } P &\apart V)
  % &(\text{if}\!\not\exists \many{W},~ V &= P\subst{BV(P)}{\many{W}})
  \\
  \Rec x = V
  &=
  V\subst{x}{(\Rec x = V)}
\end{align*}

Observational axioms:
\begin{align*}
  % \lambda x. (V ~ x)
  % &=
  % V
  % &(\text{if } x &\notin FV(V))
  % \\
  (\lambda x. E[x]) ~ M
  &=
  E[M]
  \\
  E\left[
    \begin{aligned}
      &\Match M \With \\
      &\qquad\set{\many{P \to N}}
    \end{aligned}
  \right]
  &=
  \begin{aligned}
    &\Match M \With \\
    &\qquad \set{\many{P \to E[N]}}
  \end{aligned}
  &(\text{if } BV(P) \cap FV(E) = \emptyset)
\end{align*}

Apartness between patterns and values ($P \apart V$):
\begin{gather*}
  \infer
  {K ~ P_1 \dots P_n \apart V}
  {V \neq K ~ W_1 \dots W_n}
  \qqqquad
  \infer
  {K~P_1 \dots P_n \apart K ~ V_1 \dots V_n}
  {1 \leq j \leq n & P_j \apart V_j}
\end{gather*}

\caption{Untyped equational axioms of the target language.}
\label{fig:target-equality}
\end{figure}

\begin{figure}
\centering

\begin{alignat*}{2}
  % \mathit{TemplateValue} &\ni{}& B_v
  % &::= \varepsilon
  % \mid O_v; B_v
  % \mid \Continue x \to V
  % \\
  % \mathit{ExtensionValue} &\ni{}& O_v
  % &::= O_f
  % \mid O_f; O_v
  % \mid \Nest O_v
  % \mid \Try x \to B_v
  % \\
  \mathit{ExtensionFunc} &\ni{}& F
  &::= Q[x ~ P] ~ O
  \mid \lambda P. O
  \\
  \mathit{Value} &\ni{}& V
  &::= \dots
  \mid \lamstar (F; B)
  \mid \Template B
  \mid \Extension O
  \\
  % \mathit{NonRecTemplate} &\ni{}& B_{nr}
  % &::= O_{nr}; B_{nr}
  % \mid \Else \to M
  % \\
  % \mathit{NonRecExtension} &\ni{}& O_{nr}
  % &::= \varepsilon
  % \mid O_{nr}; O'_{nr}
  % \mid Q[\_] ~ O_{nr}
  % \mid \lambda P. O_{nr}
  % \\
  % &&&\phantom{:=}
  % \mid \Match P \gets M ~ O_{nr}
  % \mid \Nest O
  % \mid \Try x \to B_{nr}
  % \\
  % \mathit{RecCxt} &\ni{}& R
  % &::= \hole
  % \mid R; O
  % \mid O; R
  % \mid Q[x] ~ R
  % \mid \lambda P. R
  % \mid \Match P \gets M ~ O
  % \mid \Try x \to R
\end{alignat*}

Identity, associativity, and annihilation laws of composition:
\begin{align*}
  \varepsilon; O &= O = O; \varepsilon
  &
  (O_1; O_2); O_3 &= O_1; (O_2; O_3)
  &
  \Do M; O &= \Do M
  \\
  \varepsilon; B &= B
  &
  (O_1; O_2); B &= O_1; (O_2; B)
  &
  \Do M; B &= \Else M
\end{align*}

% Decomposing patterns and copatterns:
% \begin{align*}
%   (Q[x] ~ P) ~ O
%   &=
%   Q[x] ~ (\lambda P. O)
%   &
%   \_ ~ O
%   &=
%   O
%   &
%   \lambda P. O
%   &=
%   \lambda x. (\Match P \gets x ~ O)
% \end{align*}

% Factoring out recursion ($x \neq y$ and $x \notin BV(P)$):
% \begin{align*}
%   \lambda y. (x ~ O)
%   &=
%   x ~ (\lambda y. O)
%   &
%   \Match P \gets M ~ (x ~ O)
%   &=
%   x ~ (\Match P \gets M ~ O)
%   \\
%   (x ~ O); O'
%   &=
%   x ~ (O; O')
%   &
%   O; (x ~ O')
%   &=
%   x ~ (O; O')
% \end{align*}

Instantiating templates and recursive $\lamstar$:
\begin{align*}
  % (\Extension O) ~ V
  % &=
  % \Template{} (O; \Continue x \to (V ~ x))
  % \\
  % (\Template R[Q[x] ~ O]) ~ V
  % &=
  % (\Template R[Q[\_] ~ O\subst{x}{V}]) ~ V
  % \\
  % (\Template R[\Continue x \to M]) ~ V
  % &=
  % (\Template R[\Else \to M\subst{x}{V}]) ~ V
  % \\
  % (\Template B_{nr}) ~ V
  % &=
  % (\Template B_{nr}) ~ W
  % \\
  \lamstar (F; B)
  &=
  (\Template F; B) ~ (\lamstar (F; B))
  \\
  (\Template O; B) ~ V
  &=
  (\Extension O) ~ (\Template B) ~ V
  \\
  (\Template \varepsilon) ~ V
  &=
  \mathit{fail}~V
  \\
  (\Template \Continue x \to M) ~ V
  &=
  M\subst{x}{V}
  \\
  (\Extension \Try x \to B) ~ V
  &=
  \Template B\subst{x}{V}
\end{align*}

Pattern and copattern matching:
\begin{align*}
  \Match P \gets V ~ O
  &=
  O\subst{\many{x}}{\many{W}}
  &(\text{if } P\subst{\many{x}}{\many{W}} &= V)
  \\
  \Match{} P \gets V ~ O
  &=
  \varepsilon
  &(\text{if } P &\apart V)
  \\[1ex]
  (\Template{} (\lambda P. \Do M); B) ~ V' ~ V
  &=
  M\subst{\many{x}}{\many{W}}
  &(\text{if } P\subst{\many{x}}{\many{W}} &= V)
  \\
  (\Template{} (\lambda P. O); B) ~ V' ~ V
  &=
  (\Template B) ~ V' ~ V
  &(\text{if } P &\apart V)
  \\[1ex]
  C[(\Template{} (Q[y] = M); B) ~ V]
  &=
  M\subst{y}{V}\subst{\many{x}}{\many{W}}
  &(\text{if } Q\subst{\many{x}}{\many{W}} &= C)
  \\
  C[(\Template{} (Q[y] ~ O); B) ~ V]
  &=
  C[(\Template B) ~ V]
  &(\text{if } Q &\apart C)
  \\[1ex]
  C[\lamstar (Q[y] = M); B]
  &=
  \begin{aligned}[t]
    M
    &\subst{y}{(\lamstar (Q[y] = M); B)}
    \\
    &\subst{\many{x}}{\many{W}}
  \end{aligned}
  &(\text{if } Q\subst{\many{x}}{\many{W}} &= C)
  \\
  C[\lamstar (Q[y] ~ O); B]
  &=
  C[\lamstar B]
  &(\text{if } Q &\apart C)
\end{align*}

Apartness between copatterns and contexts ($Q \apart C$):
\begin{gather*}
  \infer
  {\hole ~ P_1 \dots P_n \apart \hole ~ V_1 \dots V_n}
  {P_n \apart V_n}
  \qqqquad
  \infer
  {Q ~ P \apart C}
  {Q \apart C}
  \qqqquad
  \infer
  {Q \apart C ~ V}
  {Q \apart C}
\end{gather*}

\caption{Some equalities of copattern extensions.}
\label{fig:source-equality}
\end{figure}

\begin{lemma}
  The following instances of translation are all values up to the equational
  theory of the target language in \cref{fig:target-equality}:
  \begin{enumerate}[(a)]
  \item $T\den{B} = \lambda s. M$ for some term $M$,
  \item $E\den{O} = \lambda b. \lambda s. M$ for some term $M$,
  \item $E\den{F} = \lambda b. \lambda s. \lambda x. M$ for some term $M$,
  \item $\den{V} = W$ for some value $W$.
  \end{enumerate}
\end{lemma}

\begin{proposition}[Conservative Extension]
  If $M = N$ in the equational theory of the target
  (\cref{fig:target-equality}), then so too does $\den{M} = \den{N}$.
\end{proposition}

\begin{proposition}[Soundness]
  The equational axioms given in \cref{fig:source-equality} are sound with
  respect to the translation in \cref{fig:translation},
  \begin{align*}
    M &= M' &&\implies & \den{M} &= \den{M'} \\
    B &= B' &&\implies & T\den{B} &= T\den{B'} \\
    O &= O' &&\implies & E\den{O} &= E\den{O'}
  \end{align*}
  up to the equational theory of the target language in
  \cref{fig:target-equality}.
\end{proposition}

\section{Optimizing Away Administrative Reductions} \label{sec-opt}

\begin{figure}
\centering
Translating new terms:  
\begin{align*}
  \den{\lamstar B}()
  &=
  \Rec \mathit{self} = T\den{B}(\lambda x. \mathit{self} ~ x)
  \\
  \den{\Template B}()
  &=
  \lambda s. T\den{B}(s)
  \\
  \den{\Extension O}()
  &=
  \lambda b. \lambda s. E\den{O}(b, s)
  \\
  \den{M}()
  &=
  \text{by induction}
  &(\text{otherwise})
\end{align*}
Translating templates:
\begin{align*}
  T\den{\varepsilon}(V)
  &=
  \mathit{fail}~V
  \\
  T\den{O; B}(V)
  &=
  (\lambda b. E\den{O}(b, V)) ~ (\lambda s. T\den{B}(s))
  \\
  T\den{\Continue x \to M}(V)
  &=
  \den{M}()\subst{x}{V}
\end{align*}

Translating copattern-matching and pattern-matching functions:
\begin{align*}
  E\den{(Q[x] ~ P) ~ O}(W, V)
  &=
  E\den{Q[x] ~ (\lambda P. O)}(W, V)
  \\
  E\den{x ~ O}(W, V)
  &=
  E\den{O}(W, V)\subst{x}{W}
  \\
  E\den{\lambda P. O}(W, V)
  &=
  E\den{\lambda x. \Match P \gets x ~ O}(W, V)
  &(\text{if } P \notin \mathit{Variable})
\end{align*}

Translating other extensions:
\begin{align*}
  E\den{\varepsilon}(W, V)
  &=
  W(V)
  \\
  E\den{O; O'}(W, V)
  &=
  (\lambda b. E\den{O}(b, V)) ~ (\lambda s. E\den{O'}(W, s))
  \\
  E\den{\lambda x. O}(W, V)
  &=
  \lambda x. E\den{O}((\lambda s'. W(s') ~ x), V)
  \\
  E\den{\Match P \gets M ~ O}(W, V)
  &=
  \Match \den{M}() \With
  \set{P \to E\den{O}(W, V); \_ \to W(V)}
  \\
  E\den{\Nest O}(W, V)
  &=
  \Rec s' = E\den{O}((\lambda \_. W(V)), (\lambda x. s' ~ x))
  \\
  E\den{\Try x \to B}(W, V)
  &=
  T\den{B}(V)\subst{x}{W}
\end{align*}

Inlining administrative $\lambda$-abstractions:
\begin{align*}
  (\lambda x. M)(W, \many{V})
  &=
  M(\many{V})\subst{x}{W}
  \\
  M(\many{V})
  &=
  M ~ \many{V}
  &(\text{otherwise})
\end{align*}

\caption{Inlining version of the translation.}
\label{fig:inline-translation}
\end{figure}

\begin{proposition}
  Up to the equational theory of the target language in
  \cref{fig:target-equality},
  \begin{align*}
    \den{M}() &= \den{M}
    \\
    \lambda s. T\den{B}(s) &= T\den{B}
    \\
    \lambda b. \lambda s. E\den{O}(b, s) &= E\den{O}
  \end{align*}  
\end{proposition}

\section{Related and Future Work} \label{sec-future}

\section{Conclusion} \label{sec-conclusion}

\begin{credits}
\subsubsection{\ackname}
% 
This material is based upon work supported by the National Science Foundation
under Grant No. 2245516.

\subsubsection{\discintname}
%
The authors have no competing interests to declare that are
relevant to the content of this article.
\end{credits}
%
% ---- Bibliography ----
%
% BibTeX users should specify bibliography style 'splncs04'.
% References will then be sorted and formatted in the correct style.
%
\bibliographystyle{splncs04}
\bibliography{refs}

\end{document}
