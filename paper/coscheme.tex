% This is samplepaper.tex, a sample chapter demonstrating the
% LLNCS macro package for Springer Computer Science proceedings;
% Version 2.21 of 2022/01/12
%

\documentclass[runningheads]{llncs}
%
\usepackage[T1]{fontenc}
% T1 fonts will be used to generate the final print and online PDFs,
% so please use T1 fonts in your manuscript whenever possible.
% Other font encondings may result in incorrect characters.
%
\usepackage{graphicx}
% Used for displaying a sample figure. If possible, figure files should
% be included in EPS format.
\usepackage{xcolor}
\usepackage[shortlabels]{enumitem}

% llncs.cls clashes with amsthm.
% Save the LNCS proof environment defined by the class
\let\lncsproof\proof \let\lncsendproof\endproof \let\lncsqed\qed
% Remove the definitions in order to load amsthm
\let\proof\relax\let\endproof\relax
% Load AMS styles
\usepackage{amsmath}
\usepackage{amsthm}
\usepackage{amssymb}
% restore the LNCS class defined proof
\let\proof\lncsproof \let\endproof\lncsendproof \let\qed\lncsqed

\usepackage{stmaryrd}
\usepackage{braket}
\usepackage{proof}

\usepackage{minted}

\usepackage{hyperref}
\usepackage{cleveref}

\usepackage{preamble}

% If you use the hyperref package, please uncomment the following two lines
% to display URLs in blue roman font according to Springer's eBook style:
\usepackage{color}
\renewcommand\UrlFont{\color{blue}\rmfamily}
\urlstyle{rm}

% Unicode characters:
\DeclareUnicodeCharacter{3BB}{$\lambda$}

\begin{document}
%
\title{CoScheme: Compositional Copatterns in Scheme}
%
%\titlerunning{Abbreviated paper title}
% If the paper title is too long for the running head, you can set
% an abbreviated paper title here
%
\author{Paul Downen\inst{1}\orcidID{0000-0003-0165-9387}}
%
\authorrunning{P. Downen}
% First names are abbreviated in the running head.
% If there are more than two authors, 'et al.' is used.
%
\institute{University of Massachusetts Lowell, Lowell MA 01854, USA \\
\email{Paul\_Downen@uml.edu}}
%
\maketitle              % typeset the header of the contribution
%
\begin{abstract}
The abstract should briefly summarize the contents of the paper in
150--250 words.

\keywords{Codata \and Copatterns \and Scheme \and Macros \and Composition \and Expression Problem.}
\end{abstract}
%
%
%
\section{Introduction} \label{sec-intro}

For decades, functional programmers have had a reliable and versatile method for representing tree-shaped structures: inductive data types.
These can model data of any size --- for example, lists of an arbitrary length --- but each instance must be \emph{finite}.
Infinite information --- like a stream of input that can go on forever --- does not fit into an inductive type, so the programmer must use some other representation to model potentially infinite objects.

Fortunately, the inductive data types used by functional programmers every day have a polar opposite: \emph{coinductive codata types}.
The \emph{coinductive} descriptor signals that values of the type may contain infinite information.
Haskell programmers are already well-versed in these types, since non-strict languages blur the line between induction and coinduction.
For example, consider the usual example of the infinite list of Fibonacci numbers in Haskell:
\begin{minted}{haskell}
fibs = 0 : 1 : zipWith (+) fib (tail fib)
\end{minted}
\hs|fibs| cannot be fully evaluated because it has no base case --- it would eventually expand out to \hs|0 : 1 : 1 : 2 : 3 : 5 : 8 : ...| forever --- but this is no problem in a non-strict language that only evaluates as much as needed.

In contrast, \emph{codata} describes types that are defined by primitive \emph{destructors} that \emph{use} values of the codata type, as opposed to the primitive constructors that define how to build values of a data type.
For example, the usual \agda|Stream a| codata type of infinite \agda|a|'s is defined by two destructors: \agda|Head : Stream a -> a| extracts the first element and \agda|Tail : Stream a -> Stream a| discards the first element and returns the rest.
To define new streams, we can describe how they react to different combinations of \agda|Head| and \agda|Tail| destructions using \emph{copatterns} \cite{APTS2013C}.  The copattern-based definition of the \agda|fibs| function above is:
\begin{minted}{agda}
fibs : Stream Nat
Head fibs = 0
Head (Tail fibs) = 1
Tail (Tail fibs) = zipWith _+_ fibs (Tail fibs)
\end{minted}
\adriano{This example is funny, it works but emits 3 warnings, two related to the termination checker and one suggestion to turn on a flag (--guardedness). However, if you turn this flag on those warnings become errors. After typesetting the proofs, I will try to look for more failing examples in Agda.}
Even though copatterns provide an elegant way to define codata, their adoption is not universal.
To illustrate, this previous definition does not fully satisfy Agda's termination checker.
Although we can force Agda to accept this instance, not every copattern is valid due to technical limitations.
OCaml is another language that supports copatterns.
However, the fork that added this feature was never merged and stopped being updated to newer OCaml versions.

In this scenario, we have chosen Scheme as our implementation language.
Customizability is one of the key features of lisps dialects, and by taking leverage on that, we can produce a new solution to the expression problem \cite{ExpressionProblem}.
Therefore, by choosing a well-established dialect, we produce a standalone, composable, and reusable implementation of copatterns for a new audience.
\adriano{Should we talk about the expression problem here in the introduction?}


Our primary contributions are:
\begin{itemize}
    \item An encoding of copatterns in Scheme and Racket.
    Our abstraction provides a syntax that enables equational reasoning and emphasizes both the vertical and horizontal composition of copatterns.
    We provide three flavors for our implementation: A R6RS version where we implement our own pattern matching, A naive Racket version, and a cleaver Racket version where we optimize the number of administrative reductions;
    \item A core calculus with high-level features such as nested copatterns, self-referential objects, recursion templates, and composable extensions.
    This calculus depicts our implementation, and we prove that it is a conservative extension of the $\lambda$-calculus with pattern-matching and recursion;
\end{itemize} 
The remainder of the article has the following structure: First, we introduce our implementation by explaining meaningful examples (Section \ref{sec-examples}).
Second, we specify a core language with high-level features representing our implementation. Then we describe a translation into a target $\lambda$-calculus (Section \ref{sec-translation}).
Third, we scrutinize our implementation, comparing each provided flavor (Section \ref{sec-macro}).
Fourth, we present the properties of our system (Section \ref{sec-correctness}).
Last, we explain the details of our optimized racket implementation (Section \ref{sec-opt}).


%%% Local Variables:
%%% mode: LaTeX
%%% TeX-master: "coscheme"
%%% End:


\section{Programming with Composable Copatterns in Scheme}
\label{sec-examples}

\section{Translating Composable Copatterns} \label{sec-translation}

\subsection{Challenges}

Even though the behavior of small examples may be straightforward to understand, there are several challenges to correctly implementing copatterns in the general case.
Some of these challenges are specific to Scheme --- a dynamically-typed, call-by-value language --- which forces us to carefully resolve the timing of when and which copatterns are matched.
Other challenges are specific to our extensions to copatterns --- the ability to compose copattern matching in two different directions --- which also brings in the notion of the recursive ``self.''

\subsubsection{Timing and the order of copattern matching}
\label{sec:timing-challenges}

Copatterns may have ambiguous cases where two different overlapping copattern equations match the same application.
For example, this following function moves a number by \scm|1| away from \scm|0| --- positives are incremented and negatives are decremented:
\begin{minted}{scheme}
(define* [(away-from0 x) (try-if (>= x 0)) = (+ x 1)]
 [(away-from0 x) (try-if (<= x 0)) = (- x 1)])
\end{minted}
Consequently, we must interpret the programmer's code as it is written since we cannot gain any information from a static type system.
In the previous example, the two different equations overlap for \scm|0| itself: either one matches the call \scm|(away-from0 0)|.
To disambiguate overlapping copatterns, the listed equations are always tried top-down, and the first full match ``wins,'' as is typical in functional languages.
In this case, the first line wins, so \scm|(away-from0 0)| is \scm|1|.
Furthermore, guards like \scm|try-if| and \scm|try-match| are run left-to-right with shortcircuiting --- the moment a copattern or a guard fails, everything to the right is skipped.
This makes it possible to protect potentially-erroneous guards with another safety guard to its left, such as \scm|(try-if (not (= y 0)))| followed by \scm|(try-if (> (/ x y) z))|.

However, there are more timing issues besides these usual choices for disambiguation and short-circuiting.
First of all, since we are in a call-by-value language, we have to handle cases where an object is used in a context that doesn't fully match a copattern \emph{yet}, but could in the future --- and possibly multiple different times.
This can happen for instances like curried functions that take arguments in multiple different calls.
Just like with ordinary curried functions, using such an object in a calling context passing only the first list of arguments --- but not the second --- builds a \emph{value} which closes over the parameters so far.
For example, consider this simple counter object that can add or get its current internal state.
\begin{minted}{scheme}
(define* [((counter x) 'add y) = (counter (+ x y))]
         [((counter x) 'get)   = x])
\end{minted}
The call \scm|((counter 4) 'get)| matches the second equation, which is \scm|4|, but \scm|(counter 4)| on its own is not enough information to definitively match either copattern, so it is just a value remembering that \scm|x = 4| and waiting for another call.
Similarly, the call \scm|(counter (+ x y))| on the right-hand side is \emph{also} incomplete in the same sense, so it, too, is a value.
This definition gives us an object with the following behavior:
\begin{minted}{scheme}
> (define c0 (counter 4))
> (define c1 (c0 'add 1))
> ((c1 'add 2) 'get)
7
> (c1 'get)
5
\end{minted}

So far, what we have seen so far seems similar to pattern-matching functions in languages that are curried-by-default.
One way in which copatterns generalize curried functions is that each equation can take a \emph{different} number of arguments.
For example, consider this reordering of the \scm|stutter| stream from \cref{sec-examples}:
\begin{minted}{scheme}
(define* [(((stutter n) 'tail) 'tail) = (stutter (+ n 1))]
         [(((stutter n) 'tail) 'head) = n]
         [ ((stutter n) 'head)        = n])
\end{minted}
Since none of the copatterns overlap, its behavior is exactly the same as before.
But notice the extra complication here: calling \scm|((stutter 10) 'head)| with two arguments (\scm|10| and \scm|'head|) should immediately return \scm|10|.
However, the first equation is waiting for three arguments (an \scm|n| and two \scm|'tail|s passed separately).
That means the underlying code implementing \scm|stutter| \emph{cannot} ask for three arguments in three different calls and then check that the last two are \scm|'tail|.
Instead, it has to eagerly match the arguments its given against the patterns and try each of the guards to see if the current line fails --- and only \emph{after} that all succeed, it may ask for more arguments and continue the copattern match.

\subsubsection{Composition and the dimensions of extensibility}
\label{sec:composition-challenges}

The second set of challenges is due to the new notions of object composition that we develop here.
In particular, we want to be able to combine objects in two different directions:
\begin{itemize}
\item
  \emph{vertical composition} is an ``either or'' combination of two or more objects, such as \scm|(o1 'compose o2 ...)| that acts like \scm|o1| or \scm|o2|, \etc[,] depending on which one knows how to respond to the context.
  Textually, vertical composition of \scm|(object line-a1 ...)| and \scm|(object line-b1 ...)| behaves as if we copied all each line of copattern-matching equations internally used to define the two objects and pasted them vertically into the newly-composed object as:
\begin{minted}{scheme}
(object line-a1 ...
        line-b1 ...)
\end{minted}
\item
  \emph{horizontal composition} is an ``and then'' combination of objects in a copattern-matching line, such as \scm|[(self 'method1) (try-object o1)]| defining a \scm|'method1| that continues to act like \scm|o1| when \scm|o1| knows how to respond to the surrounding context, and otherwise tries the next line.
  Textually, the vertical composition of a \scm|'method1| followed by trying another object with its own copattern-matching contexts \scm|Q1 Q2 ...| acts as if the two copatterns are combined, and the inner object is inlined into the outer one like so:
\begin{minted}{scheme}
(object [(self 'method1) (try-object (object [Q1 = response1]
                                             [Q2 = response2]
                                             ...))]
        ...)
=
(object [(self 'method1) (comatch Q1) = response1]
        [(self 'method1) (comatch Q2) = response2]
        ...)
\end{minted}
\end{itemize}

Even though we can visually understand the two directions of composition by the textual manipulations above, in reality, both of these compositions are done at run-time (\ie with arbitrary procedural values), as opposed to ``compile-time'' transformation (\ie macro-expansion time manipulations of code).
This means we need an extensible representation of run-time object values that allows for automatically switching from one object to another in the case of copattern-match failure, as well as correctly keeping track of what to try next.

The basic idea of this representation can be understood as an extension of an idiom in ordinary functional programming.
In order to define an open-ended, pattern-matching function, we can give the cases we know how to handle now by matching on the arguments and include a default ``catch-all'' case at the end for the other behavior.
In Haskell, this might look like
\begin{minted}{haskell}
f next PatA1 PatA2 ... = expr1
f next PatB1 PatB2 ... = expr2
...
f next x1    x2    ... = next x1 x2 ...
\end{minted}

For example, consider the single-line \scm|eval-add| evaluator object from \cref{sec-examples}.
In order to compose \scm|eval-add| with another evaluator handling a different case, like \scm|eval-mul|, its internal extensible code takes an extra hidden argument saying what to try \scm|next| if its line does not match, analogous to:
\begin{minted}{scheme}
(define (eval-add-ext1 next)
  (lambda* [(self 'eval `(add ,l ,r)) = (+ (self 'eval l) (self 'eval r))]
           [ self                     = (next self)]))
\end{minted}
Note that, unlike the Haskell code above, the hidden \scm|next| parameter also takes \emph{another} hidden parameter: \scm|self|.
Why?
Because if the \scm|next| set of equations needs to recurse, it cannot actually jump to itself directly --- that would skip the \scm|eval-add| code entirely --- but needs to jump back to the very first equation to try.
This \scm|self| parameter holds the value of the \emph{whole} object after all compositions have been done, as it appeared in the original call site.
Thus, the internal extensible code \scm|eval-add-ext| \emph{also} takes this second \scm|self| parameter for the same reason: it may be the second component of a composition, and it is similar to the following Racket definition:
\begin{minted}{scheme}
(define ((eval-add-ext2 next) self)
  (match-lambda* [(list 'eval `(add ,l ,r)) (+ (self 'eval l) (self 'eval r))]
                 [args                      ((next self) args)]))
\end{minted}

% One final detail to note: unless otherwise stated, the \scm|self| parameter --- which is visible as the root of any copattern --- is \emph{always} the same view of the entire object.
% That means nesting multiple copatterns in sequence might not give the expected result because the \scm|self| parameter in the hole of every copattern context will be bound to the same value.
% If we instead want the parameter in the hole of every copattern context to reflect the object at that point in time --- that is, be assigned the value given by the partial applications given by the preceding copatterns --- we can use the \scm|nest| operation.
% For example, nesting copatterns in a sequence gives us a shorthand for the common functional idiom of a ``local'' loop that closes over some parameters that never change, such as this definition of mapping a function over a list:
% \begin{minted}{scheme}
% (define* [(map* f xs) = ((map* f) xs)]
%          [(map* f) (nest)
%            (extension
%             [(go null)        = null]
%             [(go (cons x xs)) = (cons (f x) (go xs))])])
% \end{minted}
% The \scm|map*| function supports both curried and uncurried applications, and they are defined to be equal.
% Its real code is given in the curried case, where the function parameter \scm|f| is bound first and never changes.
% Then, in a second step, we have the internal looping function \scm|go|, which matches over its list parameter and recurses with a new list.
% By using \scm|nest|, we have \scm|go = (map* f)|, so that \scm|f| is visible from the closure but does not need to be passed again at every step of the loop.

\subsection{Double-barrel translation}

To explain the correctness and behavior of composable copattern matching, we give a high-level translation into a conventional $\lambda$-calculus with recursion and pattern matching (given in \cref{fig:target-syntax}).
Our pattern language is modeled after a small common core found among various implementations of Scheme, which includes normal variable wildcards $x$ that can match anything, quoted symbols $\q{x}$, and lists of the form $\Null$ or $(\Cons P \, P')$.
Note that we assume all bound variables $x$ in a pattern are distinct.
% TODO: Deletion Candidate
As shorthand, we write a list of patterns $P_1 ~ P_2 ~ \dots ~ P_n$ for $(\Cons P_1 ~ (\Cons P_2 ~ \dots (\Cons P_n \Null)))$.
To model the patterns found in typed functional languages like ML and Haskell, such as constructor applications $K ~ \many{P}$, we can represent the constructor  as a quoted symbol $\q{K}$ and the application as a list $\q{K} ~ \many{P}$.
% TODO: Deletion Candidate
The patterns' specifics are surprisingly not essential to the main copattern translation and could be extended with other features found in more specific implementations.  

\begin{figure}[t]
\centering
\begin{alignat*}{2}
  % \mathit{Variable} &\ni{}& x, y, z
  % \\
  \mathit{Term} &\ni{}& M, N
  &::= x
  \mid M ~ N
  \mid \lambda x. M
  \mid K
  \mid \Match M \With \set{\many{P \to N}}
  \mid \Rec x = M
  \\
  \mathit{Pattern} &\ni{}& P
  &::= x
  \mid \q{x}
  \mid \Null
  \mid \Cons P \, P'
\end{alignat*}

\caption{Target language: pure $\lambda$-calculus with pattern-matching and recursion.}
\label{fig:target-syntax}
\end{figure}

For simplicity, this translation begins from a small source language with copatterns (given in \cref{fig:source-syntax}) separated into three main syntactic categories:
\begin{itemize}
\item[($M, N$)] \emph{Terms} represent ordinary first-class values as well as applications.
  The new forms of terms are $\lamstar B$, which gives a self-referential copattern-matching object, along with $\Template B$ and $\Extension O$ which include the other two syntactic categories as first-class values.
\item[($B$)] \emph{Templates} represent self-referential code without a fixed self.
  Instead, the ``self'' placeholder remains unbound for now, and it can be instantiated later as $\Template B ~ V$ (where the ``self'' of the template is bound to $V$) or $\lamstar B$ (where the ``self'' of the template is recursively bound to $\lamstar B$).
\item[($O$)] \emph{Extensions} represent extensible code that can be composed together both vertically and horizontally.
  Instead of failing on an unsuccessful match, will try an as-of-yet unspecified ``next'' option.
  To support recursion, the ``self'' placeholder is also unbound for now --- just like with templates --- and can be bound later when the whole object is finished being composed.
  The ``next'' thing to try can be given by the vertical composition with another extension $O; O'$ or a base-case template $O; B$.
  Arbitrary first-class values can passed in as the next option ($V$) and the self object ($W$) as $\Extension O ~ V ~ W$.
\end{itemize}

\begin{figure}[t]
\centering
\small
\begin{alignat*}{2}
  % \mathit{Variable} &\ni{}& x, y, z
  % \\
  \mathit{Term} &\ni{}& M, N
  &::= \dots
  \mid \lamstar B
  \mid \Template B
  \mid \Extension O
  \\
  \mathit{Template} &\ni{}& B
  &::= \varepsilon
  \mid O; B
  \mid \Continue x \to M
  \\
  \mathit{Extension} &\ni{}& O
  &::= \varepsilon
  \mid O; O'
  \mid Q[x] ~ O
  \mid \lambda P.~ O
  % \mid \If M ~ O
  \mid \Match P \gets M ~ O
  % \mid \Nest O
  \mid \Try x \to B
  \\
  \mathit{Copattern} &\ni{}& Q
  &::= \hole
  \mid Q ~ P
  \\
  \mathit{Pattern} &\ni{}& P
  &::= x
  \mid \q{x}
  \mid \Null
  \mid \Cons P \, P'
\end{alignat*}

Syntactic sugar:
\begin{align*}
  \Else M
  &=
  \Continue \_ \to M
  &
  (= M)
  =
  \Do M
  &=
  \Try \_ \to \Else M
  \\
  \If M ~ O
  &=
  \Match \True \gets M ~ O
  &
  (\Let x = M ~ O)
  &=
  \Match x \gets M ~ O
\end{align*}
\caption{Source language: target extended with nested copatterns,
  self-referential objects, recursion templates, and composable extensions.}
\label{fig:source-syntax}
\end{figure}

The remaining new syntax gives ways to define and combine copattern-matching expressions.
Copatterns $Q[x]$ themselves are a subset of contexts, $Q$, surrounding an object internally named $x$.
Two lines separated by a semicolon ($O; O'$) is vertical composition that tries either $O$ or $O'$, and prefixing with a copattern-matching expression ($Q[x] O$) is horizontal composition that tries $Q[x]$ and then $O$.
The $\varepsilon$ represents an empty extension with respect to vertical composition: it immediately refers to the next option.
Smaller special cases of matching include pattern lambdas ($\lambda P. O$) that try to match a new argument against $P$, and pattern guards ($\Match P \gets M ~ O$) that try to match a given expression $M$ against $P$; both of which continue as $O$ if they succeed.
% $\Nest O$ allows for nesting multiple copatterns with a partially applied self object from this point.

Finally, we have the terminators for ending a sequence of matching.
A template can end in the empty $\varepsilon$ (which just fails, because there is no code to handle the case) or a $\Continue x \to M$ which serves as the default ``catch-all'' case.
The parameter $x$ bound by $\Continue x \to M$ is another way to introduce a name for the recursive reference to the object itself at the end of a template  and allows for $M$ to restart from the top and continue the computation.
The syntactic sugar $\Else M$ covers the common case where $M$ give an answer without recursively continuing.
Similarly, an extension can end with $\Try x \to B$. This gives a ``catch-all'' case that runs some other (non-extensible) template $B$.
The parameter $x$ bound by $\Try x \to B$ gives a name to the next option that would have been tried after this one and allows $B$ to explicitly move on to the next option if it needs to.
The syntactic sugar $\Do M$ covers the most common case of $\Try$ which definitively commits to a particular term $M$ to return as the result without trying any further options.
To write examples in a similar style to ML-family languages, we also use the syntactic sugar $(= M)$ with the same meaning, which looks odd out of context but expresses the equational nature of copattern matching when used in examples.

Thus, the full translation from the source (\cref{fig:source-syntax}) to target (\cref{fig:target-syntax}) is given in \cref{fig:translation}.
This translation shares many similarities to continuation-passing style (CPS) translations.
However, we explicitly avoid converting the entire program to CPS.
Notably, every syntactic form for the source language is unchanged; for example, $\den{M~N} = \den{M} ~ \den{N}$.
Instead, the only time we need to introduce an extra parameter is for the two new syntactic categories.
All templates are translated to functions that take a value for the whole object itself to a new version of that object.
Similarly, all extensions are translated to functions that take both a template as the ``base case'' to try next and a value for the whole object itself.
Even though this is dynamically-typed, we can view the type of templates as object transformers and extensions as template transformers:
\begin{align*}
  Object &= \text{some type of function}
  \\
  Template &= Object \to Object'
  \\
  Extension &= Template \to Template'
  = Template \to Object \to Object'
\end{align*}

\begin{figure}[t]
\centering
\small
Translating new terms:  
\begin{align*}
  \den{\lamstar B}
  &=
  (\Rec \mathit{self} = T\den{B} ~ \mathit{self})
  &=_\eta
  (\Rec \mathit{self} = T\den{B} ~ (\lambda x. \mathit{self} ~ x))
  \\
  \den{\Template B}
  &=
  T\den{B}
  \\
  \den{\Extension O}
  &=
  E\den{O}
  \\
  \den{M}
  &=
  \text{by induction}
  &(\text{otherwise})
\end{align*}
Translating templates:
\begin{align*}
  T\den{\varepsilon}
  &=
  \mathit{fail}
  &
  &=_\eta
  \lambda s. \mathit{fail}~s
  \\
  T\den{O; B}
  &=
  E\den{O} ~ T\den{B}
  &
  &=_\eta
  \lambda s. E\den{O} ~ T\den{B} ~ s
  \\
  T\den{\Continue x \to M}
  &=
  \lambda x. \den{M}
\end{align*}

Translating copattern-matching and pattern-matching functions:
\begin{align*}
  E\den{(Q[x] ~ P) ~ O}
  &=
  E\den{Q[x] ~ (\lambda P. O)}
  \\
  E\den{x ~ O}
  &=
  \lambda b. \lambda x. E\den{O} ~ b ~ x
  \\
  E\den{\lambda P. O}
  &=
  E\den{\lambda x. \Match P \gets x ~ O}
  &(\text{if } P \notin \mathit{Variable})
\end{align*}

Translating other extensions:
\begin{align*}
  E\den{\varepsilon}
  &=
  \lambda b. b
  &
  &=_\eta
  \lambda b. \lambda s. b ~ s
  \\
  E\den{O; O'}
  &=
  E\den{O} \comp E\den{O'}
  &
  &=_\eta
  \lambda b. \lambda s. E\den{O} ~ (E\den{O'}~b) ~ s
  \\
  E\den{\lambda x. O}
  &=
  \lambda b. \lambda s. (\lambda x. E\den{O} ~ (\lambda s'. b ~ s' ~ x) ~ s)
  \\
  E\den{\Match P \gets M ~ O}
  &=
  \rlap{$
    \lambda b. \lambda s.
    \Match \den{M} \With \set{P \to E\den{O}~b~s; \_ \to b~s}
  $}
  % \lambda b. \lambda s.
  % \begin{aligned}[t]
  %   &\Match \den{M} \With \\
  %   &\quad
  %   \begin{aligned}[t]
  %     \{~
  %     P &\to E\den{O}~b~s; \\
  %     \_ &\to b~s
  %     ~\}
  %   \end{aligned}
  % \end{aligned}
  \\
  % E\den{\Nest O}
  % &=
  % \lambda b. \lambda s. \Rec s' = E\den{O} ~ (\lambda \_. b ~ s) ~ (\lambda x. s' ~ x)
  % \\
  E\den{\Try x \to B}
  &=
  \lambda x. T\den{B}
\end{align*}
\caption{Translating copattern-based source code to the target language.}
\label{fig:translation}
\end{figure}

The interesting cases for translating terms are the new forms.
$\Template B$ and $\Extension O$ are just translated to their given forms as transformation functions.
With $\lamstar B$, we need to recursively plug its translation in for its self parameter.
Note the one detail that the recursive $\mathit{self}$ is $\eta$-expanded to in this application.
This ensures that $\lambda x. \mathit{self} ~ x$ is treated as a value in a real implementation, and is always safe assuming that $B$ describes a function (non-functional cases of $\lamstar B$ are undefined user error).

For templates and extensions, the terminators $\Continue$ and $\Try$ are translated to plain $\lambda$-abstractions that allow the programmer direct access to their implicit parameters.
% Other cases are specific to each form.
Complex copatterns ($x ~ \many{P_1} P_n ~ O$) are reduced down to a simpler sequence of pattern lambdas ($x ~ \lambda P_1. \dots \lambda P_n. ~ O$), and pattern lambdas ($\lambda P. O$) are reduced down to a simpler non-matching lambda followed by an explicit match guard ($\lambda x. \Match P \gets x ~ O$).

This leaves just the base cases of simple extension forms.
Each time an extension (of form $\lambda b. \lambda s. \dots$) ``fails,'' it calls the given next template with the given self object ($b~s$).
A match guard $\den{\Match P \gets M ~ O}$ will try to match the translation of $M$ against the pattern $P$; the success case continues as $E\den{O}$ with the same next template and self.
A non-matching lambda $\den{\lambda x. O}$ always succeeds (for now), but note that the next template to try on failure has to be changed to include the given argument.
Why?
Because the lambda has already consumed the next argument from its context, it would be gone if, later on, the following operations fail and move on to the next option.
So instead of invoking the given $b$ directly as $b~s'$ (for a potentially different future $s'$), they need to invoke $b$ applied to this argument $x$ as $b~s'~x$.
% Finally, the $\den{\Nest O}$ operation is defined by recursively creating a new value for the self parameter by recursively taking a new snapshot of how the object looks now after all the preceding applications and matchings have already occurred.

In this translation, we also give the $\eta$-reduced forms on the right-hand side when available.
This shows that the empty extension $\varepsilon$ is just the identity function (given the next thing $b$ to try, $\varepsilon$ does nothing and immediately moves on to $b$), and horizontal composition $O; O'$ is just ordinary function composition.


%%% Local Variables:
%%% mode: LaTeX
%%% TeX-master: "coscheme"
%%% End:


\section{Macro Definition} \label{sec-macro}

\section{Correctness} \label{sec-correctness}

\begin{figure}
\centering
  
\begin{alignat*}{2}
  \mathit{Value} &\ni{}& V, W
  &::= x
  \mid \lambda x. M
  \mid K~\many{V}
  \\
  \mathit{EvalCxt} &\ni{}& E
  &::= \hole
  \mid E ~ M
  \mid V ~ E
  \mid \Match E \With \set{\many{P \to N}}
  \mid \Rec x = E
\end{alignat*}

Operational axioms:
\begin{align*}
  (\lambda x. M) ~ V
  &=
  M\subst{x}{V}
  % \\
  % \begin{aligned}
  %   &\Match V \With \\
  %   &\qquad\set{\many[i]{P_i \to N_i}}
  % \end{aligned}
  % &=
  % N_k\subst{BV(P_k)}{\many{W}}
  % &
  % \begin{aligned}
  %   (&\text{if } && V = P_k\subst{BV(P_k)}{\many{W}} \\
  %   &\text{and } &&\forall 1 \leq j < k, \\
  %   &&&\not\exists \many{W'}, V = P_j\subst{BV(P_j)}{\many{W'}})
  % \end{aligned}
  \\
  \begin{aligned}
    &\Match V \With \\
    &\qquad
    \begin{aligned}[t]
    \{~ &P \to N; \\
    &\many{P' \to N'}~\}
    \end{aligned}
  \end{aligned}
  &=
  N\subst{\many{x}}{\many{W}}
  &(\text{if } P\subst{\many{x}}{\many{W}} &= V)
  % &(\text{if } \exists \many{W},~ V &= P\subst{BV(P)}{\many{W}})
  \\
  \begin{aligned}
    &\Match V \With \\
    &\qquad
    \begin{aligned}[t]
    \{~ &P \to N; \\
    &\many{P' \to N'}~\}
    \end{aligned}
  \end{aligned}
  &=
  \begin{aligned}
    &\Match V \With \\
    &\qquad \set{\many{P' \to N'}}
  \end{aligned}
  &(\text{if } P &\apart V)
  % &(\text{if}\!\not\exists \many{W},~ V &= P\subst{BV(P)}{\many{W}})
  \\
  \Rec x = V
  &=
  V\subst{x}{(\Rec x = V)}
\end{align*}

Observational axioms:
\begin{align*}
  % \lambda x. (V ~ x)
  % &=
  % V
  % &(\text{if } x &\notin FV(V))
  % \\
  (\lambda x. E[x]) ~ M
  &=
  E[M]
  \\
  E\left[
    \begin{aligned}
      &\Match M \With \\
      &\qquad\set{\many{P \to N}}
    \end{aligned}
  \right]
  &=
  \begin{aligned}
    &\Match M \With \\
    &\qquad \set{\many{P \to E[N]}}
  \end{aligned}
  &(\text{if } BV(P) \cap FV(E) = \emptyset)
\end{align*}

Apartness between patterns and values ($P \apart V$):
\begin{gather*}
  \infer
  {K ~ P_1 \dots P_n \apart V}
  {V \neq K ~ W_1 \dots W_n}
  \qqqquad
  \infer
  {K~P_1 \dots P_n \apart K ~ V_1 \dots V_n}
  {1 \leq j \leq n & P_j \apart V_j}
\end{gather*}

\caption{Untyped equational axioms of the target language.}
\label{fig:target-equality}
\end{figure}

\begin{figure}
\centering

\begin{alignat*}{2}
  % \mathit{TemplateValue} &\ni{}& B_v
  % &::= \varepsilon
  % \mid O_v; B_v
  % \mid \Continue x \to V
  % \\
  % \mathit{ExtensionValue} &\ni{}& O_v
  % &::= O_f
  % \mid O_f; O_v
  % \mid \Nest O_v
  % \mid \Try x \to B_v
  % \\
  \mathit{ExtensionFunc} &\ni{}& F
  &::= Q[x ~ P] ~ O
  \mid \lambda P. O
  \\
  \mathit{Value} &\ni{}& V
  &::= \dots
  \mid \lamstar (F; B)
  \mid \Template B
  \mid \Extension O
  \\
  % \mathit{NonRecTemplate} &\ni{}& B_{nr}
  % &::= O_{nr}; B_{nr}
  % \mid \Else \to M
  % \\
  % \mathit{NonRecExtension} &\ni{}& O_{nr}
  % &::= \varepsilon
  % \mid O_{nr}; O'_{nr}
  % \mid Q[\_] ~ O_{nr}
  % \mid \lambda P. O_{nr}
  % \\
  % &&&\phantom{:=}
  % \mid \Match P \gets M ~ O_{nr}
  % \mid \Nest O
  % \mid \Try x \to B_{nr}
  % \\
  % \mathit{RecCxt} &\ni{}& R
  % &::= \hole
  % \mid R; O
  % \mid O; R
  % \mid Q[x] ~ R
  % \mid \lambda P. R
  % \mid \Match P \gets M ~ O
  % \mid \Try x \to R
\end{alignat*}

Identity, associativity, and annihilation laws of composition:
\begin{align*}
  \varepsilon; O &= O = O; \varepsilon
  &
  (O_1; O_2); O_3 &= O_1; (O_2; O_3)
  &
  \Do M; O &= \Do M
  \\
  \varepsilon; B &= B
  &
  (O_1; O_2); B &= O_1; (O_2; B)
  &
  \Do M; B &= \Else M
\end{align*}

% Decomposing patterns and copatterns:
% \begin{align*}
%   (Q[x] ~ P) ~ O
%   &=
%   Q[x] ~ (\lambda P. O)
%   &
%   \_ ~ O
%   &=
%   O
%   &
%   \lambda P. O
%   &=
%   \lambda x. (\Match P \gets x ~ O)
% \end{align*}

% Factoring out recursion ($x \neq y$ and $x \notin BV(P)$):
% \begin{align*}
%   \lambda y. (x ~ O)
%   &=
%   x ~ (\lambda y. O)
%   &
%   \Match P \gets M ~ (x ~ O)
%   &=
%   x ~ (\Match P \gets M ~ O)
%   \\
%   (x ~ O); O'
%   &=
%   x ~ (O; O')
%   &
%   O; (x ~ O')
%   &=
%   x ~ (O; O')
% \end{align*}

Instantiating templates and recursive $\lamstar$:
\begin{align*}
  % (\Extension O) ~ V
  % &=
  % \Template{} (O; \Continue x \to (V ~ x))
  % \\
  % (\Template R[Q[x] ~ O]) ~ V
  % &=
  % (\Template R[Q[\_] ~ O\subst{x}{V}]) ~ V
  % \\
  % (\Template R[\Continue x \to M]) ~ V
  % &=
  % (\Template R[\Else \to M\subst{x}{V}]) ~ V
  % \\
  % (\Template B_{nr}) ~ V
  % &=
  % (\Template B_{nr}) ~ W
  % \\
  \lamstar (F; B)
  &=
  (\Template F; B) ~ (\lamstar (F; B))
  \\
  (\Template O; B) ~ V
  &=
  (\Extension O) ~ (\Template B) ~ V
  \\
  (\Template \varepsilon) ~ V
  &=
  \mathit{fail}~V
  \\
  (\Template \Continue x \to M) ~ V
  &=
  M\subst{x}{V}
  \\
  (\Extension \Try x \to B) ~ V
  &=
  \Template B\subst{x}{V}
\end{align*}

Pattern and copattern matching:
\begin{align*}
  \Match P \gets V ~ O
  &=
  O\subst{\many{x}}{\many{W}}
  &(\text{if } P\subst{\many{x}}{\many{W}} &= V)
  \\
  \Match{} P \gets V ~ O
  &=
  \varepsilon
  &(\text{if } P &\apart V)
  \\[1ex]
  (\Template{} (\lambda P. \Do M); B) ~ V' ~ V
  &=
  M\subst{\many{x}}{\many{W}}
  &(\text{if } P\subst{\many{x}}{\many{W}} &= V)
  \\
  (\Template{} (\lambda P. O); B) ~ V' ~ V
  &=
  (\Template B) ~ V' ~ V
  &(\text{if } P &\apart V)
  \\[1ex]
  C[(\Template{} (Q[y] = M); B) ~ V]
  &=
  M\subst{y}{V}\subst{\many{x}}{\many{W}}
  &(\text{if } Q\subst{\many{x}}{\many{W}} &= C)
  \\
  C[(\Template{} (Q[y] ~ O); B) ~ V]
  &=
  C[(\Template B) ~ V]
  &(\text{if } Q &\apart C)
  \\[1ex]
  C[\lamstar (Q[y] = M); B]
  &=
  \begin{aligned}[t]
    M
    &\subst{y}{(\lamstar (Q[y] = M); B)}
    \\
    &\subst{\many{x}}{\many{W}}
  \end{aligned}
  &(\text{if } Q\subst{\many{x}}{\many{W}} &= C)
  \\
  C[\lamstar (Q[y] ~ O); B]
  &=
  C[\lamstar B]
  &(\text{if } Q &\apart C)
\end{align*}

Apartness between copatterns and contexts ($Q \apart C$):
\begin{gather*}
  \infer
  {\hole ~ P_1 \dots P_n \apart \hole ~ V_1 \dots V_n}
  {P_n \apart V_n}
  \qqqquad
  \infer
  {Q ~ P \apart C}
  {Q \apart C}
  \qqqquad
  \infer
  {Q \apart C ~ V}
  {Q \apart C}
\end{gather*}

\caption{Some equalities of copattern extensions.}
\label{fig:source-equality}
\end{figure}

\begin{lemma}
  The following instances of translation are all values up to the equational
  theory of the target language in \cref{fig:target-equality}:
  \begin{enumerate}[(a)]
  \item $T\den{B} = \lambda s. M$ for some term $M$,
  \item $E\den{O} = \lambda b. \lambda s. M$ for some term $M$,
  \item $E\den{F} = \lambda b. \lambda s. \lambda x. M$ for some term $M$,
  \item $\den{V} = W$ for some value $W$.
  \end{enumerate}
\end{lemma}

\begin{proposition}[Conservative Extension]
  If $M = N$ in the equational theory of the target
  (\cref{fig:target-equality}), then so too does $\den{M} = \den{N}$.
\end{proposition}

\begin{proposition}[Soundness]
  The equational axioms given in \cref{fig:source-equality} are sound with
  respect to the translation in \cref{fig:translation},
  \begin{align*}
    M &= M' &&\implies & \den{M} &= \den{M'} \\
    B &= B' &&\implies & T\den{B} &= T\den{B'} \\
    O &= O' &&\implies & E\den{O} &= E\den{O'}
  \end{align*}
  up to the equational theory of the target language in
  \cref{fig:target-equality}.
\end{proposition}

\section{Optimizing Away Administrative Reductions} \label{sec-opt}

\begin{figure}
\centering
Translating new terms:  
\begin{align*}
  \den{\lamstar B}()
  &=
  \Rec \mathit{self} = T\den{B}(\lambda x. \mathit{self} ~ x)
  \\
  \den{\Template B}()
  &=
  \lambda s. T\den{B}(s)
  \\
  \den{\Extension O}()
  &=
  \lambda b. \lambda s. E\den{O}(b, s)
  \\
  \den{M}()
  &=
  \text{by induction}
  &(\text{otherwise})
\end{align*}
Translating templates:
\begin{align*}
  T\den{\varepsilon}(V)
  &=
  \mathit{fail}~V
  \\
  T\den{O; B}(V)
  &=
  (\lambda b. E\den{O}(b, V)) ~ (\lambda s. T\den{B}(s))
  \\
  T\den{\Continue x \to M}(V)
  &=
  \den{M}()\subst{x}{V}
\end{align*}

Translating copattern-matching and pattern-matching functions:
\begin{align*}
  E\den{(Q[x] ~ P) ~ O}(W, V)
  &=
  E\den{Q[x] ~ (\lambda P. O)}(W, V)
  \\
  E\den{x ~ O}(W, V)
  &=
  E\den{O}(W, V)\subst{x}{W}
  \\
  E\den{\lambda P. O}(W, V)
  &=
  E\den{\lambda x. \Match P \gets x ~ O}(W, V)
  &(\text{if } P \notin \mathit{Variable})
\end{align*}

Translating other extensions:
\begin{align*}
  E\den{\varepsilon}(W, V)
  &=
  W(V)
  \\
  E\den{O; O'}(W, V)
  &=
  (\lambda b. E\den{O}(b, V)) ~ (\lambda s. E\den{O'}(W, s))
  \\
  E\den{\lambda x. O}(W, V)
  &=
  \lambda x. E\den{O}((\lambda s'. W(s') ~ x), V)
  \\
  E\den{\Match P \gets M ~ O}(W, V)
  &=
  \Match \den{M}() \With
  \set{P \to E\den{O}(W, V); \_ \to W(V)}
  \\
  E\den{\Nest O}(W, V)
  &=
  \Rec s' = E\den{O}((\lambda \_. W(V)), (\lambda x. s' ~ x))
  \\
  E\den{\Try x \to B}(W, V)
  &=
  T\den{B}(V)\subst{x}{W}
\end{align*}

Inlining administrative $\lambda$-abstractions:
\begin{align*}
  (\lambda x. M)(W, \many{V})
  &=
  M(\many{V})\subst{x}{W}
  \\
  M(\many{V})
  &=
  M ~ \many{V}
  &(\text{otherwise})
\end{align*}

\caption{Inlining version of the translation.}
\label{fig:inline-translation}
\end{figure}

\begin{proposition}
  Up to the equational theory of the target language in
  \cref{fig:target-equality},
  \begin{align*}
    \den{M}() &= \den{M}
    \\
    \lambda s. T\den{B}(s) &= T\den{B}
    \\
    \lambda b. \lambda s. E\den{O}(b, s) &= E\den{O}
  \end{align*}  
\end{proposition}

\section{Related and Future Work} \label{sec-future}

\section{Conclusion} \label{sec-conclusion}

\begin{credits}
\subsubsection{\ackname}
% 
This material is based upon work supported by the National Science Foundation
under Grant No. 2245516.

\subsubsection{\discintname}
%
The authors have no competing interests to declare that are
relevant to the content of this article.
\end{credits}
%
% ---- Bibliography ----
%
% BibTeX users should specify bibliography style 'splncs04'.
% References will then be sorted and formatted in the correct style.
%
\bibliographystyle{splncs04}
\bibliography{refs}

\end{document}
