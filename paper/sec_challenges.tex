Even though the behavior of small examples may be straightforward to understand, there are several challenges to correctly implementing copatterns in the general case.
Some of these challenges are specific to Scheme --- a dynamically-typed, call-by-value language --- which forces us to carefully resolve the timing of when and which copatterns are matched.
Other challenges are specific to our extensions to copatterns --- the ability to compose copattern matching in two different directions --- which also brings in the notion of the recursive ``self.''

\subsubsection{Timing and the order of copattern matching}
\label{sec:timing-challenges}

Copatterns may have ambiguous cases where two different overlapping copattern equations match the same application.
For example, this following function moves a number by \scm|1| away from \scm|0| --- positives are incremented and negatives are decremented:
\begin{minted}{scheme}
(define* [(away-from0 x) (try-if (>= x 0)) = (+ x 1)]
 [(away-from0 x) (try-if (<= x 0)) = (- x 1)])
\end{minted}
Consequently, we must interpret the programmer's code as it is written since we cannot gain any information from a static type system.
In the previous example, the two different equations overlap for \scm|0| itself: either one matches the call \scm|(away-from0 0)|.
To disambiguate overlapping copatterns, the listed equations are always tried top-down, and the first full match ``wins,'' as is typical in functional languages.
In this case, the first line wins, so \scm|(away-from0 0)| is \scm|1|.
Furthermore, guards like \scm|try-if| and \scm|try-match| are run left-to-right with shortcircuiting --- the moment a copattern or a guard fails, everything to the right is skipped.
This makes it possible to protect potentially-erroneous guards with another safety guard to its left, such as \scm|(try-if (not (= y 0)))| followed by \scm|(try-if (> (/ x y) z))|.

However, there are more timing issues besides these usual choices for disambiguation and short-circuiting.
First of all, since we are in a call-by-value language, we have to handle cases where an object is used in a context that doesn't fully match a copattern \emph{yet}, but could in the future --- and possibly multiple different times.
This can happen for instances like curried functions that take arguments in multiple different calls.
Just like with ordinary curried functions, using such an object in a calling context passing only the first list of arguments --- but not the second --- builds a \emph{value} which closes over the parameters so far.
For example, consider this simple counter object that can add or get its current internal state.
\begin{minted}{scheme}
(define* [((counter x) 'add y) = (counter (+ x y))]
         [((counter x) 'get)   = x])
\end{minted}
The call \scm|((counter 4) 'get)| matches the second equation, returning the answer \scm|4|, but \scm|(counter 4)| on its own is not enough information to definitively match either copattern, so it is just a value remembering that \scm|x = 4| and waiting for another call.
Similarly, the call \scm|(counter (+ x y))| on the right-hand side is \emph{also} incomplete in the same sense, so it, too, is a value.
This definition gives us an object with the following behavior:
\begin{minted}{scheme}
> (define c0 (counter 4))
> (define c1 (c0 'add 1))
> ((c1 'add 2) 'get)
7
> (c1 'get)
5
\end{minted}

So far, what we have seen so far seems similar to pattern-matching functions in languages that are curried-by-default.
One way in which copatterns generalize curried functions is that each equation can take a \emph{different} number of arguments.
For example, consider this reordering of the \scm|stutter| stream from \cref{sec-examples}:
\begin{minted}{scheme}
(define* [(((stutter n) 'tail) 'tail) = (stutter (+ n 1))]
         [(((stutter n) 'tail) 'head) = n]
         [ ((stutter n) 'head)        = n])
\end{minted}
Since none of the copatterns overlap, its behavior is exactly the same as before.
But notice the extra complication here: calling \scm|((stutter 10) 'head)| with two arguments (\scm|10| and \scm|'head|) should immediately return \scm|10|.
However, the first equation is waiting for three arguments (an \scm|n| and two \scm|'tail|s passed separately).
That means that the underlying code implementing \scm|stutter| \emph{cannot} ask for three arguments in three different calls and then checks that the last two are \scm|'tail|.
Instead, it has to eagerly match the arguments that it is given against the patterns and try each of the guards to see if the current line fails --- and only \emph{after} that all succeed, it may ask for more arguments and continue the copattern match.

\subsubsection{Composition and the dimensions of extensibility}
\label{sec:composition-challenges}

The second set of challenges is due to the new notions of object composition that we develop here.
In particular, we want to be able to combine objects in two different directions:
\begin{itemize}
\item
  \emph{vertical composition} is an ``either or'' combination of two or more objects, such as \scm|(o1 'compose o2 ...)| that acts like \scm|o1| or \scm|o2|, \etc[,] depending on which one knows how to respond to the context.
  Textually, the vertical composition of \scm|(object line-a1 ...)| and \scm|(object line-b1 ...)| behaves as if we copied all each line of copattern-matching equations internally used to define the two objects and pasted them vertically into the newly-composed object as:
\begin{minted}{scheme}
(object line-a1
        ...
        line-b1
        ...)
\end{minted}
\item
  \emph{horizontal composition} is an ``and then'' combination of objects in a copattern-matching line, such as \scm|[(self 'method1) (o1 'unplug)]| defining a \scm|'method1| that continues to act like \scm|o1| when \scm|o1| knows how to respond to the surrounding context, and otherwise tries the next line.
  Textually, the vertical composition of a \scm|'method1| followed by trying another object with its own copattern-matching contexts \scm|Q1 Q2 ...| acts as if the two copatterns are combined, and the inner object is inlined into the outer one like so:
\begin{minted}{scheme}
(object [(self 'method1) ((object [Q1 = response1]
                                  [Q2 = response2]
                                  ...) 'unplug)]
        ...)
=
(object [(self 'method1) (comatch Q1) = response1]
        [(self 'method1) (comatch Q2) = response2]
        ...)
\end{minted}
\end{itemize}

Even though we can visually understand the two directions of composition by the textual manipulations above, in reality, both of these compositions are done at run-time (\ie with arbitrary procedural values), as opposed to ``compile-time'' transformation (\ie macro-expansion time manipulations of code).
This means we need an extensible representation of run-time object values that allows for automatically switching from one object to another in the case of copattern-match failure, as well as correctly keeping track of what to try next.

The basic idea of this representation can be understood as an extension of an idiom in ordinary functional programming.
In order to define an open-ended, pattern-matching function, we can give the cases we know how to handle now by matching on the arguments and include a default ``catch-all'' case at the end for the other behavior.
In Haskell, this might look like
\begin{minted}{haskell}
f next PatA1 PatA2 ... = expr1
f next PatB1 PatB2 ... = expr2
...
f next x1    x2    ... = next x1 x2 ...
\end{minted}

For example, consider the single-line \scm|eval-add| evaluator object from \cref{sec-examples}.
In order to compose \scm|eval-add| with another evaluator handling a different case, like \scm|eval-mul|, its internal extensible code takes an extra hidden argument saying what to try \scm|next| if its line does not match, analogous to:
\begin{minted}{scheme}
(define (eval-add-ext1 next)
  (lambda*
    [(self 'eval `(add ,l ,r)) = (+ (self 'eval l) (self 'eval r))]
    [ self                     = (next self)]))
\end{minted}
Note that, unlike the Haskell code above, the hidden \scm|next| parameter also takes \emph{another} hidden parameter: \scm|self|.
Why?
Because if the \scm|next| set of equations needs to recurse, it cannot actually jump to itself directly --- that would skip the \scm|eval-add| code entirely --- but needs to jump back to the very first equation to try.
This \scm|self| parameter holds the value of the \emph{whole} object after all compositions have been done, as it appeared in the original call site.
Thus, the internal extensible code \scm|eval-add-ext| \emph{also} takes this second \scm|self| parameter for the same reason: it may be the second component of a composition, and it is similar to the following Racket definition:
\begin{minted}{scheme}
(define ((eval-add-ext2 next) self)
  (match-lambda*
    [(list 'eval `(add ,l ,r)) (+ (self 'eval l) (self 'eval r))]
    [args                      ((next self) args)]))
\end{minted}

One final detail to note: unless otherwise stated, the \scm|self| parameter --- which is visible as the root of any copattern --- is \emph{always} the same view of the entire object.
That means nesting multiple copatterns in sequence might not give the expected result because the \scm|self| parameter in the hole of every copattern context will be bound to the same value.
If we instead want the parameter in the hole of every copattern context to reflect the object at that point in time --- that is, be assigned the value given by the partial applications given by the preceding copatterns --- we can use the \scm|nest| operation.
For example, nesting copatterns in a sequence gives us a shorthand for the common functional idiom of a ``local'' loop that closes over some parameters that never change, such as this definition of mapping a function over a list:
\begin{minted}{scheme}
(define* [(map* f xs) = ((map* f) xs)]
         [(map* f) (nest)
           (extension
            [(go null)        = null]
            [(go (cons x xs)) = (cons (f x) (go xs))])])
\end{minted}
The \scm|map*| function supports both curried and uncurried applications, and they are defined to be equal.
Its real code is given in the curried case, where the function parameter \scm|f| is bound first and never changes.
Then, in a second step, we have the internal looping function \scm|go|, which matches over its list parameter and recurses with a new list.
The definition of \scm|go| is given directly in the form of an \scm|extension| --- the result of \scm|'unplug|ging an object --- which is composed with the curried application \scm|(map* f)|.
By using \scm|nest|, we have \scm|go = (map* f)|, so that \scm|f| is visible from the closure but does not need to be passed again at every step of the loop.


%%% Local Variables:
%%% mode: LaTeX
%%% TeX-master: "coscheme"
%%% End:
