\begin{figure}[t!]
\centering
  
\begin{alignat*}{2}
  \mathit{Value} &\ni{}& V, W
  &::= x
  \mid \lambda x. M
  \mid \Null
  \mid \Cons V \, W
  \mid \q{x}
  \\
  \mathit{EvalCxt} &\ni{}& E
  &::= \hole
  \mid E ~ M
  \mid V ~ E
  \mid \Match E \With \set{\many{P \to N}}
  \mid \Rec x = E
\end{alignat*}
% Operational axioms:
\begin{align*}
  (\beta)
  &&
  (\lambda x. M) ~ V
  &=
  M\subst{x}{V}
  % \\
  % \begin{aligned}
  %   &\Match V \With \\
  %   &\qquad\set{\many[i]{P_i \to N_i}}
  % \end{aligned}
  % &=
  % N_k\subst{BV(P_k)}{\many{W}}
  % &
  % \begin{aligned}
  %   (&\text{if } && V = P_k\subst{BV(P_k)}{\many{W}} \\
  %   &\text{and } &&\forall 1 \leq j < k, \\
  %   &&&\not\exists \many{W'}, V = P_j\subst{BV(P_j)}{\many{W'}})
  % \end{aligned}
  \\
  (\mathit{match})
  &&
  \begin{aligned}
    &\Match V \With \\
    &\qquad
    \begin{aligned}[t]
    \{~ &P \to N; \\
    &\many{P' \to N'}~\}
    \end{aligned}
  \end{aligned}
  &=
  N\subst{\many{x}}{\many{W}}
  &(\text{if } P\subst{\many{x}}{\many{W}} &= V)
  % &(\text{if } \exists \many{W},~ V &= P\subst{BV(P)}{\many{W}})
  \\
  (\mathit{apart})
  &&
  \begin{aligned}
    &\Match V \With \\
    &\qquad
    \begin{aligned}[t]
    \{~ &P \to N; \\
    &\many{P' \to N'}~\}
    \end{aligned}
  \end{aligned}
  &=
  \begin{aligned}
    &\Match V \With \\
    &\qquad \set{\many{P' \to N'}}
  \end{aligned}
  &(\text{if } P &\apart V)
  % &(\text{if}\!\not\exists \many{W},~ V &= P\subst{BV(P)}{\many{W}})
  \\
  (\mathit{rec})
  &&
  (\Rec x = V)
  &=
  V\subst{x}{(\Rec x = V)}
\end{align*}

% Observational axioms:
% \begin{align*}
%   % \lambda x. (V ~ x)
%   % &=
%   % V
%   % &(\text{if } x &\notin FV(V))
%   % \\
%   (\lambda x. E[x]) ~ M
%   &=
%   E[M]
%   \\
%   E\left[
%     \begin{aligned}
%       &\Match M \With \\
%       &\qquad\set{\many{P \to N}}
%     \end{aligned}
%   \right]
%   &=
%   \begin{aligned}
%     &\Match M \With \\
%     &\qquad \set{\many{P \to E[N]}}
%   \end{aligned}
%   &(\text{if } BV(P) \cap FV(E) = \emptyset)
% \end{align*}

Apartness between patterns and values ($P \apart V$):
\begin{gather*}
  \infer
  {\q{x} \apart V}
  {V \notin \mathit{Variable} \cup \set{\q{x}}}
  \qquad
  \infer
  {\Null \apart V}
  {V \notin \mathit{Variable} \cup \set{\Null}}
  \\[1ex]
  \infer
  {\Cons P ~ P' \apart V}
  {V \notin \mathit{Variable} \cup \set{\Cons W ~ W' \mid W, W' \in \mathit{Value}}}
  \\[1ex]
  \infer
  {\Cons P ~ P' \apart \Cons W ~ W'}
  {P \apart W}
  \qquad
  \infer
  {\Cons P ~ P' \apart \Cons W ~ W'}
  {P' \apart W'}
  % \infer
  % {K ~ P_1 \dots P_n \apart V}
  % {V \neq K ~ W_1 \dots W_n}
  % \qqqquad
  % \infer
  % {K~P_1 \dots P_n \apart K ~ V_1 \dots V_n}
  % {1 \leq j \leq n & P_j \apart V_j}
\end{gather*}

\caption{Untyped equational axioms of the target language.}
\label{fig:target-equality}
\end{figure}

We already used the translation to a core $\lambda$-calculus as a specification for implementing compositional copatterns, but the translation is also useful for another purpose: checking the expected meaning of copattern-matching code.
With that in mind, we now look for some laws that let us equationally reason about some programs to make sure they behave as expected.

First, the core target language --- a standard call-by-value $\lambda$-calculus extended with pattern-matching and recursion --- has the equational theory shown in \cref{fig:target-equality}, which is the \emph{reflexive}, \emph{symmetric}, \emph{transitive}, and \emph{compatible} (\ie equalities can be applied in \emph{any} context) closure of the listed rules.
It has the usual $\beta$ axiom (restricted to substituting value arguments), two axioms for handling pattern-match success ($\mathit{match}$) and failure ($\mathit{apart}$), and an axiom for unrolling recursive values ($\mathit{rec}$).
Values ($V, W$) include the usual ones for call-by-value $\lambda$-calculus ($x$ and $\lambda x. M$) as well as lists ($\Null$ and $\Cons V ~ W$) and symbolic literals ($\q x$).
Matching a value $V$ against a pattern $P$ will succeed if the variables ($\many{x}$) in the pattern can be replaced by other values ($\many{W}$) to generate exactly that $V$: $P\subst{\many{x}}{\many{W}} = V$.
In contrast, matching fails if the two are known to be \emph{apart}, written $P \apart V$ and defined in \cref{fig:target-equality}, which implies that all possible substitutions of $P$ will \emph{never} generate $V$.
Note that while matching and apartness are mutually exclusive, there are some values that are neither matching or apart from some patterns.
For example, compare the value $x$ against the pattern $\Null$; $x$ may indeed stand for $\Null$ or another value like $\lambda y. M$.

The first usual property is that the translation specified in \cref{fig:translation} is a \emph{conservative extension}: any two terms that are equal by the target equational theory are still equal after translation.
Because the translation is hygienic and compositional by definition,  we can follow the proof strategy in \cite{DownenAriola2014CSCC}.

\begin{restatable}[Conservative Extension]{proposition}{thmconservativeextension}
  \label{thm:conservative-extension}
  If $M = N$ in the equational theory of the target
  (\cref{fig:target-equality}), then so too does $\den{M} = \den{N}$.
\end{restatable}

\begin{figure}[t!]
\centering

\begin{alignat*}{2}
  % \mathit{TemplateValue} &\ni{}& B_v
  % &::= \varepsilon
  % \mid O_v; B_v
  % \mid \Continue x \to V
  % \\
  % \mathit{ExtensionValue} &\ni{}& O_v
  % &::= O_f
  % \mid O_f; O_v
  % \mid \Nest O_v
  % \mid \Try x \to B_v
  % \\
  \mathit{ExtensionFunc} &\ni{}& F
  &::= Q[x ~ P] ~ O
  \mid \lambda P. O
  \\
  \mathit{Value} &\ni{}& V
  &::= \dots
  \mid \lamstar (F; B)
  \mid \Template B
  \mid \Extension O
  \\
  % \mathit{NonRecTemplate} &\ni{}& B_{nr}
  % &::= O_{nr}; B_{nr}
  % \mid \Else \to M
  % \\
  % \mathit{NonRecExtension} &\ni{}& O_{nr}
  % &::= \varepsilon
  % \mid O_{nr}; O'_{nr}
  % \mid Q[\_] ~ O_{nr}
  % \mid \lambda P. O_{nr}
  % \\
  % &&&\phantom{:=}
  % \mid \Match P \gets M ~ O_{nr}
  % \mid \Nest O
  % \mid \Try x \to B_{nr}
  % \\
  % \mathit{RecCxt} &\ni{}& R
  % &::= \hole
  % \mid R; O
  % \mid O; R
  % \mid Q[x] ~ R
  % \mid \lambda P. R
  % \mid \Match P \gets M ~ O
  % \mid \Try x \to R
\end{alignat*}

Identity, associativity, and annihilation laws of composition:
\begin{align*}
  \varepsilon; O &= O % = O; \varepsilon
  &
  (O_1; O_2); O_3 &= O_1; (O_2; O_3)
  &
  \Do M; O &= \Do M
  \\
  \varepsilon; B &= B
  &
  (O_1; O_2); B &= O_1; (O_2; B)
  &
  \Do M; B &= \Else M
\end{align*}

% Decomposing patterns and copatterns:
% \begin{align*}
%   (Q[x] ~ P) ~ O
%   &=
%   Q[x] ~ (\lambda P. O)
%   &
%   \_ ~ O
%   &=
%   O
%   &
%   \lambda P. O
%   &=
%   \lambda x. (\Match P \gets x ~ O)
% \end{align*}

% Factoring out recursion ($x \neq y$ and $x \notin BV(P)$):
% \begin{align*}
%   \lambda y. (x ~ O)
%   &=
%   x ~ (\lambda y. O)
%   &
%   \Match P \gets M ~ (x ~ O)
%   &=
%   x ~ (\Match P \gets M ~ O)
%   \\
%   (x ~ O); O'
%   &=
%   x ~ (O; O')
%   &
%   O; (x ~ O')
%   &=
%   x ~ (O; O')
% \end{align*}

Instantiating templates and recursive $\lamstar$:
\begin{align*}
  % (\Extension O) ~ V
  % &=
  % \Template{} (O; \Continue x \to (V ~ x))
  % \\
  % (\Template R[Q[x] ~ O]) ~ V
  % &=
  % (\Template R[Q[\_] ~ O\subst{x}{V}]) ~ V
  % \\
  % (\Template R[\Continue x \to M]) ~ V
  % &=
  % (\Template R[\Else \to M\subst{x}{V}]) ~ V
  % \\
  % (\Template B_{nr}) ~ V
  % &=
  % (\Template B_{nr}) ~ W
  % \\
  \lamstar (F; B)
  &=
  (\Template F; B) ~ (\lamstar (F; B))
  % \\
  % \lambda x. (\lamstar (F; B)) ~ x
  % &=
  % \lamstar (F; B)
  \\
  (\Template O; B) ~ V
  &=
  (\Extension O) ~ (\Template B) ~ V
  \\
  (\Template \varepsilon) ~ V
  &=
  \mathit{fail}~V
  \\
  (\Template \Continue x \to M) ~ V
  &=
  M\subst{x}{V}
  \\
  (\Extension \Try x \to B) ~ V
  &=
  \Template B\subst{x}{V}
\end{align*}

Pattern and copattern matching:
\begin{align*}
  \Match P \gets V ~ O
  &=
  O\subst{\many{x}}{\many{W}}
  &(\text{if } P\subst{\many{x}}{\many{W}} &= V)
  \\
  \Match{} P \gets V ~ O
  &=
  \varepsilon
  &(\text{if } P &\apart V)
  \\[1ex]
  (\Template{} (\lambda P. \Do M); B) ~ V' ~ V
  &=
  M\subst{\many{x}}{\many{W}}
  &(\text{if } P\subst{\many{x}}{\many{W}} &= V)
  \\
  (\Template{} (\lambda P. O); B) ~ V' ~ V
  &=
  (\Template B) ~ V' ~ V
  &(\text{if } P &\apart V)
  \\[1ex]
  C[(\Template{} (Q[y] = M); B) ~ V]
  &=
  M\subst{y}{V}\subst{\many{x}}{\many{W}}
  &(\text{if } Q\subst{\many{x}}{\many{W}} &= C)
  \\
  C[(\Template{} (Q[y] ~ O); B) ~ V]
  &=
  C[(\Template B) ~ V]
  &(\text{if } Q &\apart C)
  \\[1ex]
  C[\lamstar (Q[y] = M); B]
  &=
  \begin{aligned}[t]
    M
    &\subst{y}{(\lamstar (Q[y] = M); B)}
    \\
    &\subst{\many{x}}{\many{W}}
  \end{aligned}
  &(\text{if } Q\subst{\many{x}}{\many{W}} &= C)
  \\
  C[\lamstar (Q[y] ~ O); \Else M]
  &=
  C[M]
  &(\text{if } Q &\apart C)
\end{align*}

Apartness between copatterns and contexts ($Q \apart C$):
\begin{gather*}
  \infer
  {Q ~ P \apart C ~ V}
  {Q\subst{\many{x}}{\many{W}} = C & P \apart V}
  \qqqquad
  \infer
  {Q ~ P \apart C}
  {Q \apart C}
  \qqqquad
  \infer
  {Q \apart C ~ V}
  {Q \apart C}
\end{gather*}

\caption{Some equalities of copattern extensions.}
\label{fig:source-equality}
\end{figure}

To reason about the new features in the source language --- introduced by $\lamstar$, $\Template$, and $\Extension$ --- we introduce additional axioms given in \cref{fig:source-equality}, so that the source equational theory is the \emph{reflexive}, \emph{symmetric}, \textit{transitive}, and \emph{compatible} closure of these rules in both \cref{fig:target-equality,fig:source-equality}.
The purpose of these new equalities is to peform some reasoning about programs using copatterns, and in particular, to check that the syntactic use of \scm|=| really means equality.
For example, a special case is
\begin{math}
  Q[\lamstar(Q[y] = M); B] = M\subst{y}{\lamstar(Q[y]=M);B}
  ,
\end{math}
which says a $\lamstar$ appearing in \emph{exactly} the same context as the left-hand side of an equation will unroll (recursively) to the right-hand side.
Other equations describe algebraic laws of copattern alternatives, and how to fill in templates and extensions when applied.
This source equational theory is \emph{sound} with respect to translation.

\begin{restatable}[Soundness]{proposition}{thmsoundness}
  \label{thm:soundness}
  The translation is \emph{sound} w.r.t. the source and target equational theories (\eg $M = N$ in \cref{fig:source-equality} implies $\den{M} = \den{N}$ in \cref{fig:target-equality}).
  % The equational axioms given in \cref{fig:source-equality} are sound with
  % respect to the translation in \cref{fig:translation},
  % \begin{align*}
  %   M &= M' &&\implies & \den{M} &= \den{M'} \\
  %   B &= B' &&\implies & T\den{B} &= T\den{B'} \\
  %   O &= O' &&\implies & E\den{O} &= E\den{O'}
  % \end{align*}
  % up to the equational theory of the target language in
  % \cref{fig:target-equality}.
\end{restatable}


%%% Local Variables:
%%% mode: LaTeX
%%% TeX-master: "coscheme"
%%% End:
