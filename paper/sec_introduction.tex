For decades, functional programmers have had a reliable and versatile method for representing tree-shaped structures: inductive data types.
These can model data of any size --- for example, lists of an arbitrary length --- but each instance must be \emph{finite}.
Infinite information --- like a stream of input that can go on forever --- does not fit into an inductive type, so the programmer must use some other representation to model potentially infinite objects.

Fortunately, the inductive data types used by functional programmers every day have a polar opposite: \emph{coinductive codata types}.
The \emph{coinductive} descriptor signals that values of the type may contain infinite information.
Haskell programmers are already well-versed in coinductive styles of types since non-strict languages blur the line between induction and coinduction.
For example, consider the usual example of the infinite list of Fibonacci numbers in Haskell:
\begin{minted}{haskell}
fibs = 0 : 1 : zipWith (+) fib (tail fib)
\end{minted}
\hs|fibs| cannot be fully evaluated because it has no base case --- it would eventually expand out to \hs|0 : 1 : 1 : 2 : 3 : 5 : 8 : ...| forever --- but this is no problem in a non-strict language that only evaluates as much as needed.

In contrast, \emph{codata} describes types defined by primitive \emph{destructors} that \emph{use} values of the codata type, as opposed to the primitive constructors that define how to build values of a data type.
For example, the usual \agda|Stream a| codata type of infinite \agda|a|'s is defined by two destructors: \agda|Head : Stream a -> a| extracts the first element and \agda|Tail : Stream a -> Stream a| discards the first element and returns the rest.
To define new streams, we can describe how they react to different combinations of \agda|Head| and \agda|Tail| destructions using \emph{copatterns}~\cite{APTS2013C}.  The copattern-based definition of the same \agda|fibs| function above is:
\begin{minted}{agda}
fibs : Stream Nat
Head fibs = 0
Head (Tail fibs) = 1
Tail (Tail fibs) = zipWith _+_ fibs (Tail fibs)
\end{minted}
Unfortunately, if we want to actually produce working code in this style, our choices are limited.
Agda gives the most full-fledged implementation of copatterns in a real system~\cite{ElaboratingDependentCopatterns}.
However, Agda is primarily a proof assistant rather than a general-purpose programming language, and as such, has different concerns than an ordinary functional programmer.
In particular, Agda currently does not understand if \agda|fibs| is well-founded --- it is --- and so \agda|fibs| is rejected by the termination checker.
There is also some support for copatterns in OCaml~\cite{LaforgueR17}, but as an unofficial extension that has not been merged into the main compiler.

We want to be able to write this kind of copattern-based code and have it fully integrated into a real, general-purpose programming language.
The easiest way to do so is to start with a programmable programming language~\cite{ProgrammablePL} and provide a new language extension as a library~\cite{LanguageLibrary}. We focus, in particular, on Scheme and Racket, which offers a robust macro system for seamlessly implementing new language features.

We show a new method for implementing copatterns in Scheme-like languages.
Rather than just recapping the previous macro-based definition of copatterns~\cite{LaforgueR17}, we also use macros, but we focus on providing new methods of extensibility not available before.
In particular, code defined by our copattern macros can be composed in a variety of ways,
offering a new solution to the expression problem~\cite{ExpressionProblem}.
Not only do copatterns allow us to easily define code using equational reasoning in Scheme, but these new dimensions of composition also allow us to capture some ``design patterns'' used by functional programmers as first-class abstractions.

Our primary contributions are organized as follows:
\begin{itemize}
\item \Cref{sec-examples} shows examples of programming with copattern equations in Scheme-like languages, including new forms of program composition --- vertical and horizontal --- that allows us to solve familiar examples of the expression problem~\cite{ExpressionProblem} through a fusion of functional and object-oriented techniques.
\item \Cref{sec-translation} presents a theory for how to translate copatterns into a small core target language --- untyped $\lambda$-calculus with recursion and patterns --- with a local double-barrel transformation reminiscent of selective continuation-passing style transformation.
  Importantly, only the new language constructs are transformed, while existing ones in the target language are unchanged.
\item \Cref{sec-macro} explains how to implement the high-level translation above in real code, and specifically how the implementation differs between Racket and a standard R${}^6$RS-compliant Scheme.
\item \Cref{sec-correctness} demonstrates correctness in terms of an equational theory for reasoning about copattern-matching code in the source language, which is a conservative extension of the target language, and we prove that it is sound with respect to translation.
\end{itemize} 
% The remainder of the article has the following structure: First, we introduce our implementation by explaining meaningful examples (Section \ref{sec-examples}).
% Second, we specify a core language with high-level features representing our implementation. Then we describe a translation into a target $\lambda$-calculus (Section \ref{sec-translation}).
% Third, we scrutinize our implementation, comparing each provided flavor (Section \ref{sec-macro}).
% Fourth, we present the properties of our system (Section \ref{sec-correctness}).
% Last, we explain the details of our optimized racket implementation (Section \ref{sec-opt}).


%%% Local Variables:
%%% mode: LaTeX
%%% TeX-master: "coscheme"
%%% End:
