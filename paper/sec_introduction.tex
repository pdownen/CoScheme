
Composition is one of the great promises of functional programming to combat complexity.
As opposed to monolothic solutions, functional programming languages encourage us to decompose large problems into small, reusable, and reliable parts and then to recompose them back into a whole solution \cite{Hughes1989WFPM}.
This practice is encouraged through tools like higher-order functions to abstract out common patterns and laziness to separate generation, selection, and consumption of information.
Rather than implementing a complex algorithm as a single special-purpose loop, functional programming lets us express the same solution as the composition of simple domain-specific operations and generic combinators: maps, filters, folds, and unfolds.

However, the \emph{expression problem}~\cite{ExpressionProblem} is a familiar foe that still resists this (de)compositional approach.
It captures the common problem that arises when we want to maintain code --- such as an evaluator for the syntax trees of an expression language --- by extending it in two different directions: adding new forms of data (\ie classes of objects) and new operations (\ie methods) on them.
Traditionally, functional languages can easily add new operations over any given data type, but adding a new constructor requires a major rewrite that can potentially alter the rest of the code.
Conversely, object-oriented languages make it easy to add a new class of object, but extending a base class with a new method again requires major rewriting.
Being a common obstacle in the way of maintaining, extending, and decomposing code, the expression problem has garnered many solutions in the object-oriented \cite{GangOfFour,wehr_javagi_2011} and functional \cite{swierstra_data_2008,keep_nanopass_2013} worlds, and especially hybrid languages that mix both \cite{BrachaC90Mixins,flatt1998mixin}.

This work presents a novel solution to the expression problem: composable copatterns.
Copatterns~\cite{APTS2013C} are often associated with codata types for expressing infinite objects, but their use is not limited to just that.
Their composition, in particular, allows us to define programs by performing equational reasoning in the evaluation context.
Performing the ``substitution of equals for equals''~\cite{wadler_critique_1987} enhances the predictability and composability of our programs since we can analyze our part code in isolation.

Previous implementations of copatterns can be found in strongly typed languages which impose prescribed restrictions on their use.
For example, Agda gives the most full-fledged implementation of copatterns in a real system~\cite{ElaboratingDependentCopatterns}.
However, Agda is primarily a proof assistant rather than a general-purpose programming language, and as such, has different concerns than an ordinary functional programmer.
There is also some support for copatterns in OCaml~\cite{LaforgueR17}, but as an unofficial extension that has not been merged into the main compiler.

The copatterns implemented here are also implemented as macros like \cite{LaforgueR17}; however, we present a different encoding that focuses on providing new methods of extensibility that were not available before, and can be desugared without any static typing information.
To achieve that, and to fully integrate it into a practical general-purpose programming language, we choose a programmable programming language~\cite{ProgrammablePL} and provide a new language extension as a library~\cite{LanguageLibrary}.
We focus, in particular, on Scheme and Racket, which offers a robust macro system for seamlessly implementing new language features.

Our extension presents three different composition flavors, allowing us to capture some ``design patterns'' used by functional programmers as first-class abstractions.
First, we have \emph{vertical} composition, which permits us to gather a collection of alternative options with failure handling.
Second, we have \emph{horizontal} composition, which permits us to compose a sequence of steps, parameters, matching, or guards.
Third, we have \emph{circular} composition, which allows us to recurse back on the entire composition itself.

Our primary contributions are organized as follows:
\begin{itemize}
\item \Cref{sec-examples} shows examples of programming with copattern equations in Scheme-like languages, including new forms of program composition --- vertical and horizontal --- that allows us to solve familiar examples of the expression problem~\cite{ExpressionProblem} through a fusion of functional and object-oriented techniques.
\item \Cref{sec-api} exposes the challenges related to implementing copatterns in this scenario, introduces our library API, shows how we can desugar our abstractions into a set of primitives and how the implementations differ between Racket and a standard R${}^6$RS-compliant Scheme.
% \item \Cref{sec-macro} explains how to implement the high-level translation above in real code, and specifically how the implementation differs between Racket and a standard R${}^6$RS-compliant Scheme.
\item \Cref{sec-translation} presents a theory for how to translate copatterns into a small core target language --- untyped $\lambda$-calculus with recursion and patterns --- with a local double-barrel transformation reminiscent of selective continuation-passing style transformation.
  Importantly, only the new language constructs are transformed, while existing ones in the target language are unchanged.
\item \Cref{sec-correctness} demonstrates correctness in terms of an equational theory for reasoning about copattern-matching code in the source language, which is a conservative extension of the target language, and we prove that it is sound with respect to translation.
\end{itemize} 
% The remainder of the article has the following structure: First, we introduce our implementation by explaining meaningful examples (Section \ref{sec-examples}).
% Second, we specify a core language with high-level features representing our implementation. Then we describe a translation into a target $\lambda$-calculus (Section \ref{sec-translation}).
% Third, we scrutinize our implementation, comparing each provided flavor (Section \ref{sec-macro}).
% Fourth, we present the properties of our system (Section \ref{sec-correctness}).
% Last, we explain the details of our optimized racket implementation (Section \ref{sec-opt}).


%%% Local Variables:
%%% mode: LaTeX
%%% TeX-master: "coscheme"
%%% End:
