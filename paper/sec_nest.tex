Since our implementation did not follow a monolithic macro approach --- instead favoring first-class values defined by small macros or ordinary procedures --- the programmer can easily expand copatterns with new features that combine seamlessly with the rest.
To ilustrate, we can add new operations, such as:

\subsection{Nest: Redefine yourself!}
One remark about the hidden \scm|self| parameter --- which is visible as the root of any copattern --- is \emph{always} the same view of the entire object.
That means nesting multiple copatterns in sequence might not give the expected result because the \scm|self| parameter in the hole of every copattern context will be bound to the same value.
Nevertheless, if we want the parameter in the hole of every copattern context to reflect the object at that point in time --- that is, be assigned the value given by the partial applications given by the preceding copatterns --- we can define a new operation, \scm|nest|.
This can be useful because nesting copatterns in a sequence gives us a shorthand for the common functional idiom of a ``local'' loop that closes over some parameters that never change, such as this definition of mapping a function over a list:
\begin{minted}{scheme}
(define*
  [(map* f xs) = ((map* f) xs)]
  [(map* f) (nest)
    (extension
     [(go `())         = `()]
     [(go `(,x . ,xs)) = `(,(f x) . ,(go xs))])])
\end{minted}
The \scm|map*| function supports both curried and uncurried applications, and they are defined to be equal.
Its real code is given in the curried case, where the function parameter \scm|f| is bound first and never changes.
Then, in a second step, we have the internal looping function \scm|go|, which matches over its list parameter and recurses with a new list.
By using \scm|nest|, we have \scm|go = (map* f)|, so that \scm|f| is visible from the closure but does not need to be passed again at every step of the loop.

On the implementation side, the \scm|nest| operation is a regular function from extensions to extensions.
This operation updates an extension's current view of itself from \scm{here} --- the partial application of all the arguments seen so far --- for its remainder by providing a new recursive biding to \scm{here} and using it as the new self.
\begin{minted}{scheme}
(define (nest ext)
  (try next there
    (letrec ([here (apply-extension
                    ext
                    (non-rec (next there))
                    (lambda args (apply here args)))])
      here)))
\end{minted}

In the theory side, we can extend the extensions of our source language with one more form: $\Nest O$, which allows us to nest multiple copatterns with a partially applied self object. 
Finally, the translation of the $\den{\Nest O}$ operation is defined by recursively creating a new value for the self parameter by recursively taking a new snapshot of how the object looks now after all the preceding applications and matchings have already occurred.
\begin{align*}
  E\den{\Nest O}
  &=
  \lambda b. \lambda s. \Rec s' = E\den{O} ~ (\lambda \_. b ~ s) ~ (\lambda x. s' ~ x)
\end{align*}


%%% Local Variables:
%%% mode: LaTeX
%%% TeX-master: "coscheme"
%%% End:
