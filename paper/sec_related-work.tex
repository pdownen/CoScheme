While we use the expression problem as a motivating application of compositional copatterns, it has multiple previous solutions using various features and spreading on multiple paradigms.
The Visitor pattern~\cite{GangOfFour} is a classic solution in the object-oriented world.
However, we have other options such as~\cite{wehr_javagi_2011}, which brings Java interfaces closer to Haskell's type classes, and~\cite{hutchison_extensibility_2012}, which presents the abstraction of object algebras that are easily integrated into OO languages, since they do not depend upon fancy features.
On the other hand, on the functional paradigm side, we have solutions like~\cite{swierstra_data_2008} and~\cite{keep_nanopass_2013}.
The former utilizes Haskell's type system, specifically type classes, to provide a modular way to construct data types and functions.
However, this approach relies on a type system.
The latter is a compiler framework that provides tools to specify different languages for the compiler passes. The framework achieves this by automatically handling a portion of the mapping between two given entities.

Previously, copatterns have been developed exclusively from the perspective of statically-typed languages.
Much of the work has been for dependently typed languages like Agda \cite{ElaboratingDependentCopatterns}, which use a type-driven approach to elaborate copatterns \cite{UnnestingCopatterns,ThibodeauMasters}.
The closest related works are the \cite{jeannin_cocaml_2017}, which implements an OCaml extension that supports regular coinductive data types, and \cite{LaforgueR17}, which implements copatterns as an OCaml macro.
CoCaml\cite{jeannin_cocaml_2017} supports only a subset of coinductive data types, and \cite{LaforgueR17} is too concerned with type system ramifications.
Here, we show how to implement copatterns with no typing information and focus instead on composition and equational reasoning.
The literature also presents other examples of open recursion and inheritance in a functional setting; for instance, \cite{Brown2009FunctionI} implements memorization for monadic computations using inheritance and mixins.

The translation in \cref{fig:translation} is reminiscent of ``double-barreled CPS'' \cite{DoubleBarrelCPS} used to define control effects like delimited control \cite{AbstractingControl} and exceptions \cite{KimYiDanvy98}.
In our case, rather than a ``successful return path'' continuation, there is a ``resume recursion'' continuation.
Expressions that return successfully just return as normal, similar to a selective CPS \cite{SelectiveCPS}, which makes it possible to implement as a macro expansion.
% In this sense, copatterns preserve the expressive power of the language \cite{ExpressivePower}.
A ``next case'' continuation --- to handle copattern-matching failure --- is introduced to make each line of a copattern-based definition a separate first-class value.
From that point, the ``recursive self'' must be a parameter because no one sliver of a definition suffices to describe the whole.

Theories of object-oriented languages \cite{abadi96,CookP94} also model the ``self'' keyword as a parameter later instantiated by recursion; either as an explicit recursive binding, or encoded as self-application.
This is done to handle the implicit composition of code from inheritance, whereas here, we need to handle explicit composition of first-class extensible objects.
The full connection between copatterns --- as we describe here --- and object-oriented languages remains to be seen.
In terms of the Lisp family of languages, the approach here seems closest to a first-class generalization of \emph{mixins} \cite{BrachaC90Mixins,flatt1998mixin} with a simple dispatch mechanism (matching), in contrast to class-based frameworks focused on complex dispatch \cite{CLOS,Ingalls86,chambers1992}.


%%% Local Variables:
%%% mode: LaTeX
%%% TeX-master: "coscheme"
%%% End:
