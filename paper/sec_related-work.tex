Previously, copatterns have been developed exclusively from the perspective of statically-typed languages.
Much of the work has been for dependently typed languages like Agda \cite{ElaboratingDependentCopatterns}, which use a type-driven approach to elaborate copatterns \cite{UnnestingCopatterns,ThibodeauMasters}.
The closest related work is the implementation of copatterns as an OCaml macro \cite{LaforgueR17}, but this, too, is concerned with type system ramifications.
Here, we show how to implement copatterns with no typing information, and focus instead on composition and equational reasoning.

The translation in \cref{fig:translation} is reminiscent of ``double-barrelled CPS'' \cite{DoubleBarrelCPS} used to define control effects like delimited control \cite{AbstractingControl} and exceptions \cite{KimYiDanvy98}.
In our case, rather than a ``successful return path'' continuation, there is a ``resume recursion'' continuation.
Expressions that return successfully just return as normal, similar to a selective CPS \cite{SelectiveCPS}, which makes the it possible to implement as a macro expansion.
In this sense, copatterns preserve the expressive power of the language \cite{ExpressivePower}.
A ``next case'' continuation --- to handle copattern-matching failure --- is introduced to make each line of a copattern-based definition a separate first-class value.
From that point, the ``recursive self'' must be a parameter, because no one sliver of a definition suffices to describe the whole.

Theories of object-oriented languages \cite{abadi96,CookP94} also model the ``self'' keyword as a parameter later instantiated by recursion; either as an explicit recursive binding, or encoded as self-application.
This is done to handle the implicit composition of code from inheritance, whereas here we need to handle explicit composition of first-class extensible objects.
The full connection between copatterns --- as we describe here --- and object-oriented languages remains to be seen.
In terms of the lisp family of languages, the approach here seems closest to a first-class generalization of \emph{mixins} \cite{BrachaC90Mixins,flatt1998mixin} with a simple dispatch mechanism (matching), in contrast to class-based frameworks focused on complex dispatch \cite{CLOS,Ingalls86,chambers1992}.


%%% Local Variables:
%%% mode: LaTeX
%%% TeX-master: "coscheme"
%%% End:
