% \documentclass[aspectratio=169]{beamer}
\documentclass{beamer}

\usepackage{stmaryrd}
\usepackage{cmll}
\usepackage{braket}
\usepackage{tikz}
\usepackage{tikz-cd}
\usepackage{listings}
\usepackage{minted}

\usepackage{preamble}
\usepackage{slides-preamble}

\DeclareUnicodeCharacter{3BB}{$\lambda$}

\setlength{\jot}{0pt}

\setbeamersize{text margin left = 1em}

\setminted{fontsize=\footnotesize}

\setbeamercolor{alerted text}{fg=orange}
\colorlet{whisper}{gray}
\colorlet{good}{green}
\colorlet{bad}{red}

\newcommand{\whisper}[1]{{\color{whisper}{#1}}}

\title{CoScheme: Compositional Copatterns in Scheme}
\subtitle{Or, ``Equals'' Means Equals}
\author{
  {\large\bf \emph{Paul Downen} and Adriano Corbelino II}
  \\
  University of Massachusetts Lowell
}
\date{TFP --- Thursday, January 16, 2025}

\begin{document}
\maketitle

% 25 minute time limit (~20 minute talk, ~5 minute Q&A)

\begin{frame}
  \frametitle{Why Copatterns in Scheme?}
  \framesubtitle{Composition, Composition, Composition!}

  \begin{pointed}
  \item Programs defined by equational reasoning on their context \\ (\ala ML, Haskell)
  \item Composition of extensible fragments at run-time
    \begin{pointed}
    \item \alert{Vertical} --- \textit{either or} --- compose alternative
      options, handling failure
    \item \alert{Horizontal} --- \textit{and then} --- compose sequence of
      steps, parameters, matching, guards
    \item \alert{Circular} --- \textit{self} --- recursion back on the entire
      composition itself
    \end{pointed}
  \item Library of composable macros
  \item Side benefit: supports infinite objects, some OO-style designs
  \end{pointed}
\end{frame}

\section{Copatterns in Scheme}

\begin{frame}[fragile]
\frametitle{Encoding Functional Equations}
\framesubtitle{Procedural style via manual operations}

\begin{haskell}
zip (x:xs) (y:ys) = (x, y) : zip xs ys
zip xs     ys     = []
\end{haskell}

\vspace{2.75em}
\pause

\begin{scheme}
(define (zip xs ys)
  (cond
    [(and (pair? xs) (pair? ys))
     (cons (cons (car xs) (car ys))
           (zip (cdr xs) (cdr ys)))]
    [else '()]))
\end{scheme}
\end{frame}

\begin{frame}[fragile]
\frametitle{Encoding Functional Equations}
\framesubtitle{Hybrid style via pattern matching}

\begin{haskell}
zip (x:xs) (y:ys) = (x, y) : zip xs ys
zip xs     ys     = []
\end{haskell}

\vspace{0.5em}

\begin{scheme}
(define (zip xs ys)
  (match xs
    [`(,x . ,xs-rest)
     (match ys
       [`(,y . ,ys-rest)
        `((,x . ,y) . ,(zip xs-rest ys-rest))]
       [_ '()])]
    [_ '()]))
\end{scheme}
\end{frame}

\begin{frame}[fragile]
\frametitle{Encoding Functional Equations}
\framesubtitle{Equational style via copattern matching}

\begin{haskell}
zip (x:xs) (y:ys) = (x, y) : zip xs ys
zip xs     ys     = []
\end{haskell}

\vspace{5.25em}

\begin{scheme}
(define*
  [(zip `(,x . ,xs) `(,y . ,ys))
  = `((,x . ,y) . ,(zip xs ys))]
  [(zip xs ys) = `()])
\end{scheme}
\end{frame}

\begin{frame}[fragile]
\frametitle{Encoding Infinite Objects}
\framesubtitle{Streams observed through head and tail projecections}

Stuttering stream from 0:
\begin{alignat*}{5}
  &stutter~&&0 &&= 0, 0, 1, 1, 2, 2, 3, 3, \dots
\end{alignat*}
Stuttering stream from n:
\begin{alignat*}{5}
  &stutter~&&n &&= n, n, n+1, n+1, n+2, n+2, n+3, n+3, \dots
\end{alignat*}

\pause

$\Stream a = (\mathtt{'head} \to a) ~\&~ (\mathtt{'tail} \to \Stream a)$

\begin{scheme}
(define*
  [ ((stutter n) 'head)        = n]
  [(((stutter n) 'tail) 'head) = n]
  [(((stutter n) 'tail) 'tail) = (stutter (+ n 1))])
\end{scheme}
\end{frame}

\begin{frame}[fragile]
\frametitle{Equational Reasoning Pop Quiz}
\framesubtitle{Counter objects}

\begin{scheme}
(define*
  [((counter x) 'add y) = (counter (+ x y))]
  [((counter x) 'get)   = x])
\end{scheme}

\scm|((counter 5) 'get) = | \pause \scm|5|

\pause

\scm|(((counter 5) 'add 6) 'get)|
\pause

\scm|= ((counter 11) 'get)  = 11|

\pause

\begin{scheme}
(define c (counter 10))
\end{scheme}

\scm|(list ((c 'add 2) 'get) ((c 'add 4) 'get))|

\pause

\scm|= (list ((counter 12) 'get) ((counter 14) 'get))|

\scm|= '(12 14)|
\end{frame}

\section{The \\ Expression Problem}

\begin{frame}[fragile]
\frametitle{A Simple Evaluator}
\framesubtitle{Arithmetic expressions with just numbers and addition}  

\begin{scheme}
;; Expr = `(num ,Number)
;;      | `(add ,Expr ,Expr)

;; expr0 : Expr
(define expr0
  '(add (num 1) (add (num 2) (num 3))))

;; eval : Expr -> Number
(define*
  [(eval `(num ,n))    = n]
  [(eval `(add ,l ,r)) = (+ (eval l) (eval r))])
\end{scheme}
\end{frame}

\begin{frame}[fragile]
\frametitle{The Expression Problem}

Easy to add new operations:
\begin{scheme}
;; expr? : Any -> Bool
(define*
  [(expr? `(num ,n))    = (number? n)]
  [(expr? `(add ,l ,r)) = (and (expr? l) (expr? r))]
  [(expr? _)            = #f])

;; list-nums : Expr -> List Number
(define*
  [(list-nums `(num ,n))    = (list n)]
  [(list-nums `(add ,l ,r))
  = (append (list-nums l) (list-nums r))])
\end{scheme}

Hard to add new expression cases:
\begin{scheme}
;; Expr = ... | `(mul ,Expr ,Expr)
???
\end{scheme}
\end{frame}

\begin{frame}[fragile]
\frametitle{A De/Re-composed Evaluator}
\framesubtitle{As extensible objects}

\begin{scheme}
(define-object
  [(eval-num `(num ,n)) = n])

(define-object
  [(eval-add `(add ,l ,r))
  = (+ (eval-add l) (eval-add r))])

;; same as before, but now defined via composition
(define eval-obj (eval-num 'compose eval-add))
\end{scheme}

\pause

\begin{scheme}
eval-obj expr = eval expr

eval-obj = (object
             [(eval-num `(num ,n))    = n]
             [(eval-add `(add ,l ,r))
             = (+ (eval-add l) (eval-add r))])
\end{scheme}
\end{frame}

\begin{frame}[fragile]
\frametitle{Simple Extensions of the Expression Language}
\framesubtitle{Vertical composition}

\begin{scheme}
;; Expr = ... | `(mul ,Expr ,Expr)

(define-object
  [(eval-mul `(mul ,l ,r))
  = (* (eval-mul l) (eval-mul r))])

(define eval-arith
  (eval-obj 'compose eval-mul))
\end{scheme}

\pause

\begin{scheme}
eval-arith
=
(object
  [(eval-num `(num ,n))    = n]
  [(eval-add `(add ,l ,r))
  = (+ (eval-add l) (eval-add r))]
  [(eval-mul `(mul ,l ,r))
  = (* (eval-mul l) (eval-mul r))])
\end{scheme}
\end{frame}

\begin{frame}[fragile]
\frametitle{Challenge: Adding Variables \& Environments}

Evaluating variable expressions requires an environment
\begin{scheme}
;; Expr = ... | `(var ,Symbol)

(define-object
  [(eval-var env `(var ,x)) = (dict-ref env x)])
\end{scheme}

\begin{scheme}
(eval-var '((x . 10) (y . 20)) '(var y)) = 20
\end{scheme}

How to compose (binary) \scm|eval-var| with (unary) \scm|eval-arith|?
\end{frame}

\begin{frame}[fragile]
\frametitle{Solution 1: Fixing the Environment}
\framesubtitle{Run-time vertical composition}

\begin{scheme}
(define (fix-environment alg-evaluator env)
  (object
    [(_ expr)
     (try-apply-forget alg-evaluator env expr)]))
;; try-apply-forget attemps an application,
;; if it fails to match, continue with next option
   
(define-object
  [(eval-alg env expr)
  = (((fix-environment (eval-var 'unplug) env)
      'compose eval-arith)
     expr)])
;; 'unplug is an inherited-by-default method of objects
;; that converts it to a composable extension
\end{scheme}
\end{frame}

\begin{frame}[fragile]
\frametitle{Solution 2: Threading the Environment}
\framesubtitle{Horizontal composition \& first-class failure continuation}

\vspace{-2em}
\begin{alignat*}{2}
  &eval ~ expr &&= \dots eval ~ subexpr \dots \\
  &&&\Downarrow \\
  &eval ~ env ~ expr &&= \dots eval ~ env ~ subexpr \dots
\end{alignat*}

\pause

\begin{scheme}[fontsize=\scriptsize]
(define (with-environment arith-evaluator)
  (object
   [(self env expr)
    (with-self
        (override-λ* self
          [(_ sub-expr) = (self env sub-expr)])
      (try-apply-forget arith-evaluator expr))]))
;; Implement environment extension by hiding environment
;; and overriding its concept of "self" to bring it back

(define eval-alg
  ((with-environment (eval-arith 'unplug))
   'compose eval-var))
\end{scheme}
\end{frame}

\begin{frame}[fragile]
\frametitle{Challenge: Constant Folding}
\framesubtitle{Doing what you can}

Instead of using an environment to evaluate variables, just leave them alone:
\begin{scheme}
(constant-fold '(mul (add (num 1) (num 2))
                     (add (var x) (num 4))))
=
               '(mul (num 3)
                     (add (var x) (num 4)))
\end{scheme}
\end{frame}

\begin{frame}[fragile]
\frametitle{Intermezzo: More Cautious Error Handling}
\framesubtitle{Safer arithmetic}

\begin{scheme}
(define-object eval-add-safe
  [(self `(add ,l ,r))
  = (self 'add (self l) (self r))]
  [(self 'add x y)
   (try-if (and (number? x) (number? y)))
  = (+ x y)])

(define-object eval-mul-safe
  [(self `(mul ,l ,r))
  = (self 'mul (self l) (self r))]
  [(self 'mul x y)
   (try-if (and (number? x) (number? y)))
  = (* x y)])

(define eval-arith-safe
  (eval-num 'compose eval-add-safe eval-mul-safe))
\end{scheme}
\end{frame}

\begin{frame}[fragile]
\frametitle{Leave Variables Alone}
\framesubtitle{And reform blocked expressions}

\begin{scheme}
(define-object
  [(leave-variables `(var ,x)) = `(var ,x)])

\end{scheme}
\pause
\begin{scheme}
(define (operation? s)
  (or (equal? s 'add) (equal? s 'mul)))

(define-object reform-operations
  [(reform op l r) (try-if (operation? op))
                   (try-if number? l)
  = (reform op `(num ,l) r)]
  [(reform op l r) (try-if (operation? op))
                   (try-if number? r)
  = (reform op l `(num ,r))]
  [(reform op l r) (try-if (operation? op))
  = (list op l r)])
\end{scheme}
\end{frame}

\begin{frame}[fragile]
\frametitle{Solution: Constant Folding Made Easy}
\framesubtitle{Simple vertical composition}

\begin{scheme}
(define constant-fold
  (eval-arith-safe 'compose
                   leave-variables
                   reform-operations))
\end{scheme}
\pause
\begin{scheme}
(define expr3
  '(add (add (num 1) (num 1))
        (mul (var x)
             (mul (num 2) (add (num 2) (num 3))))))

(constant-fold expr3)
= '(add (num 2) (mul (var x) (num 10)))
\end{scheme}
\end{frame}

\section{(How) \\ Does It Work?}

\begin{frame}
\frametitle{The Tower of Extensibility}
\framesubtitle{Objects, Templates, Extensions}

\begin{align*}
  Object
  &=
  \text{some type of usable function}
  \\\\
  Template
  &=
  \alt<1>{Object}{(self : Object)} \to Object'
  \\\\
  Extension
  &=
  \alt<1-2>{Template}{(next : Template)} \to Template'
  \\
  &=
  \alt<1-2>{Template}{(next : Template)}
  \to \alt<1-2>{Object}{(self : Object)}
  \to Object'
\end{align*}

\pause\pause
\end{frame}

\let\mintscheme\scheme
\let\mintendscheme\endscheme
\let\scheme\relax
\let\endscheme\relax
\lstnewenvironment{scheme}{\lstset{language=Scheme}}{}

\begin{frame}[fragile]
\frametitle{Expanding Definition Macros}

\begin{scheme}
(define* name clause ...)
= (define name (λ* clause ...))

(define-object name clause ...)
= (define name (object clause ...))
\end{scheme}
\pause
\begin{scheme}
(λ* clause ...) = (introspect (template clause ...))

(object (<: mod) clause ...)
= (plug (mod (extension clause ...)))

(object clause ...)
= (object (<: (default-object-modifier)) clause ...)
\end{scheme}
\end{frame}

\begin{frame}[fragile]
\frametitle{Expanding ``Big'' Macros}

\begin{scheme}
(template clause ...)
= (closed-cases (extension clause ...))

(extension [copat step ...] ...)
= (compose [chain (comatch copat) step ...] ...)
\end{scheme}
\pause
\begin{scheme}
(chain (op ...) step ... ext)
= (op ... (chain step ... ext))
(chain = expr)
= (always-is expr)
\end{scheme}
\end{frame}

\begin{frame}[fragile]
\frametitle{Some ``Small'' Macros}

\begin{scheme}[fontsize=\small]
(try next self expr)
= (λ(next) (λ(self) expr))

(always-is expr) = (try _ _ expr)

(try-if check ext)
= (try next self
       (if check
           ((ext next) self)
           (next self)))
\end{scheme}
\pause
\begin{scheme}[fontsize=\small]
(try-λ x ext)
= (try next self
       (λ x
         ((ext (λ(s) (apply (next s) x)))
          self)))
\end{scheme}
\end{frame}

\begin{frame}
\frametitle{Soundness of Equational Reasoning}

\begin{pointed}
\item Model macros as a (selective) CPS-like translation
  % on a specified syntactic grammar
\item Target: $\lambda$-calculus + recursion + symbols + lists + patterns
\item Source: Target + copatterns + templates + extensions + $\lamstar$
\item Translations from Source to Target:
  \begin{align*}
    \den{\_}  &: Term \to Target && \text{(only translates new forms)} \\
    T\den{\_} &: Template \to Target && \text{(always a unary function)} \\
    E\den{\_} &: Extension \to Target && \text{(always a binary function)}
  \end{align*}
\end{pointed}
\pause
\begin{theorem}[Conservative Extension]
  If $M = N$ in the \alert{target} theory, then $M = N$ in the \alert{source}
  theory.
\end{theorem}
\begin{theorem}[Soundness]
  If $M = N$ in the \alert{source} theory, then $\den{M} = \den{N}$ in the
  \alert{target} theory.
\end{theorem}
\end{frame}

\begin{frame}[fragile]
\frametitle{(Co)Pattern Matching Equalities}

\begin{pointed}
\item Patterns $P$ against \emph{values} $V$:
  \begin{itemize}
  \item Matching iff $P[\many{W}/\many{x}] = V$
  \item Apart iff $P \apart V$ (inductively defined over structure of $P$)
  \end{itemize}
  \pause
\item Copatterns $Q$ against \emph{contexts} $C$
  \begin{itemize}
  \item Matching iff $Q[\many{W}/\many{x}] = C$
  \item Apart iff $Q \apart C$ (inductively defined over structure of $Q$)
  \end{itemize}
\end{pointed}
\pause
\begin{scheme}[fontsize=\small]
(try-match P V ext)
= ext[W.../x...]      ; if P[W.../x...] = V
(try-match P V ext)
= empty-extension     ; if P # V

C[(λ* [Q[y] = expr] clause ...)]
= expr[W.../x...]     ; if Q[W.../x...] = C
C[(λ* [Q[y] = expr] [else deflt])]
= C[deflt]            ; if Q # C
\end{scheme}
\end{frame}

\let\scheme\mintscheme
\let\endscheme\mintendscheme

\begin{frame}
\frametitle{CoScheme: Composable, Equational, Copatterns!}
\framesubtitle{Try It Yourself!}

\begin{itemize}
\item \url{https://github.com/pdownen/CoScheme}
\begin{center}
\includegraphics[keepaspectratio,width=0.75\textwidth,height=0.5\textheight]{qrcode.png}
\end{center}
\item Racket
\item R${}^6$RS
\end{itemize}
\end{frame}

% \appendix

\section{Factoring Copatterns}

\begin{frame}[fragile]
\frametitle{Nested definitions}
\framesubtitle{Sharing constant parameters of a loop}  

\begin{haskell}
map f []     = []
map f (x:xs) = f x : map f xs

map f xs = go xs                -- map f = go
  where go []     = []
        go (x:xs) = f x : go xs
\end{haskell}

\pause
\vspace{2em}

\begin{scheme}
(define*
  [(map f xs) = ((map f) xs)]  ; (un)curried forms equal
  [(map f) (nest)              ; map f = go
    (extension
     [(go `())         = `()]
     [(go `(,x . ,xs)) = `(,(f x) . ,(go xs))])])
\end{scheme}
\end{frame}

\begin{frame}[fragile]
\frametitle{Refactoring a Common Prefix}
\framesubtitle{Sharing copatterns and computations}

\begin{scheme}
(define-object reform-operations
  [(reform op l r) (try-if (operation? op))
                   (try-if number? l)
  = (reform op `(num ,l) r)]
  [(reform op l r) (try-if (operation? op))
                   (try-if number? r)
  = (reform op l `(num ,r))]
  [(reform op l r) (try-if (operation? op))
  = (list op l r)])
\end{scheme}
common prefix: \scm|(reform op l r) (try-if (operation? op))|
\pause
\begin{scheme}
(define-object reform-operations
  [(reform op l r) (try-if (operation? op))
   (extension
    [_ (try-if (number? l)) = (reform op `(num ,l) r)]
    [_ (try-if (number? r)) = (reform op l `(num ,r))]
    [_                      = (list op l r)])])
\end{scheme}
\end{frame}

\end{document}
